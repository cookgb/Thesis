\chapterhead{Chapter~\the\chapnum:  The Hamiltonian Constraint and\cr
Boundary Conditions}
%%%%%%%%%%%%%%%%%%%
The final step in the construction of initial-data sets via the conformal
imaging approach is the solution of the Hamiltonian constraint (4.8).  As shown
in Chapter 4, the domain in which a solution of the Hamiltonian constraint must
be found is $E^3-\{a_\alpha\}$.  That is, a three-dimensional Euclidean
flat space with $N$ spheres cut out, one for each black hole.  If proper boundary
conditions can be posed on the surfaces of these deleted spheres and at
infinity, then the Hamiltonian constraint can be posed as a quasi-linear,
elliptic boundary value problem.  Bowen and York [1980] have derived such
boundary conditions and these will be described below.  I will also discuss some
known, exact solutions to the Hamiltonian constraint and the approach taken for
finding numerical solutions.

In order to pose the Hamiltonian constraint as a boundary value problem,
boundary conditions must be found for all surfaces.  The boundary condition at
infinity has been discussed in Chapter~4.  It is fixed by the assumption that
the initial-data slice is asymptotically flat.  This, together with the choice
of a flat background metric, demands that the conformal factor behave like
(4.10).  Thus, the first boundary condition is that the conformal factor is
{\it one} at infinity.  Along with the demand that the metric be regular, this
implies that the conformal factor always be greater than zero.  In practice,
infinity is often not part of the domain of solution for the conformal factor
when numerical solutions are found.  In this case, an approximate boundary
condition must be used at large distances from the holes.  York and Piran [1982]
have suggested and used an approximate boundary condition which has become
widely accepted.  I will discuss it later in this chapter along with an
improvement which I have derived.

In order for the solution of the Hamiltonian constraint to be inversion
symmetric, it must satisfy the isometry condition (4.43) for every hole in the
system.  Consider the $\alpha^{th}$ coordinate patch and take a derivative of the
isometry condition.  The result is
$$
{\bar{D}}{}_{i}\psi \left({\bf x}\right) =-{{a}_{\alpha } \over
{r}_{\alpha }^{2}}{n}^{\alpha }_{i}\psi \left({{\bf J}_{\alpha
}\left({\bf x}\right)}\right) +{{a}_{\alpha } \over {r}_{\alpha
}}{\left({{\bf J}_{\alpha
}}\right)}{}_{i}{}^{j}{\left[{{\bar{D}}{}_{j}\psi \left({\bf
x}\right)}\right]}_{{\bf J}_{\alpha }\left({\bf x}\right)}. \eqn{}
$$
If this is contracted with the $\alpha^{th}$ unit normal and evaluated on the
$\alpha^{th}$ inversion boundary, the result is
$$
{\left.{{n}_{\alpha }^{i}{\bar{D}}{}_{i}\psi }\right|}_{{a}_{\alpha
}}=-{\left.{{\psi  \over 2{r}_{\alpha }}}\right|}_{{a}_{\alpha }}, \eqn{}
$$
where the fact that $r_\alpha = a_\alpha$ is a fixed point set of the isometry
has been used.  That is, ${\bf J}_\alpha(a_\alpha) = a_\alpha$.  (7.2)
represents a boundary condition which can be used on the surface of each deleted
sphere.  This reduces directly to the boundary condition derived by Bowen and
York [1980] in the case of a single hole and to the form shown by Kulkarni
{\it et al}. [1983] for the case of multiple holes.

Bowen and York [1980] have shown, in the case of a single hole, that the area of
the inversion boundary, or ``throat'', is extremal.  This result holds for the
case of multiple holes as well and can be seen as follows.  Consider the area of
a sphere of radius $r_\alpha$ centered around the $\alpha^{th}$ throat.
$$
{A}_{\alpha }=\oint_{}^{}{\psi }^{4}{r}_{\alpha }^{2}{d}^{2}{\Omega }_{\alpha
} \eqn{}
$$
The condition for extremal area is found by varying the area and setting the
variation to zero.
$$
\delta {A}_{\alpha }=2\oint_{}^{}{r}_{\alpha }{\psi }^{3}\left({2{r}_{\alpha
}{\bar{D}}{}_{i}\psi +\psi {n}^{\alpha }_{i}}\right){\delta
x}{}^{i}{d}^{2}{\Omega }_{\alpha }=0 \eqn{}
$$
Considering only changes normal to the surface, (7.4) gives
$$
{n}_{\alpha }^{i}{\bar{D}}{}_{i}\psi +{\psi  \over 2{r}_{\alpha }}=0,
\eqn{}
$$
which, together with (7.2), shows that the throats must be extremal surfaces.

The Hamiltonian constraint, together with the boundary conditions of asymptotic
flatness at infinity and inversion symmetry on the throats, constitutes a
quasi-linear, elliptic boundary value problem.  In order for this boundary value
problem to be well posed, it is necessary to show that any solution satisfying
the boundary conditions is unique.  Two difficulties arise in trying to prove
this.  One arises from the non-linearity of the Hamiltonian constraint and the
other from the inversion symmetry boundary condition (7.2).  In order to deal
with the non-linearity, one can consider a local uniqueness proof.  That is, it
is shown that no other solutions lie in the neighborhood of a given solution. 
This does not preclude the existence of other solutions which are
``significantly different''.  If a standard, local uniqueness proof is attempted,
it is found that the relative sign in (7.2) prevents any conclusion.  This is
because (7.2) is an ``anti-Robin'' boundary condition, although it appears at
first sight to be a standard Robin condition ({\it cf}. Bergman and Schiffer
[1953]).  The problem arises because $n^i_\alpha$ points inward, into the domain
of the solution, not outward.

In spite of these problems, York [1989] has proven that the solution for a
single hole is locally unique.  This proof depends explicitly on the inversion
symmetry of both the conformal factor and the background extrinsic curvature. 
My attempts to generalize this proof to the case of multiple holes have been
unsuccessful.  The presence of extra holes makes the method inconclusive.  I
believe that the solution may depend on limits on the maximum hole-radius to
hole-separation ratio as do the convergence proofs of Misner [1963]  and
Kulkarni [1984] for the case of the infinite series which arise when multiple
holes are present.

In spite of the lack of any uniqueness proof in the case of multiple holes, I
will continue on the belief any solutions found will be locally unique and search
for solutions.  I will begin by looking at the asymptotic behavior of the
solutions to the Hamiltonian constraint.  In doing this, I will follow Bowen,
Rauber, and York [1984] and use a ``Newtonian'' potential $\Phi$ instead of the
conformal factor, where $\Phi$ is defined by
$$
\psi =1-{1 \over 2}\Phi . \eqn{}
$$

The Hamiltonian constraint is very much like a Poisson equation except that the
source term is coupled non-linearly to the conformal factor.  In terms of
$\Phi$, the Hamiltonian constraint is written
$$
{\overline{\nabla }}^{2}\Phi ={1 \over 4}{\psi
}^{-7}{\bar{A}}_{ij}{\bar{A}}{}^{i}{}^{j}\equiv {\kappa \over 2}{\rho
}_{\scriptscriptstyle eff}, \eqn{}
$$
where $\rho_{\scriptscriptstyle eff}$ is defined as an effective energy density
source.  If we assume for a moment that $\rho_{\scriptscriptstyle eff}$
has compact support, then a multipole expansion of $\Phi$ is well defined
outside the support and is given by 
$$
\Phi =-{\kappa  \over 8\pi }\left[{{E \over r}+{{d}{}_{i}{n}{}^{i} \over
{r}^{2}}+{3 \over 2}{{I}{}_{i}{}_{j}\left({{n}{}^{i}{n}{}^{j}-{1 \over
3}{f}{}^{i}{}^{j}}\right) \over {r}^{3}}+\Order\left({{r}^{-4}}\right)}\right].
\eqn{}
$$

The monopole coefficient $E$ is the total ADM energy contained in the
initial-data slice ({\it cf}. Arnowitt {\it et al}. [1962]).  The other multipole
coefficients are associated with the energy distribution on the initial slice. 
These multipole coefficients are defined in terms of the following standard
Cartesian surface integrals:
$$
E={2 \over \kappa }\oint_{\infty }^{}{\bar{D}}{}^{i}\Phi
{d}^{2}{\bar{S}}{}_{i}, \eqn{}
$$
$$
{d}{}_{i}={2 \over \kappa }\oint_{\infty
}^{}\left({{x}{}_{i}{\bar{D}}{}^{j}\Phi -{\delta }_{i}^{j}\Phi
}\right){d}^{2}{\bar{S}}{}_{j}, \eqn{}
$$
and
$$
{I}{}_{i}{}_{j}={2 \over \kappa }\oint_{\infty
}^{}\left[{\left({{x}{}_{i}{x}{}_{j}-{1 \over
3}{r}^{2}{f}{}_{i}{}_{j}}\right){\bar{D}}{}^{k}\Phi
-\left({{x}{}_{i}{\delta }_{j}^{k}+{x}{}_{j}{\delta }_{i}^{k}-{2 \over
3}{x}{}^{k}{f}{}_{i}{}_{j}}\right)\Phi }\right]{d}^{2}{\bar{S}}{}_{k}.
\eqn{}
$$

The effective source does not, of course, have compact support.  Because of
this, the multipole expansion will only be well defined through some given order
defined by the radial fall off rate of the source.  From the conditions for
asymptotic flatness (4.10) and (4.11), it is seen that the effective source
behaves as
$$
{\rho }_{\scriptscriptstyle eff}=\Order\left({{r}^{-4}}\right). \eqn{}
$$
This means that the monopole term, and thus the total energy, is always well
defined on an asymptotically flat slice.

The well behaved nature of the multipole expansion through monopole order is the
basis of an approximate boundary condition proposed by York and Piran [1982]. 
The conformal factor takes the form
$$
\psi =1+{\kappa  \over 16\pi }{E \over r}+\Order\left({{r}^{-2}}\right). \eqn{}
$$
Taking the radial derivative of (7.13) and eliminating the total energy term
results in the following Robin boundary condition:
$$
{\partial \psi  \over \partial r}={1-\psi  \over r}+{1 \over
r}\Order\left({{r}^{-2}}\right). \eqn{}
$$
Applied at large distances from the holes, (7.14) provides a good approximate
outer boundary condition.  Exactly how far away the outer boundary must be for
(7.14) to be a good approximation depends on the explicit form of the effective
source.

Evans [1989] has suggested that an improvement in the order of accuracy of the
outer boundary condition would be of great use in numerical solutions as it would
allow the outer boundary to be placed closer to the holes and reduce the size of
the numerical problem.  Prompted by this, I have explored the possibility of
incorporating higher order terms into the approximation.  In general, (7.8) is a
valid expansion for $\Phi$ only through the monopole term.  In order to make it
valid for higher order terms, it must be modified.  Consider a function
$\Lambda$ which is not annihilated by the Laplacian and which satisfies the
following: $$
{\overline{\nabla }}^{2}\Lambda =\Order\left({{r}^{-4}}\right)=4\pi {\rho
}_{\scriptscriptstyle eff}+\Order\left({{r}^{-5}}\right). \eqn{}
$$
$\Lambda$ is thus chosen to compensate for the highest order effects of the
effective source.  I demand that it have the following properties, in addition to
(7.15): 
$$
\Lambda =\Order\left({{r}^{-2}}\right) \eqn{}
$$
and
$$
{\bar{D}}{}_{i}\Lambda =\Order\left({{r}^{-3}}\right). \eqn{}
$$

Now consider expanding $\Phi$ as
$$
\Phi =-{\kappa  \over 8\pi }\left[{{E \over r}+{{\tilde{d}}{}_{i}{n}{}^{i}
\over {r}^{2}}-\Lambda +\Order\left({{r}^{-3}}\right)}\right]. \eqn{}
$$
The total energy given by the monopole coefficient is still defined by (7.9)
because of the fall-off condition (7.17).  The dipole term is modified and, in
terms of (7.10), is given by
$$
{\tilde{d}}{}_{i}={d}{}_{i}-{2 \over \kappa }\oint_{\infty
}^{}\left({{x}{}_{i}{\bar{D}}{}^{j}\Lambda -{\delta }_{i}^{j}\Lambda
}\right){d}^{2}{\bar{S}}{}_{j}. \eqn{}
$$
Given the fall-off conditions (7.16) and (7.17), the correction to the dipole
moment will not, in general, vanish and, thus, the ``dipole moment'' of (7.18) no
longer represents a multipole moment of the initial energy distribution.  If the
integral over $\Lambda$ in (7.19) were to vanish either by some symmetry or
because (7.16) and (7.17) fall off faster than necessary, then the dipole moment
would retain its original interpretation and the multipole expansion would be
well defined through dipole order.

To proceed, we must have an explicit form for the function $\Lambda$.  In order to
determine this, we must explore the asymptotic form of the effective source.
Since the potential $\Phi$ is always well defined through monopole order for
asymptotically flat initial data, the asymptotic form of the effective source is
given by
$$
{\rho }_{\scriptscriptstyle eff}={1 \over 2\kappa
}{\bar{A}}{}_{i}{}_{j}{\bar{A}}{}^{i}{}^{j}+\Order\left({{r}^{-5}}\right).
\eqn{}
$$
Thus, in order to fix $\Lambda$, we need only consider the behavior of the
background extrinsic curvature.  The asymptotic behavior of the extrinsic
curvature is 
$$
{\bar{A}}{}_{i}{}_{j}={3\kappa  \over 16\pi
{r}^{2}}\sum\limits_{\alpha =1}^{N} \left[{{P}^{\alpha
}_{i}{n}{}_{j}+{P}^{\alpha
}_{j}{n}{}_{i}-\left({{f}{}_{i}{}_{j}-{n}{}_{i}{n}{}_{j}}\right){P}_{\alpha
}^{k}{n}{}_{k}}\right]+\Order\left({{r}^{-3}}\right). \eqn{}
$$
Using (7.21) in (7.20), the effective energy density is
$$
{4\pi \rho }_{\scriptscriptstyle eff}={9\kappa  \over 64\pi
{r}^{4}}\sum\limits_{\alpha =1}^{N} \sum\limits_{\beta =1}^{N}
\left\{{{P}^{\alpha }_{i}{P}_{\beta }^{i}+2{P}^{\alpha
}_{i}{n}{}^{i}{P}^{\beta
}_{j}{n}{}^{j}}\right\}+\Order\left({{r}^{-5}}\right). \eqn{} 
$$
The solution of (7.15) for $\Lambda$ is found by inspection to be
$$
\Lambda ={9\kappa  \over 128\pi {r}^{2}}\sum\limits_{\alpha =1}^{N}
\sum\limits_{\beta =1}^{N} \left\{{{P}^{\alpha }_{i}{P}_{\beta
}^{i}-2{P}^{\alpha }_{i}{n}{}^{i}{P}^{\beta
}_{j}{n}{}^{j}}\right\}. \eqn{}
$$
Taking the gradient of $\Lambda$ gives
$$
{\bar{D}}{}^{k}\Lambda =-{2{n}{}^{k} \over r}\Lambda -{9\kappa{n}{}^{k} \over
64\pi {r}^{3}}\sum\limits_{\alpha =1}^{N} \sum\limits_{\beta
=1}^{N} \left\{{{P}_{\alpha }^{k}{P}^{\beta }_{i}{n}{}^{i}+{P}_{\beta
}^{k}{P}^{\alpha }_{i}{n}^{i}-2{n}{}^{k}{P}^{\alpha
}_{i}{n}^{i}{P}^{\beta }_{j}{n}^{j}}\right\}, \eqn{}
$$
and so (7.16) and (7.17) are satisfied.

We can now examine the effect of (7.23) on the dipole term defined by (7.19). 
Straightforward calculation shows
$$
\eqalign{{\tilde{d}}{}_{i}&={d}{}_{i}+{6 \over \kappa }\oint_{\infty
}^{}{n}{}_{i}\Lambda {r}^{2}{d}^{2}\Omega \cr &={d}{}_{i}+{27 \over
16}\sum\limits_{\alpha =1}^{N} \sum\limits_{\beta =1}^{N}
\left\{{{P}^{\alpha }_{j}{P}_{\beta
}^{j}\left\langle{{n}{}_{i}}\right\rangle-2{P}^{\alpha }_{j}{P}^{\beta
}_{k}\left\langle{{{{n}{}_{i}n}{}^{j}n}{}^{k}}\right\rangle}\right\}\cr
&={d}{}_{i},\cr} \eqn{}
$$
where $\langle\bullet\rangle$ denotes averaging over a unit two-sphere.  Since
the correction to the dipole term vanishes, we find that the dipole moment is
well defined and retains its original meaning as a moment of the initial energy
distribution, and we find that the asymptotic form for the conformal factor can
be written as 
$$
\psi =1+{\kappa  \over 16\pi }\left\{{{E \over r}+{{d}{}_{i}{n}{}^{i} \over
{r}^{2}}-{9 \over 16{r}^{2}}\sum\nolimits\limits_{\alpha =1}^{N}
\sum\limits_{\beta =1}^{N} \left({{P}^{\alpha }_{i}{P}_{\beta
}^{i}-2{P}^{\alpha }_{i}{n}{}^{i}{P}^{\beta
}_{j}{n}{}^{j}}\right)}\right\}+\Order\left({{r}^{-3}}\right). \eqn{}
$$

I should note that both of the $\Order(r^{-2})$ terms in (7.26) can be made to
vanish by an appropriate choice for the frame of the observer.  The dipole term
arises if the observer is not in the ``center of energy'' frame.  The total
energy is the lowest order, non-vanishing multipole moment and is, thus,
independent of the choice of the origin of coordinates.  The dipole moment
explicitly depends on the location of the origin of coordinates and can be made
to vanish by an appropriate choice for the origin of coordinates.  The second
term arises if the observer is not in the ``center of momentum'' frame.  If the
observer is boosted into a frame in which the sum of the linear momenta for all
$N$ holes vanishes, then the effective energy density is, asymptotically,
$\Order(r^{-6})$.  In this case, the multipole expansion (7.8) is well defined
through quadrupole order with no correction terms.

We can use (7.26) as the basis of an approximate outer boundary condition.  As
with (7.13), by taking the radial derivative of (7.26) and eliminating the
monopole term, we find
$$
{\partial \psi  \over \partial r}={1-\psi  \over r}-{\kappa  \over 16\pi
{r}^{3}}\left\{{{d}{}_{i}{n}{}^{i}-{9 \over 16}\sum\limits_{\alpha
=1}^{N} \sum\limits_{\beta =1}^{N} \left({{P}^{\alpha
}_{i}{P}_{\beta }^{i}-2{P}^{\alpha }_{i}{n}{}^{i}{P}^{\beta
}_{j}{n}{}^{j}}\right)}\right\}+{1 \over r}\Order\left({{r}^{-3}}\right).
\eqn{} 
$$
This boundary condition is of higher order than (7.14) whenever the system is
not in a center of energy {\it and} a center of momentum frame.  The extra
accuracy, however, comes at a cost.  This boundary condition depends explicitly
on the dipole moment which can only be determined from the solution to the
Hamiltonian constraint.  Thus, the boundary condition depends on the solution. 
This does not, however, prevent (7.27) from being a useful approximation for the
outer boundary condition.  The boundary condition can be used in an iterative
fashion in numerical solutions to the Hamiltonian constraint.  Because the
dipole term in (7.27) occurs at one orders in $r$ below the dominant term, it
will constitute only a small correction and the iterative scheme will be
convergent.

Before resorting to numerical methods, all analytic avenues should be explored. 
The only known analytic solutions to the inversion-symmetric Hamiltonian
constraint which represent physical situations are time symmetric.  That is, the
extrinsic curvature vanishes identically and the Hamiltonian constraint reduces
to a {\it linear} Laplace equation.  In the case of a single hole, the
inversion-symmetric, time-symmetric solution of the Hamiltonian constraint is a
slice of the Schwarzschild solution in isotropic coordinates: 
$$
\psi =1+{\kappa E \over 16\pi r}. \eqn{}
$$
$E$ is the total energy of the solution as defined by (7.9) and there are no
higher order multipole moments.

For the case of multiple holes, Misner [1963] has described the construction of
inversion-symmetric, time-symmetric solutions to the Hamiltonian constraint.  In
the case of two holes of equal size, Misner's solution can be cast into the form
of an infinite series of terms, each consisting of combinations of hyperbolic
and trigonometric functions.  I will generalize this to the case of two holes of
any arbitrary size through the use of recurrence relations.  I have not seen
this calculation anywhere else, although it may have been carried out before.

In order to be inversion symmetric, the conformal factor must satisfy (4.43) for
every hole.  As with the extrinsic curvature tensor, we can define an inversion
operator which acts on scalars.  I define the inversion operator acting through
the $\alpha^{th}$ hole by
$$
{\aleph }_{\alpha }\actson\psi \left({\bf x}\right) \equiv {{a}_{\alpha }
\over {r}_{\alpha }}\psi \left({{\bf J}_{\alpha }\left({\bf
x}\right)}\right) \eqn{}
$$
and
$$
{\aleph }_{\alpha }\actson1\equiv {{a}_{\alpha } \over {r}_{\alpha }}. \eqn{}
$$
An $N$ hole inversion-symmetric and time-symmetric solution of the Hamiltonian
constraint is then represented formally by
$$
\psi \left({\bf x}\right)=1+\sum\nolimits\limits_{\left\{{{\alpha
}_{i}}\right\}}^{} \left({\prod\nolimits\limits_{i=1}^{N} {\aleph }_{{\alpha
}_{i}}}\right). \eqn{}
$$
In order to evaluate (7.31) for the case of two holes, I will follow Chapter 6
and use cylindrical coordinates scaled relative to the radius of the first hole,
$a_1$.  The positions of the two holes will be given by (6.16) and the radius of
the second hole will be defined by (6.27).  A straightforward but tedious
calculation (Bowen {\it et al}. [1984] defines a useful notation for simplifying
the calculation) reduces (7.31) to
$$
\psi \left({\bf x}\right) =1+\sum\nolimits\limits_{n=1}^{\infty }
\left\{{{{W}_{1}^{n} \over {r}_{1}^{n}}+{{W}_{2}^{n} \over
{r}_{2}^{n}}}\right\}, \eqn{}
$$
based on the following recursively defined quantities.
%$$
%{\zeta }_{1}^{n}\equiv \left\{{\eqalign{{1 \over {\zeta }_{1}^{n-1}-{\zeta
%}_{1}}+{\zeta }_{1}\quad&:\quad\hbox{for $n$ odd}\cr {{\alpha }^{-2} \over
%{\zeta }_{1}^{n-1}-{\zeta }_{2}}+{\zeta }_{2}\quad&:\quad\hbox{for $n$ even}\cr
%{\zeta }_{1}\quad&:\quad\hbox{for $n=1$}\cr}}\right. \eqn{}
%$$
$$
{\zeta }_{1}^{n}\equiv \cases{{1 \over {\zeta }_{1}^{n-1}-{\zeta }_{1}}+{\zeta
}_{1}&: for $n$ odd\cr {{\alpha }^{-2} \over {\zeta }_{1}^{n-1}-{\zeta
}_{2}}+{\zeta }_{2}&: for $n$ even\cr \hfill{\zeta }_{1}&: for $n=1$\cr} \eqn{}
$$
$$
{\zeta }_{2}^{n}\equiv \cases{{{\alpha }^{-2} \over {\zeta }_{2}^{n-1}-{\zeta
}_{2}}+{\zeta }_{2}&: for $n$ odd\cr {1 \over {\zeta }_{2}^{n-1}-{\zeta
}_{1}}+{\zeta }_{1}&: for $n$ even\cr \hfill{\zeta }_{2}&: for $n=1$\cr} \eqn{} 
$$
$$
{W}_{1}^{n}\equiv \cases{{{W}_{1}^{n-1} \over \left|{{\zeta }_{1}^{n-1}-{\zeta
}_{1}}\right|}&: for $n$ odd\cr {{\alpha }^{-1}{W}_{1}^{n-1} \over \left|{{\zeta
}_{1}^{n-1}-{\zeta }_{2}}\right|}&: for $n$ even\cr \hfill 1&: for $n=1$ \cr}
\eqn{}
$$
$$
{W}_{2}^{n}\equiv \cases{{{\alpha }^{-1}{W}_{2}^{n-1} \over \left|{{\zeta
}_{2}^{n-1}-{\zeta }_{2}}\right|}&: for $n$ odd\cr {{W}_{2}^{n-1} \over
\left|{{\zeta }_{2}^{n-1}-{\zeta }_{1}}\right|}&: for $n$ even\cr \hfill{\alpha
}^{-1}&: for $n=1$\cr} \eqn{}
$$
$$
{r}_{1}^{n}\equiv \sqrt {{\rho }^{2}+{\left({z-{\zeta }_{1}^{n}}\right)}^{2}}
\eqn{}
$$
$$
{r}_{2}^{n}\equiv \sqrt {{\rho }^{2}+{\left({z-{\zeta }_{2}^{n}}\right)}^{2}}
\eqn{}
$$

The series expansion for the conformal factor (7.32) differs for the expansion
for the extrinsic curvature (6.15) in the crucial point that each term in (7.32)
is evaluated at the same spatial point.  In (6.15), the point of evaluation for
each term is imaged.  This complication was necessary in the case of the
extrinsic curvature in order to make the solution tractable.  Since the position
dependence of (7.32) is not imaged, the series expansion can be used directly in
the integrals for the multipole moments.  The results for the total energy and
dipole moment are
$$
{E \over {a}_{1}}={16\pi\over\kappa}\sum\nolimits\limits_{n=1}^{\infty }
\left({{W}_{1}^{n}+{W}_{2}^{n}}\right) \eqn{}
$$
and
$$
{{d}{}_{z} \over {a}_{1}^{2}}
={16\pi\over\kappa}\sum\nolimits\limits_{n=1}^{\infty } \left({{W}_{1}^{n}{\zeta
}_{1}^{n}+{W}_{2}^{n}{\zeta }_{2}^{n}}\right). \eqn{}
$$

Since I know of no analytic methods for finding solutions to the Hamiltonian
constraint in time-asymmetric situations, I will turn now to numerical
techniques.  The problem of solving a partial differential equation consists of
two basic parts:  the discretization of the continuous problem into a set of
algebraic equations and the solution of the resulting large set of possibly
non-linear equations.  Though distinct, the two components are often treated
together ({\it cf}. Ames [1977], Mitchel and Griffiths [1980], and Allen
[1954]).  The problem of discretization can be handled by several different
techniques.  The two main techniques are the method of finite differences and
the method of finite elements and  I choose to use the method of finite
differences.

The problem of solving the large set of algebraic equations is the prime
difficulty in finding numerical solutions.  In the case of an elliptic partial
differential equation, one is always faced, in the end, with the task of
solving a very large matrix equation.  The size of the matrix is determined by
the number of spatial dimensions on which the equation depends and on the spatial
resolution of the discretization, the latter determining the accuracy of the
approximation.  In order to reduce the size of the matrix which must be
inverted, one often considers problems with sufficient symmetry to reduce the
dimensional dependence of the differential equation to one or two dimensions. 
In subsequent chapters, I will do just this by considering only axisymmetric
configurations.  In the case of a single hole, this is not a serious
restriction.  If you do not consider the situation of a hole with {\it both}
linear and angular momenta, then the situation is naturally axisymmetric.  In
the case of two holes, the restriction to axisymmetry eliminates, among other
things, the possibility of exploring initial-data sets with non-zero orbital
angular momentum.  Such configurations are of great physical interest and will
certainly be examined in the future, however, in subsequent chapters I will
consider only axisymmetric configurations which can be dealt with by
two-dimensional numerical techniques.
\vfill
\eject