\chapterhead{Chapter~\the\chapnum:  Inversion-Symmetric, Single-Hole\cr
Initial-Data Sets}
%%%%%%%%%%%%%%%%%%%%%%%%%%%%%%%%%%%%%%%
In Chapter~8, a technique for solving the Hamiltonian constraint and for
extracting physically relevant information for a single hole with linear or
angular momentum was described.  In this chapter, I will discuss the
interpretation of the content of these initial-data sets.  As mentioned in
Chapter~8, this has been done previously by several authors.  I am including a
discussion of these single-hole initial-data sets, both for completeness, and
because of new information concerning the location of the apparent horizons
surrounding the holes.  As discussed in Chapter~12, the behavior of apparent
horizons for a single boosted hole has not been understood correctly in the past,
and the new understanding of the apparent-horizon structure does affect the
interpretation of the initial-data sets.

Because all of the single-hole initial-data sets described here are constructed
following the conformal imaging approach, they necessarily exist on a
{\it complete}, space-like hypersurface consisting of two asymptotically-flat
``sheets'' joined by a single throat.  The existence of an event horizon in the
spacetime is confirmed by the existence of an inversion-symmetric pair of
apparent horizons, one for each ``universe'' or ``sheet''.

Associated with each initial-data set is a total energy for the system.  The mass
associated with this energy is, in fact, the Keplerian mass which an observer at
infinity on either sheet would measure by examining the orbit of a test
particle.  It is quite likely that the total energy in any of these initial-data
sets is not entirely bound to the black hole.  Some of the energy in the initial
slice can be in the form of gravitational radiation which may escape to
infinity.  With this in mind, it is useful to consider the {\it irreducible
mass} $M_{\scriptscriptstyle IR}$ of the black hole.  This mass is given by
(8.33) and is associated with the area of the event horizon which (classically)
can never decrease.

Given the initial data, observers at infinity on either slice can determine the
total linear and angular momentum of the slice in addition to the total energy. 
For a boosted hole, the total energy in the slice consists of the kinetic energy
of the boosted hole in addition to the energy bound to the hole and any
gravitational radiation outside the hole.  In this case, the rest energy of the
hole will be of more importance than the total energy since it does not include
the kinetic energy.  For a hole in simple translation with linear momentum $P$,
the rest energy can be calculated simply from
$$
{E}_{total }^{2}={E}_{rest}^{2}+{P}^{2}. \eqn{}
$$
For a spinning hole with angular momentum $S$, some of the energy in the system
is in the form of rotational kinetic energy.  Instead of associating it with the
total energy of the system, however, it is traditional, and somewhat more
natural, to associate this extra energy with the mass of the hole.  Using the
Christodoulou [1970] formula, the mass of the hole can be defined from
$$
{M}^{2}={M}_{\scriptscriptstyle IR}^{2}+ {{S}^{2} \over 4{M}_{\scriptscriptstyle
IR}^{2}}. \eqn{} $$

By taking translational and rotational kinetic energy into account as in (13.1)
and (13.2), and if there are no other sources of energy, then the rest energy of
the system and the mass of the hole should be equivalent.  If they are not
equal, then, for a vacuum configuration, the only other way that energy can be
present in the system is in the form of gravitational radiation.  This radiation
may or may not be bound to the black hole.  If it is, then as the data is
evolved in time, the radiation will fall into the hole and the mass of the hole
will increase.  If it is not bound to the hole, then eventually, it will
propagate to infinity.

It is possible to obtain an upper limit on the amount of gravitational radiation
which can reach infinity.  It is not possible to determine the {\it irreducible
mass} of a hole in these initial-data set since this requires knowledge of the
area of the {\it event} horizon.  However, we do know that the event horizon will
necessarily be outside of or coincident with the apparent horizon, the mass of
which is a lower limit on the irreducible mass.  Using the total energy in the
slice, the value of the linear and angular momenta at infinity and the mass of
the {\it apparent} horizon, the upper limit to the amount of gravitational
radiation which can escape to infinity is given by
$$
{E}_{rad}\le \sqrt {{E}_{total }^{2}-{P}^{2}} -\sqrt {{M}_{\scriptscriptstyle
AH}^{2}+{{S}^{2} \over 4{M}_{\scriptscriptstyle AH}^{2}}}. \eqn{}
$$
Similarly, the maximum ``efficiency'' with  which a given configuration can
produce gravitational radiation is given by
$$
\hbox{Efficiency} \le {{E}_{rad} \over \sqrt {{E}_{total }^{2}-{P}^{2}}}.
\eqn{}
$$

Figures 13.1, 13.2 and 13.3 display the physical content of the initial data
sets corresponding, respectively, to a hole with linear momentum generated by
$\bar{A}^+_{ij}$, a hole with linear momentum generated by $\bar{A}^-_{ij}$, and
a hole with angular momentum generated by $\bar{A}_{ij}$.  I should note that
the data in these figures are presented differently than they have been in the
past.  The data is displayed in terms of {\it naturally} dimensionless
quantities.  Specifically, all quantities are scaled relative to the mass of the
hole as given in (13.2) where $M_{\scriptscriptstyle IR}$ is replaced by the mass
of the apparent horizon which is a very good approximation to the irreducible
mass.  Historically ({\it cf}. York and Piran [1982], Choptuik [1984], and Cook
and York [1990]), all quantities have been examined, either directly or
effectively, scaled relative to the conformal radius of the minimal surface, ie.
$E/a$, $P/a$, etc.  This seems, initially, to be of trivial importance in the
case of a single hole.  However, it can lead to subtle misinterpretations of
the physical content of the initial-data sets.  For example, for a given value
of $P/a$ as measured at infinity, the initial-data sets constructed from
$\bar{A}^+_{ij}$ and $\bar{A}^-_{ij}$ give different value for the total energy
of the slice $E/a$.  This fact was used in Cook and York [1990] to argue that
the two initial-data slices cannot be different slicings of the same global
space-time.  While the conclusion may be true, it cannot be made based on this
argument because the mass (and thus the proper area) of the apparent horizons in
the two cases are also different.  The two physical configurations are not
directly comparable.

Postponing for the moment any comparison between the two, consider the initial
data for a hole with linear momentum generated by either $\bar{A}^+_{ij}$ or
$\bar{A}^-_{ij}$.  Figures~13.1 and 13.2 display, respectively for these cases,
the {\it naturally} scaled total energy, rest energy, maximum radiation energy,
and radiation efficiency as a function of the {\it naturally} scaled linear
momentum.  Also shown for each computed value of the momentum is the velocity of
the hole.  This quantity is plotted adjacent to the value of the total energy
and is given by $V = P/E$.  We see in both cases, that the total energy follows
the general behavior of a boosted point particle.  In addition, the velocity of
the hole is less than one in all cases and the general trend suggests that it
will not exceed unity.

%\figlabel{3.375truein}{Figure~13.1:  Energy and velocity for a boosted black
%hole with linear\cr momentum $P$ generated from the extrinsic curvature
%$\bar{A}^+_{ij}$.}
\figlabelpdf{3.375truein}{Figure~13.1:  Energy and velocity for a boosted black
hole with linear\cr momentum $P$ generated from the extrinsic curvature
$\bar{A}^+_{ij}$.}{H1plots/P+.pdf}

%\figlabel{3.375truein}{Figure~13.2:  Energy and velocity for a boosted black
%hole with linear\cr momentum $P$ generated from the extrinsic curvature
%$\bar{A}^-_{ij}$.}
\figlabelpdf{3.375truein}{Figure~13.2:  Energy and velocity for a boosted black
hole with linear\cr momentum $P$ generated from the extrinsic curvature
$\bar{A}^-_{ij}$.}{H1plots/P-.pdf}

If the initial data being examined were that of a simple, boosted Schwarzschild
black hole, then the naturally scaled rest energy should be unity for all values
of the momentum.  We see immediately in Figures~13.1 and 13.2 that this is not
the case and so there must be gravitational radiation present in the
initial-data slice.  This was anticipated because, as Bowen and York [1980]
pointed out, the boosted Schwarzschild metric is not conformally flat.  Since it
is the conformal three-geometry which carries the gravitational degrees of
freedom for the metric, the choice of conformal flatness requires the presence of
gravitational radiation in the initial slice.  Plotted along with the maximum
radiation content of the slice in Figures~13.1 and 13.2 is the maximum
percentage of the rest energy which is present in the form of radiation.

Figure~13.3 displays the naturally scaled total energy, maximum radiation
energy, and radiation efficiency as a function of the {\it naturally} scaled
angular momentum.  For a spinning hole, the total energy equals the rest energy
since the hole has no linear momentum.  In the absence of gravitational
radiation in the slice, the naturally scaled total energy should be {\it one}
since the mass of the hole (13.2) to which the energy is scaled includes the
rotational kinetic energy.  From Figure~13.3, we see that the scaled total
energy does deviate from unity and so there must be gravitational radiation in
the slice.  This, too, was anticipated because, as Bowen and York [1980] pointed
out, a $t=constant$ slice (Boyer-Lindquist time) of the Kerr metric, like the
boosted Schwarzschild metric, is not conformally flat.  The maximum radiation
content of the slice is plotted in Figure~13.3 to a different scale than the
total energy.  Plotted along with the maximum radiation content of the slice in
Figure~13.3 is the maximum percentage of the rest energy which is present in the
form of radiation.

%\figlabel{3.375truein}{Figure~13.3:  Energy of a spinning black hole with angular
%momentum $S$.}
\figlabelpdf{3.375truein}{Figure~13.3:  Energy of a spinning black hole with angular
momentum $S$.}{H1plots/Jc.pdf}

Note that in Figures~13.1 and 13.2, it appears that the scaled linear momentum
may increase without bound.  On the other hand, in Figure~13.3, the scaled
angular momentum is asymptotic to unity.  This behavior is explained by noting
that an extreme Kerr metric occurs when $S/M^2 = 1$.  The asymptotic behavior of
the scaled angular momentum simply indicates that initial-data sets for rotating
holes constructed via the conformal-imaging approach will not violate cosmic
censureship.

Returning to the case of a hole with linear momentum, it is useful to compare in
more detail the differences between the two sets of initial data generated from
$\bar{A}^+_{ij}$ and $\bar{A}^-_{ij}$.  In Chapter~12, it was discovered
that the apparent horizons for the initial-data sets associated with
$\bar{A}^+_{ij}$ and $\bar{A}^-_{ij}$ behave differently.  For the case of
$\bar{A}^-_{ij}$, the apparent horizon for the top sheet is coincident with the
minimal surface.  From Chapter~12, we know that the inversion-symmetric
counterpart for this apparent horizon will be an apparent horizon for the bottom
sheet which is also coincident with the minimal surface.  This apparent horizon
structure is exactly that found in $t=constant$ slices of the Schwarzschild
metric in isotropic coordinates.

For the case of $\bar{A}^+_{ij}$, the position of the apparent horizon in the top
sheet is given by (12.40).  We see that half of the apparent horizon is on one
side of the minimal surface and half is on the other.  For an observer at
infinity on the top sheet, a positive value of the linear momentum will be
pointed in the positive $z$ (or $\theta=0$) direction.  From (12.40), we see that
the the apparent horizon on the ``leading'' side of the hole ($\theta=0$) is
positioned interior to the minimal surface from the point of view of an observer
on the top sheet.  The apparent horizon crosses the minimal surface at
$\theta=\pi/2$ and is exterior to it on the ``trailing'' side.  Figure~13.4
illustrates this behavior by plotting the position of the apparent horizon in
the conformal background space using standard spherical coordinates.  These
coordinates are the natural coordinate patch for the top sheet.  The bottom
sheet, as mentioned in Chapter~4, is the region interior to the minimal surface
and infinity on the bottom sheet is compactified to the origin. 

The inversion-symmetric counterpart for this apparent horizon, as described in
Chapter~12, is positioned analogously from the point of view of an observer at
infinity on the ``bottom'' sheet.  Examining $\bar{A}^+_{ij}$ carefully, we find
that an observer at infinity on the bottom sheet also finds a positive value of
the momentum to point toward infinity in the $\theta=0$ direction (see
Figure~13.5).  This observer will  also find that the apparent horizon is
interior to the minimal surface on the leading edge of the hole, crosses the
minimal surface at $\theta=\pi/2$, and is exterior to the minimal surface on the
trailing edge.

%\figlabel{3.375truein}{Figure~13.4:  The location of the top and bottom apparent
%horizons (for data\cr sets generated from $\bar{A}^+_{ij}$) and the minimal
%surface in the conformal background\cr space.  Also plotted are the spatial
%projections of various ``outgoing''\cr null vectors with negative expansion.}
\figlabelpdf{3.375truein}{Figure~13.4:  The location of the top and bottom apparent
horizons (for data\cr sets generated from $\bar{A}^+_{ij}$) and the minimal
surface in the conformal background\cr space.  Also plotted are the spatial
projections of various ``outgoing''\cr null vectors with negative expansion.}{Figures/Figure13_4.pdf}

As seen from either Figure~13.4 or 13.5, the behavior of the apparent horizons
means that an observer on one sheet will ``see'' the apparent horizon for the
other sheet on the leading edge of the hole.  This alternate apparent horizon is
more correctly a ``past'' apparent horizon.  Interpreted in this way, we see that
the initial-data hypersurface resulting from $\bar{A}^+_{ij}$ is a slice of a
global space-time which intersects the past apparent horizon for values of
$\theta<\pi/2$, intersects the bifurcation point at $\theta=\pi/2$, and
intersects the future apparent horizon for $\theta>\pi/2$.  In contrast to this,
the initial-data hypersurface resulting from $\bar{A}^-_{ij}$ is a slice of a
global space-time which intersects the bifurcation point for all $\theta$.

%\figlabel{4.375truein}{Figure~13.5:  The location of the apparent horizons
%(for data sets generated\cr from $\bar{A}^+_{ij}$) and the minimal surface
%illustrated on an embedding\cr diagram of the geometry.}
\figlabelpdf{4.375truein}{Figure~13.5:  The location of the apparent horizons
(for data sets generated\cr from $\bar{A}^+_{ij}$) and the minimal surface
illustrated on an embedding\cr diagram of the geometry.}{Figures/Figure13_5.pdf}

To examine the connection between these two families of initial-data sets for
boosted black holes, and to understand them better, it is useful to return to
the example of the first-order, boosted Schwarzschild initial data discussed in
Chapter~12.  While the metric, to first order, is simply the Schwarzschild
metric, the extrinsic curvature and apparent horizon structure can be given
either by $\bar{A}^+_{ij}$ or $\bar{A}^-_{ij}$ and their corresponding apparent
horizons as described above.  The choice of $\bar{A}^+_{ij}$ or $\bar{A}^-_{ij}$
is determined by the choice of the isometry condition which the lapse function
must satisfy and thus, the isometry condition which the extrinsic curvatures
must satisfy.  In the case of these first-order initial-data sets, we find that
the choice of the isometry condition simply distinguishes two different maximal
slices of the same global space-time and the evolution of either initial-data
set must generate the same global space-time.

The question now is whether or not the two initial data sets given by
$\bar{A}^+_{ij}$ or $\bar{A}^-_{ij}$ and their respective solutions of the
Hamiltonian constraint are two different slicings of the same global
space-time.  Cook and York [1990] concluded that this was not the case since the
total energy for the two cases, with the same linear momentum, were different. 
As I have discussed above, this conclusion cannot be made on this basis.  If the
naturally scaled initial data is compared by using cubic-spline interpolation to
facilitate a direct comparison, then one finds that the scaled energies differ
on at a level of less than 0.1\%.  This level of error is too close to the
level of the discretization error in the solution for a conclusion to be drawn. 
This question may only be answerable by evolving the initial-data sets and
examining the gravitational radiation emitted during the two evolutions.

The apparent-horizon structure found for the $\bar{A}^+_{ij}$ initial data in
Chapter~12, in addition to illuminating the structure of the initial-data slice,
also lends support to the use of the conformal imaging approach.  As mentioned
in Chapter~4,  Thornburg [1987] has proposed using $N+1$ sheeted initial-data
hypersurfaces as opposed to the two sheeted initial-data hypersurfaces of the
conformal imaging approach.  The use of these hypersurfaces is accompanied by
the use of an apparent-horizon inner boundary condition in place of the minimal
surface boundary condition of the conformal imaging approach.  By using the
apparent-horizon boundary condition, no information about the geometry interior
to the apparent horizon can be determined.  It is argued that this information is
irrelevant since no information can propagate through the apparent horizon and
so it can have no effect on the future evolution of the initial data.  What this
argument ignores, however, is the fact that the geometry interior to the
apparent horizon does affect the exterior geometry in the sense that they are
both part of a global construction.  We see from the example of the apparent
horizon structure for $\bar{A}^+_{ij}$ that knowing only the incomplete initial
data exterior to the apparent horizon would not allow the past apparent horizon
to be found.  The construction of initial data sets on {\it complete} manifolds
is crucial to assuring that the maximum amount of information can be extracted
both from the initial-data set and from its evolution.

\vfill
\eject
