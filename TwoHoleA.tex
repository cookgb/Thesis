\chapterhead {Chapter~\the\chapnum:  Inversion-Symmetric Extrinsic Curvature\cr
 for Two Black Holes}
%%%%%%%%%%%%%%%%%%%%%%%%%%%%%%%%
The formalism described in the previous chapter describes a method for
constructing inversion-symmetric solutions to the momentum constraint equation
for $N$ black holes.  This chapter is devoted to a method for evaluating these
inversion-symmetric solutions in the case of two black holes.  For the
construction of initial-data sets, it is sufficient to be able to evaluate
numerically the formal infinite series solution described in the previous
chapter (5.21).  In this light, the search for an analytic expression for each
term in the series is abandoned.  Instead, let us consider the evaluation of the
infinite series at a specific point in the domain exterior to the two holes.

Since I deal here with the case of two holes, the hole label, $\alpha$, takes on
the values of 1 and 2.  Let ${\bf x}$ denote the point at which the extrinsic
curvature will be evaluated, and with it define the following quantities:
$$
{\bf x}_{1}\equiv {\bf J}_{1}\left({\bf x}\right),\quad{\bf x}_{ 21}\equiv {\bf
J}_{2}\left({{\bf x}_{1}}\right),\quad{\bf x}_{121}\equiv {\bf J}_{1}\left({{\bf
x}_{21}}\right),\ldots. \eqn{}
$$
and
$$
{\bf x}_{2}\equiv {\bf J}_{2}\left({\bf x}\right),\quad{\bf x}_{ 12}\equiv {\bf
J}_{1}\left({{\bf x}_{2}}\right),\quad{\bf x}_{212}\equiv {\bf J}_{2}\left({{\bf
x}_{12}}\right),\ldots. \eqn{}
$$
With $\hat{A}_{ij}$ defined by (5.22), consider the first few terms of the
infinite series (5.21).
$$
{\Re }_{1}^{\pm }\actson \hat{A}\left({\bf x}\right) =\pm {\left({{{a}_{1}
\over {r}_{1}}}\right)}_{\left({\bf x}\right)}^{2}{\left[{{\bf J}_{
1}}\right]}_{\left({\bf x}\right)}{\hat{A}}_{\left({{\bf x}_{
1}}\right)}{\left[{{\bf J}_{1}}\right]}_{\left({\bf x}\right)}. \eqn{}
$$
$$
{\Re }_{2}^{\pm }\actson {\Re }_{1}^{\pm }\actson \hat{A}\left({\bf
x}\right) ={\left({{{a}_{2} \over {r}_{2}}}\right)}_{\left({\bf
x}\right)}^{2}{\left({{{a}_{1} \over {r}_{1}}}\right)}_{\left({{\bf x}_{
2}}\right)}^{2}{\left[{{\bf J}_{2}}\right]}_{\left({\bf
x}\right)}{\left[{{\bf J}_{1}}\right]}_{\left({{\bf x}_{
2}}\right)}{\hat{A}}_{\left({{\bf x}_{12}}\right)}{\left[{{\bf J}_{
1}}\right]}_{\left({{\bf x}_{2}}\right)}{\left[{{\bf J}_{
2}}\right]}_{\left({\bf x}\right)}. \eqn{}
$$
$$
\eqalign{{\Re }_{1}^{\pm }\actson {\Re }_{2}^{\pm }\actson {\Re }_{1}^{\pm
}\actson \hat{A}\left({\bf x}\right) =\pm &{\left({{{a}_{1} \over
{r}_{1}}}\right)}_{\left({\bf x}\right)}^{2}{\left({{{a}_{2} \over
{r}_{2}}}\right)}_{\left({{\bf x}_{1}}\right)}^{2}{\left({{{a}_{1} \over
{r}_{1}}}\right)}_{\left({{\bf x}_{21}}\right)}^{2}\times \cr &{\left[{{\bf
J}_{1}}\right]}_{\left({\bf x}\right)}{\left[{{\bf J}_{
2}}\right]}_{\left({{\bf x}_{1}}\right)}{\left[{{\bf J}_{
1}}\right]}_{\left({{\bf x}_{21}}\right)}{\hat{A}}_{\left({{\bf x}_{
121}}\right)}{\left[{{\bf J}_{1}}\right]}_{\left({{\bf x}_{
21}}\right)}{\left[{{\bf J}_{2}}\right]}_{\left({{\bf x}_{
1}}\right)}{\left[{{\bf J}_{1}}\right]}_{\left({\bf x}\right)}.\cr} \eqn{}
$$
The extension to higher order terms is obvious.  Now, recursively define the two
vectors ${\bf x}^n_1$ and ${\bf x}^n_2$ as follows.
$$
{\bf x}_{1}^{n}\equiv \cases{{\bf J}_{1}\left({{\bf x}_{ 1}^{n-1}}\right)&: for
$n$ odd \cr {\bf J}_{\rm 2}\left({{\bf x}_{\rm 1}^{n-1}}\right)&: for $n$
even\cr \hfill{\bf x}&: for $n=0$.\cr} \eqn{}
$$
$$
{\bf x}_{2}^{n}\equiv \cases{{\bf J}_{\rm 2}\left({{\bf x}_{ 2}^{n-1}}\right)&:
for $n$ odd\cr {\bf J}_{1}\left({{\bf x}_{ 2}^{n-1}}\right)&: for $n$ even\cr
\hfill{\bf x} &: for $n=0$.\cr} \eqn{} 
$$
In terms of ${\bf x}^n_1$ and ${\bf x}^n_2$, recursively define the following
two matrices. 
$$
{M}_{1}^{n}\equiv \cases{{M}_{1}^{n-1}{\left[{{\bf J}_{ 1}}\right]}_{\left({{\bf
x}_{1}^{n-1}}\right)}&: for $n$ odd\cr {M}_{1}^{n-1}{\left[{{\bf
J}_{2}}\right]}_{\left({{\bf x}_{ 1}^{n-1}}\right)}&: for $n$ even\cr
\hfill\Unitmatrix&: for $n=0$.\cr} \eqn{} 
$$
$$
{M}_{2}^{n}\equiv \cases{{M}_{2}^{n-1}{\left[{{\bf J}_{ 2}}\right]}_{\left({{\bf
x}_{2}^{n-1}}\right)}&: for $n$ odd\cr {M}_{2}^{n-1}{\left[{{\bf
J}_{1}}\right]}_{\left({{\bf x}_{ 2}^{n-1}}\right)}&: for $n$ even\cr
\hfill\Unitmatrix&: for $n=0$.\cr} \eqn{} 
$$
Finally, recursively define the following two scalar quantities.
$$
{F}_{1}^{n}\equiv \cases{{F}_{1}^{n-1}{\left({{{a}_{1} \over
{r}_{1}}}\right)}_{\left({{\bf x}_{1}^{n-1}}\right)}^{2}&: for $n$ odd\cr
{F}_{1}^{n-1}{\left({{{a}_{2} \over {r}_{2}}}\right)}_{\left({{\bf
x}_{1}^{n-1}}\right)}^{2}&: for $n$ even\cr \hfill 1&: for $n=0$.\cr} \eqn{} 
$$
$$
{F}_{2}^{n}\equiv \cases{{F}_{2}^{n-1}{\left({{{a}_{2} \over
{r}_{2}}}\right)}_{\left({{\bf x}_{2}^{n-1}}\right)}^{2}&: for $n$ odd\cr
{F}_{2}^{n-1}{\left({{{a}_{1} \over {r}_{1}}}\right)}_{\left({{\bf x}_{
2}^{n-1}}\right)}^{2}&: for $n$ even\cr \hfill 1&: for $n=0$.\cr} \eqn{}
$$

In terms of these quantities, the first few terms in the series (5.19) can be
written as
$$
{\Re }_{1}^{\pm }\actson \hat{A}\left({\bf x}\right) ={\left({\pm
1}\right)}^{1}{F}_{1}^{1}{M}_{1}^{1}{\hat{A}}_{\left({{\bf x}_{
1}^{1}}\right)}{\tilde{M}}_{1}^{1}, \eqn{}
$$
$$
{\Re }_{2}^{\pm }\actson {\Re }_{1}^{\pm }\actson \hat{A}\left({\bf
x}\right) ={\left({\pm
1}\right)}^{2}{F}_{2}^{2}{M}_{2}^{2}{\hat{A}}_{\left({{\bf x}_{
2}^{2}}\right)}{\tilde{M}}_{2}^{2}, \eqn{}
$$
and
$$
{\Re }_{1}^{\pm }\actson {\Re }_{2}^{\pm }\actson {\Re }_{1}^{\pm }\actson
\hat{A}\left({\bf x}\right) ={\left({\pm
1}\right)}^{3}{F}_{1}^{3}{M}_{1}^{3}{\hat{A}}_{\left({{\bf x}_{
1}^{3}}\right)}{\tilde{M}}_{1}^{3}. \eqn{}
$$
Note that $\tilde{M}^n_\alpha$ denotes the transpose of $M^n_\alpha$.  By using
the recursion relations defined above, it is easy to prove that (6.12), (6.13),
and (6.14) are equivalent, respectively, to (6.3), (6.4), and (6.5).  The pattern
for a general term in the series (5.21) is now clear.  The set of quantities to
use, $({\bf x}^n_1,M^n_1,F^n_1)$ or $({\bf x}^n_2,M^n_2,F^n_2)$, is determined
by the label of the left-most inversion operator, and the index of the
recursion, $n$, is determined by the number of inversion operators.  The
two-hole, inversion-symmetric solution to the momentum constraint equation can
now be written as
$$
\eqalign{\bar{A}\left({\bf x}\right) &={\Re }^{\pm }\actson\hat{A}\left({\bf
x}\right)\cr &=\hat{A}\left({\bf x}\right) +\sum\limits_{n=1}^{\infty }
{\left({\pm 1}\right)}^{n}\left\{{{F}_{1}^{n}{M}_{1}^{n}{\hat{A}}_{\left({{\bf
x}_{
1}^{n}}\right)}{\tilde{M}}_{1}^{n}+{F}_{2}^{n}{M}_{2}^{n}{\hat{A}}_{\left({{\bf
x}_{2}^{n}}\right)}{\tilde{M}}_{2}^{n}}\right\}.\cr} \eqn{}
$$

Formulated in this way, any desired numerical accuracy can be obtained for the
components of $\bar{A}_{ij}$.  To obtain the $n^{th}$ term in the series, one
only needs to know the six quantities ${\bf x}^{n-1}_1$, ${\bf x}^{n-1}_2$,
$M^{n-1}_1$, $M^{n-1}_2$, $F^{n-1}_1$, and $F^{n-1}_2$ which have already been
computed in order to obtain the $(n-1)^{th}$ term.  From a computational point
of view, it is a simple iterative matter to keep adding terms until the value of
$\bar{A}_{ij}$ converges to machine accuracy.

This formulation of the extrinsic curvature is completely general.  It can be
used to compute the components of the extrinsic curvature when the two holes
have any relative sizes and separation, and for any values of the linear and
angular momentum for each hole.  In later chapters, I will explore solutions to
the Hamiltonian constraint in the case of axisymmetry.  In preparation for this,
I will describe (6.15) explicitly for this special case.

Let us consider first the coordinate system to be used.  Since the configuration
is to be axisymmetric, the obvious choice is to use cylindrical coordinates.  It
is best to use dimensionless coordinates, so it is desirable to scale the
coordinates relative to some physical length scale of the problem.  There are
three natural length scales in the problem:  the size of each of the two holes
and the separation between the two holes.  I choose to use the radius of the
first hole, $a_1$, as the natural length scale.

The two holes will be centered on the $z$-axis and will be positioned so that
$$
{\bf C}_{1}=\left[{\matrix{0&0&{a}_{1}{\zeta
}_{1}\cr}}\right]\qquad\hbox{and}\qquad{\bf
C}_{2}=\left[{\matrix{0&0&{a}_{1}{\zeta }_{2}\cr}}\right]. \eqn{} 
$$
In order to maintain axisymmetry, we choose the linear and angular momentum
vectors for the two holes to have components only in the $z$ direction.  The
physical, unscaled linear and angular momentum vectors will be
$$
{\bf P}_{1}=\left[{\matrix{0&0&{P}_{1}\cr}}\right] \qquad\hbox{and}\qquad{\bf
P}_{ 2}=\left[{\matrix{0&0&{P}_{2}\cr}}\right] \eqn{}
$$
and
$$
{\bf S}_{1}=\left[{\matrix{0&0&{S}_{1}\cr}}\right] \qquad\hbox{and}\qquad{\bf
S}_{ 2}=\left[{\matrix{0&0&{S}_{2}\cr}}\right]. \eqn{}
$$
The cylindrical coordinate components of $\hat{A}_{ij}$ can now be written in
dimensionless form as
$$
{{\hat{A}}{}_{\rho }{}_{\rho } \over {a}_{1}}=-{3\kappa  \over 16\pi
}\left\{{{\left({{{P}_{1} \over {a}_{1}}}\right){\left({z-{\zeta
}_{1}}\right)}^{3} \over {\left({{\rho }^{2}+{\left({z-{\zeta
}_{1}}\right)}^{2}}\right)}^{5/2}}+{\left({{{P}_{2} \over
{a}_{1}}}\right){\left({z-{\zeta }_{2}}\right)}^{3} \over {\left({{\rho
}^{2}+{\left({z-{\zeta }_{2}}\right)}^{2}}\right)}^{5/2}}}\right\}, \eqn{}
$$
$$
{{\hat{A}}{}_{\phi }{}_{\phi } \over {a}_{1}}=-{3\kappa  \over 16\pi }{\rho
}^{2}\left\{{{\left({{{P}_{1} \over {a}_{1}}}\right)\left({z-{\zeta
}_{1}}\right) \over {\left({{\rho }^{2}+{\left({z-{\zeta
}_{1}}\right)}^{2}}\right)}^{3/2}}+{\left({{{P}_{2} \over
{a}_{1}}}\right)\left({z-{\zeta }_{2}}\right) \over {\left({{\rho
}^{2}+{\left({z-{\zeta }_{2}}\right)}^{2}}\right)}^{3/2}}}\right\}, \eqn{}
$$
$$
\eqalign{{{\hat{A}}{}_{z}{}_{z} \over {a}_{1}}={3\kappa  \over 16\pi
}&\left\{{{\left({{{P}_{1} \over {a}_{1}}}\right)\left({z-{\zeta
}_{1}}\right)\left({{\rho }^{2}+2{\left({z-{\zeta }_{1}}\right)}^{2}}\right)
\over {\left({{\rho }^{2}+{\left({z-{\zeta
}_{1}}\right)}^{2}}\right)}^{5/2}}}\right.\cr
&\qquad\qquad\qquad+\left.{{\left({{{P}_{2} \over
{a}_{1}}}\right)\left({z-{\zeta }_{2}}\right)\left({{\rho
}^{2}+2{\left({z-{\zeta }_{2}}\right)}^{2}}\right) \over {\left({{\rho
}^{2}+{\left({z-{\zeta }_{2}}\right)}^{2}}\right)}^{5/2}}}\right\},\cr} \eqn{} 
$$
$$
{{\hat{A}}{}_{\rho }{}_{z} \over {a}_{1}}={3\kappa  \over 16\pi }\rho
\left\{{{\left({{{P}_{1} \over {a}_{1}}}\right)\left({{\rho
}^{2}+2{\left({z-{\zeta }_{1}}\right)}^{2}}\right) \over {\left({{\rho
}^{2}+{\left({z-{\zeta }_{1}}\right)}^{2}}\right)}^{5/2}}+{\left({{{P}_{2} \over
{a}_{1}}}\right)\left({{\rho }^{2}+2{\left({z-{\zeta }_{2}}\right)}^{2}}\right)
\over {\left({{\rho }^{2}+{\left({z-{\zeta
}_{2}}\right)}^{2}}\right)}^{5/2}}}\right\}, \eqn{}
$$
$$
{{\hat{A}}{}_{\rho }{}_{\phi } \over {a}_{1}}={3\kappa  \over 16\pi }{\rho
}^{3}\left\{{{\left({{{S}_{1} \over {a}_{1}^{2}}}\right) \over {\left({{\rho
}^{2}+{\left({z-{\zeta }_{1}}\right)}^{2}}\right)}^{5/2}}+{\left({{{S}_{2} \over
{a}_{1}^{2}}}\right) \over {\left({{\rho }^{2}+{\left({z-{\zeta
}_{2}}\right)}^{2}}\right)}^{5/2}}}\right\}, \eqn{}
$$
and
$$
{{\hat{A}}{}_{\phi }{}_{z} \over {a}_{1}}={3\kappa  \over 16\pi }{\rho
}^{2}\left\{{{\left({{{S}_{1} \over {a}_{1}^{2}}}\right)\left({z-{\zeta
}_{1}}\right) \over {\left({{\rho }^{2}+{\left({z-{\zeta
}_{1}}\right)}^{2}}\right)}^{5/2}}+{\left({{{S}_{2} \over
{a}_{1}^{2}}}\right)\left({z-{\zeta }_{2}}\right) \over {\left({{\rho
}^{2}+{\left({z-{\zeta }_{2}}\right)}^{2}}\right)}^{5/2}}}\right\}. \eqn{}
$$

Before describing the explicit form which the isometry map and the Jacobian will
take, it is useful to examine the square of the background extrinsic curvature
which is the quantity of interest in the Hamiltonian constraint.  Expressed as a
dimensionless quantity, it takes the form
$$
\eqalign{{a}_{1}^{2}{\bar{A}}_{ij}&{\bar{A}}^{ij}={\left[{{\left({{\Re
}^{\pm }\actson\!\left({{\hat{A} \over
{a}_{1}}}\right)}\!\!\right)}_{\!\!\!\rho\rho}}\right]}^{2}+{\left[{{1 \over
{\rho }^{2}}{\left({{\Re }^{\pm }\actson\!\left({{\hat{A} \over
{a}_{1}}}\right)}\!\!\right)}_{\!\!\!\phi\phi}}\right]}^{2}+{\left[{{\left({{\Re
}^{\pm }\actson\!\left({{\hat{A} \over
{a}_{1}}}\right)}\!\!\right)}_{\!\!\!zz}}\right]}^{2}\cr &+
2{\left[{{\left({{\Re }^{\pm }\actson \left({{\hat{A} \over
{a}_{1}}}\right)}\!\!\right)}_{\!\!\!\rho z}}\right]}^{2}+2{\left[{{1 \over
\rho }{\left({{\Re }^{\pm }\actson\! \left({{\hat{A} \over
{a}_{1}}}\right)}\!\!\right)}_{\!\!\!\rho\phi }}\right]}^{2}+2{\left[{{1 \over
\rho }{\left({{\Re }^{\pm }\actson\! \left({{\hat{A} \over
{a}_{1}}}\right)}\!\!\right)}_{\!\!\!\phi z}}\right]}^{2}.\cr} \eqn{} 
$$
It is clear that care will have to be taken in computing the second, fifth, and
sixth terms in (6.25) so as not to encounter errors resulting from division by
zero (not to mention computational roundoff errors) as the $\rho=0$ axis is
approached.  The second term in (6.25) is handled easily by the property that
$\bar{A}_{ij}$ is traceless.  This fact immediately gives
$$
{1 \over {\rho }^{2}}{\left({{\Re }^{\pm }\actson\!\left({{\hat{A} \over
{a}_{1}}}\right)}\!\!\right)}_{\!\!\!\phi\phi}=-\left({\left[{{\left({{\Re
}^{\pm }\actson\!\left({{\hat{A} \over
{a}_{1}}}\right)}\!\!\right)}_{\!\!\!\rho\rho }}\right]+\left[{{\left({{\Re
}^{\pm }\actson \left({{\hat{A} \over
{a}_{1}}}\right)}\!\!\right)}_{\!\!\!zz}}\right]}\right), \eqn{} 
$$
so we never actually need to compute this term.  The fifth and sixth terms
of (6.25) cannot be handled so easily.  Their resolution depends on the explicit
form of the isometry and the Jacobian in the scaled cylindrical coordinate
system.

In order to define the isometry conditions, one more definition must be made. 
The scaled centers of the holes have already been defined by (6.16).  The scaled
radius of the first hole is just {\it one} since that length sets the scale for
all quantities.  All that remains is to define a dimensionless quantity to give
the size of the second hole.  I define this quantity by the following:
$$
{a}_{1}=\alpha {a}_{2}. \eqn{}
$$

A straightforward calculation shows that the isometry conditions take the
following dimensionless forms:
$$
{\bf J}_{1}\left({\bf x}\right) =\left[{\matrix{{\rho  \over {\rho
}^{2}+{\left({z-{\zeta }_{1}}\right)}^{2}}&,\phi &,{z-{\zeta }_{1} \over {\rho
}^{2}+{\left({z-{\zeta }_{1}}\right)}^{2}}+{\zeta }_{1}\cr}}\right] \eqn{}
$$
and
$$
{\bf J}_{2}\left({\bf x}\right) =\left[{\matrix{{{\alpha }^{-2}\rho 
\over {\rho }^{2}+{\left({z-{\zeta }_{2}}\right)}^{2}}&,\phi &,{{\alpha
}^{-2}\left({z-{\zeta }_{2}}\right) \over {\rho }^{2}+{\left({z-{\zeta
}_{2}}\right)}^{2}}+{\zeta }_{2}\cr}}\right]. \eqn{}
$$
Based on these two equations, the Jacobians are given by
$$
{\left[{{\bf J}_{1}}\right]}_{\left({\bf x}\right)}
=\left[{\matrix{{{\left({z-{\zeta }_{1}}\right)}^{2}-{\rho }^{2} \over
{\left({{\rho }^{2}+{\left({z-{\zeta }_{1}}\right)}^{2}}\right)}^{2}}&0&{-2\rho
\left({z-{\zeta }_{1}}\right) \over {\left({{\rho }^{2}+{\left({z-{\zeta
}_{1}}\right)}^{2}}\right)}^{2}}\cr 0&1&0\cr {-2\rho \left({z-{\zeta
}_{1}}\right) \over {\left({{\rho }^{2}+{\left({z-{\zeta
}_{1}}\right)}^{2}}\right)}^{2}}&0&{{\rho }^{2}-{\left({z-{\zeta
}_{1}}\right)}^{2} \over {\left({{\rho }^{2}+{\left({z-{\zeta
}_{1}}\right)}^{2}}\right)}^{2}}\cr}}\right] \eqn{}
$$
and
$$
{\left[{{\bf J}_{2}}\right]}_{\left({\bf x}\right)}
=\left[{\matrix{{{\alpha }^{-2}\left({{\left({z-{\zeta }_{2}}\right)}^{2}-{\rho
}^{2}}\right) \over {\left({{\rho }^{2}+{\left({z-{\zeta
}_{2}}\right)}^{2}}\right)}^{2}}&0&{-2{\alpha }^{-2}\rho \left({z-{\zeta
}_{2}}\right) \over {\left({{\rho }^{2}+{\left({z-{\zeta
}_{2}}\right)}^{2}}\right)}^{2}}\cr 0&1&0\cr {-2{\alpha }^{-2}\rho
\left({z-{\zeta }_{2}}\right) \over {\left({{\rho }^{2}+{\left({z-{\zeta
}_{2}}\right)}^{2}}\right)}^{2}}&0&{{\alpha }^{-2}\left({{\rho
}^{2}-{\left({z-{\zeta }_{2}}\right)}^{2}}\right) \over {\left({{\rho
}^{2}+{\left({z-{\zeta }_{2}}\right)}^{2}}\right)}^{2}}\cr}}\right] \eqn{}
$$
For completeness, I display the dimensionless form for the recursion relations
(6.10) and (6.11):
$$
{F}_{1}^{n}=\cases{{\left.{{{F}_{1}^{n-1} \over {\rho }^{2}+{\left({z-{\zeta
}_{1}}\right)}^{2}}}\right|}_{\left({{\bf x}_{ 1}^{n-1}}\right)}&: for $n$
odd\cr {\left.{{{\alpha }^{-2}{F}_{1}^{n-1} \over {\rho }^{2}+{\left({z-{\zeta
}_{2}}\right)}^{2}}}\right|}_{\left({{\bf x}_{ 1}^{n-1}}\right)}&: for $n$
even\cr \hfill 1&: for $n=0$\cr} \eqn{} 
$$
and
$$
{F}_{2}^{n}=\cases{{\left.{{{\alpha }^{-2}{F}_{2}^{n-1} \over {\rho
}^{2}+{\left({z-{\zeta }_{2}}\right)}^{2}}}\right|}_{\left({{\bf x}_{
2}^{n-1}}\right)}&: for $n$ odd\cr {\left.{{{F}_{2}^{n-1} \over {\rho
}^{2}+{\left({z-{\zeta }_{1}}\right)}^{2}}}\right|}_{\left({{\bf x}_{
2}^{n-1}}\right)}&: for $n$ even\cr \hfill 1&: for $n=0$.\cr} \eqn{}
$$

Consider now, in light of the explicit form of the recurrence relations, the
form of the $n^{th}$ term in the series (6.15).  Explicit calculation shows that
the $\rho\rho$, $\rho z$, and $zz$ terms depend only on $\hat{A}_{\rho\rho}$,
$\hat{A}_{\rho z}$, and $\hat{A}_{zz}$.  The $\rho\phi$ and $\phi z$ terms
depend only on $\hat{A}_{\rho\phi}$ and $\hat{A}_{\phi z}$.  Finally, the
$\phi\phi$ term depends only on $\hat{A}_{\phi\phi}$.  This means that the linear
and angular momentum pieces of the inversion-symmetric extrinsic curvature are
completely uncoupled and can be considered separately.  

The two terms of concern in (6.25),
$$
{1 \over \rho }{\left({{\Re }^{\pm }\actson\!\left({{\hat{A} \over
{a}_{1}}}\right)}\!\!\right)}_{\!\!\!\rho\phi}\qquad\hbox{and}\qquad{1 \over
\rho }{\left({{\Re }^{\pm }\actson\!\left({{\hat{A} \over
{a}_{1}}}\right)}\!\!\right)}_{\!\!\!\phi z}, \eqn{}
$$
are angular momentum terms and, as mentioned above, depend only on
$\hat{A}_{\rho\phi}$ and $\hat{A}_{\phi z}$.  Looking at (6.23) and (6.24), we
see that both terms contain overall factors of $\rho$.  This indicates that the
factors of $1/\rho$ should be ``commuted'' with the inversion operator.  Consider
the $n^{th}$ term of the first expression in (6.34). 
$$ {1 \over \rho
}{F}_{1}^{n}{\left[{{M}_{1}^{n}{\hat{A}}_{\left({{\bf x}_{\rm
1}^{n}}\right)}{\tilde{M}}_{1}^{n}}\right]}_{\rho\phi}+{1 \over \rho
}{F}_{2}^{n}{\left[{{M}_{2}^{n}{\hat{A}}_{\left({{\bf x}_{\rm
2}^{n}}\right)}{\tilde{M}}_{2}^{n}}\right]}_{\rho\phi}. \eqn{} 
$$
If we denote the $\rho$ component of ${\bf x}^n_\alpha$ as $\rho^n_\alpha$, then
it is straightforward to verify that 
$$
{1 \over \rho }={{F}_{\alpha }^{n} \over {\rho }_{\alpha }^{n}}. \eqn{}
$$
Equation (6.35) now becomes
$$
{\left({{F}_{1}^{n}}\right)}^{2}{\left[{{M}_{1}^{n}{\left[{{\hat{A} \over \rho
}}\right]}_{\left({{\bf x}_{\rm
1}^{n}}\right)}{\tilde{M}}_{1}^{n}}\right]}_{\rho\phi
}+{\left({{F}_{2}^{n}}\right)}^{2}{\left[{{M}_{2}^{n}{\left[{{\hat{A} \over \rho
}}\right]}_{\left({{\bf x}_{\rm
2}^{n}}\right)}{\tilde{M}}_{2}^{n}}\right]}_{\rho\phi}, \eqn{}
$$
which is completely regular as the $\rho=0$ axis is approached since it depends
only on $\hat{A}_{\rho\phi}$ and $\hat{A}_{\phi z}$, which both have overall
factors of $\rho$.  The second expression in (6.34) can be handled in exactly the
same manner.

The components of the background extrinsic curvature can now be accurately
calculated in cylindrical coordinates, with special care being given to the
angular momentum terms.  If axisymmetry is not present, then a similar
construction can be obtained in terms of Cartesian coordinates.  In this case,
the linear and angular momentum components will not decouple, but there will be
no coordinate singularities to deal with.
\vfill
\eject