\chapterhead{Chapter~{\the\chapnum}: The Conformal and Transverse-Traceless\cr
Decompositions}
%%%%%%%%%%%%%%%%%%%%%
The Hamiltonian or scalar constraint equation, (2.82), and the momentum or
vector constraint equation, (2.83), represent integrability conditions which the
two fields $\gamma_{ij}$ and $K_{ij}$ and any matter fields must satisfy on a
spacelike hypersurface $\Sigma$.  These fields are the initial data which must
be given to solve Einstein's equations as a Cauchy problem and, as a necessary
condition, they must satisfy the constraints.  Since Einstein's equations are
dynamical, the constraints cannot restrict all of the components of the metric
and extrinsic curvature.  Since $\gamma_{ij}$ and $K_{ij}$ are symmetric,
three-dimensional tensors, they each have six independent components.  There are
four constraint equations and so four of the 12 components are not independent. 
But which four?  York has developed a method of  breaking up the metric and
extrinsic curvature into constrained and unconstrained pieces.  This method
involves the conformal decomposition of the initial data and relies on a certain
decomposition of symmetric tensors (York [1972], [1973a], and [1979]).

For generality, I start by considering an $n$ dimensional spacelike hypersurface
$\Sigma$, and restrict $n$ later to be three.  To begin, we conformally relate
the metric $\gamma_{ij}$ to a conformal background metric $\bar\gamma_{ij}$ by
$$
\gamma_{ij} \equiv \psi^{4/(n-2)} \bar\gamma_{ij} \qquad \hbox{and}  \qquad
\gamma^{ij} \equiv {\psi }^{-4/(n-2)}\bar\gamma ^{ij}.\eqn{}
$$
Here, to avoid confusion, I mark tensors in the conformal background space with
an overbar.  This relationship forces other relationships between certain
quantities on the physical hypersurface and the background space.  For example,
from the definition of the connection,
$$
\eqalign{{\Gamma^i}_{jk}&={1 \over 2}\gamma^{i\ell}
\left({\gamma_{j\ell,k}+\gamma_{k\ell,j}-\gamma_{jk,\ell}}\right)\cr
&=\bar\Gamma^{i}{}_{jk}+{2 \over n-2}{\psi }^{-1}\left({{\delta
}_{j}^{i}{\overline{\nabla }}{}_{k}\psi +{\delta }_{k}^{i}{\overline{\nabla
}}{}_{j}\psi -{\bar{\gamma }}{}_{j}{}_{k}{\overline{\nabla }}{}^{i}\psi
}\right),\cr}\eqn{}
$$
we define the change in the connection as
$$
\delta {\Gamma }{}^{i}{}_{j}{}_{k}\equiv {\Gamma
}{}^{i}{}_{j}{}_{k}-{\bar{\Gamma }}{}^{i}{}_{j}{}_{k}={2 \over
n-2}\left({{\delta }_{j}^{i}{\overline{\nabla }}{}_{k}\ln\psi +{\delta
}_{k}^{i}{\overline{\nabla }}{}_{j}\ln\psi -{\bar{\gamma
}}{}_{j}{}_{k}{\bar{\gamma }}{}^{i}{}^{\ell }{\overline{\nabla }}{}_{\ell
}\ln\psi }\right),\eqn{}
$$
where $\overline\nabla_k$ is the covariant derivative compatible with
$\bar\gamma_{ij}$.  If we consider the Riemann tensor
$$
{R_{ijk}}^\ell = {\Gamma^\ell}_{ki,j} - {\Gamma^\ell}_{kj,i} +
{\Gamma^m}_{ki}{\Gamma^\ell}_{mj} - {\Gamma^m}_{kj}{\Gamma^\ell}_{mi}, \eqn{}
$$
we find
$$
\delta{R_{ijk}}^\ell \equiv {R_{ijk}}^\ell - \bar{R}_{ijk}{}^\ell =
\delta{\Gamma^\ell}_{ki,j} - \delta{\Gamma^\ell}_{kj,i} +
\delta{\Gamma^m}_{ki}\delta{\Gamma^\ell}_{mj} -
\delta{\Gamma^m}_{kj}\delta{\Gamma^\ell}_{mi}. \eqn{}
$$
The change in the Ricci tensor is defined easily by
$$
\eqalign{\delta {R}{}_{i}{}_{j}&\equiv {R}{}^{\ell }{}_{i}{}_{\ell
}{}_{j}-{\bar{R}}{}^{\ell }{}_{i}{}_{\ell }{}_{j}={R}{}_{i}{}_{\ell
}{}_{j}{}^{\ell }-{\bar{R}}{}_{i}{}_{\ell }{}_{j}{}^{\ell }=\delta
{R}{}_{i}{}_{\ell }{}_{j}{}^{\ell }\cr &=\delta {\Gamma }{}^{\ell
}{}_{i}{}_{j}{}_{;\ell }-\delta {\Gamma }{}^{\ell }{}_{j}{}_{\ell
}{}_{;i}+\delta {\Gamma }{}^{m}{}_{j}{}_{i}\delta {\Gamma }{}^{\ell }{}_{\ell
}{}_{m}-\delta {\Gamma }{}^{m}{}_{\ell }{}_{i}\delta {\Gamma }{}^{\ell
}{}_{j}{}_{m}\cr &={2 \over n-2}\left[{\left({2-n}\right){\overline{\nabla
}}{}_{i}{\overline{\nabla }}{}_{j}\ln\psi -{\bar{\gamma
}}{}_{i}{}_{j}{\bar{\gamma }}{}^{\ell }{}^{m}{\overline{\nabla }}{}_{\ell
}{\overline{\nabla }}{}_{m}\ln\psi }\right]\cr &\qquad+{4 \over
n-2}\left[{\left({{\overline{\nabla }}{}_{i}\ln\psi }\right){\overline{\nabla
}}{}_{j}\ln\psi -{\bar{\gamma }}{}_{i}{}_{j}{\bar{\gamma }}{}^{\ell
}{}^{m}\left({{\overline{\nabla }}{}_{\ell }\ln\psi }\right){\overline{\nabla
}}{}_{m}\ln\psi }\right].\cr} \eqn{}
$$
Finally, if we consider the effect on the Ricci scalar, we find
$$
\eqalign{R&={\gamma }{}^{i}{}^{j}{R}{}_{i}{}_{j}={\psi
}^{-4/(n-2)}{\bar{\gamma }}{}^{i}{}^{j}{R}{}_{i}{}_{j}\cr &={\psi
}^{-4/(n-2)}\bar{R}-{4\left({n-1}\right) \over n-2}{\psi
}^{-\left({n+2}\right)/\left({n-2}\right)}{\overline{\nabla }}^{2}\psi. \cr}
\eqn{}
$$

With these relations, we can consider a conformal decomposition of the
constraint equations of Einstein's theory.  Here, we will take $n=3$ and the
covariant derivative compatible with $\bar\gamma_{ij}$ is $\bar D_i$.  We also
specify conformal transformations for the extrinsic curvature, the energy
density, and the momentum density.  The trace and trace-free parts of the
extrinsic curvature are considered separately so
$$
{K}{}^{i}{}^{j}={A}{}^{i}{}^{j}+{1 \over 3}{\gamma }{}^{i}{}^{j}K. \eqn{}
$$
For now, we assign the following arbitrary weights to the conformal
transformations.  Values for these will be determined later:
$$
{A}{}^{i}{}^{j}\equiv {\psi }^{\alpha }{\bar{A}}{}^{i}{}^{j}\qquad
\hbox{or}\qquad{A}{}_{i}{}_{j}={\psi }^{\alpha +8}{\bar{A}}{}_{i}{}_{j},\eqn{} 
$$
$$
K\equiv {\psi }^{\beta }\bar{K}, \eqn{}
$$
$$
\rho \equiv {\psi }^{\gamma }\bar{\rho }, \eqn{}
$$
and
$$
{j}{}^{i}\equiv {\psi }^{\delta }{\bar{\jmath}}{}^{i}.
\eqn{}
$$

The Hamiltonian constraint, (2.82), becomes
$$
8{\overline{\nabla }}^{2}\psi -\psi \bar{R}+\epsilon {2 \over 3}{\psi
}^{2\beta +5}{\bar{K}}^{2}-\epsilon {\psi }^{2\alpha
+13}{\bar{A}}{}_{i}{}_{j}{\bar{A}}{}^{i}{}^{j}=\epsilon
2\kappa {\psi }^{\gamma +5}\bar{\rho }. \eqn{}
$$
The momentum constraint equation, (2.83), takes the form
$$
{D}{}_{j}\left({{A}{}^{i}{}^{j}-{2 \over 3}{\gamma
}{}^{i}{}^{j}K}\right)=-\epsilon \kappa {j}{}^{i}. \eqn{}
$$
If we consider the first term on the left-hand side, we find that for any
symmetric, traceless tensor $A^{ij}$ which transforms as (3.9),
$$
{D}{}_{j}{A}{}^{i}{}^{j}={\psi }^{-10}{\bar{D}}{}_{j}\left({{\psi
}^{10+\alpha }{\bar{A}}{}^{i}{}^{j}}\right). \eqn{}
$$
Thus, if we choose $\alpha=-10$, we see that
$$
{D}{}_{j}{A}{}^{i}{}^{j}={\psi
}^{-10}{\bar{D}}{}_{j}{\bar{A}}{}^{i}{}^{j}, \eqn{}
$$
and if $\bar A^{ij}$ is divergenceless, then $A^{ij}$ will be as well.  With
(3.16), the momentum constraint becomes
$$
{\psi }^{-10}{\bar{D}}{}_{j}{\bar{A}}{}^{i}{}^{j}-{2 \over 3}{\psi
}^{\beta -4}{\bar{\gamma }}{}^{i}{}^{j}{\bar{D}}{}_{j}\bar{K}-{2
\over 3}\beta {\psi }^{\beta -5}\bar{K}{\bar{\gamma
}}{}^{i}{}^{j}{\bar{D}}{}_{j}\psi =-\epsilon \kappa {\psi }^{\delta
}{\bar{\jmath}}{}^{i}. \eqn{}
$$
If we choose $\beta=0$ so that the trace of the extrinsic curvature has no
conformal scaling and choose $\delta=-10$, then the momentum constraint
simplifies to
$$
{\bar{D}}{}_{j}{\bar{A}}{}^{i}{}^{j}-{2 \over 3}{\psi
}^{6}{\bar{\gamma }}{}^{i}{}^{j}{\bar{D}}{}_{j}K=-\epsilon \kappa
{\bar{\jmath}}{}^{i}. \eqn{}
$$
The conformal relations of all quantities have now been defined except for
those of $\rho$.  At this point, the Hamiltonian constraint takes the form
$$
8{\overline{\nabla }}^{2}\psi -\psi \bar{R}+\epsilon {2 \over 3}{\psi
}^{5}{K}^{2}-\epsilon {\psi
}^{-7}{\bar{A}}{}_{i}{}_{j}{\bar{A}}{}^{i}{}^{j}=\epsilon
2\kappa {\psi }^{\gamma +5}\bar{\rho }. \eqn{}
$$
A simplifying choice for $\gamma$ would be $\gamma=-5$.  However, this choice
does not lead to a well posed problem for the linearized Hamiltonian
constraint on asymptotically flat hypersurfaces ({\it cf}. York [1979]).  This would
require $\gamma<-5$.  A good choice for $\gamma$ can be found by demanding
that the dominance of energy condition be maintained locally.  This requires
that
$$
{\rho }^{2}-{\gamma }{}_{i}{}_{j}{j}{}^{i}{j}{}^{j}={\psi }^{-16}\left({{\psi
}^{2\gamma +16}{\bar{\rho }}^{2}-{\bar{\gamma
}}{}_{i}{}_{j}{\bar{\jmath}}{}^{i}{\bar{\jmath}}{}^{j}}\right)\ge 0,
\eqn{}
$$
and we take $\gamma=-8$.  The Hamiltonian constraint now takes the form
$$
8{\overline{\nabla }}^{2}\psi -\psi \bar{R}+\epsilon {2 \over 3}{\psi
}^{5}{K}^{2}-\epsilon {\psi
}^{-7}{\bar{A}}{}_{i}{}_{j}{\bar{A}}{}^{i}{}^{j}=\epsilon
2\kappa {\psi }^{-3}\bar{\rho }. \eqn{}
$$
The final step which may be taken in simplifying the constraints is to note that
in the momentum constraint, $\bar A^{ij}$ only occurs in a divergence term so
that the transverse part of $\bar A^{ij}$ is not restricted by the equations. 
Thus, we may split the tensor into its transverse and longitudinal parts. 
Recalling that $\bar A^{ij}$ is traceless, the longitudinal part of $\bar
A^{ij}$ is defined as the symmetric, traceless gradient of a vector ``potential''
$W^i$:
$$
{\bar{A}}_{\scriptscriptstyle L}^{ij}=
{\bar{D}}{}^{i}{W}{}^{j}+{\bar{D}}{}^{j}{W}{}^{i}-{2 \over 3}{\bar{\gamma
}}{}^{i}{}^{j}{\bar{D}}{}_{k}{W}{}^{k}\equiv {\left({LW}\right)}{}^{i}{}^{j}.
\eqn{}
$$
We then find
$$
{\bar{A}}{}^{i}{}^{j}= \bar{A}_{\scriptscriptstyle
TT}^{ij}+{\left({LW}\right)}{}^{i}{}^{j}, \eqn{}
$$
where $\bar A_{\scriptscriptstyle TT}^{ij}$ is the transverse traceless part of
$\bar A^{ij}$ and satisfies 
$$
{\bar{D}}{}_{j} \bar A_{\scriptscriptstyle TT}^{ij}=0. \eqn{}
$$
The divergence of the traceless part of the extrinsic curvature then reduces to
$$
\eqalign{{\bar{D}}{}_{j}{\bar{A}}{}^{i}{}^{j}&={\bar{D}}{}_{j}
{\left({LW}\right)}{}^{i}{}^{j}={\bar{D}}{}_{j}{\bar{D}}{}^{j}{W}{}^{i}+
{1\over3}{\bar{D}}{}^{i}\left({{\bar{D}}{}_{j}{W}{}^{j}}\right)+
{\bar{R}}{}_{j}{}^{i}{W}{}^{j}\cr&\equiv
{\left({{\Delta }_{\scriptscriptstyle L} W}\right)}{}^{i},\cr} \eqn{} 
$$
and the momentum constraint is
$$
{\left({{\Delta }_{\scriptscriptstyle L} W}\right)}{}^{i}-{2 \over 3}{\psi
}^{6}{\bar{\gamma }}{}^{i}{}^{j}{\bar{D}}{}_{j}K=-\epsilon \kappa
{\bar{\jmath}}{}^{i}. \eqn{}
$$

While the background field $\bar A^{ij}$ is naturally split into its transverse
and longitudinal parts, it should be noted that the split in the physical field
is not so distinct.  Equation (3.16) guarantees that the transversality of
$\bar{A}_{\scriptscriptstyle TT}^{ij}$ is preserved under the conformal
transformation to the physical space.  However, York [1973b] has shown that the
``longitudinal'' part of $A^{ij}$, given by
$$
A_{\scriptscriptstyle L}^{ij}={\psi }^{-10}{\left({LW}\right)}{}^{i}{}^{j},
\eqn{}
$$
is not orthogonal to $\psi^{-10}\bar{A}_{\scriptscriptstyle TT}^{ij}$ ({\it
cf}. Evans [1984]) and so $\bar{A}_{\scriptscriptstyle L}^{ij}$ is {\it not}, in
general, purely longitudinal.

The constraint equations (3.21) and (3.26) determine the conformal factor $\psi$
and the vector potential $W^i$.  We are free to choose the remaining parts of the
gravitational initial data in any way we like.  Specifically, the conformal
background metric $\bar\gamma_{ij}$, the trace of the extrinsic curvature $K$, and
the transverse-traceless part of the background extrinsic curvature $\bar
A_{\scriptscriptstyle TT}^{ij}$ are all freely specifiable fields, as are the
background sources $\bar\rho$ and $\bar \jmath^i$.  In fact, these {\it must} be
specified before the constraint equations can be solved.

If we recall that $\gamma_{ij}$ has six independent components and use the
Hamiltonian constraint to fix the conformal factor $\psi$, then the background
metric $\bar\gamma_{ij}$ must have five independent components.  These are not,
however, all dynamical degrees of freedom.  We must remember that there is
coordinate freedom within the hypersurface $\Sigma$ and this eliminates three of
the remaining five independent components from being dynamical degrees of
freedom.  This tells us that the equivalence class of conformal three-geometries
represents the two freely specifiable, dynamical degrees of freedom of the metric
$\gamma_{ij}$.  Similarly, with the extrinsic curvature, the momentum constraint
governs three of the six independent components of $K^{ij}$.  Remaining are the
transverse-traceless part,  $\bar A_{\scriptscriptstyle TT}^{ij}$, with two
degrees of freedom and the trace, $K$, with one.  The trace of the extrinsic
curvature is taken as a condition on the time coordinate and is not considered as
a dynamical degree of freedom.  Thus, the two freely specifiable, dynamical
degrees of freedom of the extrinsic curvature are carried by its
transverse-traceless parts.
\vfill
\eject