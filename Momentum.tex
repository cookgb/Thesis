\chapterhead {Chapter~\the\chapnum:  Solutions of the Momentum Constraint}
%%%%%%%%%%%%%%%%%%%%%%%%%%%%%%%%%%
Solving the momentum constraint equations is the first step in constructing
inversion-symmetric gravitational initial data.  If the initial-data slice is
chosen to be maximal, or more generally if the trace of the extrinsic curvature
is a constant on the initial slice, then the momentum constraint (4.2) decouples
from the Hamiltonian constraint (4.1).  This is one of the prime strengths of
the conformal decomposition of the constraints, and can be exploited in
certain cases to find analytic solutions for the background extrinsic
curvature.  When the initial slice is conformally and asymptotically flat, this
decoupling also allows for the determination of two of the physical properties
of the initial-data slice without knowing the complete initial data, as is shown
below.

The momentum contained in an asymptotically flat, initial-data slice can be
calculated from the integral
$$
{\Pi }{}_{i}{\xi }{}^{i}{}_{(k)}={1 \over \kappa }\oint_{\infty
}^{}\left({{K}{}_{i}{}^{j}-{\delta }_{i}^{j}K}\right){\xi
}{}^{i}{}_{(k)}{d}^{2}{S}{}_{j}, \eqn{}
$$
where ${\xi^i}_{(k)}$ is a Killing vector of the spatial metric $\gamma_{ij}$ (York
[1980]).  In the conformal imaging approach, the initial slice is conformally flat,
as well as asymptotically flat.  Since the background metric is flat and the
conformal factor must behave like (4.10) at infinity, the integral (5.1) can be
expressed equivalently in terms of the background extrinsic curvature and the flat
metric.  Recalling (4.5), we can now rewrite (5.1) as
$$
{\Pi }{}_{i}{\bar{\xi }}{}^{i}{}_{(k)}={1 \over \kappa }\oint_{\infty
}^{}{\bar{A}}{}_{i}{}^{j}{\bar{\xi
}}{}^{i}{}_{(k)}{d}^{2}{\bar{S}}{}_{j}. \eqn{}
$$
Here, the integral is in terms of the flat background metric, and
${\bar\xi^i}{}_{(k)}$ is a Killing vector of the flat background metric $f_{ij}$ to
which $\gamma_{ij}$ is asymptotic. If ${\bar\xi^i}{}_{(k)}$ is a translational
Killing vector, then ${\Pi_i}{\bar\xi^i}{}_{(k)}$ is the linear momentum $P_k$.  In
terms of Cartesian coordinates, the total {\it physical} linear momentum of a slice
is given by 
$$
{P}{}^{i}={1 \over \kappa }\oint_{\infty
}^{}{\bar{A}}{}^{i}{}^{j}{d}^{2}{\bar{S}}{}_{j}. \eqn{}
$$
If ${\bar\xi^i}{}_{(k)}$ is a rotational Killing vector, then
${\Pi_i}{\bar\xi^i}{}_{(k)}$ is the angular momentum $S_k$.  In terms of Cartesian
coordinates, the total {\it physical} angular momentum of a slice is given by
$$
{S}{}_{i}={1 \over \kappa }{\epsilon }{}_{i}{}_{j}{}_{k}\oint_{\infty
}^{}{x}{}^{j}{\bar{A}}{}^{k}{}^{\ell }{d}^{2}{\bar{S}}{}_{\ell }.
\eqn{}
$$
It should be noted that the angular momentum at infinity is not always well
defined.  The angular momentum $S_k$ will be asymptotically gauge invariant,
however, if ${\bar\xi^i}{}_{(k)}$ is an exact symmetry of the the physical metric
$\gamma_{ij}$ ({\it cf}. York [1980]).  So, given a solution for the background
extrinsic curvature, we see from (5.3) and (5.4) that two important properties of
the physical initial-data set can be determined without knowing the full initial
data, i.e., without solving the Hamiltonian constraint.

In the context of the conformal imaging approach, we would like to find
solutions to the momentum constraint (4.9) which satisfy the isometry condition
(4.45) and represent black holes with linear momentum $P^i_\alpha$ and angular
momentum $S^i_\alpha$ as determined by (5.3) and (5.4).  For the case of a single
black hole, Bowen and York [1980] have found such solutions and they take the
following form:
$$
\eqalign{{\bar{A}}^{\pm }_{ij}={3\kappa  \over 16\pi
{r}^{2}}&\left[{{P}{}_{i}{n}{}_{j}+{P}{}_{j}{n}{}_{i}-\left({{f}{}_{i}{}_{j}-{n}{}_{i}{n}{}_{j}}\right){P}{}^{k}{n}{}_{k}}\right]\cr
&\mp{3\kappa {a}^{2} \over 16\pi
{r}^{4}}\left[{{P}{}_{i}{n}{}_{j}+{P}{}_{j}{n}{}_{i}+\left({{f}{}_{i}{}_{j}-5{n}{}_{i}{n}{}_{j}}\right){P}{}^{k}{n}{}_{k}}\right]\cr}
\eqn{}
$$
and
$$
{\bar{A}}_{ij}={3\kappa  \over 8\pi {r}^{3}}\left[{{\epsilon
}{}_{k}{}_{i}{}_{\ell }{S}{}^{\ell }{n}{}^{k}{n}{}_{j}+{\epsilon
}{}_{k}{}_{j}{}_{\ell }{S}{}^{\ell }{n}{}^{k}{n}{}_{i}}\right]. \eqn{}
$$
Note, that we have dropped the subscript $\alpha$ on $a$, $r$, $n^i$, $P^i$, and
$S^i$ since there is only one hole.  Equation (5.5) gives the background extrinsic
curvature for a hole with linear momentum $P^i$ and is actually two solutions.  The
solution $\bar{A}^+_{ij}$ satisfies the isometry condition (4.45) with the plus
sign, and the solution $\bar{A}^-_{ij}$ satisfies the isometry condition (4.45)
with the minus sign.  Only the $\Order(r^{-2})$ term in (5.5) carries linear momentum as
can be seen by evaluating (5.3).  The second term is required for inversion
symmetry.  The two terms are individually solutions of the momentum constraint
(4.9a).  Equation (5.6) gives the background extrinsic curvature for a hole with
angular momentum $S^i$.  This solution satisfies the isometry condition with the
minus sign only and is its own inverse.

The solutions (5.5) and (5.6) are longitudinal since they can be derived from a
vector potential.  \'{O} Murchada and York [1976a] found solution to (4.9b) for
the vector potential $W^i$ for the first term in (5.5) and for (5.6).  Bowen
[1979a] found the solution giving the second term in (5.5).  The first term in
(5.5), the part carrying linear momentum, is derived from the vector potential
$$
{W}{}^{i}=-{\kappa  \over 32\pi
r}\left[{7{P}{}^{i}+{n}{}^{i}{n}{}_{j}{P}{}^{j}}\right]. \eqn{}
$$
The second part of (5.5) can also be derived from a vector potential.  It takes
the form
$$
{W}{}^{i}={\kappa {a}^{2} \over 32\pi
{r}^{3}}\left[{{P}{}^{i}-3{n}{}^{i}{n}{}_{j}{P}{}^{j}}\right]. \eqn{}
$$
As previously mentioned, this term carries no linear momentum and is present
only to enforce inversion symmetry.  The solution representing a black hole with
spin, (5.6), is derived from the vector potential
$$
W^i={\kappa  \over 8\pi r^2}{\epsilon^{ijk}}{n_j}{S_k}. \eqn{}
$$

The solutions (5.5) and (5.6) satisfy all of the conditions required by the
conformal imaging approach and can be used with the Hamiltonian constraint to
construct inversion-symmetric initial data for a single hole with linear and
angular momentum.  Such initial data has been constructed by many researchers ({\it
cf}.  York and Piran [1982], Choptuik and Unruh [1986], Rauber [1985], and Cook and
York [1990]) and will be examined in later chapters.  I will proceed here with the
question of how to construct an inversion-symmetric solution of the momentum
constraint for systems with multiple black holes.

This problem was addressed formally by Kulkarni, Shepley, and York [1983].  The
approach is centered on the fact that the momentum constraint (4.9) is linear. 
Since this is the case, a solution representing $N$ black holes, each with linear
and angular momentum can be obtained from (5.5) and (5.6) as
$$
\eqalign{ \bar{A}_{ij}={3\kappa  \over 8\pi }\sum\limits_{\alpha=1}^{N}\left({
{1 \over 2 r_\alpha^2} \left[{ P^\alpha_i n^\alpha_j+ P^\alpha_j n^\alpha_i- \left({
f_{ij}-n^\alpha_i n^\alpha_j }\right) P_\alpha^k n^\alpha_k }\right] }\right.&\cr
\left.{ +{1 \over r_\alpha^3} \left[{ \epsilon_{ki\ell} S_\alpha^\ell
n_\alpha^k n^\alpha_j +\epsilon_{kj\ell} S_\alpha^\ell n_\alpha^k n^\alpha_i
}\right] }\right)&.\cr } \eqn{} 
$$ 
Note that the second term in (5.5) has been omitted here.  The $N$ black holes are
given individual linear and angular momenta $P^i_\alpha$ and $S^i_\alpha$ in the
sense that if the holes are ``sufficiently far apart'', then they will have such
momenta as measured by (5.3) and (5.4).  If the holes are close together, then all
that can be measured is the total linear and angular momentum of the system.

The background extrinsic curvature given by (5.10) is longitudinal and traceless
and is a solution of the momentum constraint (4.9), however, it is not inversion
symmetric through any of the holes as is required.  This is why the second
term in (5.5) was not included for each hole in (5.10).  Even though (5.10) is not
inversion symmetric, Kulkarni {\it et al}. [1983] have shown how to take this
non-invertable extrinsic curvature and make it inversion symmetric through a method
of images applicable to tensors.  I will outline the formal approach below.

In order for the extrinsic curvature to be inversion symmetric, it must satisfy
$$
\eqalign{\bar{A}_{ij}\left({\bf x}\right) &=\Re_\alpha^\pm
\actson\bar{A}_{ij}\left({\bf x}\right)\cr &=\pm {\left({{{a}_{\alpha } \over
{r}_{\alpha }}}\right)}^{2}{\left({{\bf J}_{\alpha }}\right)}{}_{i}{}^{\ell
}{\bar{A}}{}_{\ell }{}_{m}\left({{\bf J}_{\alpha }\left({\bf
x}\right)}\right){\left({{\bf J}_{\alpha }}\right)}{}_{j}{}^{m}\cr}
\eqn{}
$$
for all $\alpha$.  In order to simplify and clarify notation, I will drop the
component indices and write (5.11) as
$$
\eqalign{\bar{A}\left({\bf x}\right) &={\Re }_{\alpha }^{\pm }\actson
\bar{A}\left({\bf x}\right)\cr &=\pm {\left({{{a}_{\alpha } \over
{r}_{\alpha }}}\right)}_{\left({\bf x}\right)}^{2}{\left[{{\bf J}_{\alpha
}}\right]}_{\left({\bf x}\right)}\bar{A}\left({{\bf J}_{\alpha
}\left({\bf x}\right)}\right){\left[{{\bf J}_{\alpha }}\right]}_{\left({\bf
x}\right)}.\cr} \eqn{}
$$
Note that in order to distinguish the Jacobian from the transformation ${\bf
J}_\alpha$, the Jacobian is enclosed in square brackets.  Let $M_{ij}({\bf x})$ be
any second rank tensor field on $E^3$.  It is straightforward to show the following
properties of the inversion operator $\Re^\pm_\alpha$ ({\it cf}. Kulkarni {\it et
at} [1983]). 
$$
{\Re }_{\alpha }^{\pm }\actson {\Re }_{\alpha }^{\pm }=\Unitmatrix\,. \eqn{}
$$
$$
{\left({{\Re }_{\alpha }^{\pm }\actson M}\right)}^{t}={\Re }_{\alpha }^{\pm
}\actson \left({{M}^{t}}\right), \eqn{}
$$
where $M^t$ is the transpose of $M$.
$$
{\left[{Tr\left({{\Re }_{\alpha }^{\pm }\actson
M}\right)}\right]}_{\left({\bf x}\right)} =\pm {\left({{{a}_{\alpha } \over
{r}_{\alpha }}}\right)}_{\left({\bf x}\right)}^{6}\left[{{\left({Tr\,
M}\right)}_{\left({{\bf J}_{\alpha }\left({\bf x}\right)}\right)}}\right], \eqn{}
$$
where $Tr\, M$ is the trace of $M$.
$$
\eqalign{ {\left[{ \bar{D}_j\left({\Re_\alpha^\pm\actson{M}^{ij}}\right)
}\right]}_{\left({\bf x}\right)}  =\pm {\left({a_\alpha \over
r_\alpha}\right)}_{\left({\bf x}\right)}^{6} &\Biggl\{{ {{\left({ {\bf J}_\alpha
}\right)}_{\left( {\bf x} \right)}}_{k}{}^{i} {\left[{ \bar{D}_\ell M^{k\ell}
}\right]}_{\left({{\bf J}_\alpha\left({\bf x}\right)}\right)} }\Biggr.\cr &
\quad +\Biggl.{ 2{\left({{{n}_{\alpha }^{i} \over {r}_{\alpha
}}}\right)}_{\left({\bf x}\right)}{\left[{ Tr\, M}\right]}_{\left({{\bf J}_{\alpha
}\left({\bf x}\right)}\right)}}\Biggr\}.\cr} \eqn{} 
$$

If we start with a symmetric and traceless solution, $\bar{A}_{ij}$, to the momentum
constraint (4.9a), then (5.14), (5.15), and (5.16) guarantee that the ``image'' of
it, $\Re^\pm_\alpha\actson\bar{A}_{ij}$, will also be symmetric, traceless, and
satisfy the momentum constraint.  Now consider the following operator:
$$
{\Re }^{\pm }=\Unitmatrix+\sum\limits_{\left\{{{\alpha }_{i}}\right\}}^{}
\left({\prod\limits_{i=1}^{m} {\Re }_{{\alpha }_{i}}^{\pm }}\right).
\eqn{}
$$
The notation here needs some elaboration.  The $\alpha_i$ label the holes as usual
and the subscript $i$ takes on the values of $1,\ldots,N$.  The sum is over all
unique sequences $\{\alpha_i\}$ of length $m$.  For any sequence, $\{\alpha_i\}$,
one takes the product of ``image'' operators corresponding to that sequence.  From
property (5.13), a sequence is unique if and only if $\alpha_i \not=
\alpha_{i+1}$.  As an example of this operator, consider the case of a single
hole.  The operator (5.17) takes, in this case, the form
$$
{\Re }^{\pm }=\Unitmatrix+{\Re }_{{\alpha }_{1}}^{\pm }. \eqn{}
$$
For two holes, it takes the form
$$
\eqalign{{\Re }^{\pm }=\Unitmatrix &+{\Re }_{{\alpha }_{1}}^{\pm }+{\Re }_{{\alpha
}_{1}}^{\pm }\actson{\Re }_{{\alpha }_{2}}^{\pm }+{\Re }_{{\alpha }_{1}}^{\pm
}\actson{\Re }_{{\alpha }_{2}}^{\pm }\actson{\Re }_{{\alpha }_{1}}^{\pm }+{\Re
}_{{\alpha }_{1}}^{\pm }\actson{\Re }_{{\alpha }_{2}}^{\pm }\actson{\Re }_{{\alpha
}_{1}}^{\pm }\actson{\Re }_{{\alpha }_{2}}^{\pm }+\cdots\cr &+{\Re }_{{\alpha
}_{2}}^{\pm }+{\Re }_{{\alpha }_{2}}^{\pm }\actson{\Re }_{{\alpha }_{1}}^{\pm
}+{\Re }_{{\alpha }_{2}}^{\pm }\actson{\Re }_{{\alpha }_{1}}^{\pm }\actson{\Re
}_{{\alpha }_{2}}^{\pm }+{{\Re }_{{\alpha }_{2}}^{\pm }\actson\Re }_{{\alpha
}_{1}}^{\pm }\actson{\Re }_{{\alpha }_{2}}^{\pm }\actson{\Re }_{{\alpha }_{1}}^{\pm
}+\cdots.\cr} \eqn{}
$$
It can be easily verified from (5.13), that
$$
{\Re }_{\alpha }^{\pm }\actson {\Re }^{\pm }={\Re }^{\pm } \eqn{}
$$
for all $\alpha$ and so (5.17) can be called the inversion symmetry operator.  If it
acts on any arbitrary second rank tensor $M_{ij}$, then the result will be inversion
symmetric through all $N$ holes.  In particular, if $M_{ij}$ is symmetric,
traceless, and satisfies the momentum constraint, then $\Re^\pm M_{ij}$ will be
symmetric, traceless, satisfy the momentum constraint, and be inversion symmetric.

The formal approach for obtaining an inversion-symmetric solution of the
momentum constraint (4.9a) is now obvious.  Let $\hat{A}_{ij}$ represent a sum of
single hole solutions to the momentum constraint.  The inversion-symmetric
solution is then given by
$$
{\bar{A}}{}_{i}{}_{j}={\Re }^{\pm }\actson {\hat{A}}{}_{i}{}_{j}. \eqn{}
$$
In the case of a single hole, (5.21) will be $\hat{A}_{ij}$ plus a single image
term.  Specifically, it will be a combination of (5.5) and (5.6).  If there is
more than one hole, then (5.21) will consist of $\hat{A}_{ij}$ plus an infinite
number of image terms.  In this case, one must be concerned with the convergence
of this infinite series.  Kulkarni [1984] has shown that for any reasonable
choice for $\hat{A}_{ij}$, the infinite series converges if the holes are far enough
apart.  In particular, if there are just two holes, then the infinite series
converges absolutely provided the inversion surfaces do not overlap.  This is in
direct analogy with Misner's proof of convergence for the inversion-symmetric
conformal factor on a time-symmetric initial slice (Misner [1963]).

To continue, we need to specify an explicit form for $\hat{A}_{ij}$.  A first guess
would be to use (5.10).  This would lead to an error in the computation of spins
since the $\alpha^{th}$ spin term in (5.10) is the negative of its inverse through the
$\alpha^{th}$ hole.  Using (5.10) in (5.21) would result in holes with twice the
desired spin for each hole.  There is no problem with the terms containing linear
momentum since the full, inversion-symmetric form is not used.  This error
pertaining to the spin exists in Kulkarni {\it et al}. [1983], Kulkarni [1984], and
York [1984].   Bowen, Rauber, and York [1984] correct for this error by redefining
(5.21) with an overall factor of $1/2$.  This will work provided that the full,
inversion-symmetric solutions containing linear momentum (5.5) are used.  One must
be careful, however, to choose the correct sign in (5.5) to match that of (5.21). 
The more elegant, and more computationally convenient method, is to define
$\hat{A}_{ij}$ as 
$$
\eqalign{ \hat{A}_{ij}={3\kappa  \over 16\pi }\sum\limits_{\alpha=1}^{N}\left({
{1 \over r_\alpha^2} \left[{ P^\alpha_i n^\alpha_j+ P^\alpha_j n^\alpha_i- \left({
f_{ij}-n^\alpha_i n^\alpha_j }\right) P_\alpha^k n^\alpha_k }\right] }\right.&\cr
\left.{ +{1 \over r_\alpha^3} \left[{ \epsilon_{ki\ell} S_\alpha^\ell
n_\alpha^k n^\alpha_j +\epsilon_{kj\ell} S_\alpha^\ell n_\alpha^k n^\alpha_i
}\right] }\right)&.\cr} \eqn{}
$$

Consider now, the total linear and angular momenta contained in the
inversion-symmetric background extrinsic curvature defined by (5.17), (5.21), and
(5.22).  $\bar{A}_{ij}$ will contain three types of terms.  First, it will contain
$\hat{A}_{ij}$ itself.  It will also contain the self-image of each term in
$\hat{A}_{ij}$.  Finally, it will contain arbitrary image terms.  Computing the
total linear momentum (5.3) from $\hat{A}_{ij}$ yields the sum of the individual
linear momenta $P^\alpha_i$.  As has already been mentioned, the self-image of the
linear momentum terms carry no linear or angular momentum.

If we next assume that none of the holes has any linear momentum, then the total
angular momentum (5.4) contained in $\hat{A}_{ij}$ is simply half the sum of the
individual angular momenta $S_{\alpha_i}$.  If we choose the isometry relation with
the minus sign, then the self-image of the angular momentum terms again carries half
the sum of the individual angular momenta $S_{\alpha_i}$.  This is why (5.22) is used
instead of (5.10).  If the isometry relation with the plus sign is used, the
angular momentum from the self-image terms will cancel that from $\hat{A}_{ij}$.  If
the holes have non-zero linear momentum, then the linear momentum terms in
$\hat{A}_{ij}$ can contain non-zero angular momentum although their self-image terms
cannot.

Remaining to be examined are the contributions to the total linear and angular
momentum from the general image terms.  Bowen, Rauber, and York [1984] detail an
elegant proof that no general image term can carry angular momentum.  Their
proof, however, cannot be generalized to examine their linear momentum content. 
Through tedious, explicit calculations, it can be shown that at large distances
from the holes, the leading behavior of all of the general image terms is
$\Order(r^{-6})$.  A term with this fall-off behavior contributes nothing to the
integrals (5.3) and (5.4) and thus carries no linear or angular momentum.  This is a
very powerful result.  It means that the total linear and angular momentum content
of an initial-data slice can be computed without knowing the explicit form of the
general image terms.  The only terms which contribute are the ``base'' extrinsic
curvature $\hat{A}_{ij}$ and the self-image terms of the angular momentum terms
(which are trivial).

As a final note to the formalism for constructing inversion-symmetric extrinsic
curvature solutions, it should be pointed out that the imaging is done on the
purely longitudinal field $\hat{A}_{ij}$ and not the potential $W^i$.  For the case
of a single black hole, the resulting, imaged field $\bar{A}_{ij}$ is still
longitudinal.  In the case of multiple black holes, this does not seem to be the
case.  Rauber [1985] has shown in a lengthy calculation that the resulting field
$\bar{A}_{ij}$ is not derivable from a vector potential.  This implies that
$\bar{A}_{ij}$ contains some transverse components.  This is not a problem since we
are free to specify the transverse-tracefree content of the extrinsic curvature. 
This is the reason that the less restrictive form of the momentum constraint (4.9a)
has been retained and explains further the note that the assumption (4.7) would be
relaxed.

At this point, we have a {\it formalism} for the construction of
inversion-symmetric, background extrinsic curvatures.  In order to find solutions to
the Hamiltonian constraint, we must be able to move past the formalism and find a
concrete method for computing these extrinsic curvatures.  In the case of a single
hole, the resulting inversion-symmetric, background extrinsic curvature is given in
(5.5) and (5.6).  If we consider more than one hole, the evaluation of an infinite
series must be dealt with.

The situation of primary concern is the two body problem so we need to find a
method for calculating the extrinsic curvature for two holes.  Bowen, Rauber,
and York [1984] have addressed this problem for a certain special case.  They
considered the situation in which the two holes are of equal size, have no
linear momentum, and have angular momenta which are equal in magnitude.  With
the further restriction that the direction of the angular momenta be either
aligned or anti-aligned with the axis between the two holes, they found an
analytic expression for each term in the infinite series.  Even in this
situation, which contains a high degree of symmetry, their method is tedious and
results in an expression involving complicated combinations of hyperbolic and
trigonometric functions.  If this approach is applied to the case of axially
aligned linear momentum or to the case where the linear or angular momentum is
not axisymmetric, then the method becomes completely intractable.  In order to
proceed, another avenue must be taken.  The solution which I have found to the
problem is to abandon the attempt to find an analytic expression for each term
in the infinite series.  Instead, the solution lies in a constructive,
computational approach which I detail in the next chapter.
\vfill
\eject
