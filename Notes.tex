\chapterhead{Notation and Conventions}
%%%%%%%%%%%%%%%%%%%%%%%%%
This work conforms to the notation and conventions of Misner, Thorne, and
Wheeler [1973] unless otherwise noted.  The units used in this work are kept
arbitrary throughout the development of the theoretical foundations by the use
of the proportionality constant $\kappa = 8\pi{G}/c^4$ in Einstein's equations
$G^{\mu\nu}=\kappa{T^{\mu\nu}}$ and in other dimensionfull quantities.  In the
numerical work, gravitational units ($G=c=1$; $\kappa=8\pi$) are used.

Lower case Greek indices take on the values 0,1,2,3.  If the manifold is
pseudo-Riemanian, then the ``zero'' index is associated with the timelike
direction and the signature of the metric is ($-$+++).  In Chapter~2, Riemanian
manifolds are considered in which the signature of the metric is (++++).

Lower case Latin indices take on the values 1,2,3 and indicate that the tensor
is purely spatial.  The signature of the spatial metric is (+++).  In Chapter~3,
spatial hypersurfaces of arbitrary dimension are considered.  In this case,
Latin indices take on the values 1,2,$\ldots$ and the signature of the spatial
metric is (++$\ldots$).

Partial derivatives are written either explicitly or with comma notation
$$
{\partial V^\mu \over \partial x^\nu} = \partial_\nu V^\mu \equiv
{V^\mu}_{,\nu}. $$
General covariant derivatives are expressed by either $\nabla_\nu$ or by
semicolon notation
$$
{\nabla_\nu V^\mu} \equiv {V^\mu}_{;\nu}.
$$
The induced, spatial covariant derivative on a three-dimensional, spacelike
hypersurface is expressed by $D_i$.

Beginning in Chapter~3, a background space is used which is conformally related
to the physical space.  Certain fields are used in both spaces by means of
conformal transformations.  Conformally transformed fields are expressed using
the same symbol used for the field in the physical space with the addition of an
overbar.  For example, the physical metric is denoted by $\gamma_{ij}$ and the
conformal metric by $\bar\gamma_{ij}$.  The spatial covariant derivative
compatible with the conformal metric is denoted by $\bar{D}_i$.  Also, the
conformal Laplacian is denoted by $\overline{\nabla}^2 \equiv
\bar\gamma^{ij}\bar{D}_i\bar{D}_j$.

Beginning in Chapter~4, spatial hypersurfaces which are constructed from
manifolds containing multiple ``throats'' or ``holes'' are considered.  To
indicate that a certain quantity is associated with a specific hole, a lower
case Greek index will be used.  This is not a tensor index and its position (up
or down) is not significant.  Tensor indices on these fields will be denoted
using lower case Latin indices since the fields are purely spatial.

\vfill
\eject