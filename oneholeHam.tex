\hfuzz=4pt
\chapterhead {Chapter~\the\chapnum:  Numerical Solution of the Hamiltonian
Constraint\cr for a Single, Axisymmetric Black Hole}
\hfuzz=0.1pt
%%%%%%%%%%%%%%%%%%%%%%%%%%%%%%%%
As a first example of finding numerical solutions for the Hamiltonian constraint
within the framework of the conformal imaging approach, I will examine the case
of a single black hole with axisymmetric linear or angular momentum.  In this
chapter, I will describe in detail the procedure for differencing the
Hamiltonian constraint and discuss the methods used to solve the finite
difference equations.  I will tabulate the numerical results obtained by
sweeping through the parameter space (i.e. the magnitude of the linear and
angular momenta of the hole), but I will reserve the physical interpretation
for a later chapter.  I should note that several authors (York and Piran [1982],
Choptuik [1982], and Rauber [1985]) have examined these solutions previously. 
While the differencing scheme which I use is slightly different from that of any
of the previous authors, the results I obtain are essentially the same.

I will begin by choosing a coordinate system which is appropriate for the
physical configuration.  The obvious choice is spherical polar coordinates with
the center of the black hole coincident with the origin of coordinates.  This
coordinate system is perfectly adapted for imposing the boundary conditions at
both the throat (or inversion surface) and at large distances.  For numerical
efficiency, I will use a logarithmically scaled, dimensionless radial coordinate
$x$ defined by
$$
x=\ln\left({{r \over a}}\right), \eqn{}
$$
where $a$ is the radius of the throat.  This scaling is chosen because the
curvature of the initial-data slice is expected to be greatest near the throat
of the black hole.  In the case of the conformal factor, this curvature will
show up as a gradient in the conformal factor and so the greatest changes in
the conformal factor occur in the region near the throat.  The coordinate choice
(8.1) will tend to emphasize the region near the throat in the finite difference
equations.  With the coordinate transformation (8.1), the dimensionless
coordinates covering the manifold are $(x,\theta,\phi)$ and the metric is given
by 
$$
{ds}^{2}={a}^{2}{e}^{2x}\left({{dx}^{2}+{d\theta }^{2}+{\sin}^{2}\theta {d\phi
}^{2}}\right). \eqn{}
$$
The domain of the top sheet in which the Hamiltonian constraint must be solved
is $0 \le x < \infty$ and the throat (or minimal surface) is located at $x = 0$.

To simplify the numerical problem of solving the Hamiltonian constraint, I will
restrict myself to axisymmetric configurations.  The only restriction this
imposes in the case of a single hole (as mentioned in Chapter 7) occurs in the
case in which the hole has {\it both} linear and angular momenta.  In this case,
to remain axisymmetric, the linear and angular momentum vectors must be parallel
or anti-parallel.  I will take the symmetry axis to coincide with the $z$-axis
and so the $\phi$ coordinate can be ignored.  From (8.2) and axisymmetry, the
Laplace operator takes the form
$$
{\overline{\nabla }}^{2}={a}^{-2}{e}^{-2x}\left\{{{e}^{-x}{\partial  \over
\partial x}\left({{e}^{x}{\partial  \over \partial x}}\right)+{1 \over
\sin\theta }{\partial  \over \partial \theta }\left({\sin\theta {\partial  \over
\partial \theta }}\right)}\right\}. \eqn{}
$$

The inversion-symmetric extrinsic curvatures for a single hole (as found by Bowen
and York [1980]) were given in (5.5) and (5.6).  In terms of this coordinate
system, the dimensionless components are
$$
{\bar{A}^{+}_{ij} \over a}=3\left({{P \over a}}\right) 
\left[{\matrix{2{e}^{-x}\cosh x \cos\theta &-{e}^{-x}\sinh x \sin\theta &0\cr
-{e}^{-x}\sinh x \sin\theta &-{e}^{-x}\cosh x \cos\theta &0\cr
0&0&-{e}^{-x}\cosh x \cos\theta {\sin}^{2}\theta \cr}}\right], \eqn{} 
$$
$$
{\bar{A}^{-}_{ij} \over a}=3\left({{P \over a}}\right)
\left[{\matrix{2{e}^{-x}\sinh x \cos\theta &-{e}^{-x}\cosh x \sin\theta &0\cr
-{e}^{-x}\cosh x \sin\theta &-{e}^{-x}\sinh x \cos\theta &0\cr
0&0&-{e}^{-x}\sinh x \cos\theta {\sin}^{2}\theta \cr}}\right], \eqn{}
$$
and
$$
{\bar{A}_{ij} \over a}=3\left({{S \over {a}^{2}}}\right)
\left[{\matrix{0&0&{e}^{-x}{\sin}^{2}\theta \cr 0&0&0\cr
{e}^{-x}{\sin}^{2}\theta &0&0\cr}}\right]. \eqn{} 
$$
Note that I have chosen to use gravitational units in which the proportionality
constant $\kappa=8\pi$.  Equation (8.4) gives the background extrinsic curvature
for a hole with linear momentum $P$ in the positive $z$ direction and which obeys
the extrinsic curvature isometry condition (4.45) with the plus sign.  Equation
(8.5) gives the background extrinsic curvature for the same physical
configuration, but the extrinsic curvature obeys the isometry condition with the
minus sign.  Equation (8.6) gives the background extrinsic curvature for a hole
with angular momentum $S$ in the positive $z$ direction and obeys the isometry
condition with the minus sign.  No single hole can have a non-vanishing angular
momentum at infinity and obey the isometry condition with the plus sign.

The square of the extrinsic curvature is needed in the source term  for the
Hamiltonian constraint.  The squares of (8.4), (8.5), and (8.6) in dimensionless
form are, respectively,
$$
{a}^{2}\bar{A}^{+}_{ij}\bar{A}^{+ij}=18{\left({{P
\over a}}\right)}^{2}{e}^{-6x}\left\{{{\sinh}^{2}x\left({1+2{\cos}^{2}\theta
}\right)+3{\cos}^{2}\theta }\right\}, \eqn{}
$$
$$
{a}^{2}\bar{A}^{-}_{ij}\bar{A}^{-ij}=18{\left({{P
\over a}}\right)}^{2}{e}^{-6x}\left\{{{\cosh}^{2}x\left({1+2{\cos}^{2}\theta
}\right)-3{\cos}^{2}\theta }\right\}, \eqn{}
$$
and
$$
{a}^{2}{\bar{A}}{}_{i}{}_{j}{\bar{A}}{}^{i}{}^{j}=18{\left({{S \over
{a}^{2}}}\right)}^{2}{e}^{-6x}{\sin}^{2}\theta . \eqn{}
$$
Since (8.5) and (8.6) obey the extrinsic curvature isometry condition with the
same sign, they can be considered together.  In this case, the square of the sum
is the sum of the squares since there are no cross terms and (8.8) and (8.9) can
be added together.  Examining (8.7), (8.8), and (8.9), we find that they are
also reflection symmetric through the $\theta=\pi/2$ plane.  This means that the
domain in which the Hamiltonian constraint must be solved in order to specify
initial data over the entire initial hypersurface is $0\le x < \infty$ and
$0\le\theta\le\pi/2$.

With the domain of the problem determined, boundary conditions must be imposed
on all domain boundaries.  Equation (7.5) defines the boundary condition on the
throat.  In terms of our coordinate system, this becomes
$$
{\partial \psi  \over \partial x}+{\psi  \over 2}=0\qquad\hbox{for}\qquad x=0.
\eqn{} 
$$

If the outer boundary is not taken at infinity, but at some finite radius
$x_{\scriptscriptstyle f}$, then the outer boundary condition can be approximated
by either (7.14) or (7.27).  While the accuracy of (7.27) is always equal to or
greater than the accuracy of (7.14), this boundary condition had not been
derived at the time when this problem was studied.  For this reason, I will
consider only (7.14) for this particular problem.  Equation (7.27) should be
used in any future work on the one-hole problem, and it has been used in all
work subsequent to this.  With this in mind, the approximate outer boundary
condition takes the following form in terms of our coordinate system:
$$
{\partial \psi  \over \partial x}+\psi -1=0\qquad\hbox{for}\qquad
x=x_{\scriptscriptstyle f}. \eqn{} $$

The two remaining boundaries are at $\theta=0$ and $\theta=\pi/2$.  To fix
boundary conditions here, recall first that the configuration is axisymmetric. 
In order for the solution to be axisymmetric, the derivative of the conformal
factor normal to the rotation axis must vanish.  Similarly, because of the
reflection symmetry, the derivative of the conformal factor normal to the
reflection plane must also vanish.  In terms of our coordinates, these two
boundary conditions can be written as
$$
{\partial \psi  \over \partial \theta }=0\qquad\hbox{at}\qquad\theta
=0\quad\hbox{or}\quad{\pi  \over 2}. \eqn{} $$

To difference the Hamiltonian constraint I first discretize the domain of the
solution.  The discretization will take the following form:
$$
{x}_{i}\equiv i*{h}_{x}\qquad\hbox{where}\qquad
i=0,\ldots,\Iind\quad\hbox{and}\quad{h}_{x}\equiv {{x}_{\scriptscriptstyle f}
\over\Iind} \eqn{} 
$$
and
$$
{\theta }_{j}\equiv j*{h}_{\theta }\qquad\hbox{where}\qquad
j=0,\ldots,\Jind\quad\hbox{and}\quad{h}_{\theta }\equiv {\pi /2 \over\Jind}. \eqn{}
$$
Thus, the continuous domain of the solution is approximated by a discrete
``mesh'' of points labeled by an index pair $(i,j)$.  Any function on the space
becomes a function on the mesh.  For example, the conformal factor will become
$$
\psi ({{x}_{i},{\theta }_{j}})\rightarrow {\psi }_{i,j}. \eqn{}
$$

To difference the Laplacian (8.3), I use a second order conservative
differencing scheme ({\it cf}. Adler and Piran [1984]).  I should point out that
there is no unique way to difference a differential operator.  In the limit that
the mesh spacings $h_x$ and $h_\theta$ go to zero, however, all correct
differencing schemes should converge to the same continuous differential
operator and the results of different finite difference calculations should
converge to the same continuous result.  The choice of a conservative
differencing scheme means that the finite difference equations will obey,
locally, a discrete version of Gauss' law.  As a general rule, any method for
making the finite difference equations mimic any behavior of the differential
equations should be used.

The finite difference form of the Hamiltonian constraint is expressed as follows:
$$
\eqalign{{\Ac}_{i,j}^{+}\left({{\psi }_{i+1,j}-{\psi
}_{i,j}}\right)&+{\Ac}_{i,j}^{-}\left({{\psi }_{i-1,j}-{\psi
}_{i,j}}\right)+{\Bc}_{i,j}^{+}\left({{\psi }_{i,j+1}-{\psi
}_{i,j}}\right)\cr &+{\Bc}_{i,j}^{-}\left({{\psi }_{i,j-1}-{\psi
}_{i,j}}\right)+{{e}^{2{x}_{i}} \over 8}{\psi
}_{i,j}^{-7}{\left[{{a}^{2}{\bar{A}}{}_{\ell }{}_{m}{\bar{A}}{}^{\ell
}{}^{m}}\right]}_{i,j}=0,\cr} \eqn{} 
$$
where
$$
{\Ac}_{i,j}^{\pm }\equiv {\exp\left\{{\pm {h}_{x}}\right\} \over
{h}_{x}^{2}}\qquad :\qquad\hbox{for}\quad\matrix{i=0,\ldots,\Iind\cr
j=0,\ldots,\Jind\cr} \eqn{} $$
and
$$
{\Bc}_{i,j}^{\pm }\equiv \cases{{\sin\left({{\theta }_{j}\pm {1 \over
2}{h}_{\theta }}\right) \over {h}_{\theta }^{2}\sin{\theta }_{j}}&: for
$\matrix{i=0,\ldots,\Iind\cr j=1,\ldots,\Jind\cr}$\cr \hfill{2 \over {h}_{\theta
}^{2}}&: for $\matrix{i=0,\ldots,\Iind\cr j=0\hfill\cr}$.\cr} \eqn{}
$$
Notice that a different form must be used for the $\Bc^\pm_{i,j}$ coefficients on
the $\theta=0$ axis.  This special case occurs because of the coordinate
singularity on the $z$-axis which is present in spherical coordinates.  In the
case of axisymmetry, this can be handled easily by L'H\^ opital's rule given the
boundary condition (8.12).  For $\theta=0$ $(\hbox{or\ } \pi)$, we find that
$$
\lim\limits_{\theta \rightarrow 0}^{}{1 \over \sin\theta }{\partial  \over
\partial \theta }\left({\sin\theta {\partial \psi  \over \partial \theta
}}\right)=2{{\partial }^{2}\psi  \over \partial {\theta }^{2}}, \eqn{}
$$
which fixes the special case of (8.18).  (Evans [1986a] espouses an alternate
approach for differencing near coordinate singularities based on numerical
regularization.)  Notice also, that the differenced form of the Hamiltonian
constraint has been multiplied by an overall factor of $r^2$ in order to simplify
the equations and to remove the overall $r^{-2}$ scaling in the difference
coefficients.

The difference equations, as given in (8.16), for points on the boundary of the
domain depend on values of the conformal factor outside the computational
domain.  These unknowns are fixed by the boundary conditions which have, as yet,
not been used.  I include the boundary conditions in the one-hole problem by the
{\it explicit} use of ``virtual'' boundary points.  That is, the computational
domain is expanded to include four sets of ``virtual'' mesh points, with the
values of these mesh points being determined by the boundary conditions.  On the
inner boundary, virtual points are added for $i=-1$ and $j=0,\ldots,\Jind$.  The
second order finite difference form of (8.10) fixes these virtual boundary
points by $$
{\psi }_{-1,j}={\psi }_{1,j}+{h}_{x}{\psi }_{0,j}. \eqn{}
$$
On the outer boundary, virtual points are added for $i=\Iind+1$ and
$j=0,\ldots,\Jind$.  The second order finite difference form of (8.11) fixes
these virtual boundary points by
$$
{\psi }_{\Iind+1,j}={\psi }_{\Iind-1,j}-2{h}_{x}\left({{\psi }_{\Iind,j}-1}\right).
\eqn{} $$
Finally, the virtual points outside $\theta=0$ and $\pi/2$ are fixed by the
second order finite difference form of (8.12).  These virtual points are located
at $j=-1$ or $\Jind+1$ and $i=0,\ldots,\Iind$, and are defined by
$$
{\psi }_{i,-1}={\psi }_{i,1} \eqn{}
$$
and
$$
{\psi }_{i,\Jind+1}={\psi }_{i,\Jind-1}. \eqn{}
$$

The computational mesh used can be visualized as in Figure~8.1.  The collection
of points which ``sample'' the computational domain (and the virtual boundary
points) are thought of as a two-dimensional object mimicking the two-dimensional
domain.  Each point is uniquely identified by an index pair.  It is often
convenient to think of the collection of points as a one-dimensional {\it
vector} instead of a two-dimensional mesh.  In this case, some ordering of the
points is assumed and each mesh point (and virtual boundary point) maps to a
unique element of the vector.  When viewed in this manner, objects such as the
conformal factor will carry a single index instead of two as in (8.15).

%\figlabel{3.625truein}{Figure~8.1:  Computational mesh and ``virtual'' boundary
%points.}
\figlabelpdf{3.625truein}{Figure~8.1:  Computational mesh and ``virtual'' boundary
points.}{Figures/Figure8_1.pdf}

If we think of the conformal factor as a vector, then the set of equations given
by (8.16), (8.20), (8.21), (8.22), and (8.23) can be thought of abstractly as a
{\it vector of equations} which acts on the vector of conformal factor values
$$
{\vecfunct}_{a}\left[{\psi }\right]=0. \eqn{}
$$
In general, $\vecfunct_a[\psi]$ will be a vector of nonlinear equations as is the
case for the Hamiltonian constraint (8.16).  If the equations were linear (i.e.
the extrinsic curvature is zero), then (8.24) can be represented by a square
matrix dotted into the vector $\psi_a$, all set equal to some constant vector
containing the terms which do not depend on the conformal factor.  The solution
of the finite difference equations for the value of the conformal factor at the
mesh points, then, reduces to the problem of inverting a large matrix.

In the case of the Hamiltonian constraint, the system is not linear and another
approach must be found.  There is no ideal approach for the solution of a set of
nonlinear algebraic equations.  One approach for solving them is the iterative
Newton-Raphson method for nonlinear systems ({\it cf}. Press {\it et al}.
[1988]).  Assume that $\psi_a$ is some initial guess at the solution to (8.24). 
Next, assume that the actual solution is given by $\psi_a+\delta\psi_a$ and
expand (8.24) in a Taylor series.  Truncating at linear order, we find
$$
{\vecfunct}_{a}\left[{\psi +\delta \psi }\right]={\vecfunct}_{a}\left[{\psi
}\right]+\sum\nolimits\limits_{b}^{} {\left.{{\partial {\vecfunct}_{a} \over \partial
{\psi }_{b}}}\right|}_{\psi }{\delta \psi }_{b}+\Order\left({{\delta \psi
}^{2}}\right)=0. \eqn{}
$$
If we define the linearized coefficient matrix $\coeffmatrix_{ab}$ as
$$
{\coeffmatrix}_{ab}\equiv {\left.{{\partial {\vecfunct}_{a} \over \partial {\psi
}_{b}}}\right|}_{\psi } \eqn{}
$$
and the ``residual'' vector $\residual_a$ as
$$
{\residual}_{a}\equiv -{\vecfunct}_{a}\left[{\psi }\right], \eqn{}
$$
then the solution, $\delta\psi_a$, of the linearized equation
$$
\sum\nolimits\limits_{b}^{} {\coeffmatrix}_{ab}{\delta \psi
}_{b}={\residual}_{a} \eqn{} 
$$
gives a first-order accurate correction to the initial guess $\psi_a$.  Updating
the initial guess $(\psi_a�\rightarrow \psi_a + \delta\psi_a)$ and iterating
until the correction becomes negligible everywhere will (in the absence of
pathological sets of equations) yield a solution to the nonlinear equations
(8.24) to any desired accuracy.

Direct solutions of the difference equations (i.e., those which directly invert
the coefficient or linearized coefficient matrix) are the preferred method of
solving the finite difference equations whenever it is possible.  It is often
the case, however, that the coefficient matrix is too large and requires too
much computer memory to be directly inverted.  If this is the case, other
approximation methods can be used such as simultaneous over-relaxation (SOR),
alternating direction implicit (ADI), conjugate gradient (CG), and many others
({\it cf}. Press {\it et al}. [1988] or Ames [1977]).  All such approximation
schemes iteratively converge to the solution which would be found by a
``direct'' solution.  The advantages of one over another deal with rates of
convergence, computational efficiency, and the requirement of specific
symmetries in the coefficient matrix.  Perhaps the most efficient method for
solving the difference equations is the multigrid (MG) approach.  The use of
this method for solving the Hamiltonian constraint has been discussed in
Choptuik [1982], Choptuik and Unruh [1986], and Cook [1989].  I have used the MG
approach for the solution of the finite difference equations in the case of a
single hole.  The details of the MG method are left to Appendix A and the
references therein.

Given a solution to the finite difference form of the Hamiltonian constraint, it
is important to extract any physically significant information about the initial
data.  Of fundamental interest is the total ADM energy of the initial slice
defined by (7.9).  In terms of the conformal factor and using gravitational
units, this becomes
$$
E=-{1 \over 2\pi }\oint_{\infty }^{}{\bar{D}}{}^{i}\psi
{d}^{2}{\bar{S}}{}_{i}. \eqn{}
$$
This is not, however, the best form to use for computing the total energy. 
First, we recall that the computational domain does not extend to infinity and
so this integral must be approximated by the integral over the outer boundary
at a finite radius.  Second, evaluating (8.29) from numerical data will require
that a numerical derivative be computed in order to evaluate the integrand. 
Such a numerical derivative would necessarily add error to the evaluation of
(8.29).  An alternate method for evaluating the total energy ({\it cf}. Bowen and
York [1980]) can be derived by applying Gauss' law to (8.29).  The result is
$$
\eqalign{E&={1 \over 2\pi }\oint_{r=a}^{}{\bar{D}}{}^{i}\psi
{d}^{2}{\bar{S}}{}_{i}-{1 \over 2\pi }\int_{r>a}^{}{\overline{\nabla
}}^{2}\psi {d}^{3}\bar{V}\cr &={1 \over 4\pi }\oint_{r=a}^{}{\psi  \over
r}{d}^{2}\bar{S}+{1 \over 16\pi }\int_{r>a}^{}{\psi
}^{-7}{\bar{A}}{}_{i}{}_{j}{\bar{A}}{}^{i}{}^{j}{d}^{3}\bar{V},\cr}
\eqn{}
$$
where the Hamiltonian constraint (4.8) and the inner boundary condition (7.2)
have been used.  (Note that the unit normal for the surface integral over the
inner boundary points out of the domain, i.e. toward the origin.)

This form for the total energy is much superior to (8.29).  The computational
approximation of a finite domain, however, means that the volume integral in
(8.30) will be cut off radially at the outer boundary.  Because the integrand
falls off as $\Order(r^{-2})$, the error in having a finite outer boundary
will, nonetheless, be small if the boundary is far enough out.  The major
portion of the energy results from the surface integral and the inner regions of
the volume integral.  In terms of the coordinate system being used and in the
case of axisymmetry, the total energy can be written in dimensionless form as
$$
{E \over a}\cong {1 \over 2}\int\limits_{\theta =0}^{\theta =\pi }{\psi
}_{(x=0)}\sin\theta d\theta +{1 \over
8}\int\limits_{\theta=0}^{\theta=\pi}{\int\limits_{x=0}^{x={x}_{f}}{\psi
}^{-7}\left({{{a}^{2}\bar{A}}{}_{i}{}_{j}{\bar{A}}{}^{i}{}^{j}}\right){e}^{3x}dx}d\theta
. \eqn{} $$

After computing the total energy, one might consider computing the higher order
moments in the multipole expansion.  Because of the reflection symmetry through
$\theta=\pi/2$, the dipole moment vanishes identically.  While higher moments
will not vanish identically, it is not clear as to the usefulness of these
quantities and they have not been calculated.  A quantity that {\it is} of
interest is the area of the throat or minimal surface {\it in the physical
space}.  This quantity is defined as 
$$
{A}_{\scriptscriptstyle MS}=\oint_{r=a}^{}{\psi }^{4}{d}^{2}\bar{S}. \eqn{}
$$
Associated with the area of the minimal surface is a mass defined similarly to
irreducible mass by
$$
M=\sqrt {{A \over 16\pi }}. \eqn{}
$$

In terms of the coordinate system being used and in the case of axisymmetry, the
area of the minimal surface can be written in dimensionless form as
$$
{{A}_{\scriptscriptstyle MS} \over {a}^{2}}=2\pi \int\limits_{\theta =0}^{\theta
=\pi }{\psi }_{(x=0)}^{4}\sin\theta d\theta . \eqn{}
$$

A code was constructed to solve the finite difference equations described above
using the multigrid algorithm described in Appendix A.  After obtaining a
solution, the total energy and minimal surface area and mass are integrated
numerically.  In order to test the accuracy of the code, use was made of an
exact solution to a fully non-linear ``model'' Hamiltonian constraint found by
Bowen and York [1980].  The model problem is obtained by considering a
non-physical, spherically symmetric extrinsic curvature whose square is given by
$$
{a}^{2}\bar{A}^{T}_{ij}\bar{A}^{Tij}=24{\left({{P
\over a}}\right)}^{2}{e}^{-6x}{\sinh}^{2}x. \eqn{}
$$
The solution to the model problem is
$$
{\psi }_{\scriptscriptstyle model}={e}^{-x/2}{\left[{6+2\cosh 2x+4\left({{E \over
a}}\right)\cosh x}\right]}^{1/4}, \eqn{}
$$
where
$$
{E \over a}=\sqrt {4+{\left({{P \over a}}\right)}^{2}}. \eqn{}
$$
From (8.34), the area of the minimal surface in the model problem is
$$
{{A}_{\scriptscriptstyle model} \over {a}^{2}}=16\pi \left({2+{E \over a}}\right) \eqn{}
$$
and the mass is
$$
{{M}_{\scriptscriptstyle model} \over a}=\sqrt {2+{E \over a}}. \eqn{}
$$

The code was run using (8.35) for a series of values of the momentum and using
the same computational domain as was used later for the physical problem. 
Specifically, the outer boundary was set at $x=10$.  The radial discretization
parameter was set at $\Iind=1024$ and the angular parameter at $\Jind=384$ on the
finest grid used in the multigrid solution.  Table~8.1 below summarizes the
results of the computer solution compared with the exact solution.  Full details
on the multigrid solutions to the model problem as well as the physical problem
can be found in Appendix B.  The computed solution has proven to be accurate to
roughly one part in 10,000 for all examined quantities.

Physical initial-data sets were generated with the same code using (8.7), (8.8),
and (8.9) for the square of the extrinsic curvature.  The same computational
domain and discretization were used for all cases as described above for the
model problem.  The results are summarized in Tables~8.2 and 8.3 below.  The full
details of the solutions can be found in Appendix B.  In the tables below, the
printed digits represent the significant digits of the printed values as
determined by a conservative error estimate.

\vskip 0.25truein
$$
\vbox{\offinterlineskip
\hrule
\halign{\quad\hfil#\quad&\vrule#&
\strut\quad\hfil#\quad&
\quad\hfil#\quad&\vrule#&
\strut\quad\hfil#\quad&
\quad\hfil#\quad\cr
\omit&height2pt&\omit&\omit&&\omit&\omit\cr
$P/a$ &&	$E_{exact}/a$&$E_{comp.}/a$&&	$M_{exact}/a$&	$M_{comp.}/a$\cr
\omit&height2pt&\omit&\omit&&\omit&\omit\cr
\noalign{\hrule}
\omit&height2pt&\omit&\omit&&\omit&\omit\cr
	0.0&&2.00000&1.99997&&2.00000&1.99995\cr
	1.0&&2.23607&2.23605&&2.05817&2.05812\cr
	2.5&&3.20156&3.20156&&2.28069&2.28064\cr
	5.0&&5.38517&5.38516&&2.71757&2.71750\cr
	7.5&&7.76209&7.76209&&3.12443&3.12436\cr
	10.0&&10.1980&10.1981&&3.49257&3.49249\cr
	12.5&&12.6590&12.6590&&3.82871&3.82862\cr
	15.0&&15.1328&15.1328&&4.13917&4.13908\cr
	17.5&&17.6139&17.6139&&4.42876&4.42866\cr
\omit&height2pt&\omit&\omit&&\omit&\omit\cr
\noalign{\hrule}
\noalign{\smallskip}
\multispan7 Table~8.1:  Numerical solution of the model problem compared\cr
\multispan7 \hfil with the exact solution.\hfil\cr}}
$$

$$
\vbox{\offinterlineskip
\hrule
\halign{\quad\hfil#\quad&\vrule#&
\strut\quad\qquad\hfil#\quad&
\quad\hfil#\quad\qquad&\vrule#&
\qquad\quad\hfil#\hfil\quad&
\qquad\hfil#\hfil\quad\quad\cr
\omit&height2pt&\omit&\omit&&\omit&\omit\cr
&&\multispan2\hfil$\bar{A}^+_{ij}$\hfil&&\multispan2\hfil$\bar{A}^-_{ij}$\hfil\cr
$P/a$ &&	$E/a$ &	$M_{\scriptscriptstyle MS}/a$ &&	$E/a$ &	$M_{\scriptscriptstyle MS}/a$\cr
\omit&height2pt&\omit&\omit&&\omit&\omit\cr
\noalign{\hrule}
\omit&height2pt&\omit&\omit&&\omit&\omit\cr
	0.0&&2.000&2.000&&2.000&2.000\cr
	1.0&&2.347&2.113&&2.330&2.100\cr
	2.5&&3.589&2.470&&3.545&2.430\cr
	5.0&&6.132&3.069&&6.078&3.001\cr
	7.5&&8.815&3.589&&8.760&3.502\cr
	10.0&&11.54&4.049&&11.49&3.946\cr
	12.5&&14.29&4.463&&14.24&4.348\cr
	15.0&&17.04&4.843&&17.00&4.716\cr
	17.5&&19.81&5.195&&19.76&5.058\cr
\omit&height2pt&\omit&\omit&&\omit&\omit\cr
\noalign{\hrule}
\noalign{\smallskip}
\multispan7 Table~8.2:  Energy and minimal surface mass for a single black
hole\cr \multispan7 \hfil with linear momentum.\hfil\cr}}
$$

$$
\vbox{\offinterlineskip
\hrule
\halign{\quad\hfil#\quad&\vrule#&
\strut\qquad\qquad\hfil#\hfil\qquad&
\qquad\hfil#\hfil\quad\quad\cr
\omit&height2pt&\omit&\omit\cr
	$S/a^2$ &&	$E/a$ &	$M_{\scriptscriptstyle MS}/a$\cr
\omit&height2pt&\omit&\omit\cr
\noalign{\hrule}
\omit&height2pt&\omit&\omit\cr
	0.0&&2.000&2.000\cr
	1.0&&2.048&2.033\cr
	3.0&&2.328&2.227\cr
	10.0&&3.477&3.034\cr
	30.0&&5.759&4.673\cr
	100.0&&10.41&8.071\cr
	300.0&&17.99&13.67\cr
	1000.0&&32.83&24.69\cr
	10000.0&&103.8&77.56\cr
\omit&height2pt&\omit&\omit\cr
\noalign{\hrule}
\noalign{\smallskip}
\multispan4 Table~8.3:  Energy and minimal surface mass for a\cr
\multispan4 \hfil single black hole with angular momentum.\hfil\cr}}
$$
\vfill
\eject
