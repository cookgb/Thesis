\chapterhead{Chapter~\the\chapnum:  Apparent Horizons for a Single Black Hole\cr
with Linear or Angular Momentum}
%%%%%%%%%%%%%%%%%%%%%%%%%%%%%%%%%%%%%
Given the initial data for a single black hole with non-zero linear or angular
momentum as described in Chapter~8, the apparent-horizon equation (11.11) can be
solved to locate any apparent horizons on the initial-data slice.  Because part
of the initial data is only known numerically, it will be necessary to solve
(11.11) numerically.  As described in Chapter~11, this will be accomplished by
posing the apparent-horizon equation as a boundary value problem.

As stated in Chapter~11, the location of apparent horizons is determined by
solving (11.11) for the scalar function $\tau$ of which the level surface
$\tau=\tau_0$ is the apparent horizon.  For this approach to work, the functional
form for $\tau$ must be chosen carefully.  The apparent horizons for the case of
a single hole are expected to be located near to the minimal surface.  Using the
logarithmically scaled dimensionless radial coordinates (8.1) from Chapter~8, an
appropriate choice for $\tau$ is
$$
\tau (x,\theta ,\phi )=x-h(\theta ,\phi ). \eqn{}
$$
Choosing the level surface $\tau=0$ to locate the apparent horizon then means
that the radius of the apparent horizon is given parametrically by
$$
x=h(\theta ,\phi ). \eqn{}
$$
From (11.10) we find that
$$
\lambda =a{e}^{h(\theta ,\phi )}{(1+{h}_{,\theta }^{2}+{h}_{,\phi
}^{2}/{\sin}^{2}\theta )}^{-1/2} \eqn{}
$$
and
$$
{\bar{s}}{}^{i}={1 \over a{e}^{h(\theta ,\phi )}\sqrt {1+{h}_{,\theta
}^{2}+{h}_{,\phi }^{2}/{\sin}^{2}\theta }}\left[{\matrix{1&{-h}_{,\theta
}&{-h}_{,\phi }/{\sin}^{2}\theta \cr}}\right]. \eqn{}
$$
Note that any occurrence of the coordinate $x$ must be evaluated at
$x=h(\theta,\phi)$.  From (12.4) it is seen that $\bar{s}^i$ is indeed an outward
pointing unit normal from the point of view of an observer on the top sheet $(x >
0)$.

Since the initial-data slices constructed in Chapter~8 for a single hole are
inversion symmetric, any solution of the apparent-horizon equation must have an
inversion-symmetric counterpart.  To examine it's form, it is necessary to
construct the isometry relations in our coordinate system.  From the coordinate
transformations and from (4.31) it is easily seen that the isometry map takes
the form
$$
(x=-x',\theta =\theta ',\varphi =\varphi '). \eqn{}
$$
We find then, that if (12.2) defines one apparent horizon, its inversion
symmetric counterpart is located at
$$
x=-h(\theta ,\phi ) \eqn{}
$$
and the outward pointing unit normal is
$$
{\tilde{s}}{}^{i}={1 \over a{e}^{h(\theta ,\phi )}\sqrt {1+{h}_{,\theta
}^{2}+{h}_{,\phi }^{2}/{\sin}^{2}\theta }}\left[{\matrix{-1&{-h}_{,\theta
}&{-h}_{,\phi }/{\sin}^{2}\theta \cr}}\right]. \eqn{}
$$
The outward-pointing normal in (12.7) is outward pointing from the point of view
of an observer on the bottom sheet $(x < 0)$.  Thus, the inversion-symmetric
counterpart to an apparent horizon for the top sheet is an apparent horizon for
the bottom sheet.

To proceed in locating the apparent horizons for the initial-data sets
constructed in Chapter~8, I will now restrict to the case of axisymmetry so the
apparent-horizon function $h(\theta,\phi)$ now becomes a function only of
$\theta$.  Expanding (11.11) in terms of (12.1) (and recalling $K=0$), we find
$$
\eqalign{{h}_{,\theta \theta }+\left[{\cot\theta +4{\psi }^{-1}{\psi }_{,\theta
}}\right]&{h}_{,\theta }(1 +{h}_{,\theta }^{2})+\left[{-2-4{\psi }^{-1}{\psi
}_{,x}}\right](1+{h}_{,\theta }^{2})\cr&-{e}^{-h}\sqrt {1+{h}_{,\theta
}^{2}}{\psi }^{-4}\left[{{{\bar{A}}{}_{x}{}_{x} \over
a}+{{\bar{A}}{}_{\theta }{}_{\theta } \over a}{h}_{,\theta
}^{2}-2{{\bar{A}}{}_{x}{}_{\theta } \over a}{h}_{,\theta }}\right]=0.\cr}
\eqn{}
$$
The domain in which (12.8) must be solved is $0\le\theta\le\pi$.  Axisymmetry
demands that $\partial h/\partial\theta = 0$ on the boundaries $\theta=0,\pi$. 
The demand that $\tau=0$ define the apparent horizon is imposed by evaluating all
functions of $x$ ($\psi$, $\bar{A}_{ij}$, and the exponential function) at $x =
h(\theta)$.  The functional forms for the components of the extrinsic curvature
are given in (8.4), (8.5), and (8.6).  The functional form for the conformal
factor $\psi$ is only known numerically.  This fact adds greatly to the
difficulty in finding solutions to (12.8).

A great deal about the solutions to (12.8) can be determined without actually
solving this equation.  For the case of a hole with angular momentum, examining
(8.6) shows that the three components of the extrinsic curvature which appear in
(12.8) vanish identically.  If we consider the apparent-horizon equation in the
case in which the extrinsic curvature vanishes, then from (11.7) we see that it
reduces to the equation for an extremal surface (vanishing of the divergence of
the normal vector field).  This means that in the case of a spinning hole, the
apparent-horizon equation also reduces to an extremal-surface equation.  But we
already know that the minimal surface will satisfy this equation and so we find
that the minimal surface and the apparent horizon coincide.  York and Piran
[1982] have previously noted this behavior for the apparent horizons for the
spinning hole solutions.

Solutions for a hole with linear momentum can be generated in one of two ways
depending on the sign of the isometry condition imposed on the extrinsic
curvature.  Considering the case of the isometry condition with the minus sign
(same as for a spinning hole), then from (8.5) we see that $\bar{A}_{xx}$ and
$\bar{A}_{\theta\theta}$ both vanish on the minimal surface.  If we consider the
case of $h(\theta) = constant$, then (12.8) reduces again to an extremal surface
equation and the surface $h=0$ is seen to satisfy the apparent-horizon equation. 
While there is no evidence to suggest that other solutions exist, the
non-linearity of (12.8) prevents them from being ruled out.  A numerical search
for other solutions must be conducted to determine if the minimal surface truly
is the apparent horizon.  York and Piran [1982] stated in error that 
$\bar{A}_{ij}\bar{s}^i\bar{s}^j\not= 0$ on the minimal surface in this case, and
thus that the minimal surface and apparent horizon cannot coincide.  They also
claim to have found numerical solutions to the apparent-horizon equation which
were not coincident with the minimal surface.  As I will show later, I have
found no such solutions and believe that the minimal surface and apparent
horizon do in fact coincide.

The final case to consider is that of a hole with linear momentum constructed
from an extrinsic curvature obeying the isometry condition with the plus sign. 
Examination of (8.4) shows that $\bar{A}_{xx} = 0$ only if $P/a = 0$ which is
simply the Schwarzschild solution.  This means that $h = 0$ cannot be a solution
to the apparent-horizon equation and thus the minimal surface and apparent
horizon cannot coincide.

Some understanding of the form which the apparent horizon will take in this case
can be obtained by examining the case of an infinitesimally boosted hole.  Bowen
and York [1980] examined the effect of Lorentz boosting a Schwarzschild black
hole to first order in the boost velocity.  After demanding that the boosted
slice be maximal, the metric is found to be unchanged to first order in the
boost velocity.  This metric is thus given in our coordinates by
$$
{g}{}_{i}{}_{j}={a}^{2}{\left({1+{e}^{-x}}\right)}^{4}{f}{}_{i}{}_{j}={a}^{2}{\psi
}^{4}{f}{}_{i}{}_{j}. \eqn{}
$$
Again to first order, the boosted form of the extrinsic curvature is given by
$$
{A}{}_{i}{}_{j}={\psi }^{-2}{{\bar{A}}^{\pm }}{}_{i}{}_{j} \eqn{}
$$
where the two forms of (12.10) correspond to two possible inversion symmetric
choices for the lapse function and $\bar{A}^\pm_{ij}$ is given by (5.5).  The
momentum used in (12.10) is defined by $P^i = MV^i = 2aV^i$ where $V^i$ is the
boost velocity of the hole.  A non-numerical, first order form for the apparent
horizon equation can thus be obtained by using the Schwarzschild conformal
factor from (12.9) and the extrinsic curvature $\bar{A}^+_{ij}$ given in (8.4). 
The result is
$$
\eqalign{{h}_{,\theta \theta }&+\cot\theta {h}_{,\theta }(1+{h}_{,\theta
}^{2})+\left[{-2+{2{e}^{-h/2} \over \cosh(h/2)}}\right](1+{h}_{,\theta }^{2})\cr
&-{3(P/a) \over 16{\cosh}^{2}(h/2)}\sqrt {1+{h}_{,\theta
}^{2}}\left[{\cosh h\cos\theta (2-{h}_{,\theta }^{2})+2\sinh h\sin\theta
{h}_{,\theta }}\right]=0.\cr} \eqn{}
$$
If we assume that the horizon does not deviate far from the minimal surface
during an infinitesimal boost, then we can expand (12.11) to first order in the
apparent horizon function $h$.  The result is
$$
{1 \over \sin\theta }{\partial  \over \partial \theta }\left({\sin\theta
{\partial h(\theta ) \over \partial \theta }}\right)-h(\theta )={3 \over 8}{P
\over a}\cos\theta . \eqn{}
$$
The solution to (12.12) compatible with the boundary conditions is
$$
h(\theta )=-{1 \over 8}{P \over a}\cos\theta . \eqn{}
$$
(Note that the corresponding equation in Cook and York [1990] is incorrect by a
factor of three.)  To first order in the standard radial coordinate, the
apparent horizon is located at
$$
r=a(1-{1 \over 8}{P \over a}\cos\theta ) \eqn{}
$$
and we see that the apparent horizon takes the form of a {\it translation}
$(P_\ell = \cos\theta;\ell=1)$ of the minimal surface.

With some understanding of the expected behavior of the solutions to (12.8), I
turn now to the problem of solving it numerically.  To simplify and clearify the
expression, let me define the following quantities:
$$
\Ac(h,\theta ,\psi )\equiv \cot\theta +{4 \over \psi (h,\theta
)}{\left.{{\partial \psi (x,\theta ) \over \partial \theta
}}\right|}_{x=h(\theta )}, \eqn{}
$$
$$
\Bc(h,\theta ,\psi )\equiv -2-{4 \over \psi (h,\theta )}{\left.{{\partial \psi
(x,\theta ) \over \partial x}}\right|}_{x=h(\theta )}, \eqn{}
$$
and
$$
\Cc(h,\theta ,\psi )\equiv {{e}^{-h(\theta )} \over {\psi }_{(h,\theta
)}^{4}}\sqrt {1+{\left({{\partial h \over \partial \theta }}\right)}^{2}}.
\eqn{}
$$
The apparent-horizon equation now takes the form
$$
{h}_{,\theta \theta }+(\Ac{h}_{,\theta }+\Bc)(1+{h}_{,\theta
}^{2})-\Cc\left[{{{\bar{A}}{}_{x}{}_{x} \over a}+{{\bar{A}}{}_{\theta
}{}_{\theta } \over a}{h}_{,\theta }^{2}-2{{\bar{A}}{}_{x}{}_{\theta }
\over a}{h}_{,\theta }}\right]=0\quad\hbox{for}\quad 0<\theta <\pi . \eqn{}
$$
On the boundaries at the $z$-axis, the limiting form of (12.18) can be obtained
from L'H\^{o}pital's rule and the boundary condition that $h_{,\theta} = 0$.  The
result is
$$
2{h}_{,\theta \theta }+\Bc-\Cc{{\bar{A}}{}_{x}{}_{x} \over
a}=0\quad\hbox{for}\quad\theta =0,\pi . \eqn{}
$$
To difference (12.18) and (12.19), I first discretize the domain of the
solution.  The discretization will take the following form:
$$
{\theta }_{j}\equiv j*{\delta }_{\theta
}\qquad\hbox{where}\qquad j=0,\ldots,\Jind\quad\hbox{and}\quad{\delta }_{\theta
}\equiv {\pi  \over \Jind}. \eqn{}
$$
The derivatives in (12.18) and (12.19) are differenced using second order
central differencing.  Explicitly, this gives
$$
{\left.{{\partial h \over \partial \theta }}\right|}_{\theta ={\theta
}_{j}}={{h}_{j+1}-{h}_{j-1} \over 2{\delta }_{\theta }}\equiv {\Dc}_{j} \eqn{}
$$
and
$$
{\left.{{{\partial }^{2}h \over {\partial \theta }^{2}}}\right|}_{\theta
={\theta }_{j}}={{h}_{j+1}-2{h}_{j}+{h}_{j-1} \over {\delta }_{\theta }^{2}}.
\eqn{}
$$
The set of nonlinear difference equations derived from (12.18) and (12.19) can
be written as $\vecfunct_j[h] = 0$ where
$$
\eqalign{{\vecfunct}_{j}&[h]\equiv {1 \over {\delta }_{\theta
}^{2}}({h}_{j+1}-2{h}_{j}+{h}_{j-1})+({\Ac}_{j}{\Dc}_{j}+{\Bc}_{j})(1+{\Dc}_{j}^{2})\cr
&-{\Cc}_{j}\left[{{{\bar{A}}{}_{x}{}_{x}({h}_{j},{\theta }_{j}) \over
a}+{{\bar{A}}{}_{\theta }{}_{\theta }({h}_{j},{\theta }_{j}) \over
a}{\Dc}_{j}^{2}-2{{\bar{A}}{}_{x}{}_{\theta }({h}_{j},{\theta }_{j}) \over
a}{\Dc}_{j}}\right]\cr}\quad\hbox{for}\quad j=1,\ldots,\Jind-1, \eqn{}
$$
$$
{\vecfunct}_{0}[h]\equiv {4 \over {\delta }_{\theta
}^{2}}({h}_{1}-{h}_{0})+{\Bc}_{0}-{\Cc}_{0}{{\bar{A}}{}_{x}{}_{x}({h}_{0},{\theta
}_{0}) \over a}, \eqn{}
$$
and
$$
{\vecfunct}_{\Jind}[h]\equiv {4 \over {\delta }_{\theta
}^{2}}({h}_{\Jind-1}-{h}_{\Jind})+{\Bc}_{\Jind}-
{\Cc}_{\Jind}{{\bar{A}}{}_{x}{}_{x}({h}_{\Jind},{\theta }_{\Jind}) \over a}.
\eqn{}
$$
Because of their nonlinearity, these difference equations cannot be solved
directly.  As in the case of the differenced Hamiltonian constraint, the
nonlinear difference equations can be solved iteratively by the Newton-Raphson
method for nonlinear systems described in (8.24)--(8.28).  The linearized
coefficient matrix $\coeffmatrix_{ab}$ is far more complicated for the
apparent-horizon difference equations than it is for the case of the Hamiltonian
constraint.  For clarity, I give them explicitly.  Let me first define the
following quantities:
$$
{\Ac^\prime}_{j}\equiv {\partial {\Ac} \over \partial{h}_{j}}={4 \over {\psi
}_{({\tilde{h}}_{j},{\theta }_{j})}^{2}}{\left[{-\left({{\partial \psi (x,\theta
) \over \partial x}}\right)\left({{\partial \psi (x,\theta ) \over \partial
\theta }}\right)+\psi (x,\theta )\left({{{\partial }^{2}\psi (x,\theta ) \over
\partial x\partial \theta }}\right)}\right]}_{\matrix{x={\tilde{h}}_{j}\cr
\theta ={\theta }_{j}\cr}}, \eqn{}
$$
$$
{\Bc^\prime}_{j}\equiv {\partial {\Bc} \over \partial{h}_{j}}={4 \over {\psi
}_{({\tilde{h}}_{j},{\theta }_{j})}^{2}}{\left[{{\left({{\partial \psi (x,\theta
) \over \partial x}}\right)}^{2}-\psi (x,\theta )\left({{{\partial }^{2}\psi
(x,\theta ) \over {\partial
x}^{2}}}\right)}\right]}_{\matrix{x={\tilde{h}}_{j}\cr \theta ={\theta
}_{j}\cr}}, \eqn{}
$$
$$
{\Cc^\prime}_{j}\equiv {\partial {\Cc} \over \partial{h}_{j}}={\Cc}_{j}
(1+{\Bc}_{j}), \eqn{}
$$
and
$$
{\bar{A}^\prime}{}_{\ell }{}_{m}\equiv {\left.{{\partial {\bar{A}}{}_{\ell
}{}_{m} \over \partial x}}\right|}_{\matrix{x={\tilde{h}}_{j}\cr \theta ={\theta
}_{j}\cr}} \eqn{}
$$
where $\tilde{h}$ is the current estimate for the apparent horizon function.  The
non-zero elements of the linearized coefficient matrix are
$$
\eqalign{{\coeffmatrix}_{j,j}[\tilde{h}]=-{2 \over {\delta }_{\theta
}^{2}}+({\Ac^\prime}_{j}{\Dc}_{j}+{\Bc^\prime}_{j})(1+{\Dc}_{j}^{2}) -
{\Cc}_{j}\left[{{{\bar{A}^\prime}{}_{x}{}_{x} \over a}+{{\bar{A}^\prime}{}_{\theta
}{}_{\theta } \over a}{\Dc}_{j}^{2}-2{{\bar{A}^\prime}{}_{x}{}_{\theta } \over
a}{\Dc}_{j}}\right]\hfill&\cr
+{\Cc^\prime}_{j}\left[{{{\bar{A}}{}_{x}{}_{x}({\tilde{h}}_{j},{\theta }_{j})
\over a}+{{\bar{A}}{}_{\theta }{}_{\theta }({\tilde{h}}_{j},{\theta }_{j}) \over
a}{\Dc}_{j}^{2}-2{{\bar{A}}{}_{x}{}_{\theta }({\tilde{h}}_{j},{\theta }_{j})
\over a}{\Dc}_{j}}\right]\hfill&\cr \quad\hbox{for}\quad j=1,\ldots,\Jind-1,&\cr}
\eqn{}
$$
$$
{\coeffmatrix}_{j,j}[\tilde{h}]=-{4 \over {\delta }_{\theta
}^{2}}+{\Bc^\prime}_{j}-{\Cc}_{j}\left[{{{\bar{A}^\prime}{}_{x}{}_{x} \over
a}+(1+{\Bc}_{j}){{\bar{A}}{}_{x}{}_{x}({\tilde{h}}_{j},{\theta }_{j}) \over
a}}\right]\quad\hbox{for}\quad j=0,\Jind, \eqn{}
$$
$$
\eqalign{{\coeffmatrix}_{j,j\pm 1}[\tilde{h}]={1 \over {\delta }_{\theta
}^{2}}\pm \left\{{{\Ac}_{j}{1+{\Dc}_{j}^{2} \over 2{\delta }_{\theta
}}+({\Ac}_{j}{\Dc}_{j}+{\Bc}_{j}){{\Dc}_{j} \over {\delta }_{\theta
}}-{{\Cc}_{j} \over (1+{\Dc}_{j}^{2}){\delta }_{\theta
}}\left[{{{\bar{A}}{}_{x}{}_{x}({\tilde{h}}_{j},{\theta }_{j}) \over
a}{{\Dc}_{j} \over 2}}\right.}\right.\hfill&\cr
\left.{\left.{+{{\bar{A}}{}_{\theta }{}_{\theta }({\tilde{h}}_{j},{\theta }_{j})
\over a}{(3{\Dc}_{j}^{2}+2){\Dc}_{j} \over 2}-{{\bar{A}}{}_{x}{}_{\theta
}({\tilde{h}}_{j},{\theta }_{j}) \over
a}(2{\Dc}_{j}^{2}+1)}\right]}\right\}\hfill&\cr \quad\hbox{for}\quad
j=1,\ldots,\Jind-1,&\cr} \eqn{}
$$
and
$$
{\coeffmatrix}_{0,1}[\tilde{h}]={4 \over {\delta }_{\theta
}^{2}}\qquad\hbox{and}\qquad{\coeffmatrix}_{\Jind,\Jind-1}[\tilde{h}]={4 \over
{\delta }_{\theta }^{2}}. \eqn{}
$$
Inspection of the linearized coefficient matrix shows it to be tridiagonal and
so it can be easily inverted directly by a tridiagonal matrix solver.

There is one difficult point in evaluating the intermediate functions (12.15),
(12.16), and (12.17) and their derivatives (12.26), (12.27), and (12.28).  These
functions depend upon the conformal factor $\psi$ and its derivatives which are
known only numerically on a discrete mesh.  From the form of the discretization
of the conformal factor $\psi$ (8.13) and (8.14), and from the form of the
discretization of the apparent-horizon function $h$ (12.20), it is apparent that
the discretization of $h$ can be chosen to match that of the conformal factor. 
This means that the conformal factor and its derivatives need only be evaluated
along lines of constant $\theta$ on which values of the conformal factor are
known numerically.  Unfortunately, the radial position at which the conformal
factor must be evaluated is not discretized and will, in general, not occur at a
location at which the conformal factor is known.  The evaluation of the
conformal factor will thus require interpolation in the radial direction.

The main criterion for choosing an interpolation algorithm is the requirement
that the conformal factor and its derivatives be continuous functions.  From
(12.15), (12.16), (12.26), and (12.27) we find that the highest order radial
derivative needed is a second derivative.  Interpolation by cubic splines
({\it cf}. Press {\it et al}. [1988]) has sufficient smoothness so that its
second derivatives are continuous (though not smooth) and I have chosen to use
this approach for the radial interpolation.  The radial derivatives can be
evaluated by analytically differentiating the cubic spline interpolation
formula.  Angular derivatives of the conformal factor can be evaluated using
second order centered difference approximations.

The possibility of the introduction of high frequency noise must always be
considered when numerical data is differentiated.  The use of cubic splines and
the analytic differentiation of these, rather than the numerical data itself, 
helps to eliminate possible high frequency noise in radial derivatives.  Another
step which can be taken to eliminate such effects is to factor out the gross
variations in the numerical data into an analytic prefactor (Piran [1988]).  For
the case of the one-hole initial-data sets, such a factorization can take the
form
$$
\psi (x,\theta )=f(x)\Nc(x,\theta )\qquad\hbox{where}\qquad f(x)\equiv 1+{1 \over
2}{E \over a}{e}^{-x}. \eqn{}
$$
The prefactor $f(x)$ is simply the conformal factor for a time-symmetric solution
with scaled total energy $E/a$.  By factoring this out of the numerical solution,
the resulting numerical data $\Nc(x,\theta)$ will be a more slowly varying
function of $x$ and the interpolation of this data will be less prone to high
frequency noise.  No extra effort was needed to reduce noise for the numerical
derivatives in the $\theta$ direction since the conformal factor is already a
slowly varying function in this direction.

One final point needs to be explained regarding the evaluation of the conformal
factor and its derivatives.  The Hamiltonian constraint was solved over the
finite range of $0\le x\le�x_f$ and $0\le\theta\le\pi/2$.  The range over which
the apparent-horizon equation must be solved is $0\le\theta\le\pi$.  The
extension of the numerical data to cover this region is easily achieved since
$$
\psi (x,\theta )=\psi (x,\pi -\theta )\qquad\hbox{for}\qquad\pi /2\le \theta \le
\pi . \eqn{} $$
It will also be necessary to evaluate the conformal factor for $x < 0$.  Values
in this range can be obtained from the isometry condition (4.53).  Written in
terms of the logarithmically scaled radial coordinates we find
$$
\psi (x,\theta )={e}^{\left|{x}\right|}\psi (\left|{x}\right|,\theta )\qquad\hbox{for}\qquad
x<0. \eqn{}
$$
Finally, if the conformal factor is needed in the region outside the outer
boundary of the domain $x > x_f$, this can be approximated by (7.13) (same as
$f(x)$ from (12.34)) which is the basis for the outer boundary condition imposed
at $x_f$.

A computer code based on the method described above was constructed to solve for
the apparent-horizon function $h(\theta)$ and was used to locate the apparent
horizons on all of the initial-data sets described in Chapter~8.  In addition to
computing the location of the apparent horizon, the code also computed the area
and mass (via (8.33)) of the apparent horizon.  The integral for the area of the
apparent horizon is obtained by starting with the physical metric for the
initial-data slice
$$
d{s}^{2}={a}^{2}{e}^{2x}{\psi }^{4}({dx}^{2}+{d\theta }^{2}+{\sin}^{2}\theta
{d\phi }^{2}) \eqn{}
$$
and using the parametrically defined location of the apparent horizon (12.2) to
reduce (12.37) to the induced metric on the apparent horizon
$$
d{s}^{2}={a}^{2}{e}^{2h(\theta )}{\psi }_{(h(\theta ),\theta
)}^{4}((1+{h}_{,\theta }^{2}){d\theta }^{2}+{\sin}^{2}\theta {d\phi }^{2}).
\eqn{}
$$
From (12.38), the integral for the physical area of the apparent horizon is
given in dimensionless form by
$$
{{A}_{\scriptscriptstyle AH} \over {a}^{2}}=2\pi \int_{0}^{\pi }{\psi
}_{(h(\theta ),\theta )}^{4}{e}^{2h(\theta )}\sqrt {1+{h}_{,\theta
}^{2}}\sin\theta d\theta . \eqn{}
$$

For the case of a hole with linear or angular momentum generated from an
extrinsic curvature obeying the isometry condition with the minus sign, no
solutions except the one coincident with the minimal surface were found.  Other
solutions were sought by varying the initial guess for the location of the
apparent horizon.  Convergence to the minimal surface solution was always
direct, with no indication of other possible solutions.

For the case of a hole with linear momentum generated from an extrinsic
curvature obeying the isometry condition with the plus sign, a single solution,
not coincident with the minimal surface, was found for each initial-data set. 
The apparent-horizon function for the case of $P/a = 10$ is shown in
Figure~12.1.  The general sinusoidal shape is inherent for all values of the
momentum.  In fact, to within the truncation error of the problem, the solution
is given by $$
h(\theta )=\Hc(P/a)\cos\theta \eqn{}
$$
where the amplitude $\Hc$ is a function of the scaled momentum only.  Table~12.1
lists the masses for the minimal surface $M_{\scriptscriptstyle MS}$ and the
apparent horizon $M_{\scriptscriptstyle AH}$ along with the value of the
amplitude $\Hc$ and its standard deviation $\sigma$ resulting from a
least-squares fit of the numerical data for $h(\theta)$ to the functional form
(12.40).

%%\figlabel{3.375truein}{Figure~12.1:  Apparent-horizon function $h(\theta)$ for a
%%hole with\cr linear momentum $P/a=10$ generated by $\bar{A}^+_{ij}$.}
\figlabelpdf{3.375truein}{Figure~12.1:  Apparent-horizon function $h(\theta)$ for a
hole with\cr linear momentum $P/a=10$ generated by $\bar{A}^+_{ij}$.}{AHplots/AH.pdf}

\vskip 0.25truein
$$
\vbox{\offinterlineskip
\hrule
\halign{\quad\hfil#\hfil\quad&\vrule#&
\strut\quad\hfil#\hfil\quad&
\quad\hfil#\hfil\quad&\vrule#&
\strut\quad\hfil#\hfil\quad&
\quad\hfil#\hfil\quad\cr
\omit&height2pt&\omit&\omit&&\omit&\omit\cr
$P/a$ &&	$M_{\scriptscriptstyle MS}/a$ & $M_{\scriptscriptstyle AH}/a$ &&	$\Hc$ &	$\sigma$\cr
\omit&height2pt&\omit&\omit&&\omit&\omit\cr
\noalign{\hrule}
\omit&height2pt&\omit&\omit&&\omit&\omit\cr
	1.0	&& 2.113	& 2.119	&& $-0.112291$	& 0.000012\cr
	2.5	&& 2.470	& 2.496	&& $-0.206398$	& 0.000087\cr
	5.0	&& 3.069	& 3.121	&& $-0.269071$	& 0.00021\cr
	7.5	&& 3.589	& 3.662	&& $-0.295948$	& 0.00029\cr
	10.0	&& 4.049	& 4.138	&& $-0.310731$	& 0.00034\cr
	12.5	&& 4.463	& 4.567	&& $-0.320063$	& 0.00039\cr
	15.0	&& 4.843	& 4.959	&& $-0.326486$	& 0.00041\cr
17.5	&& 5.195	& 5.324	&& $-0.331175$	& 0.00043\cr
\omit&height2pt&\omit&\omit&&\omit&\omit\cr
\noalign{\hrule}
\noalign{\smallskip}
\multispan7 \hfil Table~12.1:  Minimal-surface mass and apparent-horizon
mass\hfil\cr \multispan7 \hfil for a hole with linear momentum $P$ generatedfrom
$\bar{A}^+_{ij}$.\hfil\cr
\multispan7 \hfil Also tabulated are the  apparent-horizon amplitude $\Hc$\hfil\cr
\multispan7 \hfil and its standard deviation $\sigma$.\hfil\cr}}
$$

It is seen immediately that the area of the apparent horizon is always greater
than that of the minimal surface as it must be.  We also see that the slope of
the apparent-horizon amplitude $\Hc$ is very close to $-1/8$ at $P/a  = 1$ and,
so, matches well to the solution for an infinitesimally boosted hole. 
Further interpretation of these results will be held for the next chapter.

\vfill
\eject
