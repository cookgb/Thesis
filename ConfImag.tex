\chapterhead{Chapter~\the\chapnum:  The Conformal Imaging Approach}
%%%%%%%%%%%%%%%%%%%%%%%%
The {\it conformal imaging approach} encompasses the $(3+1)$ and conformal
decomposition techniques described in the previous chapters, along with a set of
assumptions for choosing the freely specifiable data and the topology of the
initial slice, taken in order to calculate initial-data sets which represent one
or more black holes in an astrophysical setting.  The $(3+1)$ decomposition of
Einstein's equations and York's conformal decomposition of the constraints
provide a foundation for specifying initial-data sets for a wide range of
problems of astrophysical interest.  If we restrict ourselves to considering only
four-dimensional, pseudo-Riemannian space-times foliated by spacelike
hypersurfaces, then the Hamiltonian and momentum constraints can be expressed
respectively as:
$$
8{\overline{\nabla }}^{2}\psi -\psi \bar{R}-{2 \over 3}{\psi
}^{5}{K}^{2}+{\psi
}^{-7}{\bar{A}}{}_{i}{}_{j}{\bar{A}}{}^{i}{}^{j}=-2\kappa {\psi
}^{-3}\bar{\rho }, \eqn{}
$$
and
$$
{\bar{D}}{}_{j}{\bar{A}}{}^{i}{}^{j}-{2 \over 3}{\psi
}^{6}{\bar{\gamma }}{}^{i}{}^{j}{\bar{D}}{}_{j}K=\kappa
{\bar{\jmath}}{}^{i} \eqn{a}
$$
or
$$
{\left({{\Delta }_{\scriptscriptstyle L}W}\right)}{}^{i}-{2 \over 3}{\psi
}^{6}{\bar{\gamma }}{}^{i}{}^{j}{\bar{D}}{}_{j}K=\kappa
{\bar{\jmath}}{}^{i}. \eqnb{b}
$$
Note that (4.2b) is the most restricted form of the momentum constraint,
however, we will also have use for the form given in (4.2a).

In order to solve the constraints for $\psi$ and $W^i$, we must {\it first}
specify the unconstrained portions of the initial data.  We must choose first the
conformal {\it background} metric $\bar\gamma_{ij}$.  Given this background
metric, we must choose the transverse-traceless part of the {\it background} extrinsic
curvature so as to satisfy (3.24).  Next, we must choose the trace of the
extrinsic curvature in the physical space.  Finally, we must choose values for the
{\it background} sources $\bar\rho$ and $\bar\jmath^i$.

While these choices determine all of the freely
specifiable fields present in an initial-data set, there is one other choice
which must be made.  This final choice is the prescription of the topology of
the initial-data slice.  Einstein's equations in no way determine the topology
of the initial manifold.  It can be chosen, for example, to be closed, if one is
interested in cosmological models.  Alternately, it can be chosen to be
unbounded and, perhaps, asymptotically flat.  In any case, the manifold can
be simply connected or multiply connected.  These are just some of the possible
choices and the physical consequences of these choices can be fully determined
only after initial data has been constructed.

In order to represent astrophysical situations, the first assumption is that the
initial slice is asymptotically flat.  Since we want to investigate black holes,
we choose that there be no matter sources, so
$$
\bar{\rho }=0 \eqn{}
$$
and
$$
{\bar{\jmath}}{}^{i}=0. \eqn{}
$$
We also demand that the trace of the extrinsic curvature vanish on the initial
slice, so
$$
K=0. \eqn{}
$$
This condition implies that the hypersurface is maximally embedded in the full
space-time ({\it cf}. York and Piran [1982]) and decouples the momentum constraint
from the Hamiltonian constraint.  The final assumptions on the data are that the
hypersurface be conformally flat and that the transverse-traceless part of the
background extrinsic curvature vanish.  The first assumption means we choose the
conformal background metric to be a flat metric
$$
{\bar{\gamma }}{}_{i}{}_{j}={f}{}_{i}{}_{j}. \eqn{}
$$
The second assumption is
$$
{\bar{A}}_{\scriptscriptstyle TT}^{ij}=0, \eqn{}
$$
and it should be noted here that assumption (4.7) will be relaxed when multiple
black holes are considered.  These last two choices affect the gravitational
wave content of the initial-data slice.  We recognize this from the fact that
the conformal three-geometry and the transverse-traceless part of the extrinsic
curvature represent the dynamical degrees of freedom of the gravitational
field.  In the case of a single Schwarzschild black hole, initial data can be
chosen on an asymptotically and conformally flat, maximally embedded initial
slice.  In this case, the full extrinsic curvature vanishes and the initial
slice contains no gravitational waves.  However, in general, if the
configuration is not static, there will be gravitational radiation present in
initial slices constructed with (4.6) and (4.7).  More will be said about this
later.

Given these assumptions, the constraint equations now take the following
form:
$$
{\overline{\nabla }}^{2}\psi =-{1 \over 8}{\psi
}^{-7}{\bar{A}}{}_{i}{}_{j}{\bar{A}}{}^{i}{}^{j}, \eqn{}
$$
and
$$
{\bar{D}}{}_{j}{\bar{A}}{}^{i}{}^{j}=0, \eqn{a}
$$
or
$$
{\left({{\Delta }_{\scriptscriptstyle L}W}\right)}{}^{i}=0. \eqnb{b}
$$
Note that the full trace-free background extrinsic curvature ${\bar A}_{ij}$ is
used in (4.8) and (4.9a) instead of just the longitudinal part.  This is a
reflection of the earlier comment on relaxing condition (4.7).

The only assumptions left to
be made are those which fix the topology of the initial slice.  Assume first
that the topology is taken to be a simply connected, Euclidean $E^3$ topology. 
Asymptotic flatness implies (York [1979]) that
$$
\psi =1+\Order({{ r}^{-1}}) \eqn{}
$$
and
$$
{\bar{K}}{}^{i}{}_{j}=\Order({{r}^{-2}}). \eqn{}
$$
Since there are no sources to support a non-trivial solution, the only regular
solution to Einstein's equations is empty, flat space, so there is no
gravitational field.  In order to represent black holes (and therefore strong
gravitational fields) in vacuum, non-trivial topologies must be used.  The
simplest example of this comes from the Schwarzschild solution.  If we choose a
$t = constant$ slice of the Schwarzschild solution in isotropic coordinates, the
metric and extrinsic curvature on the slice are
$$
{\gamma }{}_{i}{}_{j}={\left({1+{\kappa M \over
16\pi r}}\right)}^{4}{f}{}_{i}{}_{j} \eqn{} $$
and
$$
{K}{}_{i}{}_{j}=0. \eqn{}
$$
This slice is asymptotically and conformally flat and maximally embedded in the
full space-time.  The conformal factor is
$$
\psi =1+{\kappa M \over 16\pi r}, \eqn{}
$$
and it is singular at $r = 0$.  In order for the manifold to be regular, the
origin is cut out and the manifold becomes ``non-contractable''.

It appears at
first that this geometry simply represents a single asymptotically flat
hypersurface with the origin removed.  This situation would not represent a
complete manifold.  Consider, however, the line element of (4.12) using
spherical coordinates,
$$
{ds}^{2}={\left({1+{\kappa M \over
16\pi r}}\right)}^{4}\left({{dr}^{2}+{r}^{2}{d\Omega }^{2}}\right), \eqn{}
$$
and then change to a new radial coordinate
$$
r^\prime={\left({\kappa M \over 16\pi}\right)}^{2}{1 \over r}. \eqn{}
$$
(Note that $r = r^\prime = \kappa M/16\pi$ is a fixed point set in this
transformation and coincides with the location of the Schwarzschild event
horizon.)  In terms of this new coordinate the line element becomes
$$
{ds}^{2}={\left({1+{\kappa M \over 16\pi
r^\prime}}\right)}^{4}\left({{dr^\prime}^{2}+{r^\prime}^{2}{d\Omega
}^{2}}\right), \eqn{} 
$$
which is identical to (4.15) except that it is in terms of the new radial
coordinate $r^\prime$.  We see now that the initial slice is asymptotically flat
as $r^\prime\rightarrow \infty$ in just the same way that it is asymptotically
flat as $r \rightarrow \infty$.  $r^\prime\rightarrow \infty$ is, of course, the
same as $r \rightarrow 0$ and it can now be seen that the geometry of (4.12) is
actually that of {\it two} asymptotically flat spaces joined at the spherical
surface $r = r^\prime = \kappa M/16\pi$ (see Figure~4.1), and this implies that
the manifold is complete.

%\figlabel{3.5truein}{Figure~4.1:  Embedding diagram of time-symmetric, maximal
%slice \cr of Schwarzschild geometry in isotropic coordinates.}
\figlabelpdf{3.5truein}{Figure~4.1:  Embedding diagram of time-symmetric, maximal
slice \cr of Schwarzschild geometry in isotropic coordinates.}{Figures/Figure4_1.pdf}

These two asymptotically flat spaces are often referred to as the ``top'' and
``bottom'' sheets of the manifold and the region connecting them is known as an
Einstein-Rosen bridge (Einstein and Rosen [1935]).  The region interior to the
event horizon on the top sheet, $r < \kappa M/16\pi$, is not the interior of the
black hole, but is an identical ``second'' or ``alternate'' universe.  This can
be seen most easily by plotting the initial slice in the familiar Kruskal
coordinates (Figure~4.2).

%\figlabel{3truein}{Figure~4.2:  Time-symmetric, maximal slice of Kruskal
%space-time, \cr labeled in isotropic coordinates and showing ``top'' and
%\cr``bottom'' sheets in the two different universes.}
\figlabelpdf{3truein}{Figure~4.2:  Time-symmetric, maximal slice of Kruskal
space-time, \cr labeled in isotropic coordinates and showing ``top'' and
\cr``bottom'' sheets in the two different universes.}{Figures/Figure4_2.pdf}

This example illustrates the main principle followed in constructing vacuum,
black-hole initial-data sets:  we use multiple asymptotically flat sheets
connected by Einstein-Rosen bridges.  In doing this, two criteria must be
satisfied in the resulting initial-data sets.  The first is regularity and the
second is completeness of the manifold ({\it cf}. Brill and Lindquist [1963]).

If we want to represent more than one black hole on the initial slice, then
there are essentially two avenues which can be taken in choosing the topology of
the initial slice.  Let us assume that there are to be $N$ black holes in our
``universe''.  One approach for modeling this is to allow each of the holes to
connect to its own isolated, asymptotically flat universe through an
Einstein-Rosen bridge.  This produces an $N + 1$ sheeted manifold (Figure~4.3a). 
The second approach is to allow all $N$ holes to connect to the {\it same}
alternate, asymptotically flat universe through $N$ Einstein-Rosen bridges.  This
produces a multiply-connected, two-sheeted manifold (Figure~4.3b).

%\figlabel{2truein}{Figure~4.3:  Initial slice topologies for two black holes. 
%\cr a) Topology of an $N+1$ sheeted manifold.  \cr b) Topology of a two-sheeted
%manifold.}
\figlabelpdf{2truein}{Figure~4.3:  Initial slice topologies for two black holes. 
\cr a) Topology of an $N+1$ sheeted manifold.  \cr b) Topology of a two-sheeted
manifold.}{Figures/Figure4_3.pdf}

Analytic solutions to the initial-value equations on both topologies have been
found which represent multiple black holes at a moment of time symmetry (Misner
[1963], Lindquist [1963], and Brill and Lindquist [1963]).  If all of the holes
are momentarily at rest, then the extrinsic curvature vanishes everywhere on the
initial slice and the Hamiltonian constraint (4.8) reduces to Laplace's
equation.  A solution representing $N$ black holes is
$$
\psi =1+\sum\nolimits\limits_{\alpha =1}^{N} {{\mu }_{\alpha } \over
\left|{\bf x\rm -{\bf C}_{\alpha }}\right|}, \eqn{}
$$
where $\mu_\alpha$  are constants related to the masses of the holes and the
points ${\bf C}_\alpha$ are the positions of the centers of the holes in the
background space.  The points ${\bf C}_\alpha$ are deleted from the manifold so
that it is regular, and if they are sufficiently far apart, then each corresponds
to asymptotic infinity on a connecting sheet, as in Figure~4.3a, so the manifold
is complete.  In order to produce a two-sheeted solution, as in Figure~4.3b,
Misner [1963] demanded that the top and bottom sheets be identical.  Enforcing
this required the addition of an infinite number of poles to (4.18) with the
weights and locations of the poles determined by the method of spherical-inversion
imaging from electrostatics.  Misner showed that the infinite series converges to
an analytic function and is a solution to Laplace's equation so long as the holes
are not too close to each other.

In order to construct initial data which represents more physically interesting
situations, it is necessary to consider cases which are not time-symmetric and,
so, have non-vanishing extrinsic curvature.  This requires solving both the
Hamiltonian constraint (4.8) and the momentum constraint (4.9) on one of the two
topologies discussed above.  Analytically, the $N+1$ sheeted topology offers a
simpler approach since solutions to the constraints naturally produce this
topology with no additional effort.  Construction of initial data on the
two-sheeted manifold requires that the initial data satisfy an isometry
condition which enforces Misner's requirement that the two sheets be identical. 
On the other hand, we note that in general the Hamiltonian constraint (4.8) is a
quasi-linear, elliptic, differential equation and it will be necessary, in most
cases, to solve it numerically.  From this point of view, we find that
constructing initial data on an $N+1$ sheeted topology will require numerical
solutions on all $N+1$ asymptotically flat sheets.  The two-sheeted topology
requires solution on only one sheet since the other is identical. Thus, the
two-sheeted manifold proves to be computationally more convenient.

Thornburg [1987] has argued for the use of $N+1$ sheeted manifolds by suggesting
that the Hamiltonian constraint be solved only on the top sheet.  He argues that
the apparent horizons, which are present around the black holes, form a natural
boundary for the region in which the Hamiltonian constraint must be solved. 
Doing this, however, ignores the remainder of the initial-data slice.  While
information can never cross the apparent horizons into the top sheet during the
time evolution, this does not preclude the existence of global structures which
can be located only by knowing the initial data on the complete manifold as I
will show later ({\it cf}. Cook and York [1990]).

The final assumption of the conformal imaging approach is to assume that the
initial-data manifold has a two-sheeted topology and that the sheets are related
by an isometry which requires the two sheets, and any fields on them, to be
identical.  The approach, for the case of a single hole, was detailed by Bowen
[1979b] and Bowen and York [1980].  The extension for multiple holes was carried
out by Kulkarni, Shepley, and York [1983].  In the remainder of this chapter, I
will outline the approach, closely following Kulkarni {\it et al}. [1983].

Let $M_{\scriptscriptstyle N}$ denote the two-sheeted manifold containing $N$
Einstein-Rosen bridges, each representing a single black hole.  To construct the
manifold, take two identical, three-dimensional Euclidean spaces $E^3$ with $N$
non-intersecting spheres, located at ${\bf C}_\alpha$ and with radii $a_\alpha\,
(\alpha=1,\ldots,N)$, removed.  The two spaces will represent the top and bottom
sheets of $M_{\scriptscriptstyle N}$ and will be labeled, respectively, by $Y$
and $Z$.  If ${\bf p} = (p^1,p^2,p^3)$ represents a point in either sheet, then
the sheets are defined by
$$
Y=Z\equiv \left\{{\,{\bf p} \in {E}^{3}:\left|{{\bf p} -{\bf C}_{\alpha
}}\right|>{a}_{\alpha },\alpha =1,\ldots,N}\right\}. \eqn{}
$$
The collection of boundaries $B$ is defined as
$$
B\equiv \bigcup\limits_{\alpha =1}^{N} {B}_{\alpha
}\qquad\hbox{where}\qquad{B}_{\alpha }\equiv \left\{{\,{\bf p} \in
{E}^{3}:\left|{{\bf p} -{\bf C}_{\rm \alpha }}\right|={a}_{\alpha }}\right\}.
\eqn{}
$$
Let the union of the top sheet and the boundary be $\bar{Y}\equiv Y\cup B$, and
similarly $\bar{Z}\equiv Z\cup B$.  The manifold $M_{\scriptscriptstyle N}$ is
now defined as $M_N \equiv \bar{Y}\cup\bar{Z}$, with the boundaries $B_\alpha$
(each representing a throat or bridge) identified.  To put coordinates on the
manifold, we define a collection of coordinate maps $\Psi_\alpha$ on
$M_{\scriptscriptstyle N}$, each covering $Y$ and $Z$ through the $\alpha^{th}$
throat.  Let ${\bf x}$ denote the range of points in $E^3$ covered by the maps
and define
$$
X\equiv \left\{{\,{\bf x} \in {E}^{3}:\left|{{\bf x} -{\bf C}_{\alpha
}}\right|>{a}_{\alpha },\alpha =1,\ldots,N}\right\} \eqn{}
$$
and
$$
{S}_{\alpha }\equiv \left\{{\,{\bf x} \in {E}^{3}:\left|{{\bf x} -{\bf
C}_{\alpha }}\right|={a}_{\alpha }}\right\}. \eqn{} 
$$

We can now define the maps ${\bf J}_\alpha$ which identify the two sheets through
each of the throats
$$
{\bf J}_{\alpha }:{E}^{3}-\left\{{{a}_{\alpha }}\right\}\rightarrow
{E}^{3}-\left\{{{a}_{\alpha }}\right\} \eqn{}
$$
by
$$
{\bf J}_{\alpha }\left({\bf x}\right)\equiv \left({{{a}_{\alpha }^{2}
\over {r}_{\alpha }}}\right){\bf n}_{\alpha }+{\bf C}_{\alpha }, \eqn{}
$$
where
$$
{r}_{\alpha }=\left|{{\bf x} -{\bf C}_{\alpha }}\right| \eqn{}
$$
and
$$
{\bf n}_{\alpha }={\left({{\bf x} -{\bf C}_{\alpha }}\right) \over
{r}_{\alpha }}. \eqn{}
$$
We now define the $\alpha^{th}$ image of $X$ as
$$
{I}_{\alpha }={\bf J}_{\alpha }\left[{\bf X}\right], \eqn{}
$$
and the $\alpha^{th}$ coordinate map as
$$
{\Psi }_{\alpha }:\left({Y\cup {B}_{\alpha }\cup Z}\right)\rightarrow
\left({X\cup {S}_{\alpha }\cup {I}_{\alpha }}\right), \eqn{}
$$
where
%$$
%\eqalign{{\Psi }_{\alpha }\left({\bf p}\right)
%&=\left({{p}^{1},{p}^{2},{p}^{3}}\right)\in \left({X\cup {S}_{\alpha
%}}\right),\quad\hbox{if}\quad{\bf p} \in \left({Y\cup {B}_{\alpha }}\right)\cr
%&={\bf J}_{\alpha }\left({{p}^{1},{p}^{2},{p}^{3}}\right)\in {I}_{\alpha
%},\quad\hbox{if}\quad{\bf p} \in Z\cr}. \eqn{} 
%$$
$$
\Psi_\alpha({\bf p}) = \cases{(p^1,p^2,p^3)\in (X\cup S_\alpha),&if ${\bf p}\in
(Y\cup B_\alpha)$\cr
{\bf J}_\alpha(p^1,p^2,p^3)\in I_\alpha,&if ${\bf p}\in Z$.\cr} \eqn{}
$$
Note that ${\bf J}_\alpha^2$ is the identity operator since
$$
{\bf J}_{\alpha }\left[{{\bf J}_{\alpha }\left[{X}\right]}\right]=X,
\eqn{}
$$
and so ${\bf J}_\alpha$ is its own inverse.

For the case of one hole, we recover exactly the manifold structure seen
previously for the Schwarzschild black hole.  In this case, one coordinate patch
will smoothly cover the entire manifold.  If we consider the two sheets
separately, we have points on the top sheet labeled by $(r,\theta,\phi)$, and on
the bottom sheet by $(r^\prime,\theta^\prime,\phi^\prime)$.  From (4.29), points
on the top sheet will have coordinates $(r,\theta,\phi)$, while a point on the
bottom sheet will have coordinates ${\bf
J}_1(r^\prime,\theta^\prime,\phi^\prime)$.  In terms of spherical coordinates
this yields
$$
\left({r={{a}_{1}^{2} \over r'},\theta =\theta ',\phi =\phi '}\right), \eqn{}
$$
which is just the coordinate transformation (4.16) with $a_1 = \kappa M/16\pi$. 
So spherical coordinates with the origin removed correspond, in this case, to the
single coordinate patch for a single Einstein-Rosen bridge topology, with the
region interior to $r=\kappa M/16\pi$ being the bottom sheet.

In addition to having a two-sheeted manifold, it is required that any
fields on the two sheets of the manifold must be identical.  More precisely, we
require that the isometry between the sheets be generated by the maps ${\bf
J}_\alpha$ via the pull-back maps ${\bf
J}_\alpha^\ast$.  That is,
$$
\left({\hbox{data at}\,{\bf x} \in X}\right)=\pm {\bf J}_{\alpha
}^{\ast}\left({\hbox{data at}\,{\bf J}_{\alpha }\left({\bf x}\right)}\right).
\eqn{}
$$
The change of sign is allowed since the square of a map is the identity (4.30)
and since it will give physically meaningful results.  Since the second sheet is
covered in its entirety by $N$ different coordinate maps $I_\alpha$ (4.27), it is
required for consistency that (4.32) be satisfied not for a single $\alpha$ but
for {\it all} $\alpha=1,\ldots,N$.  This guarantees that fields on the bottom
sheet will be identical to the fields on the top sheet no matter which coordinate
map is used.

In the case of a scalar field $\Phi$ on the manifold, the isometry condition
(4.32) requires
$$
\Phi \left({\bf x}\right) =\pm \Phi \left({{\bf J}_{\alpha }\left({\bf
x}\right)}\right). \eqn{}
$$
For a one-form $\omega_i$, (4.32) requires
$$
{\omega }{}_{i}\left({\bf x}\right) =\pm {\left({{\bf J}_{\alpha
}}\right)}_{i}{}^{j}{\omega }{}_{j}\left({{\bf J}_{\alpha }\left({\bf
x}\right)}\right), \eqn{}
$$
where
$$
{\left({{\bf J}_{\alpha }}\right)}_i{}^j\equiv {\partial
{\left({{\bf J}_{\alpha }}\right)}{}^{j} \over \partial {x}{}^{i}} \eqn{}
$$
is the Jacobian of the map ${\bf J}_\alpha$.  For a vector field $V^i$, (4.32)
requires $$
{V}{}^{i}\left({\bf x}\right) =\pm {\left({{\bf J}_{\alpha
}^{-1}}\right)}_j{}^i{V}{}^{j}\left({{\bf J}_{\alpha }\left({\bf
x}\right)}\right), \eqn{}
$$
where the inverse Jacobian, defined by
$$
{\left({{\bf J}_{\alpha }^{-1}}\right)}_k{}^i{\left({{\bf J}_{
\alpha }}\right)}_j{}^k={\delta }_{j}^{k}, \eqn{}
$$
is used.  The extension to other tensor fields is obvious.

The two fields of importance for gravitational initial data are the metric and
extrinsic curvature.  These must satisfy
$$
{\gamma }{}_{i}{}_{j}\left({\bf x}\right) =\pm {\left({{\bf J}_{\alpha
}}\right)}{}_{i}{}^{k}{\left({{\bf J}_{\alpha }}\right)}{}_{j}{}^{\ell
}{\gamma }{}_{k}{}_{\ell }\left({{\bf J}_{\alpha }\left({\bf
x}\right)}\right) \eqn{}
$$
and
$$
{A}{}_{i}{}_{j}\left({\bf x}\right) =\pm {\left({{\bf J}_{\alpha
}}\right)}{}_{i}{}^{k}{\left({{\bf J}_{\alpha }}\right)}{}_{j}{}^{\ell
}{A}{}_{k}{}_{\ell }\left({{\bf J}_{\alpha }\left({\bf x}\right)}\right).
\eqn{}
$$
To explore the isometry conditions on the background fields, consider the form
that the Jacobian (4.35) and the inverse Jacobian take in Cartesian coordinates:
$$
{\left({{\bf J}_{\alpha }}\right)}{}_{i}{}^{j}={{a}_{\alpha }^{2} \over
{r}_{\alpha }^{2}}\left({{\delta }_{i}^{j}-2{n}_{\alpha }^{j}{n}^{\alpha
}_{i}}\right) \eqn{a}
$$
and
$$
{\left({{\bf J}_{\alpha }^{-1}}\right)}{}_{i}{}^{j}={{r}_{\alpha }^{2} \over
{a}_{\alpha }^{2}}\left({{\delta }_{i}^{j}-2{n}_{\alpha }^{j}{n}^{\alpha
}_{i}}\right). \eqnb{b}
$$
Using the conformal decomposition of the metric (3.1) and the assumption that
the background metric is flat (4.6), the isometry condition on the metric becomes
$$
{\psi }^{4}\left({\bf x}\right){f}{}_{i}{}_{j}=\pm {\left({{\bf J}_{
\alpha }}\right)}{}_{i}{}^{k}{\left({{\bf J}_{\alpha }}\right)}{}_{j}{}^{\ell
}{\psi }^{4}\left({{\bf J}_{\alpha }\left({\bf x}\right)}\right){
f}{}_{k}{}_{\ell }. \eqn{}
$$
Using the explicit form of the Jacobian (4.40), this reduces to
$$
{\psi }^{4}\left({\bf x}\right){f}{}_{i}{}_{j}=\pm {\left({{{a}_{\alpha }
\over {r}_{\alpha }}}\right)}^{4}{\psi }^{4}\left({{\bf J}_{\alpha
}\left({\bf x}\right)}\right){f}{}_{i}{}_{j}, \eqn{}
$$
or more simply
$$
\psi \left({\bf x}\right) ={{a}_{\alpha } \over {r}_{\alpha }}\psi
\left({{\bf J}_{\alpha }\left({\bf x}\right)}\right). \eqn{}
$$
Note that the isometry condition with a minus sign is not allowed since this
would imply the conformal factor vanishes on each throat which would make the
metric singular.  Now, if we consider the conformal transformation of the
extrinsic curvature, we find from (3.9) and (4.39)
$$
\eqalign{{\psi }^{-2}\left({\bf x}\right){\bar{
A}}{}_{i}{}_{j}\left({\bf x}\right) &=\pm {\left({{\bf J}_{\alpha
}}\right)}{}_{i}{}^{k}{\left({{\bf J}_{\alpha }}\right)}{}_{j}{}^{\ell }{\psi
}^{-2}\left({{\bf J}_{\alpha }\left({\bf x}\right)}\right){\bar{
A}}{}_{k}{}_{\ell }\left({{\bf J}_{\alpha }\left({\bf x}\right)}\right)\cr
 &=\pm {\left({{\bf J}_{\alpha }}\right)}{}_{i}{}^{k}{\left({{\bf J}_{
\alpha }}\right)}{}_{j}{}^{\ell }{\left({{{a}_{\alpha } \over {r}_{\alpha
}}}\right)}^{2}{\psi }^{-2}\left({\bf x}\right){\bar{A}}{}_{k}{}_{\ell
}\left({{\bf J}_{\alpha }\left({\bf x}\right)}\right)\cr}, \eqn{}
$$
or using (4.40),
$$
{\bar{A}}{}_{i}{}_{j}\left({\bf x}\right) =\pm {\left({{{a}_{\alpha }
\over {r}_{\alpha }}}\right)}^{6}\left({{\delta }_{i}^{k}-2{n}_{\alpha
}^{k}{n}^{\alpha }_{i}}\right)\left({{\delta }_{j}^{\ell }-2{n}_{\alpha
}^{\ell }{n}^{\alpha }_{j}}\right){\bar{A}}{}_{k}{}_{\ell
}\left({{\bf J}_{\alpha }\left({\bf x}\right)}\right). \eqn{}
$$

Equations (4.43) and (4.45) are conditions which solutions to the two constraint
equations (4.8) and (4.9)  must satisfy in order to represent inversion-symmetric
initial data on a regular, complete, two-sheeted initial slice.  Since the
background fields on the two sheets of the manifold are related by (4.43) and
(4.45), it is only necessary to solve the constraints on one sheet, provided
these relations are compatible with the constraints.  Assume that $\psi({\bf x})$
and $\bar{A}_{ij}({\it x})$ are solutions of (4.8) and (4.9a) in the region
$\bar{X}$ defined by 
$$
\bar{X}=X\,\bigcup\limits_{\alpha =1}^{N} {S}_{\alpha }, \eqn{}
$$
that is, the top sheet plus all of the throats.  Now consider the coordinate
patch $\Psi_\alpha$ and let  $\tilde\psi({\it x})$ and $\tilde{A}_{ij}({\it x})$
be defined for ${\bf x}$ in the region $I_\alpha$.  They will take values via the
two relations (4.43) and (4.45) for the $\alpha^{th}$ hole: 
$$
\tilde{\psi }\left({\bf x}\right) ={{a}_{\alpha } \over {r}_{\alpha }}\psi
\left({{\bf J}_{\alpha }\left({\bf x}\right)}\right) \eqn{}
$$
$$
{\tilde{A}}{}_{i}{}_{j}\left({\bf x}\right) =\pm {\left({{{a}_{\alpha }
\over {r}_{\alpha }}}\right)}^{6}\left({{\delta }_{i}^{k}-2{n}_{\alpha
}^{k}{n}^{\alpha }_{i}}\right)\left({{\delta }_{j}^{\ell }-2{n}_{\alpha
}^{\ell }{n}^{\alpha }_{j}}\right){\bar{A}}{}_{k}{}_{\ell
}\left({{\bf J}_{\alpha }\left({\bf x}\right)}\right) \eqn{}
$$
To simplify notation, let ${\bf x}^\prime\equiv {\bf J}_\alpha({\bf x})$.  From
(4.30), this gives 
$$
{\bf x} ={\bf J}_{\alpha }({\bf x}^\prime). \eqn{}
$$
Using the following definitions,
$$
r^\prime_{\alpha }\equiv \left|{{\bf x}^\prime-{\bf C}_{\alpha
}}\right|\qquad\hbox{and}\qquad{\bf n}^\prime_{\alpha }\equiv {\left({{\bf
x}^\prime-{\bf C}_{\alpha }}\right) \over {r^\prime_{\alpha }}}, \eqn{} 
$$
one finds:
$$
{r}_{\alpha }={r}_{\alpha }\left({{\bf J}_{\alpha }\left({{\bf
x}^\prime}\right)}\right)={{a}_{\alpha }^{2} \over {r^\prime_{\alpha }}} \eqn{} 
$$
and
$$
{\bf n}_{\alpha }={\bf n}_{\alpha }\left({{\bf J}_{\alpha
}\left({{\bf x}^\prime}\right)}\right)={\bf n}^\prime_{\alpha }. \eqn{}
$$
The data on the bottom sheet can now be written as
$$
\tilde{\psi }\left({\bf x}\right) ={r^\prime_{\alpha } \over {a}_{\alpha }}\psi
\left({{\bf x}^\prime}\right) \eqn{}
$$
and
$$
{\tilde{A}}{}_{i}{}_{j}\left({\bf x}\right) =\pm {\left({{r^\prime_{\alpha }
\over {a}_{\alpha }}}\right)}^{6}\left({{\delta }_{i}^{k}-2n^{\prime k}_{\alpha
}n^{\prime\alpha }_{i}}\right)\left({{\delta }_{j}^{\ell
}-2n^{\prime\ell}_{\alpha }n^{\prime\alpha }_{j}}\right){\bar{A}}{}_{k}{}_{\ell
}\left({{\bf x}^\prime}\right). \eqn{}
$$

Checking to see if the extrinsic curvature on the bottom sheet satisfies the
momentum constraint, we find
$$
\eqalign{{\bar{D}}{}_{j}{\tilde{A}}{}^{i}{}^{j}\left({\bf x}\right) &=\pm
{\left({{\bf J}_{\alpha
}}\right)}{}_{j}{}^{m}{\bar{D}^\prime_{m}}\left({{\left({{r^\prime_{\alpha }
\over {a}_{\alpha }}}\right)}^{6}\left({{f}{}^{k}{}^{i}-2n^{\prime k}_{\alpha
}n^{\prime i}_{\alpha }}\right)\left({{f}{}^{\ell }{}^{j}-2n^{\prime\ell}_{\alpha
}n^{\prime j}_{\alpha }}\right){\bar{A}}{}_{k}{}_{\ell }\left({{\bf
x}^\prime}\right)}\right)\cr &=\pm {\left({{r^\prime_{\alpha } \over {a}_{\alpha
}}}\right)}^{8}\left({{\delta }_{k}^{i}-2n^{\prime\alpha }_{k}n^{\prime i}_{\alpha
}}\right){\bar{D}^\prime_{j}}{\bar{A}}{}^{k}{}^{j}\left({{\bf
x}^\prime}\right),\cr} \eqn{} 
$$
where ${\bar{D}^\prime}_m$ is the background covariant derivative with respect to
${\bf x}^\prime$.  Thus, the extrinsic curvature on the bottom sheet, as given by
(4.48), satisfies the momentum constraint (4.9), if the top-sheet extrinsic
curvature defining it satisfies the momentum constraint.

To check the Hamiltonian constraint, we first
examine the right-hand side of (4.8).  From (4.54), we find
$$
\eqalign{{\tilde{A}}{}_{i}{}_{j}\left({\bf x}\right){\tilde{
A}}{}^{i}{}^{j}\left({\bf x}\right) &={f}{}^{i}{}^{k}{f}{}^{j}{}^{\ell
}{\tilde{A}}{}_{i}{}_{j}\left({\bf x}\right){\tilde{A}}{}_{k}{}_{\ell }\left({\bf
x}\right)\cr &={\left({{r^\prime_{\alpha } \over {a}_{\alpha
}}}\right)}^{12}{\bar{A}}{}_{i}{}_{j}\left({{\bf
x}^\prime}\right){\bar{A}}{}^{i}{}^{j}\left({{\bf x}^\prime}\right),\cr} \eqn{} 
$$
and, together with (4.53), this gives
$$
{\tilde{\psi }}^{-7}\left({\bf x}\right){\tilde{A}}{}_{i}{}_{j}\left({\bf
x}\right){\tilde{A}}{}^{i}{}^{j}\left({\bf x}\right) ={\left({{r^\prime_{\alpha
} \over {a}_{\alpha }}}\right)}^{5}\psi^{-7} \left({{\bf
x}^\prime}\right){\bar{A}}{}_{i}{}_{j}\left({{\bf
x}^\prime}\right){\bar{A}}{}^{i}{}^{j}\left({{\bf x}^\prime}\right). \eqn{}
$$
Next,  the Laplacian of the conformal factor gives
$$
\eqalign{{\overline{\nabla }}^{2}\tilde{\psi }\left({\bf x}\right)
&={f}{}^{i}{}^{j}{\left({{\bf J}_{\alpha
}}\right)}{}_{i}{}^{k}{\bar{D}^\prime_{k}}\left[{{\left({{\bf J}_{\alpha
}}\right)}{}_{j}{}^{\ell }{\bar{D}^\prime_{\ell }}\left({{r^\prime_{\alpha }
\over {a}_{\alpha }}}\right)\psi \left({{\bf x}^\prime}\right)}\right]\cr
&={\left({{r^\prime_{\alpha } \over {a}_{\alpha
}}}\right)}^{5}{\hbox{$\overline{\nabla }^\prime$}}^{2}\psi \left({{\bf
x}^\prime}\right).\cr} \eqn{} 
$$
Thus, the Hamiltonian constraint for the conformal factor on the bottom sheet,
$$
\eqalign{{\overline{\nabla }}^{2}\tilde{\psi }\left({\bf x}\right) +{1 \over
8}{\tilde{\psi }}^{-7}\left({\bf x}\right)&{\tilde{A}}{}_{i}{}_{j}\left({\bf
x}\right){\tilde{A}}{}^{i}{}^{j}\left({\bf x}\right)\cr
&={\left({{r^\prime_{\alpha } \over {a}_{\alpha
}}}\right)}^{5}\left({{\hbox{$\overline{\nabla }^\prime$}}^{2}\psi \left({{\bf
x}^\prime}\right)+{1 \over 8}{\psi }^{-7}\left({{\bf
x}^\prime}\right){\bar{A}}{}_{i}{}_{j}\left({{\bf
x}^\prime}\right){\bar{A}}{}^{i}{}^{j}\left({{\bf x}^\prime}\right)}\right),\cr}
\eqn{} 
$$
is satisfied if the top-sheet conformal factor defining it satisfies the
Hamiltonian constraint and if the extrinsic curvature is inversion-symmetric.

We can now summarize the conformal imaging method as an approach for computing
initial data which represents multiple black holes with non-vanishing initial
linear and angular momenta in an asypmtotically flat universe.  First, demand
that the initial-data slice consist of two asymptotically flat universes
connected via $N$ Einstein-Rosen bridges, one for each black hole.  Next, demand
that physical fields in the two universes be ``identical''.  We choose the
initial-data slice to be conformally flat and maximally embedded in the full
space-time manifold.

Working in the flat conformal background space, we start by taking  a full
three-dimensional Eulcidean three-manifold $E^3$ as the domain in which we must
determine the initial data.  For each black hole which will be present in our
universe, we choose its ``position'' ${\bf C}_\alpha$ and its ``radius''
$a_\alpha\, (\alpha=1,\ldots,N)$.  The $N$ spheres of radius $a_\alpha$ are
removed from the manifold $E^3$ and the remainder, $E^3 - \{a_\alpha\}$, is the
domain in which we must determine the initial data.

In the domain $E^3 - \{a_\alpha\}$, we first find solutions to the momentum
constraint (4.9) which satisfy the isometry condition (4.45).  By satisfying the
isometry condition (4.45), the background extrinsic curvature is said to be
inversion symmetric.  It is guaranteed to generate a solution on the second sheet
which satisfies the momentum constraint, and when combined with an
inversion-symmetric conformal factor, will yield a physical extrinsic curvature
which is regular everywhere and satisfies the condition (4.39) that it be
``identical'' on the two sheets.

Given an inversion-symmetric background extrinsic curvature, a solution
must be found for the Hamiltonian constraint (4.8) which satisfies the isometry
condition (4.43).  By satisfying the isometry condition (4.43), the conformal
factor is said to be inversion symmetric.  It is guaranteed to generate a
solution on the second sheet which satisfies the Hamiltonian constraint and, when
combined with the flat Euclidean background metric, yields a physical metric
which is regular everywhere and represents a complete manifold consisting of two
``identical'', asymptotically flat sheets.
\vfill
\eject
