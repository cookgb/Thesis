\chapterhead {Chapter~\the\chapnum:  Axisymmetric Numerical Solutions of the\cr
Hamiltonian Constraint for Two Black Holes}
%%%%%%%%%%%%%%%%%%%%%%%%%%%%%%%%%%%
To examine the case of two black holes, we must first define the configuration
in the background Euclidean space.  There are two natural length scales in the
background space:  the sizes of the throats and the separation of the centers of
the throats.  These length scales can be parameterized naturally by two
dimensionless parameters, $\alpha$ and $\beta$.  $\alpha$ has already been defined by
equation (6.27) as the ratio of the sizes of the two throats.  The second
parameter is defined as the ratio of the separation of the centers of the two
throats to the radius of the first hole.  That is,
$$
\beta \equiv {\left|{{\bf C}_{1}-{\bf C}_{2}}\right| \over
{a}_{1}}={\zeta }_{1}-{\zeta }_{2} \eqn{}
$$
where the definitions in equation (6.16) have been used.  In terms of scaled
cylindrical coordinates, the domain of solution in the case of two holes is
illustrated in Figure~9.1.

%\figlabel{3.25truein}{Figure~9.1:  Two-hole background space
%parameterization and solution domain.}
\figlabelpdf{3.25truein}{Figure~9.1:  Two-hole background space
parameterization and solution domain.}{Figures/Figure9_1.pdf}

While cylindrical coordinates are useful for describing the configuration of the
two holes in the background space, they are not well suited for use as the
coordinate system in which the Hamiltonian constraint will be differenced.  The
primary deficiency of cylindrical coordinates is that the minimal surfaces do
not lie along constant coordinate surfaces, making the differencing of the inner
boundary condition more difficult and less accurate.  In choosing a coordinate
system in which to difference the Hamiltonian constraint there are two primary
considerations.  First, the coordinate system should be orthogonal.  Second, the
coordinate system should possess constant coordinate surfaces which are
coincident with the minimal surfaces.  Because it is desired in the future to
evolve these initial-data sets numerically and examine the gravitational
radiation emitted, there is one additional condition on the coordinate system
which should be considered.  This condition is that the coordinate system should
have a constant coordinate surface which is nearly spherical surrounding the two
holes at large distances from the holes.

One coordinate system satisfying the three conditions listed above is known as
\v{C}ade\v{z} coordinates.  These coordinates were first used by \v{C}ade\v{z}
[1971] in one of the early attempts at numerically solving the two-body
problem.  While these coordinates have all of the desired properties, there are
also many difficulties associated with their use.  It was, in fact, difficulties
associated with the \v{C}ade\v{z} coordinates which hindered the last attempt at
investigating the initial-data sets for two black holes conducted by Rauber
[1985].  One difficulty associated with the use of \v{C}ade\v{z} coordinates is
the fact that the coordinate system must be generated numerically.  In addition,
the coordinate system contains a coordinate singularity positioned at a point on
the axis connecting the two holes.  

While the \v{C}ade\v{z} coordinates are well suited for the numerical evolution
of the two-body problem, the requirement of a nearly spherical outer boundary is
not necessary for the solution of the initial-value problem.  Ignoring this
requirement, we find that bispherical coordinates satisfy the two primary
conditions listed above.  While bispherical coordinates also contain a
coordinate singularity, it is located at spatial infinity and can be dealt with
easily as I will show below.  In the remainder of this chapter, I will describe
an approach for solving the Hamiltonian constraint using the bispherical
coordinate system.  Using solutions generated by this approach as a guide, I will
then investigate, in the next chapter, the solution of the Hamiltonian constraint
using \v{C}ade\v{z} coordinates.  As with the solutions to the one-hole problem,
I will reserve the interpretation of the results of the numerical calculations
for a later chapter.

The bispherical coordinate transformations to cylindrical coordinates are given
by 
$$
\rho ={C\sin\xi  \over \cosh\eta -\cos\xi }, \eqn{}
$$
$$
z={C\sinh\eta  \over \cosh\eta -\cos\xi }, \eqn{}
$$
and
$$
\phi =\phi . \eqn{}
$$
$C$ is a dimensionfull constant which sets the scale of the coordinates.  The
$\eta = constant$ surfaces are spheres centered on $z = C\coth\eta$ and with
radius $C/\sinh\eta$.  If we define $\eta_0^+$ to be the coordinate surface
coincident with the hole of radius $a_1$ and $\eta_0^-$ to be the coordinate
surface coincident with the hole of radius $a_2$, then we find that the
dimensioning constant is given by 
$$
C={a}_{1}\sinh{\eta }_{0}^{+}. \eqn{}
$$
The configuration parameters $\alpha$ and $\beta$ can be written as
$$
\alpha =-{\sinh{\eta }_{0}^{-} \over \sinh{\eta }_{0}^{+}} \eqn{}
$$
and
$$
\beta =-{\sinh\left({{\eta }_{0}^{+}-{\eta }_{0}^{-}}\right) \over \sinh{\eta
}_{0}^{-}}. \eqn{}
$$

Using a relation similar to (9.5), but for the other hole, we find
$$
{\eta }_{0}^{-}=-{\sinh}^{-1}\left[{\alpha \sinh{\eta }_{0}^{+}}\right]. \eqn{}
$$
Using (9.8) along with (9.6) and (9.7), we find the relationship
$$
\beta \sqrt {1+{\alpha }^{2}{\sinh}^{2}{\eta }_{0}^{+}}-\sqrt {1+{\alpha
}^{2}{\beta }^{2}{\sinh}^{2}{\eta }_{0}^{+}}={1 \over \alpha }. \eqn{}
$$
Solving (9.9) for $\eta_0^+$ gives
$$
{\eta }_{0}^{+}=\ln\sqrt {{\chi  \over 2}+{1 \over 2}\sqrt {{\chi
}^{2}-4}}, \eqn{}
$$
where
$$
\chi \equiv {\beta }^{2}+{1 \over {\beta }^{2}}\left({1+{1 \over {\alpha
}^{4}}}\right)-{2 \over {\alpha }^{2}}\left({1+{1 \over {\beta }^{2}}}\right).
\eqn{}
$$
Equations (9.10) and (9.8) determine the constant coordinate surfaces which are
coincident with the throats of the two black holes given the background
configuration parameters $\alpha$ and $\beta$.  The dimensionfull constant $C$ is
also fixed by the choice of these two parameters and the metric becomes
$$
{ds}^{2}={{a}_{1}^{2}{\sinh}^{2}{\eta }_{0}^{+} \over {\left({\cosh\eta -\cos\xi
}\right)}^{2}}\left({{d\eta }^{2}+{d\xi }^{2}+{\sin}^{2}\xi {d\phi
}^{2}}\right). \eqn{}
$$
The coordinates $(\eta,\xi,\phi)$ are dimensionless, and the domain in which the
Hamiltonian constraint must be solved (the top sheet) is
$\eta_0^-\le\eta\le\eta_0^+$, $0\le\xi\le\pi$, and $0\le\phi < 2\pi$. 
This coordinate range covers the entire infinite domain of the top sheet even
though the coordinates vary over a finite range.  This is possible because of a
coordinate singularity at infinity.  Examining the metric (9.12) we see that it
is conformally related to a cylindrical-like coordinate system, but the conformal
factor diverges as $\eta,\xi\rightarrow 0$.  Looking at the coordinate
transformations (9.2) and (9.3), we see that this corresponds to
$\rho,z\rightarrow\infty$.  Bispherical coordinates conformally compactify
infinity to a one dimensional line.  This behavior is the root of the problem
in using bispherical coordinates in an evolution problem.  On the other hand,
for the time-independent problem of solving the Hamiltonian constraint, it is a
bonus.  In this case, the boundary condition of asymptotic flatness can be
imposed exactly since infinity can be represented in the numerical domain.

Restricting to the case of axisymmetry, the Laplacian, written conservatively in
bispherical coordinates, takes the form
$$
\eqalign{{\overline{\nabla }}^{2}\psi ={{\left({\cosh\eta -\cos\xi }\right)}^{3}
\over {a}_{1}^{2}{\sinh}^{2}{\eta }_{0}^{+}}&\left\{{{\partial  \over \partial
\eta }\left({{1 \over \cosh\eta -\cos\xi }{\partial \psi  \over \partial \eta
}}\right)}\right.\cr&\left.{+{1 \over \sin\xi }{\partial  \over \partial \xi
}\left({{\sin\xi  \over \cosh\eta -\cos\xi }{\partial \psi  \over \partial \xi
}}\right)}\right\}.\cr}\eqn{}
$$
The boundary conditions (7.2) on the two inversion boundaries located at
$\eta=\eta_0^\pm$ take the following form
$$
{\left[{-{\left({\cosh\eta -\cos\xi }\right) \over \sinh\eta }{\partial \psi 
\over \partial \eta }+{\psi  \over 2}}\right]}_{\eta ={\eta }_{0}^{\pm }}=0.
\eqn{}
$$
Axisymmetry demands that on the $\xi=0$ and $\xi=\pi$ boundaries (except at
$\eta=\xi=0$) that 
$$
{\left.{{\partial \psi  \over \partial \xi }}\right|}_{\xi =0,\pi }=0. \eqn{}
$$
Finally, at $\eta=\xi=0$ we demand the Dirichlet boundary condition
$$
{\psi }_{(\eta =0,\xi =0)}=1, \eqn{}
$$
which fixes the asymptotic flatness boundary condition at infinity with no
approximation.  As in the one-hole case, a limiting form of the Laplacian must
be found along the $\rho=0$ axis.  Using the axisymmetry boundary condition
(9.15) and L'H\^{o}pital's rule, we find that
$$
\lim\limits_{\xi \rightarrow 0,\pi }^{}{1 \over \sin\xi }{\partial  \over
\partial \xi }\left({{\sin\xi  \over \cosh\eta -\cos\xi }{\partial \psi  \over
\partial \xi }}\right)=2{\partial  \over \partial \xi }\left({{1 \over \cosh\eta
-\cos\xi }{\partial \psi  \over \partial \xi }}\right). \eqn{}
$$

The Laplacian given by (9.13) and (9.17) along with the boundary conditions
(9.14), (9.15), and (9.16) form the base of equations by which the Hamiltonian
constraint will be differenced in bispherical coordinates.  To proceed with the
differencing, a discretization of the domain must be chosen.  The
straightforward choice is 
$$
{\eta }_{i}=i*{h}_{\eta }+{\eta
}_{0}^{-}\qquad\hbox{where}\quad i=0,\ldots,\Iind\quad\hbox{and}\quad{h}_{\eta
}={\left({{\eta }_{0}^{+}-{\eta }_{0}^{-}}\right) \over\Iind} \eqn{} $$
and
$$
{\xi }_{j}=j*{h}_{\xi }\qquad\hbox{where}\quad j=0,\ldots,\Jind
\quad\hbox{and}\quad{h}_{\xi }={\pi  \over J}. \eqn{} 
$$
This discretization cannot be used, however, because (9.18) does not guarantee
that $\eta=0$ will exist in the discretization as it must in order for the
asymptotic flatness boundary condition (9.16) to be applied.

In order to assure that $\eta=0$ is in the discrete domain, I choose $\eta$ to be
a nonlinear function of a new coordinate variable $s$.  The simplest choice is to
choose $\eta$ to be a quadratic function of the following form:
$$
\eta =\eta (s)=f*{s}^{2}+g*s, \eqn{}
$$
where $f$ and $g$ are constants to be determined so that $\eta=0$ will be contained in
the discrete domain.  Consider the following discretization of $s$,
$$
{s}_{i}=i*{h}_{s}\qquad\hbox{where}\qquad
i=-{\Sind}^{-},\ldots,0,\ldots,{\Sind}^{+} \eqn{} $$
and
$$
{h}_{s}={1 \over \Sind}\qquad\hbox{where}\qquad\Sind={\Sind}^{+}+{\Sind}^{-}.
\eqn{} 
$$
Together, (9.20) and (9.21) guarantee that $\eta(0) = 0$ is in the discrete
domain.  The two constants $f$ and $g$ can be chosen so that the end points of
the range of $s$ give the correct coordinate boundaries for $\eta$.  Let the
upper and lower limits on the range of $s$ be defined as
$$
{s}^{+}\equiv {\Sind}^{+}*{h}_{s}={\Sind}^{+}/\Sind \eqn{}
$$
and
$$
{s}^{-}\equiv -{\Sind}^{-}*{h}_{s}=-{\Sind}^{-}/\Sind. \eqn{}
$$
The two constants $f$ and $g$ are fixed by the two equations
$$
\eta ({s}^{+})={\eta }_{0}^{+}\qquad\hbox{and}\qquad\eta ({s}^{-})={\eta
}_{0}^{-}. \eqn{} $$
The result is
$$
f={\Sind \over {\Sind}^{+}}{\eta }_{0}^{+}+{\Sind \over {\Sind}^{-}}{\eta
}_{0}^{-} \eqn{} $$
and
$$
g={{\Sind}^{-} \over {\Sind}^{+}}{\eta }_{0}^{+}-{{\Sind}^{+} \over
{\Sind}^{-}}{\eta }_{0}^{-}. \eqn{} 
$$

In order for $s$ to represent an acceptable coordinate, the coordinate
transformation (9.20) must be invertible over the coordinate range in which $s$
is used.  The invertibility of the coordinate transformation can be examined
through the Jacobian of the coordinate transformation.  Let $J(s)$ represent the
Jacobian which is defined as
$$
J(s)\equiv {\partial \eta (s) \over \partial s}=2f*s+g. \eqn{}
$$
If the Jacobian is positive on both ends of the coordinate range in which $s$ is
used, then it will be positive everywhere in the range (since (9.20) is
quadratic).  We want $J(s)$ to be uniformly positive since this implies that
$\eta$ will be monotonically increasing as $s$ increases.  To ensure that any
virtual boundary points are also located in the range in which the coordinate
transformation is valid, we demand that
$$
J({s}^{-}-{h}_{s})>0\qquad\hbox{and}\qquad J({s}^{+}+{h}_{s})>0. \eqn{}
$$
From (9.29), a sufficient condition for the coordinate system to be well behaved
is
$$
{{\Sind}^{+} \over {\Sind}^{-}}{\left({2+{\Sind}^{+}}\right) \over
\left({2+3{\Sind}^{-}}\right)}<{{\eta }_{0}^{+} \over -{\eta
}_{0}^{+}}<{{\Sind}^{+} \over {\Sind}^{-}}{\left({2+3{\Sind}^{+}}\right) \over
\left({2+{\Sind}^{-}}\right)}. \eqn{} 
$$

In practice, one will choose values for $\alpha$ and $\beta$ to fix the
configuration and through (9.10), (9.11), and (9.8) these fix $\eta_0^+$ and
$\eta_0^-$.  Next, one chooses values for $\Sind^+$ and $\Sind^-$ so that (9.30)
is satisfied.  The discretization of the $(s,\xi)$-domain is then given by (9.19)
and (9.21).  The second order, conservative finite difference form of the
Hamiltonian constraint is expressed as 
$$
\eqalign{{\Ac}_{i,j}^{+}\left({{\psi }_{i+1,j}-{\psi
}_{i,j}}\right)+&{\Ac}_{i,j}^{-}\left({{\psi }_{i-1,j}-{\psi
}_{i,j}}\right)+{\Bc}_{i,j}^{+}\left({{\psi }_{i,j+1}-{\psi
}_{i,j}}\right)\cr &+{\Bc}_{i,j}^{-}\left({{\psi }_{i,j-1}-{\psi }_{i,j}}\right)
+{1 \over 8}{\psi }_{i,j}^{-7}{\left[{{a}_{1}^{2}{\bar{A}}{}_{\ell
}{}_{m}{\bar{A}}{}^{\ell }{}^{m}}\right]}_{i,j}=0,\cr} \eqn{}
$$
where
$$
\eqalign{{\Ac}_{i,j}^{\pm }\equiv {{\left({\cosh\eta ({s}_{i})-\cos{\xi
}_{j}}\right)}^{3} \over {h}_{s}^{2}J({s}_{i}){\sinh}^{2}{\eta }_{o}^{+}}{1 \over
\left({\cosh\eta ({s}_{i}\pm {h}_{s}/2)-\cos{\xi }_{j}}\right)J({s}_{i}\pm
{h}_{s}/2)}&\cr :\hbox{for all $(i,j)$ except $i=j=0$}&\cr} \eqn{}
$$
and
$$
{\Bc}_{i,j}^{\pm }\equiv \cases{{{\left({\cosh\eta ({s}_{i})-\cos{\xi
}_{j}}\right)}^{3} \over {h}_{\xi }^{2}\sin{\xi }_{j}{\sinh}^{2}{\eta
}_{o}^{+}}{\sin{(\xi }_{j}\pm {h}_{\xi }/2) \over \left({\cosh\eta
({s}_{i})-\cos{(\xi }_{j}\pm {h}_{\xi }/2)}\right)}&: for all
$i$ and $j\ne 0,\Jind$\cr {{\left({\cosh\eta ({s}_{i})-\cos{\xi
}_{j}}\right)}^{3} \over {h}_{\xi }^{2}{\sinh}^{2}{\eta }_{o}^{+}}{2 \over
\left({\cosh\eta ({s}_{i})-\cos{(\xi }_{j}\pm {h}_{\xi }/2)}\right)}&
: ${\hbox{for all $i$ and $j=0,\Jind$} \atop \hfill\hbox{except $i=j=0$.}}$\cr}
\eqn{}  $$
The square of the extrinsic curvature appearing in (9.31) can be evaluated
numerically by the technique discussed in Chapter 6.

The difference equations for points on the boundaries of the domain depend on
values of the conformal factor outside the computational domain.  These unknowns
are eliminated from the difference equations by the {\it implicit} use of
``virtual'' boundary points.  In the one-hole case, the virtual boundary points
were {\it explicitly} included in the computational mesh.  In the case of the
bispherical coordinate discretization of the Hamiltonian constraint, I have
chosen the alternate, but computationally equivalent, approach.  Put in a form
which is most convenient for eliminating the dependence on the virtual boundary
points, the inner boundary condition (9.14) at $i = -\Sind^-$ is expressed as
$$
\left({{\psi }_{-{\Sind}^{-}-1,j}-{\psi }_{-{\Sind}^{-},j}}\right)=\left({{\psi
}_{-{\Sind}^{-}+1,j}-{\psi }_{-{\Sind}^{-},j}}\right)+{1 \over
{\Cc}_{-{\Sind}^{-},j}}{\psi }_{-{\Sind}^{-},j}, \eqn{}
$$
where
$$
{\Cc}_{i,j}\equiv {-\left({\cosh\eta ({s}_{i})-\cos{\xi }_{j}}\right) \over
{h}_{s}J({s}_{i})\sinh({s}_{i})}. \eqn{}
$$
The inner boundary condition (9.14) at $i = \Sind^+$ is expressed as
$$
\left({{\psi }_{{\Sind}^{+}+1,j}-{\psi }_{{\Sind}^{+},j}}\right)=\left({{\psi
}_{{\Sind}^{+}-1,j}-{\psi }_{{\Sind}^{+},j}}\right)-{1 \over
{\Cc}_{{\Sind}^{+},j}}{\psi }_{{\Sind}^{+},j}. \eqn{}
$$
The axisymmetric boundary conditions (9.15) give
$$
\left({{\psi }_{i,-1}-{\psi }_{i,0}}\right)=\left({{\psi }_{i,1}-{\psi
}_{i,0}}\right)\qquad :\qquad\hbox{for}\quad i\ne 0 \eqn{}
$$
and
$$ \left({{\psi }_{i,\Jind+1}-{\psi }_{i,\Jind}}\right)=\left({{\psi
}_{i,\Jind-1}-{\psi }_{i,\Jind}}\right). \eqn{}
$$
Finally, the Dirichlet boundary condition (9.16) fixes the solution at infinity
to be asymptotically flat.  This final equation simply gives
$$
{\psi }_{0,0}=1. \eqn{}
$$

The differencing of the Hamiltonian constraint and boundary conditions given
above produces a set of second order, conservative difference equations.  The
equations being second order means that the {\it truncation} error in the
difference equations behaves like $\Order(h_s^2) + \Order(h_\xi^2)$.  In
general, this implies that the {\it discretization} error in the solution is also
second order so that halving the mesh spacing in both directions increases the
accuracy of the solution by a factor of four.  In the case of bispherical
coordinates, however, the presence of the coordinate singularity at $\eta=\xi=0$
spoils the behavior of the discretization error which proves to be first order.

The difference equations are solved via the Newton-Raphson method for nonlinear
systems described in Chapter 8.  The linearized equation for the iterative
correction is solved by means of the LINPACK routines for factoring (DGBFA)  and
solving (DGBSL) general banded matrices.  There are advantages and disadvantages
to using a direct solver for this set of difference equations.  The main
advantage is that the solution of the difference equations can be found to nearly
machine precision.  Using solutions on several meshes with different mesh
spacings and knowing that the leading behavior of the discretization error of the
solution is proportional to the mesh spacing, Richardson extrapolation ({\it
cf}. Press {\it et al}. [1988]) can be used to estimate the solution to the {\it
differential} equations.  This procedure is limited, however, by one of the
disadvantages of using a direct solver:  the resolution of the mesh is limited
by the physical memory of the computer.  For a direct solver, there must be
enough computer memory available to store and invert the coefficient matrix. 
Currently, the two-dimensional mesh size is limited to roughly $200\times 200$
mesh points.  Because higher order truncation error effects are totally
dominated by the lower order truncation error term only in the limit that the
mesh spacing becomes asymptotically small, the discretization error only becomes
first order asymptotically.  The limit in the mesh resolution mentioned above
thus greatly limits the accuracy with which results can be obtained by
Richardson extrapolation.

A solution to the finite difference equations for the Hamiltonian constraint can
be used in numerical integrals for various physically significant quantities. 
As in the one-hole case, the total ADM energy and the areas of the minimal
surfaces will be computed.  In the case of two holes, the dipole moment of the
energy distribution will not, in general, vanish.  Since this quantity is
important in the asymptotic expansion of the conformal factor (7.26) it will
also be computed.  One final quantity which is relevant in the case of two holes
is the proper separation of the two holes in the physical space.

The total energy integral is given by (8.29) in gravitational units.  Using
Gauss' law, one can transform it into a form similar to (8.30), which consists
of a volume integral and, in the case of two holes, two surface integrals.  In
terms of bispherical coordinates, the total energy for an axisymmetric system
can be expressed in dimensionless form as
$$
\eqalign{{E \over {a}_{1}}={\alpha {\sinh}^{2}{\eta }_{0}^{+} \over
2}\int_{0}^{\pi }{{\psi }_{(\eta ={\eta }_{0}^{-})}\sin\xi  \over
{\left({\cosh{\eta }_{0}^{-}-\cos\xi }\right)}^{2}}d\xi +{{\sinh}^{2}{\eta
}_{0}^{+} \over 2}\int_{0}^{\pi }{{\psi }_{(\eta ={\eta }_{0}^{+})}\sin\xi  \over
{\left({\cosh{\eta }_{0}^{+}-\cos\xi }\right)}^{2}}d\xi&\cr +{{\sinh}^{3}{\eta
}_{0}^{+} \over 8}\int\limits_{\xi=0}^{\xi=\pi}\int\limits_{\eta ={\eta
}_{0}^{-}}^{\eta ={\eta }_{0}^{+}}{{\psi
}^{-7}\left({{a}_{1}^{2}{\bar{A}}{}_{i}{}_{j}{\bar{A}}{}^{i}{}^{j}}\right)\sin\xi 
\over {\left({\cosh\eta -\cos\xi }\right)}^{3}}d\eta d\xi&.\cr} \eqn{}
$$
Note that the range of the volume integral in (9.40) covers the singular point
in the coordinate system.  Care must be taken in integrating this region.  An
extrapolative integral formula ({\it cf}. Press {\it et al}. [1988]) must be used
in the numerical approximation of the integral in order to avoid the divergence
of the integrand.

For an axisymmetric system, only the $z$-component of the dipole moment will not
vanish identically.  In terms of gravitational units, and written in terms of
the conformal factor, (7.10) takes the form
$$
{d}_{z}=-{1 \over 2\pi }\oint_{\infty }^{}\left\{{z{\bar{D}}{}^{k}\psi
+{\delta }_{z}^{k}\left({1-\psi }\right)}\right\}{d}^{2}{\bar{S}}{}_{k}.
\eqn{}
$$
Note that this integral must be written using Cartesian components.  Using
Gauss' law, (9.41) becomes
$$
\eqalign{{d}_{z}={1 \over 2\pi
}\oint_{{a}_{2}}^{}\left\{{z{\bar{D}}{}^{k}\psi +{\delta
}_{z}^{k}\left({1-\psi }\right)}\right\}{d}^{2}{\bar{S}}{}_{k}&+{1 \over
2\pi }\oint_{{a}_{1}}^{}\left\{{z{\bar{D}}{}^{k}\psi +{\delta
}_{z}^{k}\left({1-\psi }\right)}\right\}{d}^{2}{\bar{S}}{}_{k}\cr &+{1 \over
16\pi }\int_{V}^{}z{\psi
}^{-7}{\bar{A}}{}_{i}{}_{j}{\bar{A}}{}^{i}{}^{j}{d}^{3}\bar{V}.\cr}
\eqn{}
$$
Finally, in terms of bispherical coordinates, the $z$-component of the dipole
moment can be written in dimensionless form as
$$
\eqalign{{{d}_{z} \over {a}_{1}^{2}}=\alpha {\sinh}^{3}{\eta
}_{0}^{+}\int_{0}^{\pi }\left\{{{\left({1-\psi /2}\right)\sinh{\eta }_{0}^{-}
\over {\left({\cosh{\eta }_{0}^{-}-\cos\xi }\right)}^{3}}-{\left({1-\psi
}\right)\coth{\eta }_{0}^{-} \over {\left({\cosh{\eta }_{0}^{-}-\cos\xi
}\right)}^{2}}}\right\}\sin\xi d\xi&\cr +{\sinh}^{3}{\eta }_{0}^{+}\int_{0}^{\pi
}\left\{{{\left({1-\psi /2}\right)\sinh{\eta }_{0}^{+} \over {\left({\cosh{\eta
}_{0}^{+}-\cos\xi }\right)}^{3}}-{\left({1-\psi }\right)\coth{\eta }_{0}^{+}
\over {\left({\cosh{\eta }_{0}^{+}-\cos\xi }\right)}^{2}}}\right\}\sin\xi d\xi&
\cr +{{\sinh}^{4}{\eta }_{0}^{+} \over
8}\int\limits_{\xi=0}^{\xi=\pi}\int\limits_{\eta ={\eta }_{0}^{-}}^{\eta ={\eta
}_{0}^{+}}{{\psi
}^{-7}\left({{a}_{1}^{2}{\bar{A}}{}_{i}{}_{j}{\bar{A}}{}^{i}{}^{j}}\right) \over
{\left({\cosh\eta -\cos\xi }\right)}^{4}}\sinh\eta \sin\xi d\eta d\xi&.\cr}
\eqn{} 
$$ 
Again, care must be taken in numerically evaluating the volume
integral since the coordinate singularity is in the region of integration.

The areas of the minimal surfaces are given by two integrals of the form (8.32). 
In terms of bispherical coordinates, the two areas, $A^+_{\scriptscriptstyle MS}$
and $A^-_{\scriptscriptstyle MS}$, are given in dimensionless form by
$$
{{A}_{\scriptscriptstyle MS}^{\pm } \over {a}_{1}^{2}}={2\pi \sinh}^{2}{\eta
}_{0}^{+}\int_{0}^{\pi }{{\psi }_{\left({\eta ={\eta }_{0}^{\pm
}}\right)}^{4}\sin\xi  \over {\left({\cosh{\eta }_{0}^{\pm }-\cos\xi
}\right)}^{2}}d\xi . \eqn{}
$$

The proper separation of the two holes is given by an integral of the physical
metric along the line connecting the two holes.  In terms of bispherical
coordinates, this is given in dimensionless form by
$$
{L \over {a}_{1}}=\sinh{\eta }_{0}^{+}\int_{{\eta }_{0}^{-}}^{{\eta
}_{0}^{+}}{{\psi }_{(\xi =\pi )}^{2} \over \cosh\eta +1}d\eta . \eqn{}
$$

Unlike the one-hole case, there is no known model problem which can be used to
test the code for solving the full nonlinear difference equations.  However,
there are two useful scenarios which can be used to test the code.  First, when
the initial configuration is time-symmetric, then the exact solution is known. 
Equations (7.32) through (7.40)  describe the numerical approach for computing
the exact solution for the conformal factor as well as the total energy and
dipole moment.  The code can be run for several configurations which will test
the validity and accuracy of the code's solution of the linear problem and the
surface integral parts of the energy and dipole moment integrals.  Table~9.1
below compairs the Richardson extrapolated ADM energy from the numerical
solutions with the analytic result for several cases with $\alpha=1$.

$$
\vbox{\offinterlineskip
\hrule
\halign{\enskip\hfil#\hfil\enskip&\vrule#&
\strut\quad\hfil#\hfil\quad&
\hfil#\hfil\quad&
\hfil#\hfil\quad&\vrule#&
\quad\hfil#\hfil\quad&\vrule#&
\quad\hfil#\hfil\quad&\vrule#&
\quad\hfil#\hfil\enskip\cr
\omit&height2pt&\omit&\omit&\omit&&\omit&&\omit&&\omit\cr
\omit && \multispan{3} \hfil$E/a_1$\hfil && $E/a_1$ && $E/a_1$ && relative \cr
$\beta$ &&	$(80)$ & $(120)$ &	$(160)$ &&	$(extrap.)$ && $(analytic)$ &&
error\cr \omit&height2pt&\omit&\omit&\omit&&\omit&&\omit&&\omit\cr
\noalign{\hrule} \omit&height2pt&\omit&\omit&\omit&&\omit&&\omit&&\omit\cr
3&&6.0641&6.0902&6.1031&&6.1422&&6.1415&&$-0.01\%$\cr
4&&5.3215&5.3362&5.3434&&5.3653&&5.3642&&$-0.02\%$\cr
5&&4.9785&4.9897&4.9952&&5.0118&&5.0106&&$-0.02\%$\cr
6&&4.7777&4.7872&4.7918&&4.8059&&4.8046&&$-0.03\%$\cr
7&&4.6452&4.6538&4.6578&&4.6704&&4.6690&&$-0.03\%$\cr
8&&4.5511&4.5590&4.5627&&4.5742&&4.5727&&$-0.03\%$\cr
9&&4.4807&4.4881&4.4915&&4.5024&&4.5008&&$-0.04\%$\cr
10&&4.4261&4.4331&4.4364&&4.4466&&4.4449&&$-0.04\%$\cr
11&&4.3824&4.3892&4.3923&&4.4021&&4.4003&&$-0.04\%$\cr
12&&4.3468&4.3533&4.3562&&4.3657&&4.3639&&$-0.04\%$\cr
\omit&height2pt&\omit&\omit&\omit&&\omit&&\omit&&\omit\cr
\noalign{\hrule}
\noalign{\smallskip}
\multispan{11} \hfil Table 9.1:  ADM energy of time-symmetric solutions of the
Hamiltonian \hfil\cr
\multispan{11} \hfil constraint at $(80\times80)$, $(120\times120)$, and
$(160\times160)$ mesh resolutions, \hfil\cr
\multispan{11} \hfil Richardson extrapolated and compared with the
analytic solution \hfil\cr
\multispan{11} \hfil for the ADM energy.  $\beta$ indicates the separation of the
two holes.\hfil\cr}}
$$

To test the nonlinear aspects of the difference equations, the computation of
the extrinsic curvature, and to test the accuracy of the numerical approximations
of the volume integrals in the energy and dipole moment, the code can be run in a
configuration in which $\alpha$ and $\beta$ are both large.  This corresponds to
the case of one hole being much larger than the other and the two holes being
far apart.  Putting linear or angular momentum only on the larger hole results
in configurations which are comparable to the case of a single hole with linear
or angular momentum for which highly accurate results are available.  In
Tables~9.2, 9.3, and 9.4, I display the results for a set of runs for which
$\alpha=20$ and $\beta=50$.  The mesh was taken to be $160\times 160$.

$$
\vbox{\offinterlineskip
\hrule
\halign{\quad\hfil#\hfil\quad&\vrule#&
\strut\quad\hfil#\hfil\quad&\vrule#&
\quad\hfil#\hfil\quad&
\quad\hfil#\hfil\quad&\vrule#&
\quad\hfil#\hfil\quad\cr
\omit&height2pt&\omit&&\omit&\omit&&\omit\cr
 $P/a_1$ &&	$E/a_1$ &&	$M^+/a_1$ &	$M^-/a_1$ &&	$(E - M^-)/a_1$\cr
\omit&height2pt&\omit&&\omit&\omit&&\omit\cr
\noalign{\hrule}
\omit&height2pt&\omit&&\omit&\omit&&\omit\cr
 0.0&&2.102&&2.000&0.104&&1.998\cr
 1.0&&2.448&&2.112&0.104&&2.344\cr
 2.5&&3.689&&2.469&0.107&&3.582\cr
 5.0&&6.235&&3.066&0.112&&6.123\cr
 7.5&&8.926&&3.586&0.116&&8.810\cr
 10.0&&11.67&&4.045&0.121&&11.55\cr
 12.5&&14.43&&4.458&0.126&&14.30\cr
 15.0&&17.22&&4.838&0.130&&17.09\cr
 17.5&&20.01&&5.190&0.134&&19.88\cr
\omit&height2pt&\omit&&\omit&\omit&&\omit\cr
\noalign{\hrule}
\noalign{\smallskip}
\multispan8 \hfil Table~9.2:  Total energy and masses for two holes which are
very far apart,\hfil\cr
\multispan8 
\hfil the larger hole having linear momentum $P$.  The Extrinsic
Curvature\hfil\cr
\multispan8 \hfil obeys the isometry condition with the plus
sign.  $(\alpha=20,\beta=50)$\hfil\cr}}
$$

$$
\vbox{\offinterlineskip
\hrule
\halign{\quad\hfil#\hfil\quad&\vrule#&
\strut\quad\hfil#\hfil\quad&\vrule#&
\quad\hfil#\hfil\quad&
\quad\hfil#\hfil\quad&\vrule#&
\quad\hfil#\hfil\quad\cr
\omit&height2pt&\omit&&\omit&\omit&&\omit\cr
 $P/a_1$ &&	$E/a_1$ &&	$M^+/a_1$ &	$M^-/a_1$ &&	$(E - M^-)/a_1$\cr
\omit&height2pt&\omit&&\omit&\omit&&\omit\cr
\noalign{\hrule}
\omit&height2pt&\omit&&\omit&\omit&&\omit\cr
	0.0&&2.102&&2.000&0.104&&1.998\cr
	1.0&&2.430&&2.099&0.104&&2.326\cr
	2.5&&3.644&&2.429&0.107&&3.537\cr
	5.0&&6.180&&2.998&0.111&&6.069\cr
	7.5&&8.871&&3.499&0.116&&8.755\cr
	10.0&&11.61&&3.942&0.121&&11.49\cr
	12.5&&14.38&&4.343&0.125&&14.26\cr
	15.0&&17.17&&4.711&0.130&&17.04\cr
	17.5&&19.97&&5.052&0.134&&19.84\cr
\omit&height2pt&\omit&&\omit&\omit&&\omit\cr
\noalign{\hrule}
\noalign{\smallskip}
\multispan8 \hfil Table~9.3:  Total energy and masses for two holes which are
very far apart,\hfil\cr
\multispan8 \hfil the larger hole having linear momentum $P$.  The Extrinsic
Curvature\hfil\cr
\multispan8 \hfil obeys the isometry condition with the minus sign. 
$(\alpha=20,\beta=50)$\hfil\cr}}
$$

$$
\vbox{\offinterlineskip
\hrule
\halign{\quad\hfil#\hfil\quad&\vrule#&
\strut\quad\hfil#\hfil\quad&\vrule#&
\quad\hfil#\hfil\quad&
\quad\hfil#\hfil\quad&\vrule#&
\quad\hfil#\hfil\quad\cr
\omit&height2pt&\omit&&\omit&\omit&&\omit\cr
 $S/a_1^2$ &&	$E/a_1$ &&	$M^+/a_1$ &	$M^-/a_1$ &&	$(E - M^-)/a_1$\cr
\omit&height2pt&\omit&&\omit&\omit&&\omit\cr
\noalign{\hrule}
\omit&height2pt&\omit&&\omit&\omit&&\omit\cr
	0.0&&2.102&&2.000&0.104&&1.998\cr
	1.0&&2.149&&2.033&0.104&&2.045\cr
	3.0&&2.430&&2.227&0.104&&2.326\cr
	10.0&&3.580&&3.033&0.107&&3.473\cr
	30.0&&5.865&&4.672&0.111&&5.754\cr
	100.0&&10.52&&8.068&0.121&&10.40\cr
	300.0&&18.15&&13.67&0.138&&18.01\cr
	1000.0&&33.12&&24.68&0.172&&32.95\cr
\omit&height2pt&\omit&&\omit&\omit&&\omit\cr
\noalign{\hrule}
\noalign{\smallskip}
\multispan8 \hfil Table~9.4:  Total energy and masses for two holes which are
very far apart,\hfil\cr
\multispan8 \hfil the larger hole having angular momentum $S$.  The Extrinsic
Curvature\hfil\cr
\multispan8 \hfil obeys the isometry condition with the minus sign. 
$(\alpha=20,\beta=50)$\hfil\cr}}
$$

Assume now that the two holes are far enough apart so that the binding energy is
negligible compared to the total energy.   Subtracting the mass of the small
hole from the total energy  of the system gives an energy for the single hole
which has linear or angular momentum.  This value, along with the mass of the
larger hole, can be compared to the numerical results already calculated for a
single hole in Chapter 8.  Comparing the values of $(E - M^-)/a_1$ above against
the values of $E/a$ for the corresponding configurations in Tables 8.2 and~8.3
shows that all energies agree to within 0.4\%.  Comparing the values of
$M^+/a_1$ to $M/a$ shows even better agreement.

As in the case of the single hole, I will defer a physical interpretation of the
initial-data sets to a later chapter.  I will note now, however, that the
parameter space associated with the two-hole configuration is far too large to be
explored completely by numerical techniques.  Unlike the case of a single hole,
there is a two-dimensional parameter space (parameterized by $\alpha$ and
$\beta$) associated with the initial positions of the two holes.  There are then
three more two-dimensional parameter spaces covering the choices for the linear
and angular momentum vectors for the two holes and the choice of the isometry
condition.  If we consider these three parameter spaces separately (at most,
only two can be considered at once), then the full parameter space of the
two-hole problem is at least four-dimensional.  There is no reasonable way to
investigate the entire parameter space of the problem.  The most reasonable
approach for exploring the possible configurations is to reduce the size of the
parameter space by only considering configurations with some symmetry.  For
example, if we consider only configurations in which the two holes are of equal
size, then the parameter space is restricted to the $\alpha=1$ hyperplane.  If,
in addition, the magnitudes of the linear or angular momenta of the two holes are
equal, then the parameter space is further reduced to two, two-dimensional
parameter spaces corresponding to the cases where the momentum vectors are
either parallel or anti-parallel.  With these symmetries, the remaining parameter
space can be sampled and the results can be effectively visualized.  The
results from an extensive sampling of this reduced parameter space are
tabulated in Appendix~C and are discussed in a later chapter.

\vfill
\eject
