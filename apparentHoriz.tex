\chapterhead {Chapter~\the\chapnum:  Apparent Horizons on Inversion-Symmetric\cr
Initial-Data Sets}
%%%%%%%%%%%%%%%%%%%%%%%%%%%%%%%
The inversion-symmetric initial-data sets discussed in the preceding chapters
are all assumed to contain one or more black holes.  If we do not violate the
cosmic-censorship conjecture, then each initial-data slice will necessarily
intersect the null surface of at least one event horizon.  Ideally, we would
like to be able to determine the locations of these intersections on the
initial-data slice.  Unfortunately, determining the location of an event horizon
on any given slice requires the full future evolution of the data on that
slice.  Similar in concept, though not equivalent, to an event horizon is an
apparent horizon.  This surface, unlike the event horizon, can be located solely
in terms of the initial data.  Hawking and Ellis [1973] show that if an apparent
horizon exists, then an event horizon necessarily exists outside or coincident
with the apparent horizon.  The area of the apparent horizon thus gives a lower
bound on the area of the event horizon.  With this information, Hawking's [1971]
area theorem can be used to compute an upper limit on the total amount of energy
which can be radiated from the black-hole system via gravitational radiation. We
see, then, that in order to understand fully the physical content of an
initial-data slice, it is necessary to know the location of the apparent
horizons on the initial slice.

In order to locate apparent horizons, it is first necessary to clarify their
definition.  As discussed in Cook and York [1990], Hawking and Ellis [1973]
define an apparent horizon as the outer boundary of a connected component of a
trapped region.  A trapped region, then, is defined as the collection of all
points within all compact, orientable spacelike two-surfaces for which the
surface-orthogonal outgoing null geodesics have nonpositive expansion.  If a
spacelike hypersurface is asymptotically flat, then the expansion of null
geodesics emanating from a compact two-surface far from any holes will be
positive.  The search for apparent horizons can thus be formulated in terms of
the search for compact, orientable two surfaces on which the outgoing null
geodesics have {\it zero} expansion.

Following York [1989], consider a spacelike hypersurface $\Sigma$ with
timelike unit normal $n^\mu$.  Let $S$ be a compact, orientable spacelike
two-surface embedded in $\Sigma$ with unit normal $s^\mu$ (see Figure~11.1). 
Two orthogonal null vectors $k^\mu$ and $\ell^\mu$ can be constructed from
$n^\mu$ and $s^\mu$ as
$$
{k}{}^{\mu }\equiv {1 \over \sqrt {2}}({n}{}^{\mu }+{s}{}^{\mu
})\qquad\hbox{and}\qquad{\ell }{}^{\mu }\equiv {1 \over \sqrt {2}}({n}{}^{\mu
}-{s}{}^{\mu }). \eqn{}
$$
%\figlabel{2.75truein}{Figure~11.1:  An apparent horizon $S$ intersecting the
%initial-data surface $\Sigma$.\cr  $n^\mu$ is the timelike unit normal vector for
%the initial-data surface and $s^\mu$\cr is the {\it outward} pointing spacelike
%unit normal to the apparent horizon.}
\figlabelpdf{2.75truein}{Figure~11.1:  An apparent horizon $S$ intersecting the
initial-data surface $\Sigma$.\cr  $n^\mu$ is the timelike unit normal vector for
the initial-data surface and $s^\mu$\cr is the {\it outward} pointing spacelike
unit normal to the apparent horizon.}{Figures/Figure11_1.pdf}
\noindent The null surface $\eta$, generated by $k^\mu$, has an induced
metric $S_{\mu\nu}$ given by
$$
\eqalign{{S}{}_{\mu }{}_{\nu }&={g}{}_{\mu }{}_{\nu }+{k}{}_{\mu }{\ell }{}_{\nu
}+{k}{}_{\nu }{\ell }{}_{\mu }\cr&={g}{}_{\mu }{}_{\nu }+{n}{}_{\mu }{n}{}_{\nu
}-{s}{}_{\mu }{s}{}_{\nu }\cr&={\gamma }{}_{\mu }{}_{\nu }-{s}{}_{\mu
}{s}{}_{\nu }\cr&={P}_{\mu }^{\alpha }{P}_{\nu }^{\beta }{g}{}_{\alpha
}{}_{\beta },\cr} \eqn{}
$$
where $\gamma_{\mu\nu}$ is the induced metric of $\Sigma$ and where $P^\mu_\nu$
is the operator for projection onto the null surface generated by $k^\mu$:
$$
{P}_{\nu }^{\mu }\equiv {\delta }_{\nu }^{\mu }+{n}{}^{\mu }{n}{}_{\nu
}-{s}{}^{\mu }{s}{}_{\nu }. \eqn{}
$$
The extrinsic curvature of the null surface $\eta$ is defined similarly to
(2.19), by
$$
{\kappa }{}_{\mu }{}_{\nu }\equiv -{P}_{\mu }^{\alpha }{P}_{\nu }^{\beta
}{\nabla }{}_{(\alpha }{k}{}_{\beta )}, \eqn{}
$$
where $\nabla_\mu$ is the full, four-dimensional covariant derivative.  Because
$k^\mu$ is the null vector tangent to the outgoing null geodesic at the
surface $S$, the surface $S$ will be an apparent horizon if the expansion (trace
of (11.4)) vanishes.

Consider now the extrinsic curvature of the surface $S$ embedded in $\Sigma$.  It
is given by
$$
{\chi }{}_{\mu }{}_{\nu }\equiv -{P}_{\mu }^{\alpha }{P}_{\nu }^{\beta
}{D}{}_{(\alpha }{s}{}_{\beta )} \eqn{}
$$
where $D_\mu$ is the spatial covariant derivative induced on $\Sigma$. 
Expanding (11.4) and imposing the apparent horizon condition that the trace of
$\kappa_{\mu\nu}$ vanish gives
$$
\sqrt {2}{\kappa }_{\mu }^{\mu }={\chi }_{\mu }^{\mu }+{P}_{\mu }^{\nu
}{K}{}_{\nu }{}^{\mu }=0, \eqn{}
$$
where $K_{\mu\nu}$ is the extrinsic curvature of $\Sigma$ embedded in the full
space-time.  This can be simplified to
$$
{D}{}_{i}{s}{}^{i}-K+{K}{}_{i}{}_{j}{s}{}^{i}{s}{}^{j}=0, \eqn{a}
$$
or in terms of (3.8),
$$
{D}{}_{i}{s}{}^{i}-{2 \over 3}K+{A}{}_{i}{}_{j}{s}{}^{i}{s}{}^{j}=0. \eqnb{b}
$$
Equation (11.7), thus, represents a necessary condition that $s^i$ be the
outward-pointing unit-normal to an apparent horizon.  If the surface to which
$s^i$ is normal is the outer-most such surface, then that surface is an apparent
horizon.  Equation (11.7b) can be expressed in terms of the conformal background
space and the background quantities defined on it.  For a unit normal vector,
the natural conformal weighting is given by
$$
{s}{}^{i}={\psi }^{-2}{\bar{s}}{}^{i}, \eqn{}
$$
where $\bar{s}^i$ is the background unit-normal vector.  Using the conformal
weightings for the metric and extrinsic curvature from Chapter~3, (11.7b) becomes
$$
{\bar{D}}{}_{i}{\bar{s}}{}^{i}+4
{\bar{s}}{}^{i}{\bar{D}}{}_{i}\ln\psi -{2 \over 3}{\psi
}^{2}K+{\psi
}^{-4}{\bar{A}}{}_{i}{}_{j}{\bar{s}}{}^{i}{\bar{s}}{}^{j}=0.
\eqn{}
$$
Equation (11.9) is a single equation for a three-dimensional vector and so it
seems at first that it cannot uniquely determine $\bar{s}^i$.  Recalling that
$\bar{s}^i$ must be a unit vector removes one degree of freedom from $\bar{s}^i$
and the fact that the surface of the apparent horizon must be a closed
two-surface removes a second.  Together, these conditions demand that
$$
{\bar{s}}{}_{i}=\lambda
{\bar{D}}{}_{i}\tau\qquad\hbox{where}\qquad\lambda
={\left[{({\bar{D}}{}^{i}\tau )({\bar{D}}{}_{i}\tau )}\right]}^{-1/2},
\eqn{}
$$
and where $\tau$ is a scalar function whose level surface $\tau=\tau_0$ defines
the position of the apparent horizon.  Using (11.10) in (11.9) gives the final
form for the apparent-horizon equation
$$
\lambda {\overline{\nabla }}^{2}\tau +({\bar{D}}{}^{i}\lambda
)({\bar{D}}{}_{i}\tau )+4\lambda ({\bar{D}}{}^{i}\ln\psi
)({\bar{D}}{}_{i}\tau )-{2 \over 3}{\psi }^{2}K+{\psi }^{-4}{\lambda
}^{2}{\bar{A}}{}_{i}{}_{j}({\bar{D}}{}^{i}\tau
)({\bar{D}}{}^{j}\tau )=0. \eqn{}
$$
The apparent-horizon equation is thus in the form of a highly nonlinear partial
differential equation for the scalar function $\tau$ which can be solved
numerically by a variety of means.

If the manifold on which the initial data is specified is inversion-symmetric, as
is the case for the initial data described in earlier chapters, then the
apparent horizon must be inversion-symmetric as well.  The unit normal to the
surface of the apparent horizon, as with all vector fields defined on the
manifold, must satisfy isometry condition (4.36), so
$$
{s}{}^{i}({\bf x})=\pm {({\bf J}_{\alpha
}^{-1})}{}_{j}{}^{i}{s}{}^{j}({\bf J}_{\alpha }({\bf x})). \eqn{}
$$
In terms of the conformal background fields, this becomes
$$
{\bar{s}}{}^{i}({\bf x})=\pm {\left({{{a}_{\alpha } \over {r}_{\alpha
}}}\right)}^{2}{({\bf J}_{\alpha
}^{-1})}{}_{j}{}^{i}{\bar{s}}{}^{j}({\bf J}_{\alpha }({\bf x})).
\eqn{}
$$
To prove that apparent horizons on initial-data slices which contain
inversion-symmetric initial data are themselves inversion-symmetric, let us
assume that $\bar{s}^i({\bf x})$ is a solution of the apparent-horizon equation
(11.9), where ${\bf x}$ belongs to the set of points which satisfies $\tau({\bf
x})=\tau_0$.  The inversion-symmetric counterpart must then be the vector field
$\tilde{s}^i({\bf x})$ defined by
$$
{\tilde{s}}{}^{i}({\bf x})=\pm {\left({{{a}_{\alpha } \over {r}_{\alpha
}}}\right)}^{2}{({\bf J}_{\alpha
}^{-1})}{}_{j}{}^{i}{\bar{s}}{}^{j}({\bf J}_{\alpha }({\bf x})).
\eqn{}
$$
In terms of Cartesian coordinates and using (4.40) and (4.49)--(4.54), direct
calculation shows that
$$
{\bar{D}}{}_{i}{\tilde{s}}{}^{i}({\bf x})=\pm {\left({{r^\prime_{\alpha }
\over {a}_{\alpha
}}}\right)}^{2}\left[{\bar{D}^\prime_{i}{\bar{s}}{}^{i}({\bf
x}^\prime)-{4 \over r^\prime_\alpha}n^{\prime\alpha}_{i}{\bar{s}}{}^{i}({\bf
x}^\prime)}\right], \eqn{}
$$
$$
{\tilde{s}}{}^{i}({\bf x}){\bar{D}}{}_{i}\ln\psi ({\bf x})=\pm
{\left({{r^\prime_{\alpha } \over {a}_{\alpha
}}}\right)}^{2}{\bar{s}}{}^{i}({\bf x}^\prime)\left[{\bar{D}^\prime_{i}\ln\psi
({\bf x}^\prime)+{1 \over r^\prime_\alpha}{n}_{i}^{\prime\alpha }}\right], \eqn{}
$$
$$
{\psi }^{-2}({\bf x})K({\bf x})=\pm {\left({{r^\prime_{\alpha } \over
{a}_{\alpha }}}\right)}^{2}{\psi }^{-2}({\bf x}^\prime)K({\bf x}^\prime), \eqn{}
$$
and
$$
{\psi }^{-4}({\bf x}){\bar{A}}_{ij}({\bf x}){\tilde{s}}{}^{i}({\bf x}
){\tilde{s}}{}^{j}({\bf x})=\pm {\left({{r^\prime_{\alpha } \over {a}_{\alpha
}}}\right)}^{2}\left[{{\psi }^{-4}({\bf x}^\prime){\bar{A}}_{ij}({\bf
x}^\prime){\bar{s}}{}^{i}({\bf x}^\prime){\bar{s}}{}^{j}({\bf
x}^\prime)}\right]. \eqn{}
$$
Thus, we find that the apparent-horizon equation for $\tilde{s}^i({\bf x})$
becomes
$$
\eqalign{{\bar{D}}{}_{i}{\tilde{s}}{}^{i}({\bf x})+
4{\tilde{s}}{}^{i}({\bf x}){\bar{D}}{}_{i}\ln\psi ({\bf x})-{2 \over 3}{\psi
}^{2}({\bf x})K({\bf x})+{\psi }^{-4}({\bf x}){\bar{A}}{}_{i}{}_{j}({\bf x}
){\tilde{s}}{}^{i}(\bf x\rm ){\tilde{s}}{}^{j}(\bf x\rm )&\cr =\pm
{\left({{{r^\prime}_{\alpha } \over {a}_{\alpha
}}}\right)}^{2}\biggl[{\bar{D}^\prime_i}{\bar{s}}{}^{i}({\bf x}^\prime)+
4{\bar{s}}{}^{i}({\bf x}^\prime){\bar{D}^\prime_i}\ln\psi
({\bf x}^\prime)-{2 \over 3}{\psi }^{2}({\bf x}^\prime)K({\bf
x}^\prime)\qquad&\cr +{\psi }^{-4}({\bf x}^\prime){\bar{A}}{}_{i}{}_{j}({\bf
x}^\prime){\bar{s}}{}^{i}({\bf x}^\prime){\bar{s}}{}^{j}({\bf
x}^\prime)\biggr]&.\cr} \eqn{}
$$
Since the right-hand side of (11.19) satisfies
the apparent-horizon equation for ${\bf x}^\prime$ such that $\tau({\bf
x}^\prime)=\tau_0$, we find that $\tilde{s}^i({\bf x})$ satisfies the
apparent-horizon equation for ${\bf x}$ such that $\tau({\bf J}_\alpha({\bf
x}))=\tau_0$.

Many methods can be used to locate apparent horizons by solving the
apparent-horizon equation (11.11).  For the case of two black holes on a
time-symmetric slice, several authors (\v{C}ade\v{z} [1974] and Bishop [1982]
and [1984]) have searched for apparent horizons (minimal surfaces in the
time-symmetric case) by means of ``shooting'' methods.  Unfortunately, this
method is applicable only to axisymmetric initial-data sets since it is an
inherently one-dimensional method.  Nakamura {\it et al}. [1984] and [1985] have
developed and tested a method for locating apparent horizons based on an
expansion in spherical harmonics.  This approach is applicable to full,
three-dimensional initial-data sets.  An alternative approach which I have
developed (Cook and York [1990]) is to solve (11.11) directly as a boundary
value problem.  This approach is also applicable to three-dimensional
initial-data sets.

I have applied this approach to the problem of locating the apparent horizons in
the one-hole initial-data sets discussed in Chapter~8.  In the next chapter, I
will describe the method and the solutions found.  In the future, it will be
essential to locate the apparent horizons in the two-hole initial-data sets. 
This work has not yet been undertaken, although it is planned for the near
future.
\vfill
\eject
