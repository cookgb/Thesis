\chapterhead{Chapter~\the\chapnum:  The Hamiltonian Constraint for Two Black\cr
Holes Using \v{C}ade\v{z} Coordinates}
%%%%%%%%%%%%%%%%%%%%%%%%%%%%%%%%%%%%%%%
As mentioned in Chapter~9, the coordinate system first used by \v{C}ade\v{z}
[1971] in his early investigations into black-hole collisions is the best
coordinate system known at this time for investigating the numerical evolution
of two-hole, initial-data sets.  Unfortunately, this coordinate system is
difficult to work with because it must be constructed numerically and because it
contains a severe coordinate singularity.  In this chapter, I will describe all
aspects of the problem of defining \v{C}ade\v{z} coordinates and explore an
approach for applying them to the problem of solving the Hamiltonian constraint
for two black holes.

\v{C}ade\v{z} coordinates are defined in terms of a complex-plane
transformation ({\it cf}. Moon and Spincer [1988]).  Letting the base
coordinate system be cylindrical coordinates, the complex variable $\zeta$ is
defined as 
$$
\zeta \equiv z+i\rho. \eqn{}
$$
The cylindrical coordinate $z$ is then the real part of $\zeta$ and the
cylindrical coordinate $\rho$ is the imaginary part of $\zeta$.  The
\v{C}ade\v{z} coordinates are defined as the real and imaginary parts of a
complex function $\chi(\zeta)$.  Specifically,
$$
\eta (\rho ,z)\equiv \Re{e}\chi (\zeta)\qquad\hbox{and}\qquad\xi (\rho ,z)\equiv
\Im{m}\chi (\zeta ). \eqn{}
$$
Defined in this way, the coordinate system is guaranteed to be orthogonal
everywhere that $\chi$ is analytic.  The explicit form of the complex function
$\chi$ is taken to be
$$
\chi (\zeta )\equiv {C}_{0}^{+}\ln(\zeta -{\zeta }^{+})+{C}_{0}^{-}\ln(\zeta
-{\zeta }^{-})+\sum\nolimits\limits_{n=1}^{\infty }
\left\{{{C}_{n}^{+}{\left({\zeta -{\zeta
}^{+}}\right)}^{-n}+{C}_{n}^{-}{\left({\zeta -{\zeta
}^{-}}\right)}^{-n}}\right\}. \eqn{} 
$$
The coefficients $C^\pm_i$ are real scalars and the two complex parameters
$\zeta^\pm$ will fix the locations of the two holes.

To see the general behavior of this transformation, consider the limit that
$|\zeta|$ is large compared to $|\zeta^+|$ and $|\zeta^-|$.  In this case, (10.3)
becomes 
$$
\chi (\zeta )\approx \left({{C}_{0}^{+}+{C}_{0}^{-}}\right)\ln(\zeta )
\eqn{}
$$
and the coordinate transformations become
$$
\eta (\rho ,z)\approx \left({{C}_{0}^{+}+{C}_{0}^{-}}\right)\ln\sqrt {{\rho
}^{2}+{z}^{2}} \eqn{a}
$$
and
$$
\xi (\rho ,z)\approx \left({{C}_{0}^{+}+{C}_{0}^{-}}\right)\arg(\zeta ). \eqnb{b}
$$
We are free to scale the coordinate system so that $C^+_0 + C^-_0 = 1$ and,
in this limit, $\eta$ is simply a logarithmically scaled, spherical-polar radial
coordinate and $\xi$ is simply the spherical-polar colatitude coordinate.  This
transition to spherical-polar behavior in the coordinates is precisely the
desired behavior, at large distances from the holes, alluded to in Chapter 9.

If we consider only the logarithmic terms in (10.3) and take the limit that
$\zeta\rightarrow\zeta^\pm$, then we find that the coordinates again approach
logarithmically scaled, spherical-polar coordinates, but centered around
$\zeta^\pm$.  We see then that the $\eta = constant$ surfaces near $\zeta^\pm$
form ovals surrounding $\zeta^+$ and $\zeta^-$ and the coefficients in the
infinite series in (10.3) can be chosen to make specific $\eta = constant$
surfaces truly spherical.  This is again the desired behavior, near the holes,
for a coordinate system described in Chapter~9.

The coordinate system based on (10.3) thus has all of the properties desired
for a coordinate system to be used for a two-hole manifold.  Previous work
(\v{C}ade\v{z} [1971], Smarr {\it et al}. [1976], Rauber [1985]) done with these
coordinates has been restricted to the case in which the two holes are of equal
size.  This restriction simplifies the definition of the coordinate system
(10.3).  Here, however, I have extended the traditional definition of
\v{C}ade\v{z} coordinates so that it can address problems in which the holes
are not of equal size.  The remaining task in defining the \v{C}ade\v{z}
coordinates is to prescribe a method for fixing the coefficients $C^\pm_i$ and
parameters $\zeta^\pm$.

As in the case of bispherical coordinates, I parameterize the configuration by
the two dimensionless parameters $\alpha$ and $\beta$ ({\it cf}. Chapter~9).  In
order to maintain the ability to transform to bispherical coordinates easily, I
choose to locate the boundaries of the holes (throats) as prescribed in
Chapter~9 for bispherical coordinates.  This approach determines the two
bispherical coordinate surfaces $\eta^\pm_0$ with which the throats are
coincident.  The two positional parameters $\zeta^\pm$ are then fixed as 
$$
{\zeta }^{+}\equiv \cosh{\eta
}_{0}^{+}\qquad\hbox{and}\qquad{\zeta }^{-}\equiv {\alpha }^{-1}\cosh{\eta
}_{0}^{-}. \eqn{}
$$
In terms of the complex variable $\zeta$, the two sets of points constituting the
throats can be parameterized by $\theta^\pm$ in the following way:
$$
\zeta ({\theta }^{+})\equiv {\zeta }^{+}+\cos{\theta }^{+}+i\sin{\theta
}^{+}\qquad\hbox{for}\quad 0\le {\theta }^{+}\le \pi \eqn{}
$$
and
$$
\zeta ({\theta }^{-})\equiv {\zeta }^{-}+{\alpha }^{-1}\left({\cos{\theta
}^{-}+i\sin{\theta }^{-}}\right)\qquad\hbox{for}\quad 0\le {\theta }^{-}\le \pi.
\eqn{}
$$
Using (10.7) and (10.8), the $C^\pm_i$ coefficients are fixed by demanding that
$$
{\eta }^{+}=\Re{e}\chi (\zeta ({\theta }^{+}))\qquad\hbox{and}\qquad{\eta
}^{-}=\Re{e}\chi (\zeta ({\theta }^{-})), \eqn{}
$$
where $\eta^\pm$ are the coordinate values of the \v{C}ade\v{z} coordinate
$\eta$ for each of the throats and these can be freely chosen (the implications
of their choice will be discussed below).  The two equations in (10.9),
evaluated at a large set of points on (10.7) and (10.8), yield a large set of
equations which can be used to fix the $C^\pm_i$ coefficients numerically by a
linear least squares fit.  Finally, given values for the $C^\pm_i$ coefficients,
they, and the values for $\eta^\pm$, are rescaled so that $C^+_0 + C^-_0 =
1$.

Given a point in cylindrical coordinates, the corresponding point in
\v{C}ade\v{z} coordinates is easily obtained from (10.2).  However, given a
point $\chi_0 = \chi(\eta,x)$ in \v{C}ade\v{z} coordinates, the corresponding
point in cylindrical coordinates can only by found numerically.  The inversion
can be accomplished iteratively via Newton's method generalized for complex
functions.  The algorithm is
$$
{\zeta }^{i+1}={\zeta }^{i}-\left({\chi ({\zeta }^{i})-{\chi
}_{0}}\right){\left({{\left.{{\partial \chi  \over \partial \zeta
}}\right|}_{{\zeta }^{i}}}\right)}^{-1}. \eqn{}
$$

Because of the behavior demanded in the coordinate system around the throats and
at large distances, there is necessarily a critical point in the complex
transformation (10.3).  This point is defined by
$$
{\partial \chi (\zeta ) \over \partial \zeta }=0. \eqn{}
$$
Recalling the Cauchy-Riemann conditions
$$
{\partial \eta  \over \partial z}={\partial \xi  \over \partial \rho
}\qquad\hbox{and}\qquad{\partial \eta  \over \partial \rho }=-{\partial \xi 
\over \partial z}, \eqn{}
$$
the determinant of the Jacobian of the coordinate transformation becomes
$$
J={\partial \eta  \over \partial \rho }{\partial \xi  \over \partial
z}-{\partial \eta  \over \partial z}{\partial \xi  \over \partial \rho
}=-{\left({{\partial \eta  \over \partial \rho }}\right)}^{2}-{\left({{\partial
\eta  \over \partial z}}\right)}^{2}=-{\left|{{\partial \chi  \over \partial
\zeta }}\right|}^{2}. \eqn{}
$$
Thus, the critical point in (10.3) is a saddle-point singularity of the
coordinate system.  This singular point will be located somewhere on the
symmetry axis connecting the two throats.  The exact location will be determined
by the choices made for the values of $\eta^\pm$.  In fact, the freedom to choose
$\eta^\pm$ is used to position the singularity at some desired location.

The location of the singular point must be determined numerically.  Having fixed
the coefficients in the definition of (10.3) as describe above, equation (10.11)
can be solved iteratively via Newton's method generalized for complex
functions.  The algorithm is 
$$
{\zeta }_{s}^{i+1}={\zeta }_{s}^{i}-{\left.{{\partial \chi  \over \partial
\zeta }}\right|}_{{\zeta }_{s}^{i}}{\left({{\left.{{{\partial }^{2}\chi  \over
\partial {\zeta }^{2}}}\right|}_{{\zeta }_{s}^{i}}}\right)}^{-1}, \eqn{}
$$
where $\zeta_s$ denotes the singular value of $\zeta$.

%\figlabel{4.125truein}{Figure~10.1:  \v{C}ade\v{z} coordinates near the two holes
%showing important coordinate \cr lines and the {\it three} region nature of the
%coordinate system.}
\figlabelpdf{4.125truein}{Figure~10.1:  \v{C}ade\v{z} coordinates near the two holes
showing important coordinate \cr lines and the {\it three} region nature of the
coordinate system.}{Figures/Figure10_1.pdf}

Expanding (10.2) about the singular point and making use of the Cauchy-Riemann
conditions, the coordinate transformation is given locally by
$$
\eta (\rho ,z)={\eta }_{s}+{{\eta }_{,\rho \rho } \over 2}\left({{\rho
}^{2}-{(z-{z}_{s})}^{2}}\right)\quad\hbox{and}\quad\xi (\rho ,z)={\xi
}_{s}-{\eta }_{,\rho \rho }\rho (z-{z}_{s}). \eqn{}
$$
We see that $\xi=\xi_s$ is a double image along the $z$-axis and so must be
handled carefully.  The singular point effectively divides the \v{C}ade\v{z}
coordinatization into three regions (see Figure~10.1):  one surrounding each of
the two throats and one region surrounding both throats and extending to
infinity.  With $(\eta_s,\xi_s)$ denoting the singular point in \v{C}ade\v{z}
coordinates, the three regions are given explicitly (assuming $C^+_0 + C^-_0 =
1$) by
$$
\hbox{Region~1}\equiv \cases{{\eta }^{+}\le \eta <{\eta }_{s}\cr
0\le \xi \le {\xi }_{s}\cr}, \eqn{a}
$$
$$
\hbox{Region~2}\equiv \cases{{\eta }^{-}\le \eta <{\eta }_{s}\cr
{\xi }_{s}\le \xi \le \pi \cr}, \eqnb{b}
$$
and
$$
\hbox{Region~3}\equiv \cases{{\eta }_{s}\le \eta <\infty \cr
0\le \xi \le \pi \cr}. \eqnb{c}
$$

The presence of this coordinate singularity places one final restriction on the
definition of the \v{C}ade\v{z} coordinates. In order to handle the singular
point numerically, I will demand that it always be represented in the
discretization of the domain.  In order to maintain a constant interval in the
discretization of the $\eta$ coordinate, it is necessary to demand that
$$
{{\eta }_{s}-{\eta }^{+} \over {\eta }_{s}-{\eta }^{\_}}={{\Iind}^{+} \over
{\Iind}^{-}}, \eqn{}
$$
where $\Iind^+$ and $\Iind^-$ are positive integers.  This demand can be
satisfied iteratively by making initial guesses for $\eta^\pm$, solving for the
\v{C}ade\v{z} coefficients, and determining an initial value for $\eta_s$.  Next,
if (10.17) is not satisfied, assume $\eta^+\rightarrow\eta^++\varepsilon$ and
$\eta^-\rightarrow\eta^--\varepsilon$.  Inserting these into (10.17) and solving
for $\varepsilon$ gives a correction for $\eta^\pm$.  The new guesses for
$\eta^\pm$ are used to solve again for the \v{C}ade\v{z} coefficients and for
$\eta_s$ and the iteration is continued until (10.17) is satisfied.  This
procedure has proven to be rapidly convergent.

With the \v{C}ade\v{z} coordinates fully defined, they can now be applied to
the problem of solving the Hamiltonian constraint.  Because of the nature of the
coordinate transformations, all quantities are explicitly functions of the
cylindrical coordinates and are functions of the \v{C}ade\v{z} coordinates
only implicitly through the numerical inverse coordinate transformation.  The
Jacobian of the coordinate transformation from cylindrical to \v{C}ade\v{z}
coordinates is given by
$$
{J}_{i}^{j}=\left[{\matrix{{\eta }_{,\rho }&0&{\eta }_{,z}\cr 0&1&0\cr {\eta
}_{,z}&0&-{\eta }_{,\rho }\cr}}\right] \eqn{}
$$
and the inverse Jacobian by
$$
{\left({{J}^{-1}}\right)}_{i}^{j}=\left[{\matrix{{{\eta }_{,\rho } \over
\left({{\eta }_{,\rho }^{2}+{\eta }_{,z}^{2}}\right)}&0&{{\eta }_{,z} \over
\left({{\eta }_{,\rho }^{2}+{\eta }_{,z}^{2}}\right)}\cr  0&1&0\cr {{\eta }_{,z}
\over \left({{\eta }_{,\rho }^{2}+{\eta }_{,z}^{2}}\right)}&0&{-{\eta }_{,\rho }
\over \left({{\eta }_{,\rho }^{2}+{\eta }_{,z}^{2}}\right)}\cr}}\right]. \eqn{}
$$
The metric is given by
$$
{ds}^{2}={{a}_{1}^{2} \over {\eta }_{,\rho }^{2}+{\eta
}_{,z}^{2}}\left({{d\eta }^{2}+{d\xi }^{2}}\right)+{a}_{1}^{2}{\rho }^{2}{d\phi
}^{2}, \eqn{}
$$
where $\rho$ and $z$ are dimensionless coordinates scaled relative to $a_1$.

The Laplacian can be written in \v{C}ade\v{z} coordinates by applying the
chain rule to the Laplacian in cylindrical coordinates as done by Rauber
[1985].  The Laplacian, in this form, cannot be conservatively differenced which
is a significant disadvantage to finding numerical solutions.  However, by
making use of the explicit form of the inverse Jacobian, the Laplacian can be
rewritten in a form which can be conservatively differenced.  Restricting to the
case of axisymmetry, the Laplacian, written conservatively in \v{C}ade\v{z}
coordinates, takes the form
$$
{\overline{\nabla }}^{2}\psi ={{\eta }_{,\rho }^{2}+{\eta }_{,z}^{2} \over
{a}_{1}^{2}\rho }\left\{{{\partial  \over \partial \eta }\left({\rho {\partial
\psi  \over \partial \eta }}\right)+{\partial  \over \partial \xi }\left({\rho
{\partial \psi  \over \partial \xi }}\right)}\right\}. \eqn{}
$$
The boundary conditions (7.2) on the two inversion boundaries located at $\eta =
\eta^\pm$ take the following forms,
$$
{\left[{\sqrt {{\eta }_{,\rho }^{2}+{\eta }_{,z}^{2}}{\partial \psi  \over
\partial \eta }+{\psi  \over 2}}\right]}_{\eta ={\eta }^{+}}=0 \eqn{a}
$$
and
$$
{\left[{\sqrt {{\eta }_{,\rho }^{2}+{\eta }_{,z}^{2}}{\partial \psi  \over
\partial \eta }+{\alpha \psi  \over 2}}\right]}_{\eta ={\eta }^{-}}=0. \eqnb{b}
$$
On the $z$-axis, axisymmetry must be used to impose a boundary condition on this
coordinate boundary.  The $z$-axis exists on four separate segments in the
\v{C}ade\v{z} coordinate system.  The two boundaries for which $\xi=0$ and
$\xi=\pi$ correspond, respectively, to the $z$-axis extending to infinity from
the right of the right-most hole and to the left of the left-most hole.  The
boundaries in Regions~1 and 2 (10.16a,b) for which $\xi=\xi_s$ and
$\eta\le\eta_s$ also correspond to the $z$-axis located between the holes.  On
these boundaries, axisymmetry demands that
$$
{\left.{{\partial \psi  \over \partial \xi }}\right|}_{\rho =0}=0.\eqn{}
$$
As usual, a limiting form of the Laplacian must be found along the $\rho=0$
axis.  Using the axisymmetry boundary condition (10.23) and L'H\^{o}pital's rule,
the Laplacian takes the form
$$
{\left.{{\overline{\nabla }}^{2}\psi }\right|}_{\rho =0}={{\eta }_{,z}^{2}
\over {a}_{1}^{2}}\left\{{{{\partial }^{2}\psi  \over {\partial \eta
}^{2}}+2{{\partial }^{2}\psi  \over {\partial \xi }^{2}}}\right\}+{{\eta
}_{,\rho \rho } \over {a}_{1}^{2}}{\partial \psi  \over \partial \eta }.
\eqn{}
$$

The outer boundary of the domain in which the Hamiltonian constraint will be
found does not conveniently extend to infinity where the boundary condition of
asymptotic flatness is naturally applied.  Instead, the approximate outer
boundary condition (7.27) is applied at a finite but large radius.  For the case
of two black holes with axisymmetric linear momentum, and assuming gravitational
units, (7.27) takes the form
$$
{\partial \psi  \over \partial r}={1-\psi  \over r}-{1 \over
2{r}^{3}}\left\{{{d}_{z}\cos\theta +{9 \over
16}{\left({{P}_{1}+{P}_{2}}\right)}^{2}\cos 2\theta }\right\}+{1 \over
r}\Order({r}^{-3}). \eqn{}
$$
While \v{C}ade\v{z} coordinates do approach spherical coordinates at large
distances from the holes, they are not exactly spherical coordinates and so the
radial derivative must be handled carefully.  The radial derivative can be
written in terms of \v{C}ade\v{z} coordinates by the use of the chain rule
as follows:
$$
\eqalign{{\partial  \over \partial r}&=\cos\theta {\partial  \over \partial
z}+\sin\theta {\partial  \over \partial \rho }\cr &={\left({z{\eta }_{,z}+\rho
{\eta }_{,\rho }}\right) \over r}{\partial  \over \partial \eta }-{\left({z{\eta
}_{,\rho }-\rho {\eta }_{,z}}\right) \over r}{\partial  \over \partial \xi
}.\cr} \eqn{}
$$
The outer boundary condition (10.25) can thus be written as
$$
\left({z{\eta }_{,z}+\rho {\eta }_{,\rho }}\right){\partial \psi  \over
\partial \eta }-\left({z{\eta }_{,\rho }-\rho {\eta }_{,z}}\right){\partial
\psi  \over \partial \xi }+{{d}_{z}z \over 2{\left({{\rho
}^{2}+{z}^{2}}\right)}^{3/2}}+{9 \over
32}{\left({{P}_{1}+{P}_{2}}\right)}^{2}{\left({{z}^{2}-{\rho }^{2}}\right) \over
{\left({{\rho }^{2}+{z}^{2}}\right)}^{2}}=0, \eqn{}
$$
and is applied at an outer boundary where $\eta=\eta_f$.  As discussed in
Chapter~7, this boundary condition must be applied iteratively since it depends
on the dipole moment $d_z$ which must be determined from the solution.

The differencing of the Hamiltonian constraint is complicated by the
discretization of the computational domain which, as mentioned above, is
naturally divided into three regions.  The $\eta$ coordinate is discretized as 
$$
{\eta }_{i}=i*{h}_{\eta }+{\eta
}_{s}\qquad\hbox{where}\qquad\cases{i=-{\Iind}^{+},\ldots,-1&: Region~1\cr
i=-{\Iind}^{-},\ldots,-1&: Region~2\cr i=0,\ldots,\Iind&: Region~3\cr} \eqn{}
$$
and
$$
{h}_{\eta }={{\eta }_{f}-{\eta }_{s} \over \Iind}={{\eta }_{s}-{\eta }^{+} \over
{\Iind}^{+}}={{\eta }_{s}-{\eta }^{-} \over {\Iind}^{-}}. \eqn{}
$$
The discretization of the $\xi$ coordinate cannot be done directly.  The singular
value of $\xi$ must always occur in the discretization and this will not, in
general, be compatible with a uniform discretization.  This situation is
analogous to that already dealt with for the discretization of the $\eta$
coordinate in bispherical coordinates.  I reparameterize $\xi$ in terms of an
auxiliary variable $s$ so that
$$
\xi =\xi (s)=f*{s}^{2}+g*s+{\xi }_{s}. \eqn{}
$$
The new coordinate $s$ is then discretized as follows
$$
{s}_{j}=j*{h}_{s}\qquad\hbox{where}\qquad
j=-{\Sind}^{+},\ldots,0,\ldots,{\Sind}^{-} \eqn{}
$$
and
$$
{h}_{s}={1 \over \Sind}\qquad\hbox{where}\qquad\Sind={\Sind}^{+}+{\Sind}^{-}.
\eqn{}
$$
Together, (10.31) and (10.32) guarantee that $\xi(0) = \xi_s$ is in the discrete
domain.  The two constants $f$ and $g$ are chosen so that the end points of the
range of $s$ give the correct coordinate boundaries for $\xi$.  Let the upper and
lower limits on the range of $s$ be defined as
$$
{s}^{-}\equiv {\Sind}^{-}*{h}_{s}={\Sind}^{-}/\Sind \eqn{}
$$
and
$$
{s}^{+}\equiv -{\Sind}^{+}*{h}_{s}=-{\Sind}^{+}/\Sind. \eqn{}
$$
The two constants $f$ and $g$ are fixed by the two equations
$$
\xi ({s}^{+})=0\qquad\hbox{and}\qquad\xi ({s}^{-})=\pi. \eqn{}
$$
The result is
$$
f={\Sind \over {\Sind}^{-}}\pi -{{\Sind}^{2} \over {\Sind}^{+}{\Sind}^{-}}{\xi
}_{s} \eqn{}
$$
and
$$
g={{\Sind}^{+} \over {\Sind}^{-}}\pi
+{\Sind\left({{\Sind}^{-}-{\Sind}^{+}}\right) \over {\Sind}^{+}{\Sind}^{-}}{\xi
}_{s}. \eqn{}
$$

As was the case for bispherical coordinates, the Jacobian of the
reparameterization is given by
$$
J(s)\equiv {\partial \xi (s) \over \partial s}=2f*s+g, \eqn{}
$$
and $\xi$ will be monotonically increasing as $s$ increases if and only if
$$
J({s}^{+}-{h}_{s})>0\qquad\hbox{and}\qquad J({s}^{-}+{h}_{s})>0. \eqn{}
$$
From (10.39), a sufficient condition for the coordinate system to be well
behaved is
$$
{\Sind\left({2+\Sind}\right) \over
{\Sind}^{+}\left({2+\Sind+{\Sind}^{-}}\right)}<{\pi  \over {\xi
}_{s}}<{\Sind\left({2+\Sind}\right) \over
{\Sind}^{+}\left({2+{\Sind}^{+}}\right)}. \eqn{}
$$

In practice, one chooses values for $\alpha$ and $\beta$ to fix the configuration
and also a value for the ratio $\Iind^+/\Iind^-$ (see (10.17)) to position the
singular point.  Next, one chooses values for $\Sind^+$ and $\Sind^-$ so that
(10.40) is satisfied.  The discretization of the $(\eta,s)$-domain is then given
by (10.28) and (10.31).  The second order, conservative finite difference form of
the Hamiltonian constraint is expressed (except at the singular point $i=j=0$) as
$$
\eqalign{{\Ac}_{i,j}^{+}\left({{\psi }_{i+1,j}-{\psi
}_{i,j}}\right)&+{\Ac}_{i,j}^{-}\left({{\psi }_{i-1,j}-{\psi
}_{i,j}}\right)+{\Bc}_{i,j}^{+}\left({{\psi }_{i,j+1}-{\psi
}_{i,j}}\right)\cr&+{\Bc}_{i,j}^{-}\left({{\psi }_{i,j-1}-{\psi }_{i,j}}\right)
+{1 \over 8}{\psi }_{i,j}^{-7}{\left[{{a}_{1}^{2}{\bar{A}}{}_{\ell
}{}_{m}{\bar{A}}{}^{\ell }{}^{m}}\right]}_{i,j}=0,\cr} \eqn{}
$$
where
$$
{\Ac}_{i,j}^{\pm }\equiv \cases{{\left({{\left.{{\eta }_{,\rho
}^{2}}\right|}_{\left({{\eta }_{i},\xi ({s}_{j})}\right)}+{\left.{{\eta
}_{,z}^{2}}\right|}_{\left({{\eta }_{i},\xi ({s}_{j})}\right)}}\right) \over
{h}_{\eta }^{2}{\rho }_{\left({{\eta }_{i},\xi ({s}_{j})}\right)}}{\rho
}_{\left({{\eta }_{i}\pm {h}_{\eta }/2,\xi ({s}_{j})}\right)}&:$\hbox{for all
$(i,j)$ except $i=j=0$;} \atop \hfill{\hbox{or all $i$ and
$j=-{\Sind}^{+},{\Sind}^{-}$;} \atop \hfill\hbox{or $i<0$ and $j=0$}}$\cr
{{\left.{{\eta }_{,z}^{2}}\right|}_{\left({{\eta }_{i},\xi ({s}_{j})}\right)}
\over {h}_{\eta }^{2}}\pm {{\left.{{\eta }_{,\rho \rho }}\right|}_{\left({{\eta
}_{i},\xi ({s}_{j})}\right)} \over {2h}_{\eta }}&:$\hbox{for all $i$ and
$j=-{\Sind}^{+},{\Sind}^{-}$;} \atop \hfill\hbox{and $i<0$ and $j=0$}$\cr} \eqn{}
$$
and
$$
{\Bc}_{i,j}^{\pm }\equiv \cases{{\left({{\left.{{\eta }_{,\rho
}^{2}}\right|}_{\left({{\eta }_{i},\xi ({s}_{j})}\right)}+{\left.{{\eta
}_{,z}^{2}}\right|}_{\left({{\eta }_{i},\xi ({s}_{j})}\right)}}\right) \over
{h}_{s}^{2}J({s}_{j}){\rho }_{\left({{\eta }_{i},\xi ({s}_{j})}\right)}}{{\rho
}_{\left({{\eta }_{i},\xi ({s}_{j}\pm {h}_{s}/2)}\right)} \over J({s}_{j}\pm
{h}_{s}/2)}&:$\hbox{for all $(i,j)$ except $i=j=0$;} \atop \hfill{\hbox{or all
$i$ and $j=-{\Sind}^{+},{\Sind}^{-}$;} \atop \hfill\hbox{or $i<0$ and
$j=0$}}$\cr {\left({2{\left.{{\eta }_{,z}^{2}}\right|}_{\left({{\eta }_{i},\xi
({s}_{j})}\right)}}\right) \over {h}_{s}^{2}J({s}_{j})}{1 \over J({s}_{j}\pm
{h}_{s}/2)}&:$\hbox{for all $i$ and $j=-{\Sind}^{+},{\Sind}^{-}$;} \atop
\hfill\hbox{and $i<0$ and $j=0$}$.\cr} \eqn{}
$$
The differencing of the Laplacian at the singular point must be handled as a
special case and will be discussed below.

The boundary conditions will be handled through the {\it implicit} use of
``virtual'' boundary points as in the case of bispherical coordinates.  The
finite difference form for the  inner boundary condition for Region~1 (10.22a),
put in a form which is most convenient for eliminating the dependence on the
virtual boundary points, is given by
$$
\left({{\psi }_{-{\Iind}^{+}-1,j}-{\psi }_{-{\Iind}^{+},j}}\right)=\left({{\psi
}_{-{\Iind}^{+}+1,j}-{\psi }_{-{\Iind}^{+},j}}\right)+{{h}_{\eta }{\psi
}_{-{\Iind}^{+},j} \over \sqrt {{\left.{{\eta }_{,\rho
}^{2}}\right|}_{\left({{\eta }^{+},\xi ({s}_{j})}\right)}+{\left.{{\eta
}_{,z}^{2}}\right|}_{\left({{\eta }^{+},\xi ({s}_{j})}\right)}}}, \eqn{}
$$
and the inner boundary condition for Region~2 (10.22b) is given by
$$
\left({{\psi }_{-{\Iind}^{-}-1,j}-{\psi }_{-{\Iind}^{-},j}}\right)=\left({{\psi
}_{-{\Iind}^{-}+1,j}-{\psi }_{-{\Iind}^{-},j}}\right)+{\alpha {h}_{\eta }{\psi
}_{-{\Iind}^{-},j} \over \sqrt {{\left.{{\eta }_{,\rho
}^{2}}\right|}_{\left({{\eta }^{-},\xi ({s}_{j})}\right)}+{\left.{{\eta
}_{,z}^{2}}\right|}_{\left({{\eta }^{-},\xi ({s}_{j})}\right)}}}. \eqn{}
$$
The finite difference form for the outer boundary condition of Region 3 (10.27)
is expressed as
$$
\eqalign{({\psi }_{\Iind+1,j}&-{\psi }_{\Iind,j})=\left({{\psi
}_{\Iind-1,j}-{\psi }_{\Iind,j}}\right)+{2{h}_{\eta }\left({1-{\psi
}_{\Iind,j}}\right) \over {\left({z{\eta }_{,z}+\rho {\eta }_{,\rho
}}\right)}_{\left({{\eta }_{f},\xi ({s}_{j})}\right)}}\cr&+{{h}_{\eta } \over
{h}_{s}J({s}_{j})}{\left({{z{\eta }_{,\rho }-\rho {\eta }_{,z} \over z{\eta
}_{,z}+\rho {\eta }_{,\rho }}}\right)}_{\left({{\eta }_{f},\xi
({s}_{j})}\right)}\left({\left({{\psi }_{\Iind,j+1}-{\psi
}_{\Iind,j}}\right)-\left({{\psi }_{\Iind,j-1}-{\psi
}_{\Iind,j}}\right)}\right)\cr&-{d}_{z}{\left({{{h}_{\eta }z{\left({{\rho
}^{2}+{z}^{2}}\right)}^{-3/2} \over z{\eta }_{,z}+\rho {\eta }_{,\rho
}}}\right)}_{\left({{\eta }_{f},\xi
({s}_{j})}\right)}\cr&-{9{\left({{P}_{1}+{P}_{2}}\right)}^{2} \over
16}{\left({{{h}_{\eta }\left({{z}^{2}-{\rho }^{2}}\right){\left({{\rho
}^{2}+{z}^{2}}\right)}^{-2} \over z{\eta }_{,z}+\rho {\eta }_{,\rho
}}}\right)}_{\left({{\eta }_{f},\xi ({s}_{j})}\right)}.\cr} \eqn{} $$
The axisymmetry boundary conditions (10.23) give
$$
\left({{\psi }_{i,-{\Sind}^{+}-1}-{\psi }_{i,-{\Sind}^{+}}}\right)=\left({{\psi
}_{i,-{\Sind}^{+}+1}-{\psi }_{i,-{\Sind}^{+}}}\right), \eqn{}
$$
$$
\left({{\psi }_{i,{\Sind}^{-}+1}-{\psi }_{i,{\Sind}^{-}}}\right)=\left({{\psi
}_{i,{\Sind}^{-}-1}-{\psi }_{i,{\Sind}^{-}}}\right), \eqn{}
$$
$$
\left({{\psi }_{i,1}-{\psi }_{i,0}}\right)=\left({{\psi }_{i,-1}-{\psi
}_{i,0}}\right)\qquad\hbox{for}\quad i<0 \hbox{ in Region~1}, \eqn{}
$$
and
$$
\left({{\psi }_{i,-1}-{\psi }_{i,0}}\right)=\left({{\psi }_{i,1}-{\psi
}_{i,0}}\right)\qquad\hbox{for}\quad i<0 \hbox{ in Region~2}. \eqn{}
$$

As mentioned above, the differencing of the Laplacian at the singular point
cannot be handled in the usual manner since even the limiting behavior of the
Laplacian, as the singular point is approached, is not well defined.  One
approach to differencing the Laplacian at the singular point is obtained by
considering the mesh in the neighborhood of the singular point not as a
regularly spaced mesh in \v{C}ade\v{z} coordinates, but rather as an
irregularly spaced mesh in cylindrical coordinates.  For an axisymmetric
configuration, the Laplacian at the singular point takes the cylindrical
coordinate form of
$$
{\left.{{\overline{\nabla }}^{2}\psi }\right|}_{\left({\rho
=0,z={z}_{s}}\right)}=2{{\partial }^{2}\psi  \over {\partial \rho
}^{2}}+{{\partial }^{2}\psi  \over {\partial z}^{2}}. \eqn{}
$$
To express (10.51) by finite differences, discrete approximations to both
second derivatives must be found which can be constructed from the irregular
distribution of mesh point in the cylindrical coordinate system.

Assuming axisymmetry in the conformal factor, the general Taylor series
expansion about the singular point takes the form
$$
\eqalign{\psi ({z}_{s}+h,k)={\psi }_{s}+{\psi }_{,z}h&+{1 \over 2}{\psi
}_{,zz}{h}^{2}+{1 \over 2}{\psi }_{,\rho \rho }{k}^{2}+{1 \over 2}{\psi
}_{,z\rho \rho }h{k}^{2}+{1 \over 6}{\psi }_{,zzz}{h}^{3}\cr&+\left\{{{1 \over
4}{\psi }_{,zz\rho \rho }{h}^{2}{k}^{2}+{1 \over 24}{\psi }_{,zzzz}{h}^{4}+{1
\over 24}{\psi }_{,\rho \rho \rho \rho }{k}^{4}+\cdots}\right\}.\cr} \eqn{}
$$
If this series is truncated at some level and the difference between the value
of the conformal factor at the singular point and a sufficient number of
neighboring points are taken in (10.52), then the result is a matrix problem for
the derivatives of the conformal factor at the singular point.  In order for the
truncation error in the approximations of the second derivatives to be second
order in $h$ and $k$, all of the terms not bracketed in (10.52) must be included
in the expansion.  This means that five points neighboring the singular point
must be used.  If there happens to be a reflection symmetry in the coordinate
system in the $z$ direction through the singular point, then three of the terms
in (10.52) will vanish identically.  In order for the matrix constructed from
(10.52) not to be singular in this case, at least three pairs of points must be
used for which $h\not=0$.  This means that at least six points must be used and
thus at least one of the fourth order terms in (10.52) must be included. 
Finally, in order to symmetrize the differencing scheme as much as possible, I
have chosen to include one extra point and a second fourth order term from
(10.52).  The resulting eight-point differencing molecule is illustrated below
in Figure~10.2.

%\figlabel{2.125truein}{Figure~10.2:  Eight-point differencing molecule used for
%singular point.}
\figlabelpdf{2.125truein}{Figure~10.2:  Eight-point differencing molecule used for
singular point.}{Figures/Figure10_2.pdf}

Using the labeling in Figure~10.2, the matrix problem for the derivatives becomes
$$
\left[{\matrix{{h}_{1}&{h}_{1}^{2}&0&0&{h}_{1}^{3}&{h}_{1}^{4}&0\cr
{h}_{2}&{h}_{2}^{2}&0&0&{h}_{2}^{3}&{h}_{2}^{4}&0\cr
{h}_{3}&{h}_{3}^{2}&{k}_{3}^{2}&{h}_{3}{k}_{3}^{2}&{h}_{3}^{3}&{h}_{3}^{4}&{k}_{3}^{4}\cr
{h}_{4}&{h}_{4}^{2}&{k}_{4}^{2}&{h}_{4}{k}_{4}^{2}&{h}_{4}^{3}&{h}_{4}^{4}&{k}_{4}^{4}\cr
{h}_{5}&{h}_{5}^{2}&{k}_{5}^{2}&{h}_{5}{k}_{5}^{2}&{h}_{5}^{3}&{h}_{5}^{4}&{k}_{5}^{4}\cr
{h}_{6}&{h}_{6}^{2}&{k}_{6}^{2}&{h}_{6}{k}_{6}^{2}&{h}_{6}^{3}&{h}_{6}^{4}&{k}_{6}^{4}\cr
{h}_{7}&{h}_{7}^{2}&{k}_{7}^{2}&{h}_{7}{k}_{7}^{2}&{h}_{7}^{3}&{h}_{7}^{4}&{k}_{7}^{4}\cr}}
\right]\left[{\matrix{{\psi }_{,z}\cr {{1 \over 2}\psi }_{,zz}\cr {{1 \over
2}\psi }_{,\rho \rho }\cr {{1 \over 2}\psi }_{,z\rho \rho }\cr {{1 \over 6}\psi
}_{,zzz}\cr {{1 \over 24}\psi }_{,zzzz}\cr {{1 \over 24}\psi }_{,\rho \rho \rho
\rho }\cr}}\right]=\left[{\matrix{{\psi }_{1}-{\psi }_{0}-\Order({h}_{1}^{5})\cr
{\psi }_{2}-{\psi }_{0}-\Order({h}_{2}^{5})\cr {\psi }_{3}-{\psi
}_{0}-\Order({h}_{3}^{2}{k}_{3}^{2})\cr {\psi }_{4}-{\psi
}_{0}-\Order({h}_{4}^{2}{k}_{4}^{2})\cr {\psi }_{5}-{\psi
}_{0}-\Order({h}_{5}^{2}{k}_{5}^{2})\cr {\psi }_{6}-{\psi
}_{0}-\Order({h}_{6}^{2}{k}_{6}^{2})\cr {\psi }_{7}-{\psi
}_{0}-\Order({h}_{7}^{2}{k}_{7}^{2})\cr}}\right], \eqn{}
$$
where
$$
{h}_{i}\equiv {z}_{i}-{z}_{0}\qquad\hbox{and}\qquad{k}_{i}\equiv {\rho }_{i}.
\eqn{}
$$
Having chosen a discretization for the domain of the problem being solved, all
quantities in the matrix of (10.53) are fixed and the inverse matrix can be
computed numerically.  If the elements of the inverse matrix are denoted by
$M^{-1}_{i,j}$, then the differenced form of the Laplacian at the singular
point can be written as
$$
{\overline{\nabla }}^{2}\psi =\sum\nolimits\limits_{i=1}^{7}
\left\{{\left({2{M}_{2,i}^{-1}+4{M}_{3,i}^{-1}}\right)\left({{\psi }_{i}-{\psi
}_{0}}\right)}\right\}-\sum\nolimits\limits_{i=3}^{7}
\left\{{\left({2{M}_{2,i}^{-1}+4{M}_{3,i}^{-1}}\right)\Order({h}_{i}^{2}{k}_{i}^{2})}\right\}.
\eqn{}
$$
The overall truncation error to this scheme is $\Order(h^2 + k^2 + hk)$ and so
the rate of decrease in the error is quadratic as $h$ and $k$ decrease.  This
does not, however, accurately describe the reduction in the truncation error as
the discretization size in \v{C}ade\v{z} coordinates is reduced.  If the
discretization of the \v{C}ade\v{z} coordinates is fine enough so that the
local expansion of the coordinate transformation (10.15) is valid, then it is
seen that the reduction in the cylindrical discretization length goes as the
square root of the reduction of the \v{C}ade\v{z} discretization length. 
This means that the limiting behavior of the truncation error of (10.55) will be
first-order or linear, not quadratic.  As in the case of bispherical
coordinates, the presence of this sort of coordinate singularity leads to
first-order behavior for the discretization error.

The difference equations described above for the Hamiltonian constraint in
\v{C}ade\v{z} coordinates are solved via the Newton-Raphson method for
nonlinear systems as described in Chapter~8.  As with the bispherical
coordinates, the linearized equations for the iterative correction have been
solved directly by means of the LINPACK routines for factoring (DGBFA) and
solving (DGBSL) general banded matrices.  Because of the multiple region nature
of the \v{C}ade\v{z} coordinate system, the number of bands and the maximum
separation of the bands from the diagonal of the matrix are larger than in the
case of bispherical coordinates.  The effect of this is that the size of the
matrix which can be solved with these LINPACK routines is substantially less
than in the case of bispherical coordinates.  This means that the discretization
of the domain cannot be done as finely as is possible with bispherical
coordinates and, consequently, the accuracy of the solution will not be as
great.  An alternate approach for solving the \v{C}ade\v{z} coordinate
difference equations would be of great help in this regard.  Some work has been
done on applying the multigrid approach to this problem, but with only modest
success.  It is likely that the multiple region nature of the domain and
difficulties associated with the coordinate singularity are responsible for the
less than satisfactory behavior of the multigrid solver written for this
problem.  A detailed investigation of the applicability of the multigrid method
to the \v{C}ade\v{z} coordinate system will be necessary in order to advance
this approach.

Given a solution to the difference equations, the total ADM energy and
dipole moment as well as the minimal surface masses and the physical hole
separations can be computed via numerical integrals.  As in the case of
bispherical coordinates, care must be taken in numerically integrating near the
singular point.  Extrapolative integral formulas have been used unless otherwise
stated.  In terms of \v{C}ade\v{z} coordinates, the total energy is given in
dimensionless form by
$$
\eqalign{{E \over {a}_{1}}={1 \over 2}&\int_{0}^{{\xi }_{s}}{\left.{{\psi \rho
d\xi  \over \sqrt {{\eta }_{,\rho }^{2}+{\eta }_{,z}^{2}}}}\right|}_{\left({\eta
={\eta }^{+}}\right)}+{\alpha  \over 2}\int_{{\xi }_{s}}^{\pi }{\left.{{\psi
\rho d\xi  \over \sqrt {{\eta }_{,\rho }^{2}+{\eta
}_{,z}^{2}}}}\right|}_{\left({\eta ={\eta }^{-}}\right)}\cr&+{1 \over
8}\int\limits_{\xi=0}^{\xi ={\xi }_{s}}{\int\limits_{\eta ={\eta }^{+}}^{\eta
={\eta}_{s}}{\left({{a}_{1}^{2}{\bar{A}}{}_{i}{}_{j}{\bar{A}}{}^{i}{}^{j}}\right)
\over {\psi }^{7}}{\rho d\eta d\xi  \over {\eta }_{,\rho }^{2}+{\eta
}_{,z}^{2}}} +{1 \over 8}\int\limits_{\xi ={\xi }_{s}}^{\xi
=\pi}{\int\limits_{\eta ={\eta
}^{-}}^{\eta={\eta}_{s}}{\left({{a}_{1}^{2}{\bar{A}}{}_{i}{}_{j}{\bar{A}}{}^{i}{}^{j}}\right)
\over {\psi }^{7}}{\rho d\eta d\xi  \over {\eta }_{,\rho
}^{2}+{\eta}_{,z}^{2}}}\cr&+{1 \over 8}\int\limits_{\xi =0}^{\xi
=\pi}{\int\limits_{\eta ={\eta }_{s}}^{\eta =\infty
}{\left({{a}_{1}^{2}{\bar{A}}{}_{i}{}_{j}{\bar{A}}{}^{i}{}^{j}}\right) \over
{\psi }^{7}}{\rho d\eta d\xi  \over {\eta }_{,\rho }^{2}+{\eta
}_{,z}^{2}}}.\cr} \eqn{}
$$
The $z$ component of the dipole moment is given in dimensionless form
by
$$
\eqalign{{{d}_{z} \over {a}_{1}^{2}}=\int\limits_{0}^{{\xi
}_{s}}{\left\{{\!\!{z\psi  \over 2}+{{\eta }_{,z}(1-\psi ) \over \sqrt {{\eta
}_{,\rho }^{2}+{\eta }_{,z}^{2}}}}\right\}\!\!\left.{{\rho d\xi  \over \sqrt
{{\eta }_{,\rho }^{2}+{\eta }_{,z}^{2}}}}\right|}_{\left({\eta ={\eta
}^{+}}\right)}&\!\!\!+{1 \over 8}\int\limits_{\xi =0}^{\xi ={\xi
}_{s}}{\int\limits_{\eta ={\eta }^{+}}^{\eta
={\eta}_{s}}{\left({{a}_{1}^{2}{\bar{A}}{}_{i}{}_{j}{\bar{A}}{}^{i}{}^{j}}\right)
\over {\psi }^{7}}{z\rho d\eta d\xi  \over {\eta }_{,\rho }^{2}+{\eta
}_{,z}^{2}}}\cr +\int\limits_{{\xi }_{s}}^{\pi }{\left\{{\!\!{\alpha
z\psi  \over 2}+{{\eta }_{,z}(1-\psi ) \over \sqrt {{\eta }_{,\rho }^{2}+{\eta
}_{,z}^{2}}}}\right\}\!\!\left.{{\rho d\xi  \over \sqrt {{\eta }_{,\rho
}^{2}+{\eta }_{,z}^{2}}}}\right|}_{\left({\eta ={\eta
}^{-}}\right)}&\!\!\!+{1 \over 8}\int\limits_{\xi ={\xi }_{s}}^{\xi
=\pi}{\int\limits_{\eta ={\eta }^{-}}^{\eta ={\eta
}_{s}}{\left({{a}_{1}^{2}{\bar{A}}{}_{i}{}_{j}{\bar{A}}{}^{i}{}^{j}}\right)
\over {\psi }^{7}}{z\rho d\eta d\xi  \over {\eta }_{,\rho }^{2}+{\eta
}_{,z}^{2}}}\cr&\!\!\!+{1 \over 8}\int\limits_{\xi =0}^{\xi
=\pi}{\int\limits_{\eta ={\eta }_{s}}^{\eta =\infty
}{\left({{a}_{1}^{2}{\bar{A}}{}_{i}{}_{j}{\bar{A}}{}^{i}{}^{j}}\right) \over
{\psi }^{7}}{z\rho d\eta d\xi  \over {\eta }_{,\rho }^{2}+{\eta
}_{,z}^{2}}}.\cr} \eqn{} $$
The areas of the minimal surfaces are given in dimensionless form by
$$
{{A}_{MS}^{+} \over {a}_{1}^{2}}=2\pi \int_{0}^{{\xi }_{s}}{\left.{{{\psi
}^{4}\rho d\xi  \over \sqrt {{\eta }_{,\rho }^{2}+{\eta
}_{,z}^{2}}}}\right|}_{\left({\eta ={\eta }^{+}}\right)} \eqn{a}
$$
and
$$
{{A}_{MS}^{-} \over {a}_{1}^{2}}=2\pi \int_{{\xi }_{s}}^{\pi
}{\left.{{{\psi }^{4}\rho d\xi  \over \sqrt {{\eta }_{,\rho }^{2}+{\eta
}_{,z}^{2}}}}\right|}_{\left({\eta ={\eta }^{-}}\right)}. \eqnb{b}
$$

Finally, the proper separation of the two holes is given in dimensionless form by
$$
{L \over {a}_{1}}=-\int_{{\eta }^{+}}^{{\eta }_{s}}{\left.{{{\psi }^{2}d\eta 
\over {\eta }_{,z}}}\right|}_{\left({\xi ={\xi }_{s}}\right)}+\int_{{\eta
}^{-}}^{{\eta }_{s}}{\left.{{{\psi }^{2}d\eta  \over {\eta
}_{,z}}}\right|}_{\left({\xi ={\xi }_{s}}\right)}. \eqn{}
$$
The numerical approximation of (10.59) tends to be a poor approximation because
of the relative sparseness of mesh points between the two holes.  The accuracy
of the approximation can be greatly increased by replacing (10.59) with an
integral over three separate regions.  (10.59) consists of an integral along the
$z$-axis between the holes in Regions~1 and 2.  The third integral is taken to be
from one mesh point to the left of the singular point in Region~2 to one mesh
point to the right of the singular point in Region~1.  The two integrals in
(10.59) are adjusted so as not to include this section.  The value of the
conformal factor in the intermediate section can be approximated by a quadratic
polynomial fit to the three values of the conformal factor at the singular point
and its nearest neighbors.  This interpolated conformal factor can then be
integrated analytically in cylindrical coordinates giving a more accurate result
than can be achieved by a standard, extrapolative numerical integral.  Using the
notation of Figure~10.2, the contribution of this section to the total proper
separation is given by
$$
\eqalign{{\bar{L} \over {a}_{1}}={\psi }_{1}^{2}{({z}_{1}-{z}_{2}) \over
6({z}_{1}-{z}_{0})}(2{z}_{1}-3{z}_{0}+{z}_{2})&+{\psi
}_{0}^{2}{{({z}_{1}-{z}_{2})}^{3} \over 6({z}_{1}-{z}_{0})({z}_{0}-{z}_{2})}\cr
&-{\psi }_{2}^{2}{({z}_{1}-{z}_{2}) \over
6({z}_{0}-{z}_{2})}({z}_{1}-3{z}_{0}+2{z}_{2}).\cr} \eqn{}
$$

Because the numerical integrals for the total energy and dipole moment cannot
extend past the outer boundary of the computational domain, the integrals
(10.56) and (10.57) necessarily incorporate error beyond that introduced by the
numerical approximations.  An understanding of the nature of this error as well
as a first-order correction to the numerical result can be obtained by examining
the asymptotic forms of the integrands of (10.56) and (10.57) and estimating the
contribution from the integral over the domain exterior to the outer boundary.

For the case of two holes with axisymmetric linear and angular momenta, we find
(see (7.22)) that at large radius the square of the extrinsic curvature takes
the dimensionless form
$$
{a}_{1}^{2}{\bar{A}}{}_{i}{}_{j}{\bar{A}}{}^{i}{}^{j}\approx {9
\over
2{r}^{4}}{\left({{P}_{1}/{a}_{1}+{P}_{2}/{a}_{1}}\right)}^{2}\left({1+2{\cos}^{2}\theta
}\right). \eqn{}
$$
(Note that $r$ is dimensionless.)  Similarly, from
(7.26) we find for the conformal factor
$$
\psi \approx 1+{E/{a}_{1} \over 2r}+{{d}_{z}/{a}_{1}^{2} \over
2{r}^{2}}\cos\theta -{9 \over 32}{{({P}_{1}/{a}_{1}+{P}_{2}/{a}_{1})}^{2} \over
{r}^{2}}(1-2{\cos}^{2}\theta ). \eqn{}
$$
Using (10.61) and (10.62), the leading order contributions to the total
energy and dipole moment, resulting from the integral over the region exterior
to a sphere of radius $r_b$, are
$$
\Delta E/{a}_{1}\approx {15 \over 8}{{({P}_{1}/{a}_{1}+{P}_{2}/{a}_{1})}^{2}
\over {r}_{b}}-{105 \over 32}{{({P}_{1}/{a}_{1}+{P}_{2}/{a}_{1})}^{2} \over
{r}_{b}^{2}}{E \over {a}_{1}} \eqn{}
$$
and
$$
\Delta {d}_{z}/{a}_{1}^{2}\approx -{231 \over
160}{{({P}_{1}/{a}_{1}+{P}_{2}/{a}_{1})}^{2} \over {r}_{b}^{2}}{{d}_{z} \over
{a}_{1}^{2}}. \eqn{}
$$

If results of the numerical integration over the finite domain for the total
energy and dipole moment are denoted respectively by $E_N/a_1$ and
$d_N/a_1^2$, then first-order corrected values are given by
$$
{E \over {a}_{1}}\approx {{E}_{N}/{a}_{1}+{15 \over
8}{{({P}_{1}/{a}_{1}+{P}_{2}/{a}_{1})}^{2} \over {r}_{b}} \over 1+{105 \over
32}{{({P}_{1}/{a}_{1}+{P}_{2}/{a}_{1})}^{2} \over {r}_{b}^{2}}} \eqn{}
$$
and
$$
{{d}_{z} \over {a}_{1}^{2}}\approx {{d}_{N}/{a}_{1}^{2} \over 1+{231 \over
160}{{({P}_{1}/{a}_{1}+{P}_{2}/{a}_{1})}^{2} \over {r}_{b}^{2}}}. \eqn{}
$$
It is easy to see from (10.65) and (10.66) that configurations with a
non-vanishing net linear momentum will suffer the worst errors from the
approximation of a finite computational domain and this error can, for the most
part, be compensated for by the use of (10.65) and (10.66).

Appendix~D contains the results from computational solutions covering a large set
of configurations and grid resolutions.  Each of these computations, performed in
\v{C}ade\v{z} coordinates, can be directly compared to an identical computation
performed in bispherical coordinates and tabulated in Appendix~C.  Because of
the behavior of the discretization between the holes, configurations with small
$\beta$ require very fine discretizations in Region~3 in order to have a
moderate discretization in Regions~1 and 2.  The end result of this is that
with the current solution method, only very coarse grids can be used when
$\beta$ is small.  Consequently, the accuracy of the solution is rather poor in
these cases.  For the case of $\beta=3$, the errors in the total energy where
on the order of $20\%$ and the errors in the separation were roughly $30\%$. 
This is not at all surprising considering that only 12 discrete zones were
present between the holes while in bispherical coordinates, 160 where used in
the finest discretization.  On the other hand, when $\beta$ reached 6, the errors
were down to less than $1\%$ when there was no net momentum and to $5\%$ when
there was.  This is an encouraging level of error for the cases with no net
momenta.  The higher error in the cases with a net momentum indicate that the
outer boundary, placed at radius of roughly $200a_1$, was somewhat too close.

These preliminary solutions using \v{C}ade\v{z} coordinates, while less accurate
than those obtained with bispherical coordinates, show that there is no
fundamental problem with the use of \v{C}ade\v{z} coordinates for work on the
two-hole initial-value and evolution problems.  The differencing scheme
discussed above appears to work adequately.  The remaining difficuly lies in
finding computational routines for solving the large, poorly banded matrices
which result from the differencing scheme.

\vfill
\eject

