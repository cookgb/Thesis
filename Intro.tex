\chapterhead{Chapter~{\the\chapnum}:  Introduction}
%%%%%%%%%%%%%%%
The most widely accepted theory of the dynamics of classical gravitational
interactions is the theory of general relativity.\footnote*{Here, the term
classical gravitation is meant to imply that in any problem to which general
relativity is applied; the length, energy, and time scales are in the classical
and not in the quantum regime.}  Proposed in 1916 by Albert Einstein, it
forever changed our understanding of what gravity is and how it shapes our
universe.  The Newtonian view of gravity held it as a force which acted through
space and affected the course of matter through time.  General relativity, on
the other hand, intimately links gravity with space and time by considering our
universe as a four-dimensional, pseudo-Riemannian manifold in which curvature in
the space-time metric {\it is} the manifestation of gravity.

Any theory of gravity is meant to describe the basic attractive interaction
between massive objects.  The most fundamental dynamical interaction in any
theory of gravity, then, is that between two massive bodies.  The two-body
problem, from the point of view of Newtonian gravity, is well understood.  One
can easily specify the masses, initial positions, and initial velocities of two
point particles and determine the time development of this initial
configuration.   The Newtonian equations of motion for two massive point
particles are exactly solvable.  The analogous problem in general relativity is
not well understood.

The two-body problem of general relativity has been studied through many
approximations.  The most familiar general relativistic solution to the two-body
problem is the prediction of the precession of the perihelion of an orbit.  This
is one example of an approximate solution which can be obtained by assuming the
gravitational field is weak.  The general formalisms for obtaining such
solutions are known as the post-Newtonian and the post-Minkowskian approximation
methods.  The collision of two Schwarzschild black holes has been studied,
analytically, in the limit that one hole is much smaller than the other so that
it can be treated as a perturbation.  There are certainly other approximation
methods for studying the two-body problem, but to date there are no known exact
or approximate analytic solutions to the problem of the strong field,
relativistic interaction of two massive bodies.  Currently, the only known avenue
for the investigation of these very interesting situations is through the use of
numerical techniques.

An important consequence of the theory of general relativity is the production
and propagation of gravitational waves from dynamic gravitational
configurations.  Violent astrophysical events, such as binary coalescence, will
most likely be the strongest emitters of gravitational radiation.  It is
believed that the detection and study of gravitational waves from these events
will play an important role in the understanding of such astrophysical
phenomena.  Because of this, a vigorous experimental and theoretical
investigation of methods for detecting gravitational waves is under way,
stemming from the early 1960s.  The latest generation of detectors, known as
Laser Interferometer Gravity Wave Observatories (LIGOs), should be sensitive
enough to have a high probability of detecting gravitational waves.  As of this
writing, the National Science Foundation has included funding for the LIGOs in
the budget for fiscal year 1991, with an estimated four-year expenditure of
\$192 million.

The numerical study of Einstein's equations has been under exploration since the
early 1960s (Hahn and Lindquist [1964]), and has been vigorously pursued since
the mid 1970s.  To quote Thorne [1987], ``The effort may absorb almost as many
person-years as the development of gravitational-wave detectors; but it will be
well worthwhile:  the payoffs will include the ability to compute in detail the
waveforms from the strongest gravity-wave sources in the universe, such as the
spiraling together and coalescence of two black holes---waveforms that will be
crucial to the interpretation of gravity-wave observations and to their use in
strong-field, highly dynamical tests of general relativity.''

Considering the statements above, I feel that the study of the fully
relativistic, strong-field, two-body problem is of prime concern to the fields of
classical general relativity and relativistic astrophysics.  There are
essentially two avenues for pursuing this problem.  The first involves a study
of the collision of two compact material objects such as neutron stars.  Efforts
in this line have been made by Evans [1986b], [1987].  The second avenue
involves the purely geometrodynamic collision of two black holes.  In this case,
one of the pioneering numerical relativistic calculations was undertaken by Smarr
[1977] in exploring the head-on collision of two black holes.  In the first
case, matter is present in quantities sufficient to produce strong gravitational
fields.  The matter fields evolve in a tightly coupled manner with the
gravitational field and, given the proper initial conditions, produce a
collision.  A numerical calculation of this kind involves solving the full
Einstein field equations and the relativistic hydrodynamic equations on a simply
connected topology.  In the case of a purely geometrodynamic collision, only the
Einstein field equations need to be solved since the space-time is considered to
contain no matter.  On the other hand, to support the strong gravitational
fields, we must use non-trivial topologies.  With this second approach, the
interaction of black holes can be modeled and it is the interaction of two black
holes which is the closest relativistic analog of the Newtonian two-body problem
for point masses.

The aim of this dissertation is to advance the understanding of the relativistic
two-body problem for black holes.  A great deal of theoretical and computational
work has already been done toward this goal.  Misner [1963], Lindquist [1963],
and Brill and Lindquist [1963] found a set of analytic solutions to the
initial-value equations of general relativity which represented any number of
black holes initially at rest in a vacuum.  Hahn and Lindquist [1964] and \v
Cade\v z [1971] made early attempts to evolve this initial data numerically but
met with little success.  Besides the computational resources available, the
major hindrance in these early calculations stemmed from a poor formal
understanding of both the initial-value problem and the dynamics and kinematics
of the Einsteinian evolution equations.  York [1971], [1972], [1973a], [1979],
\'{O} Murchada and York [1973], [1974a], [1974b], [1974c] and Smarr and York
[1978a], [1978b] made many advances in the theoretical understanding of these
problems and Smarr {\it et al}. [1976] and Smarr [1977] performed the first
successful numerical simulation of black hole collisions.  Smarr's calculations
started with Misner's initial data for two black holes initially at rest and
followed the holes as they fell into a head-on collision while emitting
gravitational radiation.

In order to simulate more realistic and interesting situations, it is necessary
to specify initial conditions which represent black holes with non-zero linear
and angular momenta.  York [1979], Bowen [1979b], and Bowen and York [1980]
developed a formalism for specifying the initial data for a single black hole
with non-zero linear and angular momenta.  Many authors, (York and Piran
[1982], Choptuik [1982], Rauber [1986], and Cook and York [1990]), have studied
single-hole, axisymmetric initial-data sets using numerical methods based on
this framework.  The formalism employed for prescribing the initial data for a
single hole was generalized to allow for multiple black holes by Kulkarni {\it
et al}. [1983], York [1984], and Kulkarni [1984].  In principle, this method
allows for the individual specification of linear and angular momenta for each
hole permitting one, for example, to specify initial data for the spiralling
coalescence of two black holes.  Based on this formal approach, Bowen, Rauber,
and York [1984] detailed an approach for specifying the initial data for two
black holes with axisymmetric parallel spins.  Rauber [1986] attempted to solve
numerically for the data sets based on this approach but was unable to generate
any complete solutions.

We call the collection of the formalisms and assumptions described in the papers
listed above the {\it conformal imaging technique}.  This approach is not unique
and in fact Thornburg [1987] has found numerical solutions, based on an alternate
set of assumptions, to the initial value equations for two black holes with
axisymmetric linear and angular momenta.  I feel, however, that the conformal
imaging approach will be a more fruitful approach and proceed in this
dissertation to describe and examine solutions to the initial-value equations,
based on the conformal imaging approach, which describe two black holes with
axisymmetric linear and angular momenta.

In the chapters which follow, I begin by outlining the conformal imaging
approach.  This consists of the $(3+1)$ decomposition of Einstein's equations,
York's conformal decomposition of the constraint equations, and the choice of
the topology of the initial-data slice along with the isometry condition imposed
on this slice and its consequences for the initial-value equations.  Next, I
will discuss the analytic solutions to the momentum constraint equations for a
single black hole, the formal extension to multiple black holes, and my
algorithm for evaluating this formal solution.  The following chapters will
discuss the numerical techniques used, and the solutions found, for the case of
a single black hole with linear or angular momentum and for two black holes with
axisymmetric linear or angular momenta.  One aspect of initial-data slices
which has not been explored sufficiently is the existence and location of
apparent horizons.  I will proceed with a description of the consequences of the
conformal imaging approach on the existence and location of these apparent
horizons and detail their properties for the single black hole initial-data
sets.  Finally, I will conclude with some remarks on future work.

\vfill
\eject
