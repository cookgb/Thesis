\topskip=1truein
\noindent
GREGORY B. COOK:  Initial Data for the Two-Body Problem of General Relativity
(Under the direction of James W. York, Jr.)

\vskip 0.625truein

\centerline{\chapfnt ABSTRACT}

\vskip 0.375truein

The numerical study of Einstein's equations is of considerable interest for both
astrophysical and purely theoretical reasons as it provides the only known
avenue for the study of strong field, highly relativistic gravitational
interactions.  Of particular interest is the geometrodynamic two-body problem
since the collision of two black holes is believed to be a strong source  of
gravitational radiation.  Before studying the dynamics of such collisions, one
must first obtain complete initial-data sets which represent two black holes in
some initial configuration including non-vanishing linear and angular momenta
for each hole.  A formalism for constructing such initial-data sets is
reviewed.  Based on this formalism, computational techniques are developed which
allow for the complete specification of initial data describing both one and two
black holes with individually specifiable linear and angular momenta.  In the
case of two black holes, the initial data is constructed in two separate, base
coordinate systems.  Bispherical coordinates are used to construct very accurate
initial-data sets.  \v{C}ade\v{z} coordinates are also used because this
coordinate system is most appropriate for future work on the evolution of the
initial data.  A method for locating apparent horizons in the initial-data sets
is developed by posing the apparent-horizon equation as a boundary-value
problem.  This method is applied to the single-hole, initial-data sets leading
to a new understanding of the physical content of these initial hypersurfaces. 
The physical content of the two-hole, initial-data sets is also explored.

\vfill
\eject