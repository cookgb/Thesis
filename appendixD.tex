\chapterhead{Appendix~D:  Numerical Solutions of the Two-Hole\cr Hamiltonian
Constraint in \v{C}ade\v{z} Coordinates}
%%%%%%%%%%%%%%%%%%%%%%%%%%%%%%
This appendix contains the most important details from the numerical solution
of the two-hole Hamiltonian constraint described in Chapter~10.  The difference
equations for this approach were solved by Newton's method for systems of
equations and each linearized equation for the iterative correction was solved
directly by means of the LINPACK routines for factoring (DGBFA) and solving
(DGBSL) general banded matrices.  The solution for each configuration (except as
noted) was obtained at three different mesh resolutions.  The resolution of the
mesh is parameterized by $\Iind^+$ as defined in (10.28).  When computational
memory requirements allowed, meshes defined by $\Iind^+ = \Iind^- = 6,9,12$ were
used.  The values of the angular discretization for these three meshes were,
respectively, $\Sind^+ = \Sind^- = 16, 24, 32$. It was not always possible to use
all three meshes because the number of mesh points in Region~3 (see Chapter~10)
depends on the value of $\beta$.  For $\beta < 6$ it was not feasable to use
$\Iind^+=12$.  For $\beta=3$ it was not feasable to use $ \Iind^+=9$ either. 
This is indicated in the tables by the notation N/A.

The 60 tables which follow can be considered as ten groups of six tables.  Each
group of six tables corresponds to the six cases described in Chapter~14.  The
ten groups of six tables represent ten values of the separation parameter
$\beta=3\to12$.  Finally, each table lists results from the numerical
computations for the scaled total energy $E/a_1$, the scaled minimal surface
mass $M/a_1$, and the scaled proper separation $L/a_1$ as the linear or angular
momentum parameter is varied.

The 60 tables presented below correspond directly to configurations for each of
the tables in Appendix~C.  These can be directly compared to judge the accuracy
of solutions found using \v{C}ade\v{z} coordinates.  Note that for $\beta=8$,
all values for $\Iind^+=6$ are unreasonable.  This occurs because the  matrix
used to construct the differencing scheme for the singular point (see (10.53))
is numerically ill-conditioned for this configuration.  The difficulty can be
eliminated by choosing a different radial discretization.

\vfill
\eject

\input Tables/2-Hole.cd/cd3
\input Tables/2-Hole.cd/cd4
\input Tables/2-Hole.cd/cd5
\input Tables/2-Hole.cd/cd6
\input Tables/2-Hole.cd/cd7
\input Tables/2-Hole.cd/cd8
\input Tables/2-Hole.cd/cd9
\input Tables/2-Hole.cd/cd10
\input Tables/2-Hole.cd/cd11
\input Tables/2-Hole.cd/cd12
