\chapterhead{Appendix~B:  Numerical Solutions of the One-Hole\cr
Hamiltonian Constraint}
%%%%%%%%%%%%%%%%%%%%%%%%%%
This appendix contains the most important details from the numerical solutions
of the one-hole Hamiltonian constraint described in Chapter~8.  The code which
produced these results was a multigrid code based on the ideas outlined in
Appendix~A.  In all cases, the solutions were found on a multigrid hierarchy
consisting of seven separate grids or meshes.  The coarsest mesh, labeled level
zero, was discretized into 16 radial zones and six angular zones.  This
discretization was doubled for each finer mesh and the finest mesh on level six
was discretized into 1024 radial zones and 384 angular zones.  The outer
boundary on which the approximate outer boundary condition was specified was
placed at $x = 10$ which places it at $\sim22,000$ radii in the conformal
space.

Seven sets of output are listed below.  To gauge the accuracy of the code, the
first set of outputs lists the results from the numerical solution of the model
problem described in Chapter~8.  The total energy and minimal surface mass were
computed for nine separate values of the ``momentum'' parameter.  The values of
these quantities are listed for each multigrid level and the analytic solution
is listed at the bottom of each table.  Also listed for each level are two
values for  the $L_1$ norm of the relative error in the conformal factor.  Since
the solution is known analytically, the exact $L_1$ error is given as well as an
estimated error for level $\ell$ computed by comparing the result on level
$\ell$  with that on level $\ell-1$.  The code was configured to solve the
difference equations using the FAS limit of the code described in Appendix~A.

The next three sets of output tables contain, respectively, the results for a
hole with linear momentum based on an extrinsic curvature obeying the isometry
condition with a plus sign, a hole with linear momentum based on an extrinsic
curvature obeying the isometry condition with a minus sign, and a hole with
angular momentum based on an extrinsic curvature obeying the isometry condition
with a minus sign.  The code was configured to solve the difference equations
using adaptive scaling as described in Appendix~A. The adaptive scaling
parameter was set at $\varepsilon_s = 0.1$.  The final three sets of output
tables consist of the same solution set as the previous three except that the
code was configured to solve the difference equations using the FAS limit of the
code.
\bigskip
\bigskip
\goodbreak
\noindent{\bf $-$ Model Problem Set $-$}
\input Tables/1-Hole/1H-P.test.tex
\goodbreak
\line{\bf $-$ Physical Problem Sets (Adaptive Scaling) $-$\hfill}
\nobreak\line{\indent \bf Linear Momentum, $+$ isometry\hfill}\nobreak
\input Tables/1-Hole/1H-P+.adaptive.tex
\goodbreak
\line{\indent \bf Linear Momentum, $-$ isometry\hfill}\nobreak
\input Tables/1-Hole/1H-P-.adaptive.tex
\goodbreak
\line{\indent \bf Angular Momentum, $-$ isometry\hfill}\nobreak
\input Tables/1-Hole/1H-J.adaptive.tex
\goodbreak
\line{\bf $-$ Physical Problem Sets (FAS) $-$\hfill}
\nobreak\line{\indent \bf Linear Momentum, $+$ isometry\hfill}\nobreak
\input Tables/1-Hole/1H-P+.fas.tex
\goodbreak
\line{\indent \bf Linear Momentum, $-$ isometry\hfill}\nobreak
\input Tables/1-Hole/1H-P-.fas.tex
\goodbreak
\line{\indent \bf Angular Momentum, $-$ isometry\hfill}\nobreak
\input Tables/1-Hole/1H-J.fas.tex
\vfill
\eject
