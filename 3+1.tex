\chapterhead{Chapter~{\the\chapnum}:  The $(3+1)$ Decomposition}
%%%%%%%%%%%%%%%%%%%
Einstein's equations in full covariant form are a set of coupled partial
differential equations, the solution of which is a metric $g_{\mu\nu}$ that
represents the full pseudo-Riemannian geometry of a space-time.  This metric is
not a dynamical object in that it does not change in time.  On the contrary, the
metric represents the geometry at all times just as the metric of a two-sphere
represents the geometry of that two-sphere at all points on the sphere.  In
order for Einstein's equations to reveal their dynamical nature, we must break
the full, four-dimensional covariance and exploit the special nature of time. 
One method of doing this is the $(3+1)$ decomposition of Arnowitt, Deser, and
Misner (ADM) [1962].  The idea behind the $(3+1)$ decomposition is to divide
space-time into a sequence of ``instants'' of time.  Each instant of time holds
an ``instantaneous state'' of the gravitational field which dynamically evolves
from instant to instant.  Formally, the $(3+1)$ decomposition allows Einstein's
equations to be posed as a Cauchy problem ({\it cf}. Choquet-Bruhat and York
[1980]).  In the remainder of this chapter, I will present the $(3+1)$ splitting
of Einstein's equations based largely on York [1979], Bowen [1979b], and Evans
[1984].  The Einstein equations will be split into constraint equations which
are solved to give the initial data for a gravitational configuration on some
initial slice, and evolution equations which describe the dynamics of general
relativity and take the gravitational configuration from slice to slice. 

Consider a 4-dimensional space with metric $g_{\mu\nu}$ having signature
$(\epsilon+++)$, where $\epsilon�=�\pm$.  (For generality, we consider both
pseudo-Riemannian $(\epsilon=-)$ and Riemannian $(\epsilon=+)$ spaces.  In this
work, we will only be concerned with pseudo-Riemannian space-times, though it is
useful to keep track of where the special nature of time affects the equations. 
Also, the generality is useful considering the importance of Euclidian methods in
other areas of relativity.)  The space is assumed to be globally hyperbolic
($\epsilon=-$ case) so that it can be foliated by a family of 3-dimensional
hypersurfaces ${\Sigma}$ that fills the space $V$.  Let $\tau$ be a scalar
function such that the level surfaces of $\tau$ are the hypersurfaces
${\Sigma}$.  If the space can contain a foliation $\{{\Sigma}\}$, then the
foliation can be described by a closed one-form $\Omegatilda=\Omega_\mu
\etilda^\mu$ where $$
d\Omegatilda=0 \qquad \hbox{or}\qquad \nabla_{[\mu }\Omega_{\nu]}=0
\eqn{}
$$
and $\etilda^\mu$ is a general basis of forms.  Since $\Omega$ is
closed, it must satisfy
$$
\Omegatilda=d\tau \qquad \hbox{or}\qquad 
{\Omega }{}_{\mu }={\nabla }{}_{\mu }\tau. \eqn{}
$$
The lapse function $\alpha$ defines the norm $\|\Omega\|$ by means of the
space-time metric $g_{\mu\nu}$
$$
\|\Omega\|={g}{}^{\mu }{}^{\nu }{\Omega }{}_{\mu
}{\Omega }{}_{\nu }={g}{}^{\mu }{}^{\nu }{\nabla }{}_{\mu }\tau {\nabla }{}_{\nu
}\tau =\epsilon {\alpha }^{-2}, \eqn{}
$$
so the normalized one-form $\omegatilda$ associated with $\{\Sigma\}$ is
given by
$$
{\omega }{}_{\mu }\equiv \alpha {\Omega }{}_{\mu }=\alpha {\nabla }{}_{\mu
}\tau. \eqn{}
$$
The minus sign for pseudo-Riemannian space-times and the strictly positive
nature of $\alpha$ guarantee that the hypersurfaces defined by $\omegatilda$
will be spacelike everywhere.  The unit normal vector of a slice is given by
$$
{n}{}^{\nu }=\epsilon {g}{}^{\mu }{}^{\nu }{\omega }{}_{\mu }, \eqn{}
$$
so that $n^\mu\omega_\mu = 1$ (and so that it is timelike $n_\mu n^\mu  = -1$
for $\epsilon=-$).

The spatial metric $\gamma_{\mu\nu}$
induced by $g_{\mu\nu}$ onto a slice $\Sigma$ is given by
$$
{\gamma }{}_{\mu }{}_{\nu }={g}{}_{\mu }{}_{\nu }-\epsilon {n}{}_{\mu
}{n}{}_{\nu }={g}{}_{\mu }{}_{\nu }-\epsilon {\omega }{}_{\mu }{\omega
}{}_{\nu }, \eqn{}
$$
and the inverse spatial metric is given by
$$
{\gamma }{}^{\mu }{}^{\nu }={g}{}^{\mu }{}^{\alpha }{g}{}^{\nu }{}^{\beta
}{\gamma }{}_{\alpha }{}_{\beta }={g}{}^{\mu }{}^{\nu }-\epsilon {n}{}^{\mu
}{n}{}^{\nu }. \eqn{}
$$

Each index on a tensor can be decomposed into two pieces, one which is purely
spatial and lies entirely in the surface $\Sigma$ and one which is entirely
normal to the surface.  The projection operator $\perp$, which projects free
indices into the slice $\Sigma$, is defined as
$$
{\perp }_{\nu }^{\mu }\equiv {\gamma }_{\nu }^{\mu }={\delta }_{\nu }^{\mu
}-\epsilon {n}{}^{\mu }{n}{}_{\nu }\qquad \hbox{;}\qquad
{\perp }_{\nu }^{\mu}{n}{}_{\mu }=0. \eqn{}
$$
Similarly, the projection operator $N$, which projects free indices normal to the
slice, is defined as
$$
{N}_{\nu }^{\mu }\equiv \epsilon {n}{}^{\mu }{n}{}_{\nu }={\delta }_{\nu
}^{\mu }-{\perp }_{\nu }^{\mu }\qquad \hbox{;}\qquad
{N}_{\nu }^{\mu }{n}{}_{\mu }={n}{}_{\nu }. \eqn{}
$$
Note that $\perp$ and $N$ are idempotent projection operators since
$$
{\perp }_{\rho }^{\mu }{\perp }_{\nu }^{\rho }={\perp }_{\nu }^{\mu }
\qquad\hbox{and}\qquad{\perp
}_{\mu }^{\mu }=3, \eqn{}
$$
and
$$
{N}_{\rho }^{\mu }{N}_{\nu }^{\rho }={N}_{\nu }^{\mu }
\qquad\hbox{and}\qquad{N}_{\mu }^{\mu }=1. \eqn{}
$$

Given the spatial metric on a hypersurface $\Sigma$, we can define a spatial
covariant derivative $D_\mu$ which is compatible with the metric
$\gamma_{\mu\nu}$ and which, acting upon spatial tensors, produces spatial
tensors.  This operator is defined by taking the full covariant derivative of a
tensor and projecting all free indices into the hypersurface.  Letting $\perp$
with no indices represent the product of $n$ projection operators, one for each
free index on the object upon which it is acting, then
$$
\eqalign{{D}{}_{\mu }{T}{}^{\alpha \cdots\beta }{}_{\rho \cdots\sigma }&\equiv
\perp {\nabla }{}_{\mu }{T}{}^{\alpha \cdots\beta }{}_{\rho \cdots\sigma }\cr
&={\perp }_{\mu }^{\nu }{\perp }_{\kappa }^{\alpha }\cdots{\perp }_{\lambda
}^{\beta }{\perp }_{\rho }^{\zeta }\cdots{\perp }_{\sigma }^{\xi }{\nabla
}{}_{\nu }{T}{}^{\kappa \cdots\lambda }{}_{\zeta \cdots\xi }.\cr} \eqn{}
$$
It is easily seen that the spatial covariant derivative is compatible with the
spatial metric
$$
{D}{}_{\alpha }{\gamma }{}_{\mu }{}_{\nu }={\perp }_{\alpha }^{\beta }{\perp
}_{\mu }^{\rho }{\perp }_{\nu }^{\sigma }{\nabla }{}_{\beta }\left({{g}{}_{\rho
}{}_{\sigma }-\epsilon {n}{}_{\rho }{n}{}_{\sigma }}\right)=0. \eqn{}
$$
The requirement that the spatial covariant derivative acts only on spatial
tensors is required in order for Leibnitz's rule to hold.
$$
\eqalign{{D}{}_{\alpha }\left({{V}{}^{\mu }{W}{}_{\mu }}\right)&={\perp
}_{\alpha }^{\beta }{\nabla }{}_{\beta }\left({{V}{}^{\mu }{W}{}_{\mu
}}\right)\cr &={\perp }_{\alpha }^{\beta }{V}{}^{\mu }\left({{\perp }_{\mu
}^{\rho }+{N}_{\mu }^{\rho }}\right){\nabla }{}_{\beta }{W}{}_{\rho }+{\perp
}_{\alpha }^{\beta }{W}{}_{\mu }\left({{\perp }_{\rho }^{\mu }+{N}_{\rho }^{\mu
}}\right){\nabla }{}_{\beta }{V}{}^{\rho }\cr &={V}{}^{\mu }{D}{}_{\alpha
}{W}{}_{\mu }+{W}{}_{\mu }{D}{}_{\alpha }{V}{}^{\mu }+{\perp }_{\alpha }^{\beta
}\left({{V}{}^{\mu }{N}_{\mu }^{\rho }{\nabla }{}_{\beta }{W}{}_{\rho
}+{W}{}_{\mu }{N}_{\rho }^{\mu }{\nabla }{}_{\beta }{V}{}^{\rho }}\right)\cr}
\eqn{}
$$
which satisfies Leibnitz's rule only if $V^\mu$ and $W_\mu$ are purely spatial,
i.e. $N_\mu^\rho V^\mu = 0$ and $N_\rho^\mu W_\mu = 0$.

A slice $\Sigma$ has ``intrinsic'' curvature defined in the standard way by the
commutation of covariant derivatives.  If $N_\rho^\mu W_\mu = 0$, then
$$
\left[{{D}{}_{\mu },{D}{}_{\nu }}\right]{W}{}_{\sigma }={W}{}_{\rho
}{\dimR{3}}{}^{\rho }{}_{\sigma }{}_{\nu }{}_{\mu }
\qquad\hbox{and}\qquad{n}{}_{\rho
}{\dimR{3}}{}^{\rho }{}_{\sigma }{}_{\nu }{}_{\mu }=0. \eqn{}
$$
Here, ${{\dimR{3}}^\rho}_{\sigma\nu\mu}$ is the spatial Riemann tensor for the
hypersurface $\Sigma$, as opposed to ${{\dimR{4}}^\rho}_{\sigma\nu\mu}$ which is
the Riemann tensor for the full 4-dimensional space.  Since $\Sigma$ is an embedded
hypersurface, the shape of the slice in the full space or the ``extrinsic''
curvature of the slice is also of interest.  The extrinsic curvature of a
hypersurface is related to the projection of the gradient of the surface normal
vector field.  This naturally consists of two parts:  the symmetric part
$\Theta_{\underline{\mu\nu}}$, known as the ``strain'', and the antisymmetric
part $\omega_{\mu\nu}\kern-1em\smash{\lower3.6ex\hbox{\antisym\char20}}$, known
as the ``twist''
$$
{\Theta }{}_{\underline{\mu \nu }}\equiv \perp {\nabla }{}_{(\mu }{n}{}_{\nu)}
\qquad\hbox{and}\qquad
\omega_{\mu\nu}\kern-1em\smash{\lower3.6ex\hbox{\antisym\char20}} \equiv \perp
{\nabla }{}_{[\mu }{n}{}_{\nu]}. \eqn{}
$$
Since the vector field $n^\mu$ is surface-forming by definition
$(d\Omegatilda = 0)$, it is easy to show that
$$
\omega \wedge d\omega =0\qquad\hbox{or}\qquad{\omega }{}_{[\mu }{\nabla }{}_{\nu
}{\omega }{}_{\sigma ]}=0. \eqn{}
$$
It follows immediately that $n_{[\mu}\nabla_\nu n_{\sigma]} = 0$.  By contracting
with $n^\mu$ and projecting the remaining free indices we find
$$
\eqalign{\perp \left({{n}{}^{\mu }{n}{}_{[\mu }{\nabla }{}_{\nu }{n}{}_{\sigma
]}}\right)&={1 \over 3}\perp \left({-{\nabla }{}_{[\nu }{n}{}_{\sigma ]}+{1 \over
2}{n}{}^{\mu }\left({{n}{}_{\sigma }{\nabla }{}_{[\mu }{n}{}_{\nu ]}-{n}{}_{\nu
}{\nabla }{}_{[\mu }{n}{}_{\sigma ]}}\right)}\right)\cr &=-{1 \over 3}\perp
{\nabla }{}_{[\nu }{n}{}_{\sigma ]}=0.\cr} \eqn{}
$$
And so the twist vanishes.  The extrinsic curvature of the slice, $K_{\mu\nu}$,
is defined as minus the strain
$$
{K}{}_{\mu }{}_{\nu }\equiv -{\Theta }{}_{\underline{\mu \nu }}=-\perp {\nabla
}{}_{(\mu }{n}{}_{\nu )}. \eqn{}
$$
Note that the choice of the minus sign is a convention.  The extrinsic curvature
can be expressed in a more convenient form by expanding the gradient of the unit
normal
$$
\eqalign{{\nabla }{}_{\mu }{n}{}_{\nu }&={\delta }_{\mu }^{\alpha }{\delta
}_{\nu }^{\beta }{\nabla }{}_{\alpha }{n}{}_{\beta }=\left({{\perp }_{\mu
}^{\alpha }+\epsilon {n}{}^{\alpha }{n}{}_{\mu }}\right)\left({{\perp }_{\nu
}^{\beta }+\epsilon {n}{}^{\beta }{n}{}_{\nu }}\right){\nabla }{}_{\alpha
}{n}{}_{\beta }\cr &=\perp {\nabla }{}_{\mu }{n}{}_{\nu }+\epsilon {n}{}_{\mu
}\perp {a}{}_{\nu } \cr}\eqn{}
$$
where the acceleration of the unit normal, $a_\mu$, has been used
$$
{a}{}_{\mu }={n}{}^{\nu }{\nabla }{}_{\nu }{n}{}_{\mu }. \eqn{}
$$
Note that the acceleration is purely spatial since its contraction with the unit
normal vanishes.  This follows from the identity
$$
{\nabla }{}_{\mu }\left({{n}{}_{\nu }{n}{}^{\nu }}\right)=2{n}{}^{\nu }{\nabla
}{}_{\mu }{n}{}_{\nu }=0. \eqn{}
$$
Since the acceleration is spatial, ${\perp}a^\mu = a^\mu$ and we find
$$
{\nabla }{}_{\mu }{n}{}_{\nu }=-{K}{}_{\mu }{}_{\nu }+\epsilon {n}{}_{\mu
}{a}{}_{\nu }. \eqn{}
$$
Since the extrinsic curvature is symmetric, we can express it as
$$
{K}{}_{\mu }{}_{\nu }=-{\nabla }{}_{(\mu }{n}{}_{\nu )}+\epsilon
{n}{}_{(\mu }{a}{}_{\nu )} \eqn{}
$$
from which it is easily seen that the extrinsic curvature is also purely
spatial.  From this and the definition of the Lie derivative, it follows that
$$
\eqalign{{K}{}_{\mu }{}_{\nu }&=\perp {K}{}_{\mu }{}_{\nu }=\perp
\left({-{\nabla }{}_{(\mu }{n}{}_{\nu )}+\epsilon {n}{}_{(\mu }{a}{}_{\nu
)}}\right)=-\perp {\nabla }{}_{(\mu }{n}{}_{\nu )}\cr &=-{1 \over 2}\perp
{\hbox{\it\char36}}_{\bf n}{g}{}_{\mu }{}_{\nu }.\cr} \eqn{}
$$
Alternatively, if we look at the Lie derivative of the spatial metric, we find
$$
\eqalign{{\hbox{\it\char36}}_{\bf n}{\gamma }{}_{\mu }{}_{\nu
}&={\hbox{\it\char36}}_{\bf n}\left({{g}{}_{\mu }{}_{\nu }-\epsilon
{n}{}_{\mu }{n}{}_{\nu }}\right)\cr &=2\left({{\nabla }{}_{(\mu }{n}{}_{\nu
)}-\epsilon {n}{}_{(\mu }{a}{}_{\nu )}}\right)\cr &=-2{K}{}_{\mu }{}_{\nu
}.\cr} \eqn{}
$$
Since the unit normal is a timelike vector, the Lie derivative, alone $n^\mu$, of
the spatial metric (and thus the extrinsic curvature) is related to the
``velocity'' of the spatial metric on the slice $\Sigma$.

Given a slice $\Sigma$ in space-time, its complete geometry is described by its spatial
metric and extrinsic curvature.  Thus, the spatial metric and extrinsic
curvature represent the ``instantaneous state'' of the gravitational field and
are the dynamical quantities which will be followed to explore the evolution of a
gravitational field.  In order that a foliation of slices, $\{\Sigma\}$, can fit
into the higher dimensional space, they must satisfy certain conditions known as
the Gauss-Codazzi-Ricci conditions.  These are obtained by taking all of the
possible projections of the Riemann tensor of the full space.  We can spatially
project all of the indices of the Riemann tensor into the hypersurface, contract
one index with the unit normal and spatially project the remaining three free
indices, and finally we can contract two of the indices with unit normals and
project the remaining two.  All other combinations are identically zero by the
symmetries of the Riemann tensor.  To explore these, consider first
$$
\eqalign{{D}{}_{\mu }{D}{}_{\nu }{V}{}_{\rho }&=\perp {\nabla }{}_{\mu
}\left({\perp {\nabla }{}_{\nu }{V}{}_{\rho }}\right)\cr &=\perp {\nabla }{}_{\mu
}{\nabla }{}_{\nu }{V}{}_{\rho }+\perp \left({{\nabla }{}_{\nu }{V}{}_{\sigma
}}\right)\left({{\nabla }{}_{\mu }{\perp }_{\rho }^{\sigma }}\right)+\perp
\left({{\nabla }{}_{\sigma }{V}{}_{\rho }}\right)\left({{\nabla }{}_{\mu }{\perp
}_{\nu }^{\sigma }}\right).\cr} \eqn{}
$$
If we now expand the gradient of the projection operator, we find
$$
{\nabla }{}_{\sigma }{\perp }_{\nu }^{\mu }=\epsilon {n}{}_{\nu
}{K}{}_{\sigma }{}^{\mu }-{n}{}_{\nu }{n}{}_{\sigma }{a}{}^{\mu }+\epsilon
{n}{}^{\mu }{K}{}_{\sigma }{}_{\nu }-{n}{}^{\mu }{n}{}_{\sigma }{a}{}_{\nu }.
\eqn{}
$$
Combining (2.27) and (2.28), we find
$$
{D}{}_{\mu }{D}{}_{\nu }{V}{}_{\rho }=\perp {\nabla }{}_{\mu }{\nabla }{}_{\nu
}{V}{}_{\rho }+\epsilon \perp {K}{}_{\mu }{}_{\rho }{K}{}_{\nu }{}^{\sigma
}{V}{}_{\sigma }+\epsilon \perp {K}{}_{\mu }{}_{\nu }{n}{}^{\sigma }{\nabla
}{}_{\sigma }{V}{}_{\rho }. \eqn{}
$$
Then, using the definition of the Riemann tensor in terms of the commutation of
covariant derivatives (2.15), we get
$$
{\dimR{3}}{}_{\mu }{}_{\nu }{}_{\rho }{}^{\sigma }{V}{}_{\sigma }=\perp
{\dimR{4}}{}_{\mu }{}_{\nu }{}_{\rho }{}^{\sigma }{V}{}_{\sigma
}+\epsilon \perp {K}{}_{\mu }{}_{\rho }{K}{}_{\nu }{}^{\sigma }{V}{}_{\sigma
}-\epsilon \perp {K}{}_{\nu }{}_{\rho }{K}{}_{\mu }{}^{\sigma }{V}{}_{\sigma
}, \eqn{}
$$
which gives us the form of the fully projected Riemann tensor known as Gauss'
equation: 
$$
\perp {\dimR{4}}{}_{\mu }{}_{\nu }{}_{\rho }{}_{\sigma }={\dimR{3}}{}_{\mu
}{}_{\nu }{}_{\rho }{}_{\sigma }-\epsilon {K}{}_{\mu }{}_{\rho }{K}{}_{\nu
}{}_{\sigma }+\epsilon {K}{}_{\mu }{}_{\sigma }{K}{}_{\nu }{}_{\rho }. \eqn{}
$$

If we contract one index on the Riemann tensor with the unit normal and project
the rest, we find
$$
\perp {\dimR{4}}{}_{\mu }{}_{\nu }{}_{\rho }{}^{\sigma }{n}{}_{\sigma
}=\perp \left({{\nabla }{}_{\mu }{\nabla }{}_{\nu }{n}{}_{\rho }-{\nabla
}{}_{\nu }{\nabla }{}_{\mu }{n}{}_{\rho }}\right). \eqn{}
$$
To simplify this, consider one of the terms on the right-hand side.
$$
\eqalign{\perp {\nabla }{}_{\mu }{\nabla }{}_{\nu }{n}{}_{\rho }&=\perp {\nabla
}{}_{\mu }\left({-{K}{}_{\nu }{}_{\rho }+\epsilon {n}{}_{\nu }{a}{}_{\rho
}}\right)=-{D}{}_{\mu }{K}{}_{\nu }{}_{\rho }+\epsilon \perp
\left({{a}{}_{\rho }{\nabla }{}_{\mu }{n}{}_{\nu }}\right)\cr &=-{D}{}_{\mu
}{K}{}_{\nu }{}_{\rho }-\epsilon {a}{}_{\rho }{K}{}_{\mu }{}_{\nu }.\cr}\eqn{}
$$
It follows, then, that the singly contracted, projected Riemann tensor gives
$$
\perp {\dimR{4}}{}_{\mu }{}_{\nu }{}_{\rho }{}^{\sigma }{n}{}_{\sigma
}={D}{}_{\nu }{K}{}_{\mu }{}_{\rho }-{D}{}_{\mu }{K}{}_{\nu }{}_{\rho }, \eqn{}
$$
which is known as Codazzi's equation.

Note that Gauss' and Codazzi's equations depend only on the spatial metric,
extrinsic curvature, and their spatial derivatives.  This implies that the
Gauss-Codazzi equations represent integrability conditions which the spatial
metric and extrinsic curvature must satisfy in order for any given slice to fit
into the full space.  These equations will be directly related to the
constraints of general relativity. Finally, explicit calculation shows
$$
\eqalign{{\hbox{\it\char36}}_{\bf n}{K}{}_{\mu }{}_{\nu }&={n}{}^{\rho
}{\nabla }{}_{\rho }{K}{}_{\mu }{}_{\nu }+{K}{}_{\mu }{}_{\rho }{\nabla }{}_{\nu
}{n}{}^{\rho }+{K}{}_{\rho }{}_{\nu }{\nabla }{}_{\mu }{n}{}^{\rho }\cr
&={\dimR{4}}{}_{\mu }{}_{\rho }{}_{\nu }{}^{\sigma }{n}{}^{\rho
}{n}{}_{\sigma }-{K}{}_{\mu }{}_{\rho }{K}{}_{\nu }{}^{\rho }+\epsilon
{K}{}_{\mu }{}_{\rho }{n}{}_{\nu }{a}{}^{\rho }-{\nabla }{}_{\mu }{a}{}_{\nu
}+\epsilon {a}{}_{\mu }{a}{}_{\nu }+\epsilon {n}{}_{\mu }{n}{}^{\rho
}{\nabla }{}_{\rho }{a}{}_{\nu }.\cr} \eqn{}
$$
Projecting this into the hypersurface yields
$$
\perp {\dimR{4}}{}_{\mu }{}_{\rho }{}_{\nu }{}^{\sigma }{n}{}^{\rho
}{n}{}_{\sigma }=\perp {\hbox{\it\char36}}_{\bf n}{K}{}_{\mu }{}_{\nu
}+{K}{}_{\mu }{}_{\rho }{K}{}_{\nu }{}^{\rho }-\epsilon {a}{}_{\mu
}{a}{}_{\nu }+{D}{}_{\mu }{a}{}_{\nu }. \eqn{}
$$
If we now consider
$$
\eqalign{{n}{}^{\mu }{\hbox{\it\char36}}_{\bf n}{K}{}_{\mu }{}_{\nu
}&={n}{}^{\mu }{n}{}^{\rho }{\nabla }{}_{\rho }{K}{}_{\mu }{}_{\nu }+{n}{}^{\mu
}{K}{}_{\mu }{}_{\rho }{\nabla }{}_{\nu }{n}{}^{\rho }+{n}{}^{\mu }{K}{}_{\rho
}{}_{\nu }{\nabla }{}_{\mu }{n}{}^{\rho }\cr &=-{K}{}_{\rho }{}_{\nu }{a}{}^{\rho
}+{K}{}_{\rho }{}_{\nu }{a}{}^{\rho }=0,\cr} \eqn{}
$$
we see that ${\hbox{\it\char36}}_{\bf n}K_{\mu\nu}$ is purely spatial so we find
that the twice contracted, projected Riemann tensor gives
$$
\perp {\dimR{4}}{}_{\mu }{}_{\rho }{}_{\nu }{}^{\sigma }{n}{}^{\rho
}{n}{}_{\sigma }={\hbox{\it\char36}}_{\bf n}{\rm K}{}_{\mu }{}_{\nu }+{K}{}_{\mu
}{}_{\rho }{K}{}_{\nu }{}^{\rho }-\epsilon {a}{}_{\mu }{a}{}_{\nu
}+{D}{}_{\mu }{a}{}_{\nu }, \eqn{}
$$
which is known as Ricci's equation.  Ricci's equation involves the Lie
derivative along the normal vector field.  This is, roughly speaking, a normal
or time derivative.  However, it should be noted that ${\hbox{\it\char36}}_{\bf
n}$ is not the natural time derivative orthogonal to the hypersurface.  We want
to take the Lie derivative along a vector field $t^\mu$ which is dual to the
natural one-form associated with the hypersurface.  The timelike vector field
should satisfy
$$
{\Omega }{}_{\mu }{t}{}^{\mu}=1\qquad\hbox{or}\qquad
\left\langle{\Omegatilda,\overline{t}}\right\rangle=1. \eqn{}
$$
From (2.4) and (2.5) we find that one such vector is
$$
{t}{}^{\mu }={N}{}^{\mu }\equiv \alpha {n}{}^{\mu }. \eqn{}
$$
Of course, this is not a unique choice.  We can add to $N^\mu$ any spatial
vector, $\beta^\mu$, since
$\left\langle{\Omegatilda,\overline{\beta}}\right\rangle = 0$.  The freedom in
the definition of the timelike vector stems from the general covariance of
Einstein's equations.  In general then, the timelike vector is defined as
$$
{t}{}^{\mu }\equiv \alpha {n}{}^{\mu }+{\beta }{}^{\mu }, \eqn{}
$$
and we find that
$$
\eqalign{{\hbox{\it\char36}}_{\bf t}{K}{}_{\mu }{}_{\nu
}&={\hbox{\it\char36}}_{\alpha \bf n}{K}{}_{\mu }{}_{\nu
}+{\hbox{\it\char36}}_{\bf \beta }{K}{}_{\mu }{}_{\nu }\cr &=\alpha
{\hbox{\it\char36}}_{\bf n}{K}{}_{\mu }{}_{\nu }+{\hbox{\it\char36}}_{\bf
\beta }{K}{}_{\mu }{}_{\nu }.\cr} \eqn{}
$$
As a final simplification of Ricci's equation, consider the acceleration
$a_\mu$,
$$
\eqalign{{a}{}_{\mu }&={n}{}^{\nu }{\nabla }{}_{\nu }{n}{}_{\mu }=2{n}{}^{\nu
}{\nabla }{}_{[\nu }{n}{}_{\mu ]}=\epsilon 2{n}{}^{\nu }{\nabla }{}_{[\nu
}\left({\alpha {\Omega }{}_{\mu ]}}\right)\cr &=\epsilon {n}{}^{\nu
}\left({{\Omega }{}_{\mu }{\nabla }{}_{\nu }\alpha -{\Omega }{}_{\nu }{\nabla
}{}_{\mu }\alpha }\right)=-\epsilon {\alpha }^{-1}\left({\perp {\nabla
}{}_{\mu }\alpha }\right)\cr &=-\epsilon {D}{}_{\mu }\ln\alpha.\cr} \eqn{}
$$
Combining (2.38), (2.42), and (2.43) we find that Ricci's equation takes the
form:
$$\perp {\dimR{4}}{}_{\mu }{}_{\rho }{}_{\nu }{}^{\sigma }{n}{}^{\rho
}{n}{}_{\sigma }={\alpha }^{-1}{\hbox{\it\char36}}_{\bf t}{K}{}_{\mu
}{}_{\nu }+{K}{}_{\mu }{}_{\rho }{K}{}_{\nu }{}^{\rho }-\epsilon {\alpha
}^{-1}{D}{}_{\mu }{D}{}_{\nu }\alpha -{\alpha }^{-1}{\hbox{\it\char36}}_{\bf
\beta }{K}{}_{\mu }{}_{\nu }. \eqn{}
$$

With the Gauss-Codazzi-Ricci equations, we can decompose the 4-dimensional
Riemann tensor
$$
\eqalign{{\dimR{4}}{}_{\mu }{}_{\nu }{}_{\rho }{}_{\sigma }=&\perp
{\dimR{4}}{}_{\mu }{}_{\nu }{}_{\rho }{}_{\sigma }-\epsilon {n}{}_{\mu }\perp
{\dimR{4}}{}_{\rho }{}_{\sigma }{}_{\nu }{}_{\delta }{n}{}^{\delta }+\epsilon
{n}{}_{\nu }\perp {\dimR{4}}{}_{\rho }{}_{\sigma }{}_{\mu }{}_{\delta
}{n}{}^{\delta }\cr &-\epsilon {n}{}_{\rho }\perp {\dimR{4}}{}_{\mu }{}_{\nu
}{}_{\sigma }{}_{\delta }{n}{}^{\delta }+\epsilon {n}{}_{\sigma }\perp
{\dimR{4}}{}_{\mu }{}_{\nu }{}_{\rho }{}_{\delta }{n}{}^{\delta }\cr &+{n}{}_{\mu
}{n}{}_{\rho }\perp {\dimR{4}}{}_{\nu }{}_{\delta }{}_{\sigma }{}_{\gamma
}{n}{}^{\delta }{n}{}^{\gamma }-{n}{}_{\mu }{n}{}_{\sigma }\perp
{\dimR{4}}{}_{\nu }{}_{\delta }{}_{\rho }{}_{\gamma }{n}{}^{\delta
}{n}{}^{\gamma }\cr &+{n}{}_{\nu }{n}{}_{\sigma }\perp {\dimR{4}}{}_{\mu
}{}_{\delta }{}_{\rho }{}_{\gamma }{n}{}^{\delta }{n}{}^{\gamma }-{n}{}_{\nu
}{n}{}_{\rho }\perp {\dimR{4}}{}_{\mu }{}_{\delta }{}_{\sigma }{}_{\gamma
}{n}{}^{\delta }{n}{}^{\gamma },\cr}  \eqn{}
$$
into
$$
\eqalign{{\dimR{4}}{}_{\mu }{}_{\nu }{}_{\rho }{}_{\sigma
}&={\dimR{3}}{}_{\mu }{}_{\nu }{}_{\rho }{}_{\sigma }-\epsilon {K}{}_{\mu
}{}_{\rho }{K}{}_{\nu }{}_{\sigma }+\epsilon {K}{}_{\mu }{}_{\sigma
}{K}{}_{\nu }{}_{\rho }\cr &+\epsilon 2{n}{}_{\mu }{D}{}_{[\rho }{K}{}_{\sigma
]}{}_{\nu }+\epsilon 2{n}{}_{\nu }{D}{}_{[\sigma }{K}{}_{\rho ]}{}_{\mu
}+\epsilon 2{n}{}_{\rho }{D}{}_{[\mu }{K}{}_{\nu ]}{}_{\sigma }+\epsilon
2{n}{}_{\sigma }{D}{}_{[\nu }{K}{}_{\mu ]}{}_{\rho }\cr &+{n}{}_{\mu }{n}{}_{\rho
}\left({{\alpha }^{-1}{\hbox{\it\char36}}_{\bf t}{K}{}_{\nu }{}_{\sigma
}+{K}{}_{\nu }{}_{\delta }{K}{}_{\sigma }{}^{\delta }-\epsilon {\alpha
}^{-1}{D}{}_{\nu }{D}{}_{\sigma }\alpha -{\alpha }^{-1}{\hbox{\it\char36}}_{\bf
\beta }{K}{}_{\nu }{}_{\sigma }}\right)\cr &-{n}{}_{\nu }{n}{}_{\rho
}\left({{\alpha }^{-1}{\hbox{\it\char36}}_{\bf t}{K}{}_{\mu }{}_{\sigma
}+{K}{}_{\mu }{}_{\delta }{K}{}_{\sigma }{}^{\delta }-\epsilon {\alpha
}^{-1}{D}{}_{\mu }{D}{}_{\sigma }\alpha -{\alpha }^{-1}{\hbox{\it\char36}}_{\bf
\beta }{K}{}_{\mu }{}_{\sigma }}\right)\cr &+{n}{}_{\nu }{n}{}_{\sigma
}\left({{\alpha }^{-1}{\hbox{\it\char36}}_{\bf t}{K}{}_{\mu }{}_{\rho
}+{K}{}_{\mu }{}_{\delta }{K}{}_{\rho }{}^{\delta }-\epsilon {\alpha
}^{-1}{D}{}_{\mu }{D}{}_{\rho }\alpha -{\alpha }^{-1}{\hbox{\it\char36}}_{\bf
\beta }{K}{}_{\mu }{}_{\rho }}\right)\cr &-{n}{}_{\mu }{n}{}_{\sigma
}\left({{\alpha }^{-1}{\hbox{\it\char36}}_{\bf t}{K}{}_{\nu }{}_{\rho
}+{K}{}_{\nu }{}_{\delta }{K}{}_{\rho }{}^{\delta }-\epsilon {\alpha
}^{-1}{D}{}_{\nu }{D}{}_{\rho }\alpha -{\alpha }^{-1}{\hbox{\it\char36}}_{\bf
\beta }{K}{}_{\nu }{}_{\rho }}\right).\cr} \eqn{}
$$
The Ricci tensor is defined by contracting the first and third indices on the
Riemann tensor using the full metric.  Since the spatial Riemann tensor is
purely spatial, this is equivalent to contracting it with the spatial metric. 
This leads to
$$
\eqalign{{\dimR{4}}{}_{\mu }{}_{\nu }&={\dimR{3}}{}_{\mu }{}_{\nu
}-\epsilon K{K}{}_{\mu }{}_{\nu }+\epsilon 2{K}{}_{\mu }{}_{\rho
}{K}{}_{\nu }{}^{\rho }+\epsilon {\alpha }^{-1}{\hbox{\it\char36}}_{\bf
t}{K}{}_{\mu }{}_{\nu }-{\alpha }^{-1}{D}{}_{\mu }{D}{}_{\nu }\alpha \cr
&-\epsilon {\alpha }^{-1}{\hbox{\it\char36}}_{\bf \beta }{K}{}_{\mu
}{}_{\nu }+\epsilon {n}{}_{\mu }\left({{D}{}_{\nu }K-{D}{}_{\rho }{K}{}_{\nu
}{}^{\rho }}\right)+\epsilon {n}{}_{\nu }\left({{D}{}_{\mu }K-{D}{}_{\rho
}{K}{}_{\mu }{}^{\rho }}\right)\cr &+{n}{}_{\mu }{n}{}_{\nu }\left({{\alpha
}^{-1}{\hbox{\it\char36}}_{\bf t}K-{K}{}_{\rho }{}^{\sigma }{K}{}_{\sigma
}{}^{\rho }-\epsilon {\alpha }^{-1}{D}{}^{\rho }{D}{}_{\rho }\alpha -{\alpha
}^{-1}{\hbox{\it\char36}}_{\bf \beta}K}\right),\cr} \eqn{}
$$
where the trace of the extrinsic curvature
$$
K={g}{}^{\mu }{}^{\nu }{K}{}_{\mu }{}_{\nu }={\gamma }{}^{\mu }{}^{\nu
}{K}{}_{\mu }{}_{\nu }. \eqn{}
$$
has been used.  Finally, the Ricci scalar is defined by tracing the Ricci tensor
giving
$$
\dimR{4}=\dimR{3}-\epsilon {K}^{2}-\epsilon {K}{}_{\rho
}{}^{\sigma }{K}{}_{\sigma }{}^{\rho }+\epsilon 2{\alpha
}^{-1}{\hbox{\it\char36}}_{\bf t}K-2{\alpha }^{-1}{D}{}^{\rho }{D}{}_{\rho
}\alpha -\epsilon 2{\alpha }^{-1}{\hbox{\it\char36}}_{\bf \beta }K.
\eqn{}
$$

With these results, we can begin splitting Einstein's equations.  In fully
covariant form, Einstein's equations are written
$$
{G}{}_{\mu }{}_{\nu }={\dimR{4}}{}_{\mu }{}_{\nu }-{1 \over 2}{g}{}_{\mu
}{}_{\nu }\dimR{4}=\kappa {T}{}_{\mu }{}_{\nu }, \eqn{}
$$
where $T_{\mu\nu}$ is the stress energy tensor of the sources and $\kappa$ is the
proportionality constant which, in gravitational units $(G=c=1)$, is
$\kappa=8\pi$.  We can decompose the stress energy tensor as
$$
{T}{}_{\mu }{}_{\nu }={S}{}_{\mu }{}_{\nu }+2{n}{}_{(\mu }{j}{}_{\nu
)}+{n}{}_{\mu }{n}{}_{\nu }\rho, \eqn{}
$$
where the spatial stress $S_{\mu\nu}$, momentum density $j_\mu$, and energy
density $\rho$ are defined as
$$
{S}{}_{\mu }{}_{\nu }\equiv \perp {T}{}_{\mu }{}_{\nu },	\eqn{}
$$
$$
{j}{}_{\mu }\equiv \epsilon \perp {T}{}_{\mu }{}_{\nu }{n}{}^{\nu }, \eqn{}
$$
and
$$
\rho \equiv {n}{}^{\mu }{n}{}^{\nu }{T}{}_{\mu }{}_{\nu }. \eqn{}
$$
By tracing Einstein's equations and the decomposition of the stress energy
tensor, we find that
$$
\dimR{4}=-\kappa T=-\kappa \left({S+\epsilon \rho }\right), \eqn{}
$$
where $T$ is the trace of the stress energy tensor and $S$ is the trace of the
spatial stress tensor.  Combining (2.50), (2.51), and (2.55) Einstein's
equations take the form
$$
{\dimR{4}}{}_{\mu }{}_{\nu }=\kappa \left({{S}{}_{\mu }{}_{\nu
}+2{n}{}_{(\mu }{j}{}_{\nu )}-{1 \over 2}{\gamma }{}_{\mu }{}_{\nu
}\left({S+\epsilon \rho }\right)-{1 \over 2}\epsilon {n}{}_{\mu
}{n}{}_{\nu }\left({S-\epsilon\rho }\right)}\right). \eqn{}
$$

The final step in splitting Einstein's equations is accomplished by taking all
possible projections of (2.56).  First, if we contract both indices with the
unit normal, we find
$$
\eqalign{{\dimR{4}}{}_{\mu }{}_{\nu }{n}{}^{\mu }{n}{}^{\nu }&={\alpha
}^{-1}{\hbox{\it\char36}}_{\bf t}K-{K}{}_{\mu }{}^{\nu }{K}{}_{\nu }{}^{\mu
}-\epsilon {\alpha }^{-1}{D}{}^{\mu }{D}{}_{\mu }\alpha -{\alpha
}^{-1}{\hbox{\it\char36}}_{\bf \beta }K\cr &=-{\epsilon  \over
2}\left({\dimR{3}-\epsilon {K}^{2}+\epsilon {K}{}_{\mu }{}^{\nu
}{K}{}_{\nu }{}^{\mu }+\kappa T}\right)\cr &=-{\epsilon \kappa  \over
2}\left({S-\epsilon\rho }\right),\cr} \eqn{}
$$
where (2.49) and (2.55) have been used.  Simplifying, this gives
$$
\dimR{3}-\epsilon {K}^{2}+\epsilon {K}{}_{\mu }{}^{\nu }{K}{}_{\nu
}{}^{\mu }=-\epsilon 2\kappa \rho. \eqn{}
$$
By contracting one index with the unit normal and spatially projecting the
other, we find
$$
\eqalign{\perp {\dimR{4}}{}_{\mu }{}_{\nu }{n}{}^{\nu }&={D}{}_{\mu
}K-{D}{}_{\nu }{K}{}_{\mu }{}^{\nu }\cr &=\epsilon \kappa {j}{}_{\mu
},\cr}\eqn{}
$$
which can be expressed more conveniently as
$$
{D}{}_{\nu }\left({{K}{}^{\mu }{}^{\nu }-{\gamma }{}^{\mu }{}^{\nu
}K}\right)=-\epsilon \kappa {j}{}^{\mu }. \eqn{}
$$
The final projection of the Einstein equation comes from spatially projecting
both indices where we find
$$
\eqalign{\perp {\dimR{4}}{}_{\mu }{}_{\nu }&={\dimR{3}}{}_{\mu }{}_{\nu
}-\epsilon K{K}{}_{\mu }{}_{\nu }+\epsilon 2{K}{}_{\mu }{}_{\rho
}{K}{}_{\nu }{}^{\rho }+\epsilon {\alpha }^{-1}{\hbox{\it\char36}}_{\bf
t}{K}{}_{\mu }{}_{\nu }\cr &\qquad-{\alpha }^{-1}{D}{}_{\mu }{D}{}_{\nu }\alpha
-\epsilon {\alpha }^{-1}{\hbox{\it\char36}}_{\bf \beta }{K}{}_{\mu
}{}_{\nu }\cr &=\kappa \left({{S}{}_{\mu }{}_{\nu }-{1 \over 2}{\gamma }{}_{\mu
}{}_{\nu }\left({S+\epsilon \rho }\right)}\right),\cr} \eqn{}
$$
which can be rewritten as
$$
\eqalign{{\hbox{\it\char36}}_{\bf t}{K}{}_{\mu }{}_{\nu }&=\epsilon
{D}{}_{\mu }{D}{}_{\nu }\alpha -\epsilon \alpha {{}_{}^{(3)}R}{}_{\mu
}{}_{\nu }-2\alpha {K}{}_{\mu }{}_{\rho }{K}{}_{\nu }{}^{\rho }+\alpha
K{K}{}_{\mu }{}_{\nu }\cr &\qquad+{\hbox{\it\char36}}_{\bf \beta }{K}{}_{\mu
}{}_{\nu }+\epsilon \alpha \kappa \left({{S}{}_{\mu }{}_{\nu }-{1 \over
2}{\gamma }{}_{\mu }{}_{\nu }\left({S+\epsilon \rho }\right)}\right).\cr}
\eqn{}
$$

Equations (2.58), (2.60), and (2.62) are the projections of Einstein's
equations.  Equations (2.58) and (2.60) contain no time derivatives and are,
respectively, the Hamiltonian and momentum constraint equations.  The spatial
metric and extrinsic curvature must satisfy these equations on every slice in
the foliation.  Equation (2.62) describes the time evolution of the extrinsic
curvature.  Apparently missing from the decomposition of Einstein's equations is
the equation describing the evolution of the spatial metric.  This, however, is
simply obtained from the definition of the extrinsic curvature (2.26), which
gives us
$$
{\hbox{\it\char36}}_{\bf t}{\gamma }{}_{\mu }{}_{\nu }=-2\alpha {K}{}_{\mu
}{}_{\nu }+{\hbox{\it\char36}}_{\bf \beta }{\gamma }{}_{\mu }{}_{\nu }.
\eqn{}
$$

To this point, the basis of vectors ${\overline{e}}_\mu$ and its dual basis, the
basis of forms ${\etilda}^\mu$ defined by
$$
\left\langle{{\etilda}^\mu,{\overline{e}}{}_{\nu
}}\right\rangle={\delta }_{\nu }^{\mu }, \eqn{}
$$
have been assumed to be completely general.  To proceed, we will specialize the
basis of vectors by splitting it into a purely spatial set of basis vectors plus
a timelike basis vector which is orthogonal to the spatial basis.  We choose the
timelike basis vector to be the timelike vector $t^\mu$:
$$
\overline{t}={t}{}^{\mu }{\overline{e}}{}_{\mu }={\overline{e}}{}_{0}. \eqn{}
$$
The remaining three basis vectors are chosen to be purely spatial and must
satisfy
$$
\left\langle{\Omegatilda,{\overline{e}}{}_{i}}\right\rangle={\Omega
}{}_{\nu }\left\langle{{\etilda}{}^{\nu},{\overline{e}}{}_{i}}\right\rangle=0,
\eqn{}
$$
where $\{i=1,2,3\}$ designates the three spatial basis vectors.  The final demand
is that the timelike basis vector must commute with the spatial basis vectors
$$
\left[{\overline{t},{\overline{e}}{}_{i}}\right]={\hbox{\it\char36}}_{\bf
t}{\overline{e}}{}_{i}=0. \eqn{}
$$
If we now consider
${\hbox{\it\char36}}_{\bf
t}\left\langle{\Omegatilda,{\overline{e}}{}_{i}}\right\rangle$ and use this last
demand, we find
$$
\eqalign{{\hbox{\it\char36}}_{\bf t}\left\langle{\Omegatilda
,{\overline{e}}{}_{i}}\right\rangle&=\left\langle{{\hbox{\it\char36}}_{\bf
t}\Omegatilda,{\overline{e}}{}_{i}}\right\rangle=\left({{\hbox{\it\char36}}_{\bf
t}{\Omega }{}_{\nu }}\right)\left\langle{\etilda^{\nu
},{\overline{e}}{}_{i}}\right\rangle\cr &=\left({{t}{}^{\rho }{\nabla }{}_{\rho
}{\Omega }{}_{\nu }+{\Omega }{}_{\rho }{\nabla }{}_{\nu }{t}{}^{\rho
}}\right)\left\langle{\etilda^{\nu
},{\overline{e}}{}_{i}}\right\rangle\cr &=2{t}{}^{\rho }{\nabla }{}_{[\rho
}{\Omega }{}_{\nu ]}\left\langle{\etilda^{\nu
},{\overline{e}}{}_{i}}\right\rangle=0,\cr} \eqn{}
$$
so the spatial basis vectors remain purely spatial as they are dragged along
$t^\mu$.  Finally, because ${\overline{e}}{}_{0}$ is a coordinate basis vector
(2.65) and since it commutes with the remaining basis vectors (2.67), we find
that the effect of the Lie derivative along $t^\mu$ on {\it any} tensor is the
action of partial differentiation,
$$
{\hbox{\it\char36}}_{\bf t}{T}{}^{\alpha \cdots\beta }{}_{\rho
\cdots\sigma }={\partial }{}_{t}{T}{}^{\alpha \cdots\beta }{}_{\rho \cdots\sigma
}. \eqn{}
$$

With this choice of the bases, we can examine the components of the important
tensors.  From (2.65) we find that the components of the timelike vector $t^\mu$ are
$$
{t}{}^{\mu }=\left[{\matrix{1&0&0&0\cr}}\right]. \eqn{}
$$
Equation (2.66) requires
$$
{n}{}_{i}=0. \eqn{}
$$
Since the contraction of $n_\mu$ with any contravariant index of a spatial tensor
must vanish, this implies that any zeroth contravariant component of a spatial
tensor must vanish.  For example, the components of the shift vector must be
$$
{\beta }{}^{\mu }=\left[{\matrix{0&{\beta }{}^{i}\cr}}\right]. \eqn{}
$$
Given the definition of the timelike vector (2.41), its components (2.70), and
the components of the shift (2.72), we find that the components of the unit
normal vector must be
$$
{n}{}^{\mu }=\left[{\matrix{{\alpha }^{-1}&-{\alpha }^{-1}{\beta
}{}^{i}\cr}}\right]. \eqn{}
$$
From (2.5) we have $n_\mu n^\mu = \epsilon$, combined with (2.71) and (2.73)
these imply
$$
{n}{}_{\mu }=\left[{\matrix{\epsilon \alpha &0&0&0\cr}}\right]. \eqn{}
$$
Equation (2.71) also implies from the definition of the spatial metric (2.6)
that the spatial components of the spatial metric are identically the spatial
components of the full metric
$$
{\gamma }{}_{i}{}_{j}={g}{}_{i}{}_{j}. \eqn{}
$$
Since any zeroth component of a contravariant tensor must vanish, we find that
the components of the inverse spatial metric must be
$$
{\gamma }{}^{\mu }{}^{\nu }=\left[{\matrix{0&0\cr
0&{\gamma }{}^{i}{}^{j}\cr}}\right], \eqn{}
$$
and the components of the full inverse metric must be
$$
{g}{}^{\mu }{}^{\nu }=\left[{\matrix{\epsilon {\alpha }^{-2}&-\epsilon
{\alpha }^{-2}{\beta }{}^{j}\cr -\epsilon {\alpha }^{-2}{\beta
}{}^{i}&{\gamma }{}^{i}{}^{j}+\epsilon {\alpha }^{-2}{\beta }{}^{i}{\beta
}{}^{j}\cr}}\right]. \eqn{}
$$
If we consider $\gamma^{\mu\rho}g_{\rho\nu}=(g^{\mu\rho}-\epsilon n^\mu
n^\rho)g_{\rho\nu}=\delta_\nu^\mu-\epsilon n^\mu\delta_\nu^0$, use (2.75) and
(2.76), and restrict to spatial indices, we find
$$
{\gamma }{}^{i}{}^{k}{\gamma }{}_{k}{}_{j}={\delta }_{j}^{i}. \eqn{}
$$
This means that the spatial components of the metric and inverse metric are
three-dimensional inverses and can be used to raise and lower spatial indices on
spatial tensors.  Using this property to define the spatial, covariant form of
the shift vector, $\beta_i$, as
$$
{\beta }{}_{i}={\gamma }{}_{i}{}_{j}{\beta }{}^{j}, \eqn{}
$$
we find the components of the full metric to be
$$
{g}{}_{\mu }{}_{\nu }=\left[{\matrix{\epsilon {\alpha }^{2}+{\beta
}{}_{\ell }{\beta }{}^{\ell }&{\beta }{}_{j}\cr {\beta }{}_{i}&{\gamma
}{}_{i}{}_{j}\cr}}\right], \eqn{}
$$
and so the line element is
$$
{ds}^{2}={\gamma }{}_{i}{}_{j}\left({{dx}{}^{i}+{\beta
}{}^{i}dt}\right)\left({{dx}{}^{j}+{\beta }{}^{j}dt}\right)+\epsilon {\alpha
}^{2}{dt}^{2}. \eqn{}
$$
The entire content of the decomposed Einstein equations is now expressible as
the spatial components of (2.58), (2.60), (2.62), and (2.63):
$$
R-\epsilon {K}^{2}+\epsilon
{K}{}_{i}{}_{j}{K}{}^{i}{}^{j}=-\epsilon 2\kappa \rho, \eqn{}
$$
$$
{D}{}_{j}\left({{K}{}^{i}{}^{j}-{\gamma }{}^{i}{}^{j}K}\right)=-\epsilon
\kappa {j}{}^{i}, \eqn{}
$$
$$
\eqalign{{\partial }{}_{t}{K}{}_{i}{}_{j}&=\epsilon {D}{}_{i}{D}{}_{j}\alpha
-\epsilon \alpha \left[{{R}{}_{i}{}_{j}+\epsilon 2{K}{}_{i}{}_{\ell
}{K}{}_{j}{}^{\ell }-\epsilon K{K}{}_{i}{}_{j}-\kappa {S}{}_{i}{}_{j}+{1
\over 2}\kappa{\gamma }{}_{i}{}_{j}\left({S+\epsilon \rho }\right)}\right]\cr
&\qquad\qquad\qquad+{\beta }{}^{\ell }{D}{}_{\ell
}{K}{}_{i}{}_{j}+{K}{}_{i}{}_{\ell }{D}{}_{j}{\beta }{}^{\ell }+{K}{}_{\ell
}{}_{j}{D}{}_{i}{\beta }{}^{\ell }\cr}, \eqn{}
$$
and
$$ {\partial }{}_{t}{\gamma }{}_{i}{}_{j}=-2\alpha
{K}{}_{i}{}_{j}+{D}{}_{i}{\beta }{}_{j}+{D}{}_{j}{\beta }{}_{i}, \eqn{}
$$
where we have dropped the label on the Ricci tensor and Ricci scalar which
distinguish them from their four-dimensional counterparts. 
\vfill
\eject