\chapterhead{Chapter~\the\chapnum:  Conclusion}
%%%%%%%%%%%%%%%%%%%%%%%%
The primary motivation for undertaking an investigation of methods for computing
initial-data sets containing one or more black holes is the desire to simulate
black hole collisions.  In particular, we wish to evolve  data which represent
two black holes in a bound orbit, spiralling toward a collision.  During this
evolution, we wish to examine the gravitational radiation which is being emitted
during the spiralling infall, the collision, and the subsequent transition to
quiescence.  With this in mind, I will discuss the usefulness and limitations of
initial-data sets constructed following the conformal imaging approach.  I will
also discuss some of the problems remaining to be investigated regarding the
generation and evolution of these initial-data sets.

The most obvious advantage of the conformal imaging approach over other methods
for constructing initial-data sets containing black holes ({\it cf}. Thornburg
[1987]) is that it allows for the construction of initial data on a {\it
complete} manifold.  During an evolution of this initial data, more of the
global structure of the space-time should be accessible because the initial-data
manifold is complete.  The ``price'' which is paid for having a complete
manifold is that all of the fields which reside on the manifold must obey an
isometry condition which identifies the physical fields on the two ``sheets''
(or universes) which the manifold comprises.  At first, this identification may
seem unnatural.  The imposition of such an isometry condition is possible only
when topological vacuum solutions are considered.  If the black hole is produced
from a matter source, then there can be no ``alternate'' universe to identify
with the ``physical'' universe.  The existence of multiple, causally
disconnected, asymptotically flat regions seems to be a generic feature of
vacuum black-hole space-times.  Unless there is some reason that these
additional universes should be considered physically relevant and distinct from
the ``real'' universe, it actually seems most natural to identify
these causally disconnected regions mathematically.

One of the most important reasons for pursuing the evolution of the two black
hole collision problem is the desire to obtain accurate waveforms for the
gravitational radiation which results from the collision.  In order to be able
to compare waveforms generated by numerical means confidently against waveforms
which should be available in the future from gravity-wave detectors, it will be
important to be able to distinguish between waveforms which result from the
dynamics of the evolution and those which are present in the initial data.  As
has been shown, initial-data sets constructed by the conformal imaging approach
must contain gravitational radiation in the field exterior to the apparent
horizon.  This is not a condition inherent only to the conformal imaging
approach, however.  Any initial-data sets constructed on a space-like slice will
most likely contain gravitational radiation.  Even if the trace-free background
extrinsic curvature is purely longitudinal, the physical extrinsic curvature may
have a non-vanishing, transverse-traceless part.

Since these initial-data sets do contain gravitational radiation from the start,
it will be necessary to determine its content in some quantitative way so that
its contribution to the dynamical waveform can be determined.  It is likely
that any gravitational radiation present in the initial slice can be made to
propagate away from or fall into the holes before any of the interesting
dynamics takes place.  In addition, some hope for finding methods of examining
the gravitational radiation content of initial-data sets has been expressed by
Abrahams [1990] based on experience in extracting waveforms from numerical
evolutions of Einstein's equations ({\it cf}. Abrahams and Evans [1988],
[1990]).  I feel that this avenue of investigating the radiation content of the
initial data should be vigorously pursued.

Beyond these questions regarding the content of the initial-data sets, there are
problems which have yet to be addressed regarding their numerical generation. 
The research described in this work has dealt with the generation of
axisymmetric initial-data sets.  I have found a general, explicit approach for
evaluating the two-hole, inversion-symmetric extrinsic curvature and I have
demonstrated two approaches for solving the two-hole Hamiltonian constraint
numerically.  With regard to these axisymmetric data sets, the task of
increasing the accuracy of the numerical solutions still remains as does the
problem of locating the apparent horizons in these data sets.

In order to model the {\it spiralling} collision of two black holes,
three-dimensional initial-data sets will be required.  Many of the techniques
developed for constructing axisymmetric data sets can be applicable to the
three-dimensional problems.  The method I have developed for evaluating  the
axisymmetric, inversion-symmetric extrinsic curvature is directly generalizable
to three-dimensional situations.  The general approaches for differencing the
Hamiltonian constraint should also be applicable.  However, some effort will be
needed with regard to differencing near the coordinate singularity along the
$z$-axis in the absence of axisymmetry.  Also, much work will be required in
finding an efficient approach for solving the resulting difference equations. 
The size of this set of equations will be significantly larger than that
resulting from differencing the axisymmetric problem and will tax the memory
limitations of even the most powerful supercomputers.

I would like to turn now to the problem of evolving inversion-symmetric
initial-data sets.  As was mentioned earlier, the use of initial data on a {\it
complete} manifold should allow for more of the global space-time to be covered
during a numerical evolution.  Initial data constructed by the conformal imaging
approach allows for the efficient construction of initial data on a complete
manifold because only half of the full manifold must be considered.  The other
half is known by identification through the isometry condition.  Because the two
sheets of the manifold contain identical Cauchy data, the two causally
disconnected spacetimes which evolve from these data will be identical and
there will therefore exist a spacetime isometry between them.  If {\it efficient}
use is to be made of the initial data, however, then each spatial slice in the
subsequent evolution should {\it also} consist of a complete three-manifold on
which the data obey a spatial isometry condition which is either similar or
identical to the isometry condition on the initial-data slice.  This means that
the question of the compatibility of the isometry condition and the evolution
equations must be considered.

A full investigation of this problem is beyond the scope of this concluding
chapter, but I do wish to discuss some of the more general points.  To begin
with, all of the fields which are relevant to the evolution equations will be
required to satisfy some isometry condition.  The isometry condition includes,
in some cases, the freedom of choice in the sign of the identified field.  For
example, we have seen in Chapter~4 that the extrinsic curvature can obey the
isometry condition with either a plus or a minus sign.  The metric, on the other
hand, must obey the isometry condition only with a plus sign in order for the
metric to be non-singular on the inversion surfaces.  To determine the sign of
the isometry condition for the other fields we must examine the vacuum
Einstein equations
$$
R + {2 \over 3}K^2 - A_{ij}A^{ij} = 0, \eqn{}
$$
$$
D_j(A^{ij} - {2 \over 3}\gamma^{ij}K) = 0, \eqn{}
$$
$$
\eqalign{\partial_tA_{ij} = &-{D_i}{D_j}\alpha + \alpha R_{ij} - 2\alpha
A_{i\ell}{A_j}^{\ell} + \alpha A_{ij}({1 \over 3}K) + 3\alpha \gamma_{ij}({1
\over 3}K)^2\cr & - \gamma_{ij}\partial_t({1 \over 3}K) + \gamma_{ij}\beta^\ell
D_\ell({1 \over 3}K) + \beta^\ell D_\ell A_{ij} + A_{i\ell}D_j\beta^\ell +
A_{\ell j}D_i\beta^\ell,} \eqn{}
$$
and
$$
\partial_t\gamma_{ij} = -2\alpha A_{ij} - {2 \over 3}\alpha\gamma_{ij}K +
D_i\beta_j + D_j\beta_i. \eqn{}
$$
We see immediately that the lapse $\alpha$ and the trace of the extrinsic
curvature $K$ must obey the isometry condition with the same sign as the
extrinsic curvature
$$
\alpha({\bf x}) = \pm\alpha({\bf J}_\alpha({\bf x})) \eqn{}
$$
and
$$
K({\bf x}) = \pm K({\bf J}_\alpha({\bf x})). \eqn{}
$$
(The subscript $\alpha$ on the isometry condition ${\bf J}_\alpha$ designates
the throat through which the isometry is applied and should not be confused with
the lapse function.)  We also find that the shift vector $\beta^i$ must obey the
isometry condition with the plus sign
$$
\beta^i({\bf x}) = {({\bf J}_\alpha^{-1})_j}^i\beta^j({\bf J}_\alpha({\bf x})).
\eqn{} $$

The isometry conditions (15.5)--(15.7) along with similar conditions for the
metric and extrinsic curvature (4.38) and (4.39) can be used to generate
boundary conditions which these fields must obey at the inversion surfaces which
are fixed point sets of the isometry.  Using the isometry condition which we
know is obeyed on the initial hypersurface and going to a local spherical
coordinate system $(r,\theta,\phi)$ centered around any given throat, we can
read off boundary conditions for each component of each field.  The boundary
condition will be such that either the normal derivative of a given component
vanishes at the throat or the value of the component vanishes at the throat. 
Since there is no choice in the sign of the isometry condition for the metric,
we find immediately that the normal derivatives of $\gamma_{rr}$,
$\gamma_{\theta\theta}$, $\gamma_{\phi\phi}$, and $\gamma_{\theta\phi}$ must
vanish at a throat and $\gamma_{r\theta}$ and $\gamma_{r\phi}$ must vanish in
value at a throat.  There is also no choice in the sign of the isometry
condition for the shift vector and we find that $\beta^r$ must vanish in value
at a throat and also that the normal derivatives of $\beta^\theta$ and
$\beta^\phi$ must vanish at a throat.  The boundary condition for the remaining
quantities depends on the choice in sign for the isometry condition.  If the
isometry condition with a plus sign is chosen for $\alpha$, $K$, and $A_{ij}$,
then we find that the normal derivatives of $\alpha$, $K$, $A_{rr}$,
$A_{\theta\theta}$, $A_{\phi\phi}$, and $A_{\theta\phi}$ must vanish on a throat
and so $A_{r\theta}$ and $A_{r\phi}$ must vanish in value on a throat.  If the
isometry condition with a minus sign is chosen, then the conditions are
reversed.  Boundary conditions of this sort are being used by Bernstein and
Hobill [1990] in their investigations into the evolution of a single, perturbed
black hole.

We can now consider whether the imposition of these boundary conditions in
conjuction with the evolution equations (15.3) and (15.4) is sufficient to
guarantee that the fields on subsequent slices will be inversion symmetric.  The
imposition of the boundary conditions described above should ensure that the
surfaces identified with  fixed-point sets of the isometry remain fixed-point
sets throughout the evolution.  What it may not do, however, is determine what
the explicit form of the isometry conditions (or maps) will be.  It is the
explicit form of the isometry maps which, given a coordinate map defined by the
isometry, allows for the smooth continuation of fields through the inversion
surfaces.  This, in turn, allows global structures such as apparent horizons,
which may cross the inversion surface, to be located.

To exploit fully the conformal imaging approach, it will be necessary to 
understand better the relation between the isometry maps and the evolution
equations, and also their relation with the constraint equations on evolved
slices which will not be conformally flat.  While I do not claim to have come to
any complete understanding of this problem, I would like to discuss a few
important points.  It seems likely to me that the explicit form of the isometry
condition will be dependent on the choice of the shift vector.  This can be seen
from the fact that the isometry condition is defined to identify points in the
two identical sheets which the manifold comprises.  Since the shift vector
determines which coordinate point is associated with a physical point in the
manifold, it must be closely associated with the evolution of the isometry
condition.  Another point of consideration is the connection between the
isometry condition and the choice of hypersurfaces throughout the evolution.  It
is evident that not just any general hypersurface will allow for the
identification of data in the two sheets.  Thus, the choice of the lapse
function must be made in such a way which as to be compatible with the
isometry.  Some clue to possible constraints on the choice of the lapse function
may be found in the evolution equations.

A clear understanding of the relationships mentioned above will be very
important to our undertaking of the evolution of these inversion-symmetric
initial-data sets.  A successful combination of isometry conditions and the
evolution equations will allow for evolution on complete manifolds which I feel
will be of benefit.  We know implicitly, from the uniqueness of the evolution
of the Cauchy data, that the two universes are identical.  What technical
difficulties exist in finding an explicit, spatial isometry condition remain to
be determined.  In any event, it is clear that the work presented here is but a
small part of the effort which will be required in the quest for the evolution
of a spiralling binary coalescence.


\vfill
\eject