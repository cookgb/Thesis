\chapterhead{Chapter~\the\chapnum:  Inversion-Symmetric, Two-Hole\cr
Initial-Data Sets}
%%%%%%%%%%%%%%%%%%%%%%%%%%%%%%
The previous chapter discussed the physical content and interpretation of the
one-hole initial-data sets.  The successful interpretation of these data sets
was facilitated by the understanding of closely related analytic initial data
(the boosted Schwarzschild and the Kerr solutions) to which the numerical data
could be compared.  Furthermore, these data sets do not involve a gravitational
interaction with another object.  In the case of two holes, each with linear
or angular momentum, the task of interpreting the content of the initial data
sets is not aided in any of these ways.  There are no known analogous exact
solutions for the case of two holes with linear or angular momentum to which
the initial data can be compared.  Further, one's intuition must be suspect in
the face of the nonlinearities and non-localizability inherent in Einstein's
equations.  With this in mind, I will strive to attach some interpretation to
the content of the two-hole initial-data sets, the construction of which was
detailed in previous chapters.

The initial-data sets which will be discussed below were all constructed
following the conformal imaging approach.  The topology of the initial-data
slice is thus that of two asymptotically flat universes connected by two
Einstein-Rosen bridges.  In addition, physical fields in the two universes are
identified, leading to ``inversion symmetry'' in the initial data.  Because of
current computational limitations, only axisymmetric configurations of the two
holes are considered.  Thus, we are limited to considering a pair of holes with
linear and angular momentum vectors aligned parallel to the axis joining the
centers of the two holes.  Further, to reduce the size of the parameter space
governing the initial configuration, I will consider here only the case in
which the two holes are of equal size and have linear or angular momenta of
equal magnitude.  Recalling the choice in sign when applying the isometry
condition to the extrinsic curvature ({\it cf}.  Chapters~5 and 6) and only
considering the application of linear or angular momenta separately, we find
six separate classes of configurations to consider.  The first two cases are
defined by the holes having linear momenta generated from the extrinsic
curvature obeying the isometry condition with a plus sign.  The first case has
the linear momentum vectors aligned anti-parallel to each other so that there
is no net momentum at infinity, and the second case has the linear momentum
vectors aligned parallel to each other.  The second pair of cases are defined
by the holes having linear momenta generated from the extrinsic curvature
obeying the isometry condition with a minus sign.  Again, the two cases are
distinguished by having the linear momentum vectors aligned either
anti-parallel or parallel.  The final pair of cases represent holes with
angular momenta aligned either anti-parallel or parallel.  For clarity and
brevity in the discussion below, I will refer to these configurations,
respectively, as Cases 1 through 6.

For each of these cases, there are two remaining parameters to be explored. 
The separation between the two holes can be varied and this quantity is roughly
parameterized by the ``conformal separation'' parameter $\beta$ ({\it cf}.
equation (9.1)).  The second parameter to be varied in each initial-data set is
the magnitude of the linear momentum $P$ or angular momentum $S$  for each of
the holes.  We recall from Chapter~5 that the two-hole solution for the
inversion-symmetric extrinsic curvature was constructed from the superposition
of {\it two} single-hole solutions with the addition of image terms which carry
no momenta.  In this way, each of the holes is assigned its own value of linear
and angular momentum.  An observer at infinity will measure a total linear or
angular momentum for the system which is given by the vector sum of the
individual linear or angular momenta.  A point which must be kept in mind,
however, is that when the two holes are close to each other, the value of the
linear or angular momentum assigned to each hole may not have a well-defined
physical interpretation.

For each point in the parameter space, there are three important physical
quantities which can be directly calculated.  First is the total ADM energy $E$
of the slice.  To be completely explicit, this is equivalent to the Keplerian
mass of the system which an observer at infinity will measure by examining the
orbits of test particles.  The second quantity which can be measured is the
mass of the apparent horizon surrounding each hole.  At present, the search for
apparent horizons in these data sets has not been undertaken and we can only
approximate it with the mass of the minimal surface which will be a lower bound
on the mass of the apparent horizon.  The third quantity which is measured is
the proper separation $L$, defined here as the minimum proper distance between
the two minimal surfaces.  It would be more nearly correct to define the proper
separation as the minimum distance between the apparent horizons surrounding
the two holes; however, as mentioned above, the positions of the apparent
horizons are not yet known.  A fourth quantity which can be measured is the
dipole moment of the system.  However, because of the symmetry in the problem,
this quantity is identically zero.

Given this information about an initial-data set, the behavior of many
quantities can be explored.  As in the case of a single hole, one must be
careful when examining any quantity individually as a function of the
configuration parameters.  We would like to use some kind of {\it natural}
scaling as was used in the case of a single hole.  The separation of the two
holes, for example, is less meaningful than the separation-to-mass ratio.  The
question, of course, is how to scale the quantities in such a way as to clarify
the problem.

In addition to the total energy of the initial slice, there are two other basic
masses in the problem.  The first is the mass of the apparent horizon which is
a measure of the proper surface area of the hole.  The second is what I would
call the mass of the hole itself and is given by the Christodoulou [1970]
formula (13.2).  In scaling the total energy, it is perhaps most natural to
choose the Christodoulou mass $M$ of a single hole as the natural scaling.  If
we consider the case of two Kerr holes infinitely far apart, then $E$/M for the
system will be {\it two} regardless of the magnitude of the angular momenta on
the holes.  The correct choice for scaling the proper separation is less clear. 
For consistency, I choose to scale it with respect to the Christodoulou mass as
well.

Figures 14.1--14.6 below display the naturally scaled physical quantities
corresponding respectively to Cases 1--6.  Each plot is derived from Richardson
extrapolated initial data tabulated in Appendix~C.  Figures 14.1--14.4 show all
quantities as a function of the scaled linear momentum $P/M$ on {\it one} of the
holes.  In these cases where the holes only have linear momentum, the mass of
the hole $M$ and the mass of the apparent horizon $M_{\scriptscriptstyle AH}$
are the same.  For Figures~14.5 and 14.6, all quantities are given as a function
of the scaled angular momentum $S/M^2$ on {\it one} of the holes.  The mass in
these two cases is the Christodoulou mass of {\it one} of the holes.  It is
given by (13.2), where the irreducible mass is approximated by the mass of the
minimal surface and the spin of the individual hole is given by the angular
momentum parameter $S$.  In all plots, 10 separate configurations are
displayed, each representing a different value of the separation parameter
$\beta$.

%\figlabel{7.25truein}{Figure~14.1:  (a) total energy, (b) separation, (c)
%maximum radiation energy,\cr and (d) maximum radiation efficiency for two holes
%with linear momenta $P$\cr aligned anti-parallel to each other and generated by
%an inversion-symmetric\cr extrinsic curvature obeying the isometry condition
%with a plus sign.}
\figlabelpdf{7.25truein}{Figure~14.1:  (a) total energy, (b) separation, (c)
maximum radiation energy,\cr and (d) maximum radiation efficiency for two holes
with linear momenta $P$\cr aligned anti-parallel to each other and generated by
an inversion-symmetric\cr extrinsic curvature obeying the isometry condition
with a plus sign.}{H2plots/P+.pdf}

%\figlabel{7.25truein}{Figure~14.2:  (a)~total energy, (b)~separation,
%(c)~maximum radiation energy,\cr and (d)~maximum radiation efficiency for two
%holes with linear momenta $P$\cr aligned parallel to each other and generated
%by an inversion-symmetric\cr extrinsic curvature obeying the isometry condition
%with a plus sign.}
\figlabelpdf{7.25truein}{Figure~14.2:  (a)~total energy, (b)~separation,
(c)~maximum radiation energy,\cr and (d)~maximum radiation efficiency for two
holes with linear momenta $P$\cr aligned parallel to each other and generated
by an inversion-symmetric\cr extrinsic curvature obeying the isometry condition
with a plus sign.}{H2plots/P+2.pdf}

Each of the six figures contains four separate plots.  Plot~(a) in each figure
displays the scaled total ADM energy of the slice $E/M$.  Plot~(b) in each
figure displays the scaled proper separation of the two holes.  If we assume
that the two holes will eventually coalesce into a single boosted Schwarzschild
or Kerr hole, then we can use Hawking's [1971] area theorem to determine a
lower limit on the final mass of the resulting black hole.  Using the mass of
the minimal surface $M_{\scriptscriptstyle MS}$ as a lower limit on the masses
of the two initial event horizons, we find that the mass of the final event
horizon $M_{\scriptscriptstyle IR}$ must be larger than
$$
{M}_{\scriptscriptstyle IR}>\sqrt {2}{M}_{\scriptscriptstyle MS}.	\eqn{}
$$
If $S_1$ and $S_2$ are the signed magnitudes of the spins on the two initial
holes, then the final mass of the hole $M_f$ must be larger than
$$
{M}_{f}>{M}_{\scriptscriptstyle MS}\sqrt {2+{1 \over 8}{\left({{{S}_{1} \over
{M}_{\scriptscriptstyle MS}^{2}}+{{S}_{1} \over {M}_{\scriptscriptstyle
MS}^{2}}}\right)}^{2}}. \eqn{}
$$
If $P_1$ and $P_2$ are the signed magnitudes of the linear momenta of
the two initial holes, then an upper limit on the amount of energy which can
possibly be radiated from the system is
$$
{E}_{rad}<\sqrt {{E}^{2}-{({P}_{1}+{P}_{2})}^{2}}-{M}_{\scriptscriptstyle
MS}\sqrt {2+{1 \over 8}{\left({{{S}_{1} \over {M}_{\scriptscriptstyle
MS}^{2}}+{{S}_{1} \over {M}_{\scriptscriptstyle MS}^{2}}}\right)}^{2}}. \eqn{}
$$
The maximum possible radiation content of the slice scaled to the mass of one
of the holes $E_{rad}$ is displayed in Plot~(c).  Finally, Plot~(d) in each
figure displays the maximum radiation efficiency of each slice which is given by
$$
\hbox{efficiency}<{{E}_{rad} \over \sqrt {{E}^{2}-{({P}_{1}+{P}_{2})}^{2}}}.
\eqn{}
$$

Consider first Cases~1 and 3, which represent two holes with anti-parallel
linear momentum vectors.  These two cases, representing both choices in the
isometry, can represent either two holes headed directly toward each other or
directly away from each other.  Because the holes are of equal size and have
momenta of equal magnitude, the energy, mass, and separation are identical in
either instance.

%\figlabel{7.25truein}{Figure~14.3:  (a)~total energy, (b)~separation,
%(c)~maximum radiation energy,\cr and (d)~maximum radiation efficiency for two
%holes with linear momenta $P$\cr aligned anti-parallel to each other and
%generated by an inversion-symmetric\cr extrinsic curvature obeying the isometry
%condition with a minus sign.}
\figlabelpdf{7.25truein}{Figure~14.3:  (a)~total energy, (b)~separation,
(c)~maximum radiation energy,\cr and (d)~maximum radiation efficiency for two
holes with linear momenta $P$\cr aligned anti-parallel to each other and
generated by an inversion-symmetric\cr extrinsic curvature obeying the isometry
condition with a minus sign.}{H2plots/P-.pdf}

The first question related to these two cases is whether the two choices
for the isometry condition lead to two different global spacetimes or two
different slicings of the same global spacetime?  All four plots in each case
are qualitatively the same although there are discernible differences.  Because
we do not know the true mass of the apparent horizon, the question posed above
cannot be definitively answered.  Carefully examining the differences between
the two cases does show that if they are two different slicings of the same
spacetime, then the mass of the apparent horizon in Case~3 must increase from
its current estimate by a factor larger than the increase in Case~1.

%\figlabel{7.25truein}{Figure~14.4:  (a)~total energy, (b)~separation,
%(c)~maximum radiation energy,\cr and (d)~maximum radiation efficiency for two
%holes with linear momenta $P$\cr aligned parallel to each other and generated
%by an inversion-symmetric\cr extrinsic curvature obeying the isometry condition
%with a minus sign.}
\figlabelpdf{7.25truein}{Figure~14.4:  (a)~total energy, (b)~separation,
(c)~maximum radiation energy,\cr and (d)~maximum radiation efficiency for two
holes with linear momenta $P$\cr aligned parallel to each other and generated
by an inversion-symmetric\cr extrinsic curvature obeying the isometry condition
with a minus sign.}{H2plots/P-2.pdf}

Consider now the total energy of the system.  If the two holes are at rest
infinitely far apart, then $E/M$ should be {\it two}.  Thus, any configuration
with $E/M < 2$ should be gravitationally bound.  Plot~(a) in Figures~14.1 and
14.3 shows that if the two holes are far enough apart and have large enough
momenta, then it is energetically feasible (if the momenta are directed away
from each other) for the two holes to be unbound.  One must keep in mind,
however, that some of the excess energy may be in the form of gravitational
radiation, as in the case of a single boosted hole.

Examining Plot~(b) in Figures~14.1 and 14.3, we find that the conformal
separation parameter $\beta$ has a strong correlation with the physical
separation to mass ratio $L/M$.  For each of the ten values of $\beta$, $L/M$ is
surprisingly constant for all values of $P/M$.  This is somewhat unexpected
given the nonlinearity of Einstein's equations.

If the two holes do coalesce, then Plot~(c) in Figures~14.1 and 14.2 shows the
scaled maximum amount of energy  which can be emitted by gravitational
radiation.  As expected, $E_{rad}/M$ increases as both separation and momentum
increase.  An unusual feature is seen for $\beta=3$.  In this case, the energy
available for radiation is negative.  To understand this behavior we note that
if two Schwarzschild holes, infinitely separated and initially at rest, are
allowed to coalesce with the maximum release of energy, then $E/M$ for the
resulting hole is $\sqrt2$.  We expect, then, that an apparent horizon will
form around the two holes when their separation gives an $E/M$ value near
$\sqrt2$.  This corresponds to a value of $\beta\sim3.6$ and corresponds to
$L/E\sim2.2$, which is in rough agreement with the numerical results of Smarr
{\it et al}. [1976].  The negative value of $E_{\scriptscriptstyle rad}/M$ for
$\beta�=�3$ {\it must} stem from the inappropriate use of equation (14.3) when
the two holes are surrounded by a third apparent horizon.

Let us move now to the case of two holes with parallel linear momenta
represented in Cases~2 and 4.  The plots displaying the physical content of
these two cases are found respectively in Figures~14.2 and 14.4.  As for
Cases~1 and 3, we find that the initial data generated from the two choices of
the isometry condition are qualitatively the same although there are
discernible differences.  Again, we find that the mass of the apparent horizon
in Case~4 (minus isometry) must increase from its current estimate by a factor
larger than the increase in Case~2 if the two cases are to represent different
slicings of the same global spacetime.

Examining the behavior of the total energy in Plot~(a) of Figures~14.2 and 14.4
shows the gross behavior of a single boosted particle.  At lower values of the
momentum, the lowering of the total energy, due to increased binding energy as
the holes move closer, is also clearly visible.  Plot~(b) shows that the $L/M$
ratio is again fairly insensitive to $P/M$.  Plots~(c) and (d) show little new
information about the radiation content of the initial slice beyond that
already seen for a single boosted hole.  The primary difference results from
the effects of binding energy.

The most interesting of the cases to examine are Cases~5 and 6 which involve
two holes with, respectively, anti-parallel and parallel angular momentum
vectors.  The behavior of the scaled total energy in the slice, seen in
Plot~(a) in Figures~14.5 and 14.6, is surprising and unintuitive.  While it
clearly shows the effects of increased binding energy as the holes are brought
closer together, it also indicates that the total energy-to-mass ratio
decreases as the angular momenta increase, with the decrease being most
dramatic when the spins are anti-parallel.

%\figlabel{7.25truein}{Figure~14.5:  (a)~total energy, (b)~separation,
%(c)~maximum radiation energy,\cr and (d)~maximum radiation efficiency for two
%holes with angular momenta $S$\cr aligned anti-parallel to each other and
%generated by an inversion-symmetric\cr extrinsic curvature obeying the isometry
%condition with a minus sign.}
\figlabelpdf{7.25truein}{Figure~14.5:  (a)~total energy, (b)~separation,
(c)~maximum radiation energy,\cr and (d)~maximum radiation efficiency for two
holes with angular momenta $S$\cr aligned anti-parallel to each other and
generated by an inversion-symmetric\cr extrinsic curvature obeying the isometry
condition with a minus sign.}{H2plots/S-.pdf}

Examining the $L/M$ ratio in Plot~(b) in Figures~14.5 and 14.6 shows that it is
much more strongly affected by the spin of the holes than by their linear
momenta.  Lines of constant $\beta$ can no longer be considered to represent a
constant $L/M$ ratio.  This fact complicates the interpretation of the total
energy of the slice seen in Plot~(a) in Figures~14.5 and 14.6.  Because the
holes are getting closer, for constant $\beta$ as the angular momenta is
increased, they necessarily have a greater binding energy and the decrease in
$E/M$ must be due, in part, to this effect.  However, rough interpolations along
lines of constant $L/M$ shows clearly that the decrease in separation is not the
dominant contribution to the decrease in $E/M$.  It is possible that the
Christodoulou mass is not the correct quantity with which to scale the total
energy and separation.  On the other hand, its use in scaling the angular
momentum does have the correct limiting behavior as the angular momentum is
increased.  That is, $S/M^2$ appears to be asymptotic to {\it one}, which is the
``Kerr'' limit.

%\figlabel{7.25truein}{Figure~14.6:  (a)~total energy, (b)~separation,
%(c)~maximum radiation energy,\cr and (d)~maximum radiation efficiency for two
%holes with angular momenta $S$\cr aligned parallel to each other and generated
%by an inversion-symmetric\cr extrinsic curvature obeying the isometry condition
%with a minus sign.}
\figlabelpdf{7.25truein}{Figure~14.6:  (a)~total energy, (b)~separation,
(c)~maximum radiation energy,\cr and (d)~maximum radiation efficiency for two
holes with angular momenta $S$\cr aligned parallel to each other and generated
by an inversion-symmetric\cr extrinsic curvature obeying the isometry condition
with a minus sign.}{H2plots/S-2.pdf}

Moving on to examine the radiation content of the slices, consider first
Plots~(c) and (d) in Figure~14.5.  We see, in the case of two holes with
anti-parallel spins, that the amount of energy which is potentially available
for radiation does increase initially as the angular momentum is increased. 
However, after a certain point, the available energy rapidly diminishes with
increased angular momentum.  These effects are seen not only in Plot~(c), which
is scaled with respect to the Christodoulou mass, but also in Plot~(d), which
shows the ratio of the maximum radiation content to the total energy.  This
last ratio is independent of the choice of a scaling parameter and indicates
that the drop in total energy is a physical effect independent of the choice of
the scaling parameter.  Examining Plots~(c) and (d) in Figure~14.6 shows, in
the case of two holes with parallel spins, that the amount of energy which is
potentially available for radiation decreases continuously as the angular
momentum is increased.

%\figlabel{6.625truein}{Figure~14.7:  Comparison of the scaled total energy for
%two holes with\cr angular momenta $S$ aligned anti-parallel and parallel to
%each other\cr showing the spin-spin interaction.}
\figlabelpdf{6.625truein}{Figure~14.7:  Comparison of the scaled total energy for
two holes with\cr angular momenta $S$ aligned anti-parallel and parallel to
each other\cr showing the spin-spin interaction.}{H2plots/Scmp.pdf}

The decrease in the total energy as the spin of the holes (either parallel or
anti-parallel) is increased may indicate that there is an additional attractive
interaction between the holes which is a result of the spins and is independent
of the orientation of the spins.  Such an attractive force would add to the
binding energy of the configuration and decrease the total energy.  Beyond this,
there is clear evidence of an orientation-dependent, spin-spin interaction
between the two holes.  Figure~14.7 shows an overlay of Plot~(a) from
Figures~14.5 and 14.6 which compares the total energy of the slice for the
parallel and anti-parallel oriented spins.  Even though the sequences of
constant $\beta$ do not represent a constant $L/M$ ratio, the $L/M$ curves for
constant $\beta$ are nearly identical for both the parallel and anti-parallel
spin scenarios.  This allows us to directly compare the $E/M$ ratio in the two
cases for the same values of $\beta$.  For each value of $\beta$, we find that
the $E/M$ ratio is lower when the spins are oriented anti-parallel than when
parallel.  This indicates that there is a spin-spin force which is attractive
when the spins are anti-aligned and repulsive when aligned.  Such an
interaction is in agreement with the gravitational spin-spin interaction
investigated by Wald [1972] for the case of a spinning test particle in the
exterior field of an arbitrary, stationary, rotating source.  We can also see
that the magnitude of the spin-spin force decreases as the separation between
the holes increases.  This is apparent from the fact that the difference in the
energy levels for the aligned and anti-aligned cases decreases as the
separation increases.

As evidenced by the discussion above, the understanding of the content of the
two-hole initial-data sets is far from complete.  Many uncertainties should be
resolved when the location and areas of the apparent horizons are determined. 
This knowledge will allow for much more accurate estimates of the irreducible
mass of the holes and will greatly change many of the estimates for the
radiation content of the initial-data slices since the current method of
estimating the radiation content assumes that there is not a third horizon
surrounding both holes.  In any case, it will probably be the case that a
complete understanding of the content of these two-hole initial-data sets will
be found only by evolving them in time.
\vfill
\eject
